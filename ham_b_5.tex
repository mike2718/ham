\chapter{避免和解决射频干扰}


\noindent\textbf{问题:}我国业余电台应该遵守的关于电磁辐射污染的具体管理规定文件为: 
\begin{enumerate}[label=\Alph*), leftmargin=3em]
\item 美国FCC《射频电磁场人员暴露准则的测评方法规》
\item 国家标准GB/T 51391《通信工程建设环境保护技术标准》
\item 《业余电台管理办法》
\item 国际非电离辐射防护委员会《限制时变电场和磁场暴露的导则》
\end{enumerate}
\noindent\textbf{解说:}我国业余电台应该遵守的关于电磁辐射污染的具体管理规定文件为\textbf{国家标准GB/T 51391《通信工程建设环境保护技术标准》}。\\\textbf{答案:}B



\bigskip


\noindent\textbf{问题:}按照我国国家标准GB/T 51391《通信工程建设环境保护技术标准》,向没有屏蔽空间发射电磁场可以免于管理的有: 
\begin{enumerate}[label=\Alph*), leftmargin=3em]
\item 发射频率30MHz以下的所有业余电台
\item 发射频率30MHz以上的所有业余电台
\item 等效辐射功率在0.1-3MHz不大于300W、在3MHz-300GHz不大于100瓦的无线通信设施(设备) 
\item 所有业余电台
\end{enumerate}
\noindent\textbf{解说:}按照我国国家标准《电磁辐射防护规定》,可以免于管理的电磁辐射体(豁免水平)为\textbf{输出功率不大于15W的移动式无线电通信设备,以及等效辐射功率在0.1-3MHz不大于300W、在3MHz-300GHz不大于100瓦的辐射体}。\\\textbf{答案:}C



\bigskip


%\noindent\textbf{问题:}按照我国国家标准《电磁辐射防护规定》,凡其功率大于豁免水平(3MHz以上等效辐射功率100瓦)的一切电磁波辐射体的所有者,必须:
%\begin{enumerate}[label=\Alph*), leftmargin=3em]
%\item 业余电台可不受环境保护部门的监督管理
%\item 向所在地区的无线电管理机构交纳环境电磁辐射监测报告
%\item 向所在地区的环境保护部门缴纳环境保护费
%\item 向所在地区的环境保护部门申报、登记,并接受监督
%\end{enumerate}
%\noindent\textbf{解说:}按照我国国家标准《电磁辐射防护规定》,凡其功率大于豁免水平(3MHz以上等效辐射功率100瓦)的一切电磁波辐射体的所有者,必须\textbf{向所在地区的环境保护部门申报、登记,并接受监督}。\\\textbf{答案:}D
%
%
%\bigskip


%\noindent\textbf{问题:}按照我国国家标准《电磁辐射防护规定》,负责对超过豁免水平(3MHz以上等效辐射功率100瓦)电磁波辐射体所在的场所以及周围环境的电磁辐射水平进行监测的是:
%\begin{enumerate}[label=\Alph*), leftmargin=3em]
%\item 所在地区无线电管理机构
%\item 所在地区业余无线电协会
%\item 所在地区环境保护部门
%\item 其拥有者
%\end{enumerate}
%\noindent\textbf{解说:}按照我国国家标准《电磁辐射防护规定》,\textbf{其拥有者}负责对超过豁免水平(3MHz以上等效辐射功率100瓦)电磁波辐射体所在的场所以及周围环境的电磁辐射水平进行监测。\\\textbf{答案:}D
%
%
%
%\bigskip


%\noindent\textbf{问题:}按照我国国家标准《电磁辐射防护规定》,当监测到超过豁免水平的电磁辐射体使环境电磁辐射水平超过规定的限值时,必须:
%\begin{enumerate}[label=\Alph*), leftmargin=3em]
%\item 立即向无线电管理机构报告
%\item 立即向当地城管机构报告
%\item 尽快采取措施降低辐射水平,同时向环境保护部门报告
%\item 立即到环境保护部门缴纳电磁污染治理费
%\end{enumerate}
%\noindent\textbf{解说:}按照我国国家标准《电磁辐射防护规定》,当监测到超过豁免水平的电磁辐射体使环境电磁辐射水平超过规定的限值时,必须\textbf{尽快采取措施降低辐射水平,同时向环境保护部门报告}。\\\textbf{答案:}C
%
%
%\bigskip


%\noindent\textbf{问题:}按照我国国家标准《电磁辐射防护规定》,对超过豁免水平的电磁辐射体的环境电磁辐射水平监测应在下列地点进行:
%\begin{enumerate}[label=\Alph*), leftmargin=3em]
%\item 在发射机射频输出端口进行测量
%\item 在距辐射体天线2000米以内选择任意位置测量
%\item 在距辐射体天线100米以内最小辐射方向上选点测量
%\item 在距辐射体天线2000米以内最大辐射方向上选点测量
%\end{enumerate}
%\noindent\textbf{解说:}按照我国国家标准《电磁辐射防护规定》,对超过豁免水平的电磁辐射体的环境电磁辐射水平监测应\textbf{在距辐射体天线2000米以内最大辐射方向上选点测量}。\\\textbf{答案:}D
%
%
%
%\bigskip


%\noindent\textbf{问题:}我国国家标准《电磁辐射防护规定》所规定的电磁辐射防护限值的公众照射基本限值,其基本计量方法规是:
%\begin{enumerate}[label=\Alph*), leftmargin=3em]
%\item 一天24小时内任意6分钟内全身平均的比吸收率(SAR)应小于每公斤体重限值
%\item 任何时刻的全身平均的比吸收率(SAR)都应小于每公斤体重限值
%\item 任何时刻的瞬时辐射场强都应小于与频率无关的固定限值
%\item 任何时刻的瞬时辐射场强都应小于与频率有关的限值
%\end{enumerate}
%
%\bigskip


\noindent\textbf{问题:}我国国家标准《电磁环境控制限值》规定公众暴露控制限值中,对环境电磁辐射场强任意连续6分钟内的方均根值要求最严格的频率范围为: 
\begin{enumerate}[label=\Alph*), leftmargin=3em]
\item 300kHz-30MHz
\item 3MHz- 30MHz
\item 30MHz- 3GHz
\item 1200MHz-30GHz
\end{enumerate}
\noindent\textbf{解说:}我国国家标准《电磁环境控制限值》规定公众暴露控制限值中,对环境电磁辐射场强任意连续6分钟内的方均根值要求最严格的频率范围为\textbf{30MHz- 3GHz}。\\\textbf{答案:}C


\bigskip


\noindent\textbf{问题:}为什么电磁辐射防护规定国家标准中的照射限值随着频率的变化而不同?
\begin{enumerate}[label=\Alph*), leftmargin=3em]
\item 较低频率的无线电波相对高频率的无线电波拥有更高的能量
\item 人体会对某些特定频率的电磁波吸收量更大
\item 在自然中高频电磁波不常见
\item 较低频率的无线电波不会穿透人体
\end{enumerate}
\noindent\textbf{解说:}因为\textbf{人体会对某些特定频率的电磁波吸收量更大}。\\\textbf{答案:}B

\bigskip


\noindent\textbf{问题:}要防止HF发射机的杂散发射干扰天线附近的VHF电视机,应该发射机和天线之间串联:
\begin{enumerate}[label=\Alph*), leftmargin=3em]
\item 截止频率为300MHz左右的低通滤波器
\item 截止频率为30MHz左右的高通滤波器
\item 中心频率为30MHz左右的带通滤波器
\item 截止频率为30MHz左右的低通滤波器
\end{enumerate}
\noindent\textbf{解说:}要防止HF发射机的杂散发射干扰天线附近的VHF电视机,应该发射机和天线之间串联\textbf{截止频率为30MHz左右的低通滤波器}。\\\textbf{答案:}D


%%%???

\bigskip


\noindent\textbf{问题:}假设中继台的收、发信机共用天线,上下行频率分别为F1和F2。要防止中继台发射机对接收机产生干扰,应该对中继台设备采取下列措施:
\begin{enumerate}[label=\Alph*), leftmargin=3em]
\item 在发信机与天线间串联中心频率为F1的带阻滤波器,在收信机与天线间串接中心频率为F1的带阻滤波器
\item 在发信机与天线间串联中心频率为F2的带阻滤波器,在收信机与天线间串接中心频率为F2的带阻滤波器
\item 在发信机与天线间串联中心频率为F1的带阻滤波器,在收信机与天线间串接中心频率为F2的带阻滤波器
\item 在发信机与天线间串联中心频率为F1的带通滤波器,在收信机与天线间串接中心频率为F2的带通滤波器
\end{enumerate}
\noindent\textbf{解说:}应该\textbf{在中继台设备的发信机与天线间串联中心频率为F1的带阻滤波器,在收信机与天线间串接中心频率为F2的带阻滤波器}。\\\noindent\textbf{答案:}C


\bigskip


\noindent\textbf{问题:}要防止业余HF发射机的杂散发射干扰天线附近的电话机,应该在电话机和电话线之间之间串联:
\begin{enumerate}[label=\Alph*), leftmargin=3em]
\item 截止频率约为3kHz的高通滤波器
\item 截止频率约为3kHz的带阻滤波器
\item 截止频率不高于1MHz的低通滤波器
\item 中心频率约为3kHz的带通滤波器
\end{enumerate}
\noindent\textbf{解说:}要防止业余HF发射机的杂散发射干扰天线附近的电话机,应该在电话机和电话线之间之间串联\textbf{截止频率不高于1MHz的低通滤波器}。\\\noindent\textbf{答案:}C



\bigskip


\noindent\textbf{问题:}A、B两部HF业余电台相距很近,分别工作在A、B两个频段。为减少B电台受到来自A电台的干扰,可以在B电台与天线之间串联:
\begin{enumerate}[label=\Alph*), leftmargin=3em]
\item 中心频率为A的带阻滤波器
\item 截止频率为B的高通滤波器
\item 截止频率为A的高通滤波器
\item 中心频率为A的带通滤波器
\end{enumerate}
\noindent\textbf{解说:}可以在B电台与天线之间串联\textbf{中心频率为A的带阻滤波器}。\\\noindent\textbf{答案:}A


\bigskip


\noindent\textbf{问题:}为了减少发射设备的谐波干扰近在咫尺的接收机,可以在发射设备和天线之间串联一个LC低通滤波器。正确的说法是:
\begin{enumerate}[label=\Alph*), leftmargin=3em]
\item 滤波器的阶数越低,抑制倍频干扰的效果越好
\item 滤波器的阶数越高,抑制倍频干扰的效果越好
\item 滤波器的阶数越高,损耗的功率越小
\item 滤波器的阶数越高,耐受的功率越大
\end{enumerate}
\noindent\textbf{解说:}正确的说法是:\textbf{滤波器的阶数越高,抑制倍频干扰的效果越好}。\\\noindent\textbf{答案:}B



\bigskip


\noindent\textbf{问题:}架设业余中继台前应确定台址附近没有能与中继台下行频率形成三阶互调的发射台。如果中继台的上、下行频率分别为fR和fT,可能造成这种三阶互调的干扰频率fX是:
\begin{enumerate}[label=\Alph*), leftmargin=3em]
\item fT - fR 或 fT + fR
\item 2fT - fR 或 (fT + fR ) / 2
\item 2fT 或 2fR
\item 2(fT - fR) 或 2(fT + fR )
\end{enumerate}
\noindent\textbf{解说:}可能造成这种三阶互调的干扰频率fX是\textbf{2fT - fR 或 (fT + fR ) / 2}。\\\noindent\textbf{答案:}B



\bigskip


\noindent\textbf{问题:}如果别人报告说你的发射干扰了相邻频率的通信,此时你应当做的是:
\begin{enumerate}[label=\Alph*), leftmargin=3em]
\item 加大发射功率
\item 换用另一种调制模式工作
\item 将这种情况通知你的设备制造商
\item 检查发射机的频率指示是否准确、发射机的杂散发射指标是否合格
\end{enumerate}
\noindent\textbf{解说:}如果别人报告说你的发射干扰了相邻频率的通信,此时你应当做的是\textbf{检查发射机的频率指示是否准确、发射机的杂散发射指标是否合格}。\\\noindent\textbf{答案:}D



\bigskip


\noindent\textbf{问题:}滤去杂散发射的滤波器应该安装在什么地方?
\begin{enumerate}[label=\Alph*), leftmargin=3em]
\item 发信机和天线之间
\item 话筒上
\item 收信机和发信机之间
\item 电台的电源处
\end{enumerate}
\noindent\textbf{解说:}滤去杂散发射的滤波器应该安装在\textbf{发信机和天线之间}。\\\noindent\textbf{答案:}A


\bigskip


\noindent\textbf{问题:}在解决电视接收机被附近的144MHz业余电台的过载干扰问题的时候,应当先尝试什么措施?
\begin{enumerate}[label=\Alph*), leftmargin=3em]
\item 在电视接收机的天线端子前安装144MHz带阻滤波器
\item 在业余电台的射频输出端安装144MHz带阻滤波器
\item 在业余电台的射频输出端安装144MHz带通滤波器
\item 在电视接收机的天线端子前安装144MHz带通波器
\end{enumerate}
\noindent\textbf{解说:}应当先尝试\textbf{在电视接收机的天线端子前安装144MHz带阻滤波器}。\\\noindent\textbf{答案:}A



\bigskip


\noindent\textbf{问题:}在汽车上安装的移动电台中能听到的随着引擎转速变化的高频啸叫声的来源是?
\begin{enumerate}[label=\Alph*), leftmargin=3em]
\item 火花塞系统
\item 防抱死刹车装置的控制器
\item 发电机
\item 电动油泵
\end{enumerate}
\noindent\textbf{解说:}在汽车上安装的移动电台中能听到的随着引擎转速变化的高频啸叫声的来源是\textbf{发电机}。\\\noindent\textbf{答案:}C



\bigskip


\noindent\textbf{问题:}移动车载台的直流电源负极应当接在哪里?
\begin{enumerate}[label=\Alph*), leftmargin=3em]
\item 连接在固定住电台的挂置架上
\item 连接在天线座上
\item 连接在电池的负极或发动机的接地带
\item 可以连接在汽车的任意的金属部分
\end{enumerate}
\noindent\textbf{解说:}移动车载台的直流电源负极应当接在连接在\textbf{电池的负极或发动机的接地带}。\\\noindent\textbf{答案:}C



\bigskip


\noindent\textbf{问题:}下列哪一项可以有效减小火花塞干扰?
\begin{enumerate}[label=\Alph*), leftmargin=3em]
\item 将频率稍稍偏离一些
\item 打开电台的抑噪(NB)功能
\item 降低静噪(SQL)阀值
\item 调节电台的RIT旋钮
\end{enumerate}
\noindent\textbf{解说:}\textbf{打开电台的抑噪(NB)功能}可以有效减小火花塞干扰。\\\noindent\textbf{答案:}B



\bigskip


\noindent\textbf{问题:}一般来说,如果要解决发射机对附近有线电话的干扰,最先做的应当是:
\begin{enumerate}[label=\Alph*), leftmargin=3em]
\item 在有线电话进线处安装射频滤波器
\item 在发射机射频输出端安装高通滤波器
\item 改善发射机的接地情况
\item 在发射机射频输出端安装低通滤波器
\end{enumerate}
\noindent\textbf{解说:}一般来说,如果要解决发射机对附近有线电话的干扰,最先做的应当是\textbf{在有线电话进线处安装射频滤波器}。\\\noindent\textbf{答案:}A



\bigskip


\noindent\textbf{问题:}下列哪一种方法可以用来定位无线电噪音源或者恶意干扰源?
\begin{enumerate}[label=\Alph*), leftmargin=3em]
\item 回波定位
\item 相位锁定
\item 无线电测向
\item 多普勒雷达
\end{enumerate}
\noindent\textbf{解说:}\textbf{无线电测向}可以用来定位无线电噪音源或者恶意干扰源。\\\noindent\textbf{答案:}C




