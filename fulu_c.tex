\chapter{附录}

\section{字母解释法(节录)}

\begin{tabular}{cc}%
\begin{tabular}[t]{|c|l|}
	\hline
	\textbf{字母} & \textbf{单词} \\
	\hline
	A & Alfa \\
	\hline
	B & Bravo \\
	\hline
	C & Charlie \\
	\hline
	D & Delta \\
	\hline
	E & Echo \\
	\hline
	F & Foxtrot \\
	\hline
	G & Golf \\
	\hline
	H & Hotel \\
	\hline
	I & India \\
	\hline
	J & Juliett \\
	\hline
	K & Kilo \\
	\hline
	L & Lima \\
	\hline
	M & Mike \\
	\hline
	N & November \\
	\hline
	O & Oscar \\
	\hline
	P & Papa \\
	\hline
	Q & Quebec \\
	\hline
	R & Romeo \\
	\hline
	S & Sierra \\
	\hline
	T & Tango \\
	\hline
	U & Uniform \\
	\hline
	V & Victor \\
	\hline
	W & Whiskey \\
	\hline
	X & X-ray \\
	\hline
	Y & Yankee \\
	\hline
	Z & Zulu \\
	\hline
\end{tabular} &
\begin{tabular}[t]{|c|l|}
	\hline
	\textbf{字母} & \textbf{单词} \\
	\hline
	0 & Zero \\
	\hline
	1 & One \\
	\hline
	2 & Two \\
	\hline
	3 & Three \\
	\hline
	4 & Four \\
	\hline
	5 & Five \\
	\hline
	6 & Six \\
	\hline
	7 & Seven \\
	\hline
	8 & Eight \\
	\hline
	9 & Nine \\
	\hline
\end{tabular} \tabularnewline
\end{tabular}

\newpage

\section{常用Q简语(节录)}

\begin{longtable}{|l|l|l|}
	\hline
	 & \textbf{提问} & \textbf{回答或建议} \\
	\hline
	QRL & 你正忙着吗 & 我正忙着 \\
	\hline
	QRM & 你遇到他台干扰吗 & 我遇到他台干扰 \\
	\hline
	QRN & 你遇到天电干扰吗 & 我遇到天电干扰 \\
	\hline
	QRO & 要我增加功率吗 & \\
	\hline
	QRP & 要我减小功率吗 & \\
	\hline
	QRQ & 要我加快发送速度吗 & 请加快发送速度 \\
	\hline
	QRS & 要我减慢发送速度吗 & 请减慢发送速度 \\
	\hline
	QRT & 要我停止发送吗 & 请停止发送 \\
	\hline
	QRU & 你和我还有事吗 & 我和你无事了 \\
	\hline
	QRV & 你是否已准备好 & 我已准备好 \\
	\hline
	QRZ & 谁在呼叫我 & \\
	\hline
	QSA & 我的信号强度如何 & 你的信号强度为 × 级(1-5 级) \\
	\hline
	QSB & 我的信号有衰落吗 & 你的信号有衰落 \\
	\hline
	QSD & 我发报的手法有毛病吗 & 你发报的手法有毛病 \\
	\hline
	\multirow{2}{1em}{QSK} & 能在你的信号间隙中接收吗 & 我在发射的信号间隙中接收 \\
	& (即 QSK 插入方式)    & (即 QSK 插入方式) \\
	\hline
	QSL & 你能给我收据(或 QSL 卡片)吗 & 我给你收据(QSL 卡片)、我已收妥 \\
	\hline
	QSO & 你能直接和 ××× 电台通信吗 & 我能直接和 ××× 电台通信 \\
	\hline
	QSP & 你能传信到 ××× 电台吗 & 我能传信到 ××× 电台 \\
	\hline
	\multirow{2}{1em}{QSX} & 你将在 nnnn \si{\kHz}(或 \si{\MHz})频率 & 我将在 nnnn \si{\kHz}(或 \si{\MHz})频率 \\
	    & 守听 ××× 电台吗 & 守听 ××× 电台 \\
	\hline
	QSY & 要我将频率改到 nnnn 频率吗 & 请将频率改到 nnnn 频率 \\
	\hline
	QTH & 你的电台位置在哪里 & 我的电台位置是 ×××× \\
	\hline
\end{longtable}

\newpage



\section{业余通信常用缩语表(节录)}

说明:黑体字的缩语为考到的重要缩语。

\begin{longtable}[l]{llll}
& \textbf{缩语} & \textbf{原词} & \textbf{含义} \\
& \endfirsthead
& \textbf{缩语} & \textbf{原词} & \textbf{含义} \\
& \endhead
A & \textbf{ABT} & About & 大约 \\
& AC & Alternating current & 交流 \\
& \textbf{ADR} & Address & 地址 \\
& \textbf{ADDR} & Address & 地址 \\
& \textbf{AGC} & Automatic gain control & 收信机自动增益控制 \\
& \textbf{AGN} & Again & 再、再来一次 \\
& \textbf{AHR} & Another & 另一个 \\
& \textbf{ALC} & Automatic level control & 发信自动电平控制 \\
& \textbf{AM} & Amplitude modulation & 幅度调制 \\
& \textbf{ANT} & Antenna & 天线 \\
& \textbf{ARDF} & Amateur radio direction finding & 业余无线电测向 \\
& $\overline{\mathbf{A}\mathbf{S}}$ & & 请稍等 \\
& \textbf{AS} & & 亚洲、如同 \\
& ASK & Amplitude-shift keying & 移幅键控 \\
& \textbf{AT} & Antenna tuner & 自动天线调谐 \\
& \textbf{ATT} & Attenuated / Attenuator & 衰减/收信机输入衰减器 \\
& \textbf{ATU} & Antenna tuning unit & 天线调谐器 \\
B & \textbf{BALUN} & balanced to unbalanced / balancing unit & 平衡和不平衡/平衡单元 \\
& \textbf{BEAM} & Beam antenna & 定向天线 \\
& \textbf{BEST} & & 最好的 \\
& \textbf{BJT} & Beijing Time & 北京时间 \\
& \textbf{BK} & Break & 插入、打断 \\
& BREAK & & 我还没有说完 \\
& BREAK BREAK & & 我还没有说完 \\
& Break in & & 能在你的信号间隙中接收吗 \\
& \textbf{BURO} & QSL Bureau & QSL卡片管理局 \\
C & \textbf{C} & & 遇到、见面 \\
& \textbf{CFM} & Confirm & 确认 \\
& \textbf{CHEERIO} & & 再会、祝贺 \\
& \textbf{CL} & Close, call & 关闭(或呼叫) \\
& \textbf{CLG} & Calling & 呼叫 \\
& \textbf{CLS} & Call sign & 呼号 \\
& \textbf{CPI} & Copy & 抄收 \\
& CQ & & 普遍呼叫 \\
& \textbf{CW} & Continuous wave & 等幅电报 \\
& \textbf{CTCSS} & Continuous Tone-Coded Squelch System & 亚音调静噪 \\
D & \textbf{DATE} & & 日期 \\
& dB & decibel & 增益(单位) \\
& DC & Direct current & 直流 \\
& DR & Dear & 亲爱的 \\
& DTMF & Dual-tone multi-frequency signaling & 双音多频编码 \\
& \textbf{DP} & Dipole antenna & 偶级天线 \\
E & \textbf{EHF} & Extremely high frequency & 极高频 \\
& \textbf{EL} & & 单元(常用于天线振子) \\
& \textbf{ELE} & & 单元(常用于天线振子) \\
& \textbf{ELS} & & 单元(常用于天线振子) \\
& \textbf{ES} & & 和 \\
F & FAX & Facsimile & 传真 \\
& \textbf{FB} & Fine business & 很好的 \\
& \textbf{FER} & & 为了,对于 \\
& \textbf{FINE} & & 好的,精细的 \\
& \textbf{FM} & Frequency modulation & 频率调制、调频 \\
& \textbf{FR} & & 为了,对于 \\
& \textbf{FREQ} & Frequency & 频率 \\
& \textbf{FSK} & Frequency-shift keying & 移频键控 \\
G & GA & Go ahead & 继续、请过来 \\
& \textbf{GA} & Good afternoon & 下午好 \\
& \textbf{GB} & Goodbye & 再见 \\
& \textbf{GE} & Good evening & 晚上好 \\
& \si{\GHz} & gigahertz & 吉赫兹(单位) \\
& \textbf{GL} & Good luck & 好运气 \\
& \textbf{GLD} & Glad & 高兴 \\
& \textbf{GM} & Good morning & 早晨好 \\
& \textbf{GMT} & Greenwich Mean Time & 格林威治时间 \\ %格林尼治标准时间
& \textbf{GN} & Good night & 晚安 \\
& \textbf{GND} & Ground & 地线、地面 \\
& \textbf{GP} & Ground-plane antenna & 垂直接地天线 \\
H & HF & High frequency & 高频、即SW(Shortwave) \\
& \textbf{HPE} & Hope & 希望 \\
& \textbf{HPI} & Happy & 幸福 \\
& \textbf{HPY} & Happy & 幸福 \\
& \textbf{HR} & Here & 这里 \\
& \textbf{HR} & Hear & 听到 \\
& HST & High-speed telegraphy & 快速收发报 \\
& \textbf{HW} & How & 怎样、如何 \\
& Hz & hertz & 赫兹(单位) \\
I & \textbf{ITA1} & Baudot code & 博多码、ITA2的前身 \\
& \textbf{ITA2} & & 国际2号电报码 \\
K & \si{\kHz} & kilohertz & 千赫兹(单位) \\
& KP & & 收听 \\
L & \textbf{LCD} & Liquid-crystal display & 液晶显示器 \\
& \textbf{LED} & Light-emitting diode & 发光二极管 \\
& \textbf{LP} & Log-periodic antenna & 对数周期天线 \\
& \textbf{LW} & Long whip antenna & 长线天线 \\
M & \si{\MHz} & megahertz & 兆赫兹(单位) \\
& \textbf{MNI} & Many & 很多 \\
& \textbf{MNY} & Many & 很多 \\
& \textbf{MODE} & & 方式 \\
& \textbf{MTRS} & Meters & 米 \\
& MUF & Maximum usable frequency & 最高可用频率 \\
N & \textbf{NAME} & & 名字 \\
& \textbf{NB} & Noise blanker & 抑噪 \\
& NFM & Narrowband FM & 窄带调频 \\
& \textbf{NICE} & & 良好的 \\
& \textbf{NW} & Now & 现在 \\
O & \textbf{OM} & Old man & 老朋友 \\%朋友
& \textbf{OP} & Operator & 操作员 \\
& \textbf{OPR} & Operator & 操作员 \\
& OVER & & 我的传输已结束,期待回复 \\
P & \textbf{PM} & Phase modulation & 相位调制 \\
& \textbf{P O BOX} & Post office box & 邮政信箱 \\
& \textbf{PRE} & Preamplifier & 收信机前置放大器 \\
& \textbf{PROC} & Processor & 发信语音压缩 \\
& \textbf{PSK} & Phase-shift keying & 相位偏移调制 \\
& PSK31 & Phase Shift Keying, 31 Baud & 移相键控、31波特 \\
& \textbf{PTT} & Push-to-talk & 按键发射 \\
& \textbf{PWR} & Power & 功率 \\
Q & QPSK31 & Phase Shift Keying, 31 Baud & 移相键控、31波特 \\
R & \textbf{RCVR} & Receiver & 收信机 \\
& \textbf{RF} & Radio frequency & 无线电波 \\
& \textbf{RIG} & Equipment & 电台设备 \\
& \textbf{RIT} & Receiver incremental tuning & 接收增量调谐 \\
& \textbf{RMKS} & Remarks & 备注、注释 \\
& \textbf{ROGER} & & 回答起始语,相当于“明白”, \\
& &  & 仅在已完全抄收对方刚才发送的信息时使用 \\
& \textbf{RPRT} & Report & 报告 \\
& RST & Readability, Strength, Tone & 可辨度、信号强度、音调系统 \\
& \textbf{RTTY} & Radioteletype & \\
& \textbf{RX} & Receiver & 收信机 \\
S & \textbf{SASE} & & 写好收信人地址的信封 \\
& \textbf{SHF} & Super high frequency & 超高频 \\
& \textbf{SID} & Sudden ionospheric disturbance & 突发电离层扰动 \\
& $\overline{\mathbf{S}\mathbf{K}}$ & & 结束通信 \\
& \textbf{SQL} & Squelch & 静噪 \\
& \textbf{SRI} & Sorry & 对不起 \\
& \textbf{SRY} & Sorry & 对不起 \\
& \textbf{SSB} & Single-sideband modulation & 单边带调制 \\
& \textbf{SSN} & Sunspot number & 太阳黑子平均数 \\
& \textbf{SSTV} & Slow-scan television & 慢扫描电视 \\
& \textbf{STN} & Station & 电台 \\
& STANDBY & & 等待你的呼叫 \\
& \textbf{SURE} & & 确实 \\
& \textbf{SWL} & Short waves listener & 短波收听者 \\
T & \textbf{TEMP} & Temperature & 温度 \\
& THIS IS & & 此消息发自以下xxxx电台 \\
& TIME & & 时间 \\
& \textbf{TKS} & Thanks & 谢谢 \\
& \textbf{TNX} & Thanks & 谢谢 \\
& \textbf{TU} & Thank you & 谢谢你 \\
& \textbf{TX} & Transmitter & 发信机 \\
U & \textbf{UHF} & Ultra high frequency & 特高频 \\
& \textbf{UR} & Your & 你的,你是 \\
& \textbf{UTC} & Coordinated Universal Time & 世界协调时 \\%协调世界时
V & \textbf{VER} & Vertical antenna & 垂直天线 \\
& \textbf{VHF} & Very high frequency & 甚高频 \\
& \textbf{VIA} & & 经、由 \\
& \textbf{VOX} & Voice-operated switch & 发信机声控 \\
& \textbf{VY} & Very & 很、非常 \\
W & WFM & Wideband FM & 宽带调频 \\
& \textbf{WK} & Week & 星期 \\
& \textbf{WK} & Work & 工作 \\
& \textbf{WKD} & Worked & 联络过、工作过 \\
& \textbf{WTS} & Watts & 瓦特 \\
& \textbf{WX} & Weather & 天气 \\
X & \textbf{XIT} & Transmitter Incremental Tuning & 发射增量调谐 \\
& \textbf{XMAS} & Christmas & 圣诞节 \\
& \textbf{XMTR} & Transmitter & 发信机 \\
& \textbf{XCVR} & Transceiver & 收发信机 \\
& \textbf{XYL} & Wife & 妻子、已婚女子 \\
Y & \textbf{YAGI} & Yagi–Uda antenna & 八木天线 \\
& \textbf{YL} & Young lady & 小姐、女士 \\
& \textbf{73} & & 向对方的致意、美好的祝愿 \\
& \textbf{88} & & 向对方异性操作员的致意、美好的祝愿 \\
\end{longtable}

\newpage



\section{术语表}

\begin{longtable}[l]{ll}
	\textbf{术语} & \textbf{含义} \\
	Class of emission & 发射类别 \\
	Frequency-shift telegraphy & 频移电报技术\\
	Necessary bandwidth & 必要带宽 \\
\end{longtable}

\newpage



\section{发射类别的表示方法(节录)}

\begin{tabular}{|c|c|}
	\hline
	\multicolumn{2}{|c|}{\textbf{主载波类型(第一个符号)}} \\
	\hline
	\textbf{符号} & \textbf{描述} \\
	\hline
	A & 双边带 \\
	\hline
	F & 调频 \\
	\hline
	G & 调相 \\
	\hline
	J & 单边带、抑制载波 \\
	\hline
\end{tabular}

\bigskip

\begin{tabular}{|c|c|}
	\hline
	\multicolumn{2}{|c|}{\textbf{调制主载波的信号的性质(第二个符号)}} \\
	\hline
	\textbf{符号} & \textbf{描述} \\
	\hline
	1 & 不用调制副载波但包含量化或数字信息的单个通路 \\
	\hline
	2 & 利用调制副载波且包含量化或数字信息的单个通路 \\
	\hline
	3 & 包含模拟信息的单个通路 \\
	\hline
\end{tabular}

\bigskip

\begin{tabular}{|c|c|}
	\hline
	\multicolumn{2}{|c|}{\textbf{被发送信息类型(第三个符号)}} \\
	\hline
	\textbf{符号} & \textbf{描述} \\
	\hline
	A & 电报──用于人工收听 \\
	\hline
	B & 电报──用于自动接收 \\
	\hline
	E & 电话(包括声音广播) \\
	\hline
	F & 电视(视频) \\
	\hline
\end{tabular}

\newpage

\section{无线电频段和波段表(节录)}

\begin{tabular}{|c|c|c|c|c|}
	\hline
	\textbf{频段名称} & \textbf{缩写} & \textbf{频率范围} & \textbf{波长范围} & \textbf{波段名称} \\
	%\hline
	%反正不考
	%极低频 & ELF & 3–30 Hz & 100,000–10,000 km & 极长波 \\
	%\hline
	%超低频 & SLF & 30–300 Hz & 10,000–1,000 km & 超长波 \\
	%\hline
	%特低频 & ULF & 300–3,000 Hz & 1,000–100 km & 特长波 \\
	%\hline
	%甚低频 & VLF & 3–30 \si{\kHz} & 100–10 km & 甚长波 \\
	\hline
	低频 & LF & 30–300 \si{\kHz} & 10–1 km & 长波 \\
	\hline
	中频 & MF & 300–3,000 \si{\kHz} & 1,000–100 m & 中波 \\
	\hline
	高频 & HF & 3–30 \si{\MHz} & 100–10 m & 短波 \\
	\hline
	甚高频 & VHF & 30–300 \si{\MHz} & 10–1 m & 米波 \\
	\hline
	特高频 & UHF & 300–3,000 \si{\MHz} & 1–0.1 m & 分米波 \\
	\hline
	超高频 & SHF & 3–30 \si{\GHz} & 100–10 mm & 厘米波 \\
	\hline
	极高频 & EHF & 30–300 \si{\GHz} & 10–1 mm & 毫米波 \\
	\hline
	%反正不考
	%至高频 & THF & 300–3,000 \si{\GHz} & 1–0.1 mm & 丝米波 \\
	%\hline
\end{tabular}

\newpage






\section{各业余无线电频段名称(节录)}


\begin{longtable}{|c|c|}
	\hline
	\textbf{波段名称} & \textbf{频率范围} \\
	\hline
	160米 & 1800–2000 \si{\kHz} \\
	\hline
	80米 & 3.5–4.0 \si{\MHz} \\
	\hline
	40米 & 7.0–7.3 \si{\MHz} \\
	\hline
	20米 & 14.0–14.35 \si{\MHz} \\
	\hline
	15米 & 21–21.45 \si{\MHz} \\
	\hline
	10米 & 28–29.7 \si{\MHz} \\
	\hline
	6米 & 50–54 \si{\MHz} \\
	\hline
	2米 & 144–148 \si{\MHz} \\
	\hline
	0.7米 & 420–450 \si{\MHz} \\
	\hline
\end{longtable}

\newpage








\section{ITU字头(节录)}


\begin{longtable}{|l|l|}
	\hline
	\textbf{字头范围} & \textbf{分配给国家/地区} \\
	& \endfirsthead
	\textbf{字头范围} & \textbf{分配给国家/地区} \\
	& \endhead
	\hline
	AA--AL & United States \\
	\hline
	AM--AO & Spain \\
	\hline
	AP--AS & Pakistan \\
	\hline
	AT--AW & India \\
	\hline
	AX & Australia \\
	\hline
	AY--AZ & Argentina \\
	\hline
	A2 & Botswana \\
	\hline
	A3 & Tonga \\
	\hline
	A4 & Oman \\
	\hline
	A5 & Bhutan \\
	\hline
	A6 & United Arab Emirates \\
	\hline
	A7 & Qatar \\
	\hline
	A8 & Liberia \\
	\hline
	A9 & Bahrain \\
	\hline
	B & China \\
	\hline
	B (BM-BQ, BU-BX) & Taiwan \\
	\hline
	CA--CE & Chile \\
	\hline
	CF--CK & Canada \\
	\hline
	CN & Morocco \\
	\hline
	CO & Cuba \\
	\hline
	CP & Bolivia \\
	\hline
	CQ--CU & Portugal \\
	\hline
	CV--CX & Uruguay \\
	\hline
	CY--CZ & Canada \\
	\hline
	C2 & Nauru \\
	\hline
	C3 & Andorra \\
	\hline
	C4 & Cyprus \\
	\hline
	C5 & The Gambia \\
	\hline
	C6 & The Bahamas \\
	\hline
	C7 & World Meteorological Organization \\
	\hline
	C8--C9 & Mozambique \\
	\hline
	DA--DR & Germany \\
	\hline
	DS--DT & South Korea \\
	\hline
	DU--DZ & Philippines \\
	\hline
	D2--D3 & Angola \\
	\hline
	D4 & Cape Verde \\
	\hline
	D5 & Liberia \\
	\hline
	D6 & Comoros \\
	\hline
	D7--D9 & South Korea \\
	\hline
	EA--EH & Spain \\
	\hline
	EI--EJ & Ireland \\
	\hline
	EK & Armenia \\
	\hline
	EL & Liberia \\
	\hline
	EM--EO & Ukraine \\
	\hline
	EP--EQ & Iran \\
	\hline
	ER & Moldova \\
	\hline
	ES & Estonia \\
	\hline
	ET & Ethiopia \\
	\hline
	EU--EW & Belarus \\
	\hline
	EX & Kyrgyzstan \\
	\hline
	EY & Tajikistan \\
	\hline
	EZ & Turkmenistan \\
	\hline
	E2 & Thailand \\
	\hline
	E3 & Eritrea \\
	\hline
	E4 & Palestinian Authority \\
	\hline
	E5 & Cook Islands \\
	\hline
	E6 & Niue \\
	\hline
	E7 & Bosnia and Herzegovina \\
	\hline
	F & France \\
	\hline
	G & United Kingdom \\
	\hline
	HA & Hungary \\
	\hline
	HB & Switzerland \\
	\hline
	HB (HB0, HB3Y, HBL) & Liechtenstein \\
	\hline
	HC--HD & Ecuador \\
	\hline
	HE & Switzerland \\
	\hline
	HF & Poland \\
	\hline
	HG & Hungary \\
	\hline
	HH & Haiti \\
	\hline
	HI & Dominican Republic \\
	\hline
	HJ--HK & Colombia \\
	\hline
	HL & South Korea \\
	\hline
	HM & North Korea \\
	\hline
	HN & Iraq \\
	\hline
	HO--HP & Panama \\
	\hline
	HQ--HR & Honduras \\
	\hline
	HS & Thailand \\
	\hline
	HT & Nicaragua \\
	\hline
	HU & El Salvador \\
	\hline
	HV & Vatican City \\
	\hline
	HW--HY & France \\
	\hline
	HZ & Saudi Arabia \\
	\hline
	H2 & Cyprus \\
	\hline
	H3 & Panama \\
	\hline
	H4 & Solomon Islands \\
	\hline
	H6--H7 & Nicaragua \\
	\hline
	H8--H9 & Panama \\
	\hline
	I & Italy \\
	\hline
	JA--JS & Japan \\
	\hline
	JT--JV & Mongolia \\
	\hline
	JW--JX & Norway \\
	\hline
	JY & Jordan \\
	\hline
	JZ & Indonesia \\
	\hline
	J2 & Djibouti \\
	\hline
	J3 & Grenada \\
	\hline
	J4 & Greece \\
	\hline
	J5 & Guinea-Bissau \\
	\hline
	J6 & Saint Lucia \\
	\hline
	J7 & Dominica \\
	\hline
	J8 & Saint Vincent and the Grenadines \\
	\hline
	K & United States \\
	\hline
	LA--LN & Norway \\
	\hline
	LO--LW & Argentina \\
	\hline
	LX & Luxembourg \\
	\hline
	LY & Lithuania \\
	\hline
	LZ & Bulgaria \\
	\hline
	L2--L9 & Argentina \\
	\hline
	M & United Kingdom \\
	\hline
	N & United States \\
	\hline
	OA--OC & Peru \\
	\hline
	OD & Lebanon \\
	\hline
	OE & Austria \\
	\hline
	OF--OJ & Finland \\
	\hline
	OK--OL & Czech Republic \\
	\hline
	OM & Slovakia \\
	\hline
	ON--OT & Belgium \\
	\hline
	OU--OZ & Denmark \\
	\hline
	PA--PI & Netherlands \\
	\hline
	PJ & Netherlands Antilles \\
	\hline
	PK--PO & Indonesia \\
	\hline
	PP--PY & Brazil \\
	\hline
	PZ & Suriname \\
	\hline
	P2 & Papua New Guinea \\
	\hline
	P3 & Cyprus \\
	\hline
	P4 & Aruba \\
	\hline
	P5--P9 & North Korea \\
	\hline
	R & Russia \\
	\hline
	SA--SM & Sweden \\
	\hline
	SN--SR & Poland \\
	\hline
	SSA--SSM & Egypt \\
	\hline
	SSN--STZ & Sudan \\
	\hline
	SU & Egypt \\
	\hline
	SV--SZ & Greece \\
	\hline
	S2--S3 & Bangladesh \\
	\hline
	S5 & Slovenia \\
	\hline
	S6 & Singapore \\
	\hline
	S7 & Seychelles \\
	\hline
	S8 & South Africa \\
	\hline
	S9 & São Tomé and Príncipe \\
	\hline
	TA--TC & Turkey \\
	\hline
	TD & Guatemala \\
	\hline
	TE & Costa Rica \\
	\hline
	TF & Iceland \\
	\hline
	TG & Guatemala \\
	\hline
	TH & France \\
	\hline
	TI & Costa Rica \\
	\hline
	TJ & Cameroon \\
	\hline
	TK & France \\
	\hline
	TL & Central African Republic \\
	\hline
	TM & France \\
	\hline
	TN & Republic of the Congo \\
	\hline
	TO--TQ & France \\
	\hline
	TR & Gabon \\
	\hline
	TS & Tunisia \\
	\hline
	TT & Chad \\
	\hline
	TU & Ivory Coast \\
	\hline
	TV--TX & France \\
	\hline
	TY & Benin \\
	\hline
	TZ & Mali \\
	\hline
	T2 & Tuvalu \\
	\hline
	T3 & Kiribati \\
	\hline
	T4 & Cuba \\
	\hline
	T5 & Somalia \\
	\hline
	T6 & Afghanistan \\
	\hline
	T7 & San Marino \\
	\hline
	T8 & Palau \\
	\hline
	UJ--UM & Uzbekistan \\
	\hline
	UN--UQ & Kazakhstan \\
	\hline
	UR--UZ & Ukraine \\
	\hline
	VA--VG & Canada \\
	\hline
	VH--VN & Australia \\
	\hline
	VO & Canada \\
	\hline
	VP--VQ & United Kingdom \\
	\hline
	VR & Hong Kong \\
	\hline
	VS & United Kingdom \\
	\hline
	VT--VW & India \\
	\hline
	VX--VY & Canada \\
	\hline
	VZ & Australia \\
	\hline
	V2 & Antigua and Barbuda \\
	\hline
	V3 & Belize \\
	\hline
	V4 & Saint Kitts and Nevis \\
	\hline
	V5 & Namibia \\
	\hline
	V6 & Federated States of Micronesia \\
	\hline
	V7 & Marshall Islands \\
	\hline
	V8 & Brunei \\
	\hline
	W & United States \\
	\hline
	XA--XI & Mexico \\
	\hline
	XJ--XO & Canada \\
	\hline
	XP & Denmark \\
	\hline
	XQ--XR & Chile \\
	\hline
	XS & China \\
	\hline
	XT & Burkina Faso \\
	\hline
	XU & Cambodia \\
	\hline
	XV & Vietnam \\
	\hline
	XW & Laos \\
	\hline
	XX & Macao \\
	\hline
	XY--XZ & Burma \\
	\hline
	YA & Afghanistan \\
	\hline
	YB--YH & Indonesia \\
	\hline
	YI & Iraq \\
	\hline
	YJ & Vanuatu \\
	\hline
	YK & Syria \\
	\hline
	YL & Latvia \\
	\hline
	YM & Turkey \\
	\hline
	YN & Nicaragua \\
	\hline
	YO--YR & Romania \\
	\hline
	YS & El Salvador \\
	\hline
	YT--YU & Serbia \\
	\hline
	YV--YY & Venezuela \\
	\hline
	Y2--Y9 & Germany \\
	\hline
	ZA & Albania \\
	\hline
	ZB--ZJ & United Kingdom \\
	\hline
	ZK--ZM & New Zealand \\
	\hline
	ZN--ZO & United Kingdom \\
	\hline
	ZP & Paraguay \\
	\hline
	ZQ & United Kingdom \\
	\hline
	ZR--ZU & South Africa \\
	\hline
	ZV--ZZ & Brazil \\
	\hline
	Z2 & Zimbabwe \\
	\hline
	Z3 & North Macedonia \\
	\hline
	Z8 & South Sudan \\
	\hline
	2 & United Kingdom \\
	\hline
	3A & Monaco \\
	\hline
	3B & Mauritius \\
	\hline
	3C & Equatorial Guinea \\
	\hline
	3DA--3DM & Eswatini \\
	\hline
	3DN--3DZ & Fiji \\
	\hline
	3E--3F & Panama \\
	\hline
	3G & Chile \\
	\hline
	3H--3U & China \\
	\hline
	3V & Tunisia \\
	\hline
	3W & Vietnam \\
	\hline
	3X & Guinea \\
	\hline
	3Y & Norway \\
	\hline
	3Z & Poland \\
	\hline
	4A--4C & Mexico \\
	\hline
	4D--4I & Philippines \\
	\hline
	4J--4K & Azerbaijan \\
	\hline
	4L & Georgia \\
	\hline
	4M & Venezuela \\
	\hline
	4O & Montenegro \\
	\hline
	4P--4S & Sri Lanka \\
	\hline
	4T & Peru \\
	\hline
	4U & United Nations \\
	\hline
	4V & Haiti \\
	\hline
	4W & Timor-Leste \\
	\hline
	4X & Israel \\
	\hline
	4Y & International Civil Aviation Organization \\
	\hline
	4Z & Israel \\
	\hline
	5A & Libya \\
	\hline
	5B & Cyprus \\
	\hline
	5C--5G & Morocco \\
	\hline
	5H--5I & Tanzania \\
	\hline
	5J--5K & Colombia \\
	\hline
	5L--5M & Liberia \\
	\hline
	5N--5O & Nigeria \\
	\hline
	5P--5Q & Denmark \\
	\hline
	5R--5S & Madagascar \\
	\hline
	5T & Mauritania \\
	\hline
	5U & Niger \\
	\hline
	5V & Togo \\
	\hline
	5W & Western Samoa \\
	\hline
	5X & Uganda \\
	\hline
	5Y--5Z & Kenya \\
	\hline
	6A--6B & Egypt \\
	\hline
	6C & Syria \\
	\hline
	6D--6J & Mexico \\
	\hline
	6K--6N & South Korea \\
	\hline
	6O & Somalia \\
	\hline
	6P--6S & Pakistan \\
	\hline
	6T--6U & Sudan \\
	\hline
	6V--6W & Senegal \\
	\hline
	6X & Madagascar \\
	\hline
	6Y & Jamaica \\
	\hline
	6Z & Liberia \\
	\hline
	7A--7I & Indonesia \\
	\hline
	7J--7N & Japan \\
	\hline
	7O & Yemen \\
	\hline
	7P & Lesotho \\
	\hline
	7Q & Malawi \\
	\hline
	7R & Algeria \\
	\hline
	7S & Sweden \\
	\hline
	7T--7Y & Algeria \\
	\hline
	7Z & Saudi Arabia \\
	\hline
	8A--8I & Indonesia \\
	\hline
	8J--8N & Japan \\
	\hline
	8O & Botswana \\
	\hline
	8P & Barbados \\
	\hline
	8Q & Maldives \\
	\hline
	8R & Guyana \\
	\hline
	8S & Sweden \\
	\hline
	8T--8Y & India \\
	\hline
	8Z & Saudi Arabia \\
	\hline
	9A & Croatia \\
	\hline
	9B--9D & Iran \\
	\hline
	9E--9F & Ethiopia \\
	\hline
	9G & Ghana \\
	\hline
	9H & Malta \\
	\hline
	9I--9J & Zambia \\
	\hline
	9K & Kuwait \\
	\hline
	9L & Sierra Leone \\
	\hline
	9M & Malaysia \\
	\hline
	9N & Nepal \\
	\hline
	9O--9T & Democratic Republic of the Congo \\
	\hline
	9U & Burundi \\
	\hline
	9V & Singapore \\
	\hline
	9W & Malaysia \\
	\hline
	9X & Rwanda \\
	\hline
	9Y--9Z & Trinidad and Tobago \\
	\hline
\end{longtable}

\newpage










\section{计算公式}

直流电路欧姆定律:
\[\mbox{电流}I(\si{\ampere})=\frac{\mbox{电压}U(\si{\volt})}{\mbox{电阻值}R(\si{\ohm})}\]

峰值电压的计算公式:
\[\mbox{峰值电压}(\si{\volt})=\frac{\mbox{峰--峰值}}{2}\]

电功率的计算公式:
\begin{equation*}
    \begin{aligned}
        \mbox{电功率}P(\si{\watt}) &= U\times I \\
        & = I^2 \times R \\
        & = \frac{U^2}{R}
    \end{aligned}
\end{equation*}

正弦交流电压的有效值计算公式:
\begin{equation*}
\begin{aligned}
\mbox{交流电压的有效值}(\si{\volt})&=\frac{\mbox{峰值}}{\sqrt{2}} \\
& \approx \mbox{峰值}\times 0.707
\end{aligned}
\end{equation*}

功率增益的计算公式:
\[\mbox{功率增益}=10 \log_{10} \left( {\frac{P_{ \mbox{输出} }}{P_{ \mbox{输入} }}}\right) \si{\dB}\]

电压增益的计算公式:
\[\mbox{电压增益}=20 \log \left( {\frac{V_{ \mbox{输出} }}{V_{ \mbox{输入} }}} \right) \si{\dB}\]

%电流增益的计算公式:
%$$\mbox{电流增益}=20 \log \left( {\frac{I_{ \mbox{输出} }}{I_{ \mbox{输入} }}} \right)\ \mathrm{dB}$$

绝对增益与相对增益的关系:
\[\mbox{绝对增益} = \mbox{相对增益} + \SI{2.15}{\dB}\]

两数的乘积的对数等于两数的对数之和:
\[\log a\: b=\log a+\log b\]

两数的商的对数等于两数的对数之差:
\[\log\frac{a}{b}=\log a-\log b\]

某数的n次幂的对数等于该数的对数的n倍:
\[\log a^n =n\log a\]

\newpage








\section{常用对数表(节录)}

说明:强烈建议背诵$\log_{10} 1.0$、$\log 2.0$、……$\log 10.0$等第0列的10个数值到小数点后三位,考试闭卷,无法使用计算器,如果不背诵,将无法计算。

\begin{longtable}[c]{|c|c|}
    \hline
    \textbf{N} & \textbf{0} \\
    \hline
    %	\endfirsthead
    %	\hline
    %	\textbf{N} & \textbf{1.0} \\
    \endhead
    \num{1.0} & \num{.000} \\ \hline
    \num{2.0} & \num{.301} \\ \hline
    \num{3.0} & \num{.477} \\ \hline
    \num{4.0} & \num{.602} \\ \hline
    \num{5.0} & \num{.699} \\ \hline
    \num{6.0} & \num{.778} \\ \hline
    \num{7.0} & \num{.845} \\ \hline
    \num{8.0} & \num{.903} \\ \hline
    \num{9.0} & \num{.954} \\ \hline
    \num{10.0} & \num{1.000} \\ \hline
\end{longtable}

\newpage








\section{逻辑门的真值表}


\begin{tabular}{cc}%
    \begin{tabular}{|c|c|c|}
        \multicolumn{3}{c}{\textbf{与门}} \\
        \hline
        \multicolumn{2}{|c|}{\textbf{输入}} & \textbf{输出} \\
        \hline
        A & B & A AND B \\
        \hline
        \num{0} & \num{0} & \num{0} \\
        \hline
        \num{0} & \num{1} & \num{0} \\
        \hline
        \num{1} & \num{0} & \num{0} \\
        \hline
        \num{1} & \num{1} & \num{1} \\
        \hline
    \end{tabular} &
    \begin{tabular}{|c|c|c|}
        \multicolumn{3}{c}{\textbf{与非门}} \\
        \hline
        \multicolumn{2}{|c|}{\textbf{输入}} & \textbf{输出} \\
        \hline
        A & B & A NAND B \\
        \hline
        \num{0} & \num{0} & \num{1} \\
        \hline
        \num{0} & \num{1} & \num{1} \\
        \hline
        \num{1} & \num{0} & \num{1} \\
        \hline
        \num{1} & \num{1} & \num{0} \\
        \hline
    \end{tabular} \tabularnewline
\end{tabular}

\bigskip

\begin{tabular}{cc}%
    \begin{tabular}{|c|c|c|}
        \multicolumn{3}{c}{\textbf{或门}} \\
        \hline
        \multicolumn{2}{|c|}{\textbf{输入}} & \textbf{输出} \\
        \hline
        A & B & A OR B \\
        \hline
        \num{0} & \num{0} & \num{0} \\
        \hline
        \num{0} & \num{1} & \num{1} \\
        \hline
        \num{1} & \num{0} & \num{1} \\
        \hline
        \num{1} & \num{1} & \num{1} \\
        \hline
    \end{tabular} &
    \begin{tabular}{|c|c|c|}
        \multicolumn{3}{c}{\textbf{或非门}} \\
        \hline
        \multicolumn{2}{|c|}{\textbf{输入}} & \textbf{输出} \\
        \hline
        A & B & A NOR B \\
        \hline
        \num{0} & \num{0} & \num{1} \\
        \hline
        \num{0} & \num{1} & \num{0} \\
        \hline
        \num{1} & \num{0} & \num{0} \\
        \hline
        \num{1} & \num{1} & \num{0} \\
        \hline
    \end{tabular}  \tabularnewline
\end{tabular}

\bigskip

\begin{tabular}{cc}%
    \begin{tabular}{|c|c|c|}
        \multicolumn{3}{c}{\textbf{异或门}} \\
        \hline
        \multicolumn{2}{|c|}{\textbf{输入}} & \textbf{输出} \\
        \hline
        A & B & A XOR B \\
        \hline
        \num{0} & \num{0} & \num{0} \\
        \hline
        \num{0} & \num{1} & \num{1} \\
        \hline
        \num{1} & \num{0} & \num{1} \\
        \hline
        \num{1} & \num{1} & \num{0} \\
        \hline
    \end{tabular} &
    \begin{tabular}{|c|c|c|}
        \multicolumn{3}{c}{\textbf{异或非门}} \\
        \hline
        \multicolumn{2}{|c|}{\textbf{输入}} & \textbf{输出} \\
        \hline
        A & B & A XNOR B \\
        \hline
        \num{0} & \num{0} & \num{1} \\
        \hline
        \num{0} & \num{1} & \num{0} \\
        \hline
        \num{1} & \num{0} & \num{0} \\
        \hline
        \num{1} & \num{1} & \num{1} \\
        \hline
    \end{tabular} \tabularnewline
\end{tabular}

\newpage

\section{计算逻辑门电路输出信号题型答案的C程序}

\lstinputlisting[language=C]{jisuan.c}

\newpage



\section{将呼号转换成字母解释法的单词组合的C程序}

\lstinputlisting[language=C]{icao.c}

\newpage



\section{业余无线电台呼号所属分区信息查询C程序}

\lstinputlisting[language=C]{hhcx.c}

\newpage

\section{有用网址}

\begin{longtable}{|p{8cm}|p{8cm}|}
	\hline
	\textbf{名称} & \textbf{网址} \\
	\hline
	业余无线电台操作技术能力验证考核报名系统 & \url{http://114.115.246.55:8091/CRAC/crac/login_student.html} \\
	\hline
	复习试题和模拟考试系统下载 & \url{http://114.115.246.55:8091/CRAC/crac/pages/list_files.html} \\
	\hline
	中国无线电协会业余无线电分会资料下载 & \url{http://www.crac.org.cn/News/List?type=6} \\
	\hline
\end{longtable}
