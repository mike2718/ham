\chapter{附录}

\section{附录1 字母解释法}

\begin{tabular}{cc}%
\begin{tabular}[t]{|c|l|}
	\hline
	\textbf{字母} & \textbf{单词} \\
	\hline
	A & Alfa \\
	\hline
	B & Bravo \\
	\hline
	C & Charlie \\
	\hline
	D & Delta \\
	\hline
	E & Echo \\
	\hline
	F & Foxtrot \\
	\hline
	G & Golf \\
	\hline
	H & Hotel \\
	\hline
	I & India \\
	\hline
	J & Juliett \\
	\hline
	K & Kilo \\
	\hline
	L & Lima \\
	\hline
	M & Mike \\
	\hline
	N & November \\
	\hline
	O & Oscar \\
	\hline
	P & Papa \\
	\hline
	Q & Quebec \\
	\hline
	R & Romeo \\
	\hline
	S & Sierra \\
	\hline
	T & Tango \\
	\hline
	U & Uniform \\
	\hline
	V & Victor \\
	\hline
	W & Whiskey \\
	\hline
	X & X-ray \\
	\hline
	Y & Yankee \\
	\hline
	Z & Zulu \\
	\hline
\end{tabular} &
\begin{tabular}[t]{|c|l|}
	\hline
	\textbf{字母} & \textbf{单词} \\
	\hline
	0 & Zero \\
	\hline
	1 & One \\
	\hline
	2 & Two \\
	\hline
	3 & Three \\
	\hline
	4 & Four \\
	\hline
	5 & Five \\
	\hline
	6 & Six \\
	\hline
	7 & Seven \\
	\hline
	8 & Eight \\
	\hline
	9 & Nine \\
	\hline
\end{tabular} \tabularnewline
\end{tabular}

\newpage

\section{附录2 业余通信中常用的Q简语}

\begin{tabular}{|l|l|l|}
	\hline
	    & \textbf{提问} & \textbf{回答或建议} \\
	\hline
	QRL & 你正忙着吗 & 我正忙着 \\
	\hline
	QRM & 你遇到他台干扰吗 & 我遇到他台干扰 \\
	\hline
	QRN & 你遇到天电干扰吗 & 我遇到天电干扰 \\
	\hline
	QRO & 要我增加功率吗 & \\
	\hline
	QRP & 要我减小功率吗 & \\
	\hline
	QRQ & 要我加快发送速度吗 & 请加快发送速度 \\
	\hline
	QRS & 要我减慢发送速度吗 & 请减慢发送速度 \\
	\hline
	QRT & 要我停止发送吗 & 请停止发送 \\
	\hline
	QRU & 你和我还有事吗 & 我和你无事了 \\
	\hline
	QRV & 你是否已准备好 & 我已准备好 \\
	\hline
	QRZ & 谁在呼叫我 & \\
	\hline
	QSA & 我的信号强度如何 & 你的信号强度为 × 级(1-5 级) \\
	\hline
	QSB & 我的信号有衰落吗 & 你的信号有衰落 \\
	\hline
	QSD & 我发报的手法有毛病吗 & 你发报的手法有毛病 \\
	\hline
	\multirow{2}{1em}{QSK} & 能在你的信号间隙中接收吗 & 我在发射的信号间隙中接收 \\
	& (即 QSK 插入方式)    & (即 QSK 插入方式) \\
	\hline
	QSL & 你能给我收据(或 QSL 卡片)吗 & 我给你收据(QSL 卡片)、我已收妥 \\
	\hline
	QSO & 你能直接和 ××× 电台通信吗 & 我能直接和 ××× 电台通信 \\
	\hline
	QSP & 你能传信到 ××× 电台吗 & 我能传信到 ××× 电台 \\
	\hline
	\multirow{2}{1em}{QSX} & 你将在 nnnn KHz(或 MHz)频率 & 我将在 nnnn KHz(或 MHz)频率 \\
	    & 守听 ××× 电台吗 & 守听 ××× 电台 \\
	\hline
	QSY & 要我将频率改到 nnnn 频率吗 & 请将频率改到 nnnn 频率 \\
	\hline
	QTH & 你的电台位置在哪里 & 我的电台位置是 ×××× \\
	\hline
\end{tabular}

\newpage



\section{附录3 业余通信常用缩语表}

\newpage

\section{附录4 发射类别的表示方法}

\begin{tabular}{|c|c|}
	\hline
	\multicolumn{2}{|c|}{\textbf{主载波类型(第一个符号)}} \\
	\hline
	\textbf{符号} & \textbf{描述} \\
	\hline
	A & 双边带 \\
	\hline
	F & 调频 \\
	\hline
	G & 调相 \\
	\hline
	J & 单边带、抑制载波 \\
	\hline
\end{tabular}

\bigskip

\begin{tabular}{|c|c|}
	\hline
	\multicolumn{2}{|c|}{\textbf{调制主载波的信号的性质(第二个符号)}} \\
	\hline
	\textbf{符号} & \textbf{描述} \\
	\hline
	1 & 不用调制副载波但包含量化或数字信息的单个通路 \\
	\hline
	2 & 利用调制副载波且包含量化或数字信息的单个通路 \\
	\hline
	3 & 包含模拟信息的单个通路 \\
	\hline
\end{tabular}

\bigskip

\begin{tabular}{|c|c|}
	\hline
	\multicolumn{2}{|c|}{\textbf{被发送信息类型(第三个符号)}} \\
	\hline
	\textbf{符号} & \textbf{描述} \\
	\hline
	A & 电报──用于人工收听 \\
	\hline
	B & 电报──用于自动接收 \\
	\hline
	E & 电话(包括声音广播) \\
	\hline
	F & 电视(视频) \\
	\hline
\end{tabular}

\newpage

\section{附录5 无线电频段和波段表}

\begin{tabular}{|c|c|c|c|c|}
	\hline
	\textbf{频段名称} & \textbf{缩写} & \textbf{频率范围} & \textbf{波长范围} & \textbf{波段名称} \\
	%\hline
	%反正不考
	%极低频 & ELF & 3–30 Hz & 100,000–10,000 km & 极长波 \\
	%\hline
	%超低频 & SLF & 30–300 Hz & 10,000–1,000 km & 超长波 \\
	%\hline
	%特低频 & ULF & 300–3,000 Hz & 1,000–100 km & 特长波 \\
	%\hline
	%甚低频 & VLF & 3–30 kHz & 100–10 km & 甚长波 \\
	\hline
	低频 & LF & 30–300 kHz & 10–1 km & 长波 \\
	\hline
	中频 & MF & 300–3,000 kHz & 1,000–100 m & 中波 \\
	\hline
	高频 & HF & 3–30 MHz & 100–10 m & 短波 \\
	\hline
	甚高频 & VHF & 30–300 MHz & 10–1 m & 米波 \\
	\hline
	特高频 & UHF & 300–3,000 MHz & 1–0.1 m & 分米波 \\
	\hline
	超高频 & SHF & 3–30 GHz & 100–10 mm & 厘米波 \\
	\hline
	极高频 & EHF & 30–300 GHz & 10–1 mm & 毫米波 \\
	\hline
	%反正不考
	%至高频 & THF & 300–3,000 GHz & 1–0.1 mm & 丝米波 \\
	%\hline
\end{tabular}

\newpage






\section{附录6 n米波段的频率范围表}


\begin{tabular}{|c|c|c|}
	\hline
	\textbf{波段名称} & \textbf{频率范围} & \textbf{频率范围} \\
	\hline
	160米 & 1800–2000 kHz & 1.8-2 MHz \\
	\hline
	80米 & 3.5–4.0 MHz & 3500–4000 kHz \\
	\hline
	40米 & 7.0–7.3 MHz &  \\
	\hline
	20米 & 14.0–14.35 MHz & \\
	\hline
	15米 & 21–21.45 MHz & \\
	\hline
	10米 & 28–29.7 MHz & \\
	\hline
	6米 & 50–54 MHz & \\
	\hline
	2米 & 144–148 MHz & \\
	\hline
	0.7米 & 420–450 MHz & \\
	\hline
\end{tabular}

\newpage

\section{附录7 计算公式}

\subsection{真值表}

\begin{tabular}{cc}%
\begin{tabular}{|c|c|c|}
	\hline
	\multicolumn{3}{|c|}{\textbf{与门(AND)}} \\
	\hline
	\multicolumn{2}{|c|}{\textbf{输入}} & \textbf{输出} \\
	\hline
	A & B & A AND B \\
	\hline
	0 & 0 & 0 \\
	\hline
	0 & 1 & 0 \\
	\hline
	1 & 0 & 0 \\
	\hline
	1 & 1 & 1 \\
	\hline
\end{tabular} &
\begin{tabular}{|c|c|c|}
	\hline
	\multicolumn{3}{|c|}{\textbf{与非门(NAND)}} \\
	\hline
	\multicolumn{2}{|c|}{\textbf{输入}} & \textbf{输出} \\
	\hline
	A & B & A NAND B \\
	\hline
	0 & 0 & 1 \\
	\hline
	0 & 1 & 1 \\
	\hline
	1 & 0 & 1 \\
	\hline
	1 & 1 & 0 \\
	\hline
\end{tabular} \tabularnewline
\end{tabular}

\bigskip

\begin{tabular}{cc}%
\begin{tabular}{|c|c|c|}
	\hline
	\multicolumn{3}{|c|}{\textbf{或门(OR)}} \\
	\hline
	\multicolumn{2}{|c|}{\textbf{输入}} & \textbf{输出} \\
	\hline
	A & B & A OR B \\
	\hline
	0 & 0 & 0 \\
	\hline
	0 & 1 & 1 \\
	\hline
	1 & 0 & 1 \\
	\hline
	1 & 1 & 1 \\
	\hline
\end{tabular} &
\begin{tabular}{|c|c|c|}
	\hline
	\multicolumn{3}{|c|}{\textbf{或非门(NOR)}} \\
	\hline
	\multicolumn{2}{|c|}{\textbf{输入}} & \textbf{输出} \\
	\hline
	A & B & A NOR B \\
	\hline
	0 & 0 & 1 \\
	\hline
	0 & 1 & 0 \\
	\hline
	1 & 0 & 0 \\
	\hline
	1 & 1 & 0 \\
	\hline
\end{tabular}  \tabularnewline
\end{tabular}

\bigskip

\begin{tabular}{cc}%
\begin{tabular}{|c|c|c|}
	\hline
	\multicolumn{3}{|c|}{\textbf{异或门(XOR)}} \\
	\hline
	\multicolumn{2}{|c|}{\textbf{输入}} & \textbf{输出} \\
	\hline
	A & B & A XOR B \\
	\hline
	0 & 0 & 0 \\
	\hline
	0 & 1 & 1 \\
	\hline
	1 & 0 & 1 \\
	\hline
	1 & 1 & 0 \\
	\hline
\end{tabular} &
\begin{tabular}{|c|c|c|}
	\hline
	\multicolumn{3}{|c|}{\textbf{异或非门(NXOR)}} \\
	\hline
	\multicolumn{2}{|c|}{\textbf{输入}} & \textbf{输出} \\
	\hline
	A & B & A XNOR B \\
	\hline
	0 & 0 & 1 \\
	\hline
	0 & 1 & 0 \\
	\hline
	1 & 0 & 0 \\
	\hline
	1 & 1 & 1 \\
	\hline
\end{tabular} \tabularnewline
\end{tabular}
