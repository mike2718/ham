\chapter{无线电系统理论}

\bigskip


\noindent\textbf{问题:}电流的单位是:

\begin{enumerate}[label=\Alph*), leftmargin=3em]
	\item 欧(姆)
	\item 伏(特)
	\item 瓦(特)
	\item 安(培)
\end{enumerate}

\noindent\textbf{解说:}\textbf{安(培)}是电流的单位。\\\noindent\textbf{答案:}D


\bigskip


\noindent\textbf{问题:}电压的单位是:

\begin{enumerate}[label=\Alph*), leftmargin=3em]
	\item 瓦(特)
	\item 伏(特)
	\item 欧(姆)
	\item 安(培)
\end{enumerate}

\noindent\textbf{解说:伏(特)}是电压的单位。\\\noindent\textbf{答案:}B


\bigskip


\noindent\textbf{问题:}电阻的单位是:

\begin{enumerate}[label=\Alph*), leftmargin=3em]
	\item 伏(特)
	\item 瓦(特)
	\item 安(培)
	\item 欧(姆)
\end{enumerate}

\noindent\textbf{解说:欧(姆)}是电阻的单位。\\\noindent\textbf{答案:}D


\bigskip


\noindent\textbf{问题:}功率的单位是:

\begin{enumerate}[label=\Alph*), leftmargin=3em]
	\item 安(培)
	\item 欧(姆)
	\item 伏(特)
	\item 瓦(特)
\end{enumerate}

\noindent\textbf{解说:瓦(特)}是功率的单位。\\\noindent\textbf{答案:}D


\bigskip


\noindent\textbf{问题:}无线电常用度量单位的词头k的意义为:(“x^m”表示“x的m次方”)

\begin{enumerate}[label=\Alph*), leftmargin=3em]
	\item 10^3
	\item 10^(-6)
	\item 10^(-3)
	\item 10^6
\end{enumerate}

\noindent\textbf{解说:}无线电常用度量单位的词头k的意义为:\textbf{10^3}。\\\noindent\textbf{答案:}A


\bigskip


\noindent\textbf{问题:}无线电常用度量单位的词头m的意义为:(“x^m”表示“x的m次方”)

\begin{enumerate}[label=\Alph*), leftmargin=3em]
	\item 10^3
	\item 10^(-6)
	\item 10^(-3)
	\item 10^6
\end{enumerate}

\noindent\textbf{解说:}无线电常用度量单位的词头m的意义为:\textbf{10^(-3)}。\\\noindent\textbf{答案:}C


\bigskip


\noindent\textbf{问题:}无线电常用度量单位的词头M的意义为:(“x^m”表示“x的m次方”)

\begin{enumerate}[label=\Alph*), leftmargin=3em]
	\item 10^(-6)
	\item 10^6
	\item 10^3
	\item 10^(-3)
\end{enumerate}

\noindent\textbf{解说:}无线电常用度量单位的词头M的意义为:\textbf{10^6}。\\\noindent\textbf{答案:}B


\bigskip


\noindent\textbf{问题:}无线电常用度量单位的词头μ的意义为:(“x^m”表示“x的m次方”)

\begin{enumerate}[label=\Alph*), leftmargin=3em]
	\item 10^(-3)
	\item 10^3
	\item 10^6
	\item 10^(-6)
\end{enumerate}

\noindent\textbf{解说:}无线电常用度量单位的词头μ的意义为:\textbf{10^(-6)}。\\\noindent\textbf{答案:}D


\bigskip


\noindent\textbf{问题:}无线电常用度量单位的词头G的意义分别为:(“x^m”表示“x的m次方”)

\begin{enumerate}[label=\Alph*), leftmargin=3em]
	\item 10^9
	\item 10^6
	\item 10^(-12)
	\item 10^12
\end{enumerate}

\noindent\textbf{解说:}无线电常用度量单位的词头G的意义为:\textbf{10^9}。\\\noindent\textbf{答案:}A


\bigskip


\noindent\textbf{问题:}无线电常用度量单位的词头n的意义分别为:(“x^m”表示“x的m次方”)

\begin{enumerate}[label=\Alph*), leftmargin=3em]
	\item 10^9
	\item 10^(-9)
	\item 10^12
	\item 10^(-12)
\end{enumerate}

\noindent\textbf{解说:}无线电常用度量单位的词头n的意义为:\textbf{10^(-9)}。\\\noindent\textbf{答案:}B


\bigskip


\noindent\textbf{问题:}无线电常用度量单位的词头T的意义分别为:(“x^m”表示“x的m次方”)

\begin{enumerate}[label=\Alph*), leftmargin=3em]
	\item 10^-12
	\item 10^12
	\item 10^9
	\item 10^(-9)
\end{enumerate}

\noindent\textbf{解说:}无线电常用度量单位的词头T的意义为:\textbf{10^12}。\\\noindent\textbf{答案:}B


\bigskip


\noindent\textbf{问题:}无线电常用度量单位的词头p的意义分别为:(“x^m”表示“x的m次方”)

\begin{enumerate}[label=\Alph*), leftmargin=3em]
	\item 10^9
	\item 10^12
	\item 10^(-9)
	\item 10^(-12)
\end{enumerate}

\noindent\textbf{解说:}无线电常用度量单位的词头p的意义为:\textbf{10^(-12)}。\\\noindent\textbf{答案:}D


\bigskip


\noindent\textbf{问题:}音频所指的频率范围大致是:

\begin{enumerate}[label=\Alph*), leftmargin=3em]
	\item 16kHz – 56kHz
	\item 16Hz - 20kHz
	\item 300Hz – 3000Hz
	\item 16kHz – 20kHz
\end{enumerate}

\noindent\textbf{解说:}音频的频率范围大致是 \textbf{16Hz - 20kHz}。\\\noindent\textbf{答案:}B% 查资料得 20Hz ~ 20kHz


\bigskip


\noindent\textbf{问题:}5W可以表示为:

\begin{enumerate}[label=\Alph*), leftmargin=3em]
	\item 35dBμ
	\item 17dBm
	\item 37dBm
	\item 5dBW
\end{enumerate}

\noindent\textbf{解说:}5W可以表示为\textbf{37dBm}。\\\noindent\textbf{答案:}C %%%%%%%???


\bigskip


\noindent\textbf{问题:}0.25W可以表示为:

\begin{enumerate}[label=\Alph*), leftmargin=3em]
	\item 25dBm
	\item 6dBW
	\item 54dBμ
	\item 36dBm
\end{enumerate}

\noindent\textbf{解说:}0.25W可以表示为\textbf{54dBμ}。\\\noindent\textbf{答案:}C%%%???


\bigskip


\noindent\textbf{问题:}0.4kW可以表示为:

\begin{enumerate}[label=\Alph*), leftmargin=3em]
	\item 86dBμ
	\item 6000dBm
	\item 34dBm
	\item 400dBm
\end{enumerate}

\noindent\textbf{解说:}0.4kW可以表示为\textbf{86dBμ}。\\\noindent\textbf{答案:}A%%%???


\bigskip


\noindent\textbf{问题:}电源两端电压的方向为:

\begin{enumerate}[label=\Alph*), leftmargin=3em]
	\item 从电源的负极到正极
	\item 与电源的电动势方向相同
	\item 从电源的正极到负极
	\item 取决于负载电阻和电源内阻的相对大小
\end{enumerate}

\noindent\textbf{解说:}电源两端电压的方向为:\textbf{从电源的正极到负极}。\\\noindent\textbf{答案:}C


\bigskip


\noindent\textbf{问题:}直流电路欧姆定律是说:

\begin{enumerate}[label=\Alph*), leftmargin=3em]
	\item 流过电阻的电流I,与两端的电压U成反比,与阻值R成反比
	\item 流过电阻的电流I,与两端的电压U成正比,与阻值R成反比
	\item 流过电阻的电流I,与两端的电压U成正比,与阻值R成正比
	\item 流过电阻的电流I,与两端的电压U成反比,与阻值R成正比
\end{enumerate}

\noindent\textbf{解说:}直流电路欧姆定律是说:\textbf{流过电阻的电流I,与两端的电压U成正比,与阻值R成反比}。\\\noindent\textbf{答案:}B%%%???


\bigskip


\noindent\textbf{问题:}峰-峰值为100伏的正弦交流电压,其有效值电压为:

\begin{enumerate}[label=\Alph*), leftmargin=3em]
	\item 约70.7伏
	\item 约50伏
	\item 约141伏
	\item 约35.4伏
\end{enumerate}

\noindent\textbf{解说:}根据峰值电压的计算公式
$$\mbox{峰值}(V)=\frac{\mbox{峰-峰值}}{2}$$
套入公式得
$$\mbox{100伏的正弦交流电压的峰值}(V)=\frac{\mbox{100}}{2}=50\mbox{伏}$$
再根据正弦交流电压的有效值计算公式
$$\mbox{交流的有效值}(V)=\frac{\mbox{峰值}}{\sqrt{2}}=\mbox{峰值}\times0.707$$
套入公式得
$$\mbox{100伏的正弦交流电压的有效值}=50\times0.707=35.35\approx\textbf{35.4\mbox{伏}}$$\\\noindent\textbf{答案:}D%%%???


\bigskip


\noindent\textbf{问题:}峰值为100伏的正弦交流电压,其有效值电压为:

\begin{enumerate}[label=\Alph*), leftmargin=3em]
	\item 约50伏
	\item 约35.4伏
	\item 约141伏
	\item 约70.7伏
\end{enumerate}

\noindent\textbf{解说:}
根据正弦交流电压的有效值计算公式
$$\mbox{交流的有效值}(V)=\frac{\mbox{峰值}}{\sqrt{2}}=\mbox{峰值}\times0.707$$
套入公式得
$$\mbox{100伏的正弦交流电压的有效值}=100\times0.707=\textbf{70.7\mbox{伏}}$$\\\noindent\textbf{答案:}D



\bigskip


\noindent\textbf{问题:}峰-峰值为100伏的正弦交流电压,其平均值电压为:

\begin{enumerate}[label=\Alph*), leftmargin=3em]
	\item 约35.4伏
	\item 约70.7伏
	\item 约141伏
	\item 0
\end{enumerate}

\noindent\textbf{解说:}峰-峰值为100伏的正弦交流电压,因为其正半周和负半周的和为0,所以无论其大小,其平均值电压都为\textbf{0}。\\\noindent\textbf{答案:}D


\bigskip


\noindent\textbf{问题:}峰值为100伏的正弦交流电压,其平均值电压为:

\begin{enumerate}[label=\Alph*), leftmargin=3em]
	\item 约35.4伏
	\item 0
	\item 约141伏
	\item 约70.7伏
\end{enumerate}

\noindent\textbf{解说:}峰值为100伏的正弦交流电压,因为其正半周和负半周的和为0,所以无论其大小,其平均值电压都为\textbf{0}。\\\noindent\textbf{答案:}B


\bigskip


\noindent\textbf{问题:}相位差通常用来描述:

\begin{enumerate}[label=\Alph*), leftmargin=3em]
	\item 两个或多个同频率正弦信号之间的时间滞后或超前关系
	\item 两个或多个不同频率正弦信号之间的时间滞后或超前关系
	\item 两个或多个任意信号之间的幅度关系
	\item 两个或多个随机信号之间的时间滞后或超前关系
\end{enumerate}

\noindent\textbf{解说:}相位差通常用来描述\textbf{两个或多个同频率正弦信号之间的时间滞后或超前关系}。\\\noindent\textbf{答案:}A


\bigskip


\noindent\textbf{问题:}电源(或信号源)内阻对电路的影响是:

\begin{enumerate}[label=\Alph*), leftmargin=3em]
	\item 使电源(或信号源)的电动势降低
	\item 使电源(或信号源)的可输出功率增加
	\item 使电源(或信号源)的实际输出电压降低
	\item 减少电源(或信号源)本身的电能消耗
\end{enumerate}

\noindent\textbf{解说:}电源(或信号源)内阻对电路的影响是:\textbf{使电源(或信号源)的实际输出电压降低}。\\\noindent\textbf{答案:}C


\bigskip


\noindent\textbf{问题:}电阻元件的“额定功率”参数是指:

\begin{enumerate}[label=\Alph*), leftmargin=3em]
	\item 该元件接入任何电路时的实际消耗功率
	\item 该元件正常工作时必须正好消耗的功率
	\item 该元件正常工作时所需要的最小功率
	\item 该元件正常工作时所能承受的最大功率
\end{enumerate}

\noindent\textbf{解说:}电阻元件的“额定功率”参数是指\textbf{该元件正常工作时所能承受的最大功率}。\\\noindent\textbf{答案:}D



\bigskip


\noindent\textbf{问题:}下面哪一个术语可以用来描述交流电每秒改变方向的次数?

\begin{enumerate}[label=\Alph*), leftmargin=3em]
	\item 脉率
	\item 速率
	\item 波长
	\item 频率
\end{enumerate}

\noindent\textbf{解说:}\textbf{频率}可以用来描述交流电每秒改变方向的次数。\\\noindent\textbf{答案:}D



\bigskip


\noindent\textbf{问题:}电流使用下列哪一个单位来衡量?

\begin{enumerate}[label=\Alph*), leftmargin=3em]
	\item 欧姆
	\item 安培
	\item 伏特
	\item 瓦特
\end{enumerate}

\noindent\textbf{解说:}电流使用单位\textbf{安培}来衡量。\\\noindent\textbf{答案:}B


\bigskip


\noindent\textbf{问题:}电功率使用如下哪一个单位来衡量?

\begin{enumerate}[label=\Alph*), leftmargin=3em]
	\item 伏特
	\item 瓦特
	\item 欧姆
	\item 安培
\end{enumerate}

\noindent\textbf{解说:}电功率使用单位\textbf{瓦特}来衡量。\\\noindent\textbf{答案:}B


\bigskip


\noindent\textbf{问题:}只向一个方向流动的电流叫做什么?

\begin{enumerate}[label=\Alph*), leftmargin=3em]
	\item 平流
	\item 常流
	\item 直流
	\item 交流
\end{enumerate}

\noindent\textbf{解说:}只向一个方向流动的电流叫做\textbf{直流}。电流的大小和方向周期性变化的电流叫做交流。\\\noindent\textbf{答案:}C
%其中,干电池的直流电叫做平流。 看上去是日语术语

\bigskip


\noindent\textbf{问题:}下列哪一项是电的良导体?

\begin{enumerate}[label=\Alph*), leftmargin=3em]
	\item 木材
	\item 玻璃
	\item 铜
	\item 橡胶
\end{enumerate}

\noindent\textbf{解说:}\textbf{铜}是电的良导体。木材、玻璃、橡胶是电的良好绝缘体。\\\noindent\textbf{答案:}C


\bigskip


\noindent\textbf{问题:}下列哪一项是电的良好绝缘体?

\begin{enumerate}[label=\Alph*), leftmargin=3em]
	\item 铜
	\item 玻璃
	\item 汞
	\item 铝
\end{enumerate}

\noindent\textbf{解说:}\textbf{玻璃}是电的良好绝缘体。铜、铝、汞是电的良导体。\\\noindent\textbf{答案:}B


\bigskip


\noindent\textbf{问题:}电能消耗的速率叫做什么?

\begin{enumerate}[label=\Alph*), leftmargin=3em]
	\item 电流
	\item 电功率
	\item 电阻
	\item 电压
\end{enumerate}

\noindent\textbf{解说:}电能消耗的速率叫做\textbf{电功率}。其余选项分别为:电子的移动叫做电流,阻碍电流流动的作用叫做电阻,使电流流动的力量叫做电压。\\\noindent\textbf{答案:}B


\bigskip


\noindent\textbf{问题:}用通常的调频方式进行话音通信,必要带宽约为:

\begin{enumerate}[label=\Alph*), leftmargin=3em]
	\item 6.25kHz
	\item 12.5kHz
	\item 200Hz
	\item 2700Hz
\end{enumerate}

\noindent\textbf{解说:}用通常的调频方式进行话音通信,必要带宽约为\textbf{6.25kHz}。\\\noindent\textbf{答案:}A%%%???


\bigskip


\noindent\textbf{问题:}无线电干扰中不属于有害干扰的是:

\begin{enumerate}[label=\Alph*), leftmargin=3em]
	\item 严重地损害、阻碍按规定正常开展的无线电通信业务的干扰
	\item 危害无线电导航或其他安全业务的正常运行的干扰
	\item 一再阻断按规定正常开展的无线电通信业务的干扰
	\item 符合国家或国际上规定的干扰允许值和共用标准的干扰
\end{enumerate}

\noindent\textbf{解说:}无线电干扰中,\textbf{符合国家或国际上规定的干扰允许值和共用标准的干扰}不属于有害干扰。\\\noindent\textbf{答案:}D%%%???


\bigskip


\noindent\textbf{问题:}可以组成完整无线电接收系统的功能部件组合是:

\begin{enumerate}[label=\Alph*), leftmargin=3em]
	\item 射频滤波器、变频器、音频滤波器
	\item 接收天线、射频放大器、中频放大器
	\item 接收天线、解调器、输出部件
	\item 射频放大器、变频器、中频放大器
\end{enumerate}

\noindent\textbf{解说:}\textbf{接收天线、解调器、输出部件}可以组成完整无线电接收系统。\\\noindent\textbf{答案:}C%%%???


\bigskip


\noindent\textbf{问题:}可以组成完整无线电发信系统的功能部件组合是:

\begin{enumerate}[label=\Alph*), leftmargin=3em]
	\item 话音放大器、射频振荡器、射频功率放大器
	\item 键控电路、侧音电路、天线调谐器
	\item 射频振荡器、射频功率放大器、驻波测量电路
	\item 射频振荡器、调制器、发射天线
\end{enumerate}

\noindent\textbf{解说:}\textbf{射频振荡器、调制器、发射天线}可以组成完整无线电发信系统。\\\noindent\textbf{答案:}D%%%???


\bigskip


\noindent\textbf{问题:}无线电发射机调制部件的作用是:

\begin{enumerate}[label=\Alph*), leftmargin=3em]
	\item 调整天馈系统的参数达到阻抗匹配
	\item 以电能转换效率最高的方式控制射频功率放大器的工作点
	\item 以原始信号控制射频信号的幅度、频率、相位参数
	\item 自动控制发射信号的频谱使其保持在核准的必要带宽范围内
\end{enumerate}

\noindent\textbf{解说:}无线电发射机调制部件的作用是\textbf{以原始信号控制射频信号的幅度、频率、相位参数}。\\\noindent\textbf{答案:}C%%%???


\bigskip


\noindent\textbf{问题:}发射天线的作用是:

\begin{enumerate}[label=\Alph*), leftmargin=3em]
	\item 通过天线的增益对无线电发射机输出的射频信号加以放大
	\item 把无线电发射机输出的射频信号电流转换为热能
	\item 把无线电发射机放大后的音频话音信号转换为音频电磁场
	\item 把无线电发射机输出的射频信号电流转换为空间的电磁波
\end{enumerate}

\noindent\textbf{解说:}发射天线的作用是\textbf{把无线电发射机输出的射频信号电流转换为空间的电磁波}。\\\noindent\textbf{答案:}D%%%???


\bigskip


\noindent\textbf{问题:}接收天线系统的作用是:

\begin{enumerate}[label=\Alph*), leftmargin=3em]
	\item 把空间的有用电磁波转换为热能
	\item 把空间的有用电磁波转换为音频电压电流信号
	\item 通过天线的增益将空间有用电磁波的能量加以放大
	\item 把空间的有用电磁波转换为射频电压电流信号
\end{enumerate}

\noindent\textbf{解说:}接收天线系统的作用是\textbf{把空间的有用电磁波转换为射频电压电流信号}。\\\noindent\textbf{答案:}D%%%???


\bigskip


\noindent\textbf{问题:}保证业余无线电通信接收机优良接收能力的主要因素是:

\begin{enumerate}[label=\Alph*), leftmargin=3em]
	\item 尽量宽的接收频率覆盖范围
	\item 尽量宽而平坦的音频频率响应
	\item 尽量大的音频输出功率
	\item 良好的抗干扰能力,足够高的灵敏度,尽量低的本机噪声和信号失真
\end{enumerate}

\noindent\textbf{解说:}保证业余无线电通信接收机优良接收能力的主要因素是\textbf{良好的抗干扰能力,足够高的灵敏度,尽量低的本机噪声和信号失真}。\\\noindent\textbf{答案:}D%%%???


\bigskip


\noindent\textbf{问题:}一个频率为F的简单正弦波信号的频谱包含有:

\begin{enumerate}[label=\Alph*), leftmargin=3em]
	\item 频率为F的奇数倍的无穷多个频率分量
	\item 无穷多个连续的频率分量
	\item 频率为F的一个频率分量
	\item 频率为F的整数倍的无穷多个频率分量
\end{enumerate}

\noindent\textbf{解说:}一个频率为F的简单正弦波信号的频谱包含有\textbf{频率为F的一个频率分量}。\\\noindent\textbf{答案:}C%%%???


\bigskip


\noindent\textbf{问题:}只包含一个频率分量的信号是:
\begin{enumerate}[label=\Alph*), leftmargin=3em]
	\item 对称方波
	\item 单个无限窄脉冲
	\item 简单正弦波
	\item 连续的无限窄脉冲
\end{enumerate}

\noindent\textbf{解说:}只包含一个频率分量的信号是\textbf{简单正弦波}。\\\noindent\textbf{答案:}C%???


\bigskip


\noindent\textbf{问题:}在整个频谱内具有连续的均匀频率分量的信号是:

\begin{enumerate}[label=\Alph*), leftmargin=3em]
	\item 单个无限窄脉冲
	\item 对称方波
	\item 连续的无限窄脉冲
	\item 简单正弦波
\end{enumerate}

\noindent\textbf{解说:}在整个频谱内具有连续的均匀频率分量的信号是\textbf{单个无限窄脉冲}。\\\noindent\textbf{答案:}A%%%???


\bigskip


\noindent\textbf{问题:}包含多个频率分量的信号通过滤波器会发生下列现象:

\begin{enumerate}[label=\Alph*), leftmargin=3em]
	\item 检波
	\item 自激振荡
	\item 非线性失真
	\item 频率失真
\end{enumerate}

\noindent\textbf{解说:}包含多个频率分量的信号通过滤波器会发生\textbf{频率失真}。\\\noindent\textbf{答案:}D%%%???


\bigskip


\noindent\textbf{问题:}一个FM话音信号在频谱仪上显示为:

\begin{enumerate}[label=\Alph*), leftmargin=3em]
	\item 一条随语音闪烁的直线
	\item 一条固定的垂直线,左右伴随一组对称的随语音出现和变化的垂直线
	\item 一条复杂的周期性曲线
	\item 多条固定的直线
\end{enumerate}

\noindent\textbf{解说:}一个FM话音信号在频谱仪上显示为\textbf{一条固定的垂直线,左右伴随一组对称的随语音出现和变化的垂直线}。\\\noindent\textbf{答案:}B%%%%%%%%%%%


\bigskip


\noindent\textbf{问题:}下列几种图表中,最容易用来表达和解释模拟FM调制原理的是:

\begin{enumerate}[label=\Alph*), leftmargin=3em]
	\item 波形图
	\item 相位矢量图
	\item 星座图和眼图
	\item 频谱图
\end{enumerate}

\noindent\textbf{解说:}用\textbf{频谱图}来表达和解释模拟FM调制原理。\\\noindent\textbf{答案:}D%%%%%%%%%%%


\bigskip


\noindent\textbf{问题:}业余无线电通信最常用的三种基本调制方法,其缩写AM、FM和PM,它们的中文名称分别是:

\begin{enumerate}[label=\Alph*), leftmargin=3em]
	\item 幅度调制(调幅)、频率调制(调频)、电码调制(摩尔斯)
	\item 幅度调制(调幅)、频率调制(调频)、脉宽调制(调脉宽)
	\item 频率调制(调频)、脉码调制(调脉码)、幅度调制(调幅)
	\item 幅度调制(调幅)、频率调制(调频)、相位调制(调相)
\end{enumerate}

\noindent\textbf{解说:}AM为\textbf{幅度调制(调幅)}、FM为\textbf{频率调制(调频)}、PM为\textbf{相位调制(调相)}。\\\noindent\textbf{答案:}D%?????


\bigskip


\noindent\textbf{问题:}对于给定的FM发射设备,决定其射频输出信号实际占用带宽的因素是:

\begin{enumerate}[label=\Alph*), leftmargin=3em]
	\item 射频输出实际占用带宽为由电路决定的固定值,通信常用的是25kHz或12.5kHz
	\item 所传输信号的最高频率越高,射频输出占用带宽越宽,但与其幅度无关
	\item 所传输信号的最高频率越高、幅度越大,射频输出占用带宽越宽
	\item 所传输信号的幅度越大,射频输出占用带宽越宽,但与其频率无关
\end{enumerate}

\noindent\textbf{解说:}对于给定的FM发射设备,决定其射频输出信号实际占用带宽的因素是\textbf{所传输信号的最高频率越高、幅度越大,射频输出占用带宽越宽}。\\\noindent\textbf{答案:}C%???


\bigskip


\noindent\textbf{问题:}什么叫做“鉴频”?

\begin{enumerate}[label=\Alph*), leftmargin=3em]
	\item 判断信号频率是否发生了不应有的偏离或者漂移过程称为鉴频
	\item 对调频信号进行解调的过程称为鉴频
	\item 对调幅信号进行解调的过程称为鉴频
	\item 判断信号频率是否超过允许的频率范围的过程称为鉴频
\end{enumerate}

\noindent\textbf{解说:}所谓“鉴频”,就是指\textbf{对调频信号进行解调的过程}。\\\noindent\textbf{答案:}B
%%%???


\bigskip


\noindent\textbf{问题:}无线电发射机的效率是指:

\begin{enumerate}[label=\Alph*), leftmargin=3em]
	\item 输出到天线系统的有用信号功率与到达天线的包含杂散等无用信号的总功率之比
	\item 输出到天线系统的信号功率与发射机所消耗的电源功率之比
	\item 通信对象的接收天线得到的信号功率与发射机所消耗的电源功率之比
	\item 通信对象的接收天线得到的信号功率与发射机输出到天线系统的信号功率之比
\end{enumerate}

\noindent\textbf{解说:}无线电发射机的效率是指\textbf{输出到天线系统的信号功率与发射机所消耗的电源功率之比}。\\\noindent\textbf{答案:}B%%%???


\bigskip


\noindent\textbf{问题:}业余无线电发射机的效率总是明显低于1。所损耗的那部分能量:

\begin{enumerate}[label=\Alph*), leftmargin=3em]
	\item 绝大部分转化为热量,极小部分转化为无用信号的电磁辐射
	\item 绝大部分转化为杂散等无用信号的电磁辐射
	\item 绝大部分因阻抗失配而返回电源,极小部分转化为无用信号的电磁辐射
	\item 损耗的能量消失在电容、电感、开关器件等零部件中
\end{enumerate}

\noindent\textbf{解说:}损耗的那部分能量\textbf{绝大部分转化为热量,极小部分转化为无用信号的电磁辐射}。\\\noindent\textbf{答案:}A%%%???


\bigskip


\noindent\textbf{问题:}接收机灵敏度指标数值大小所反映的意义是:

\begin{enumerate}[label=\Alph*), leftmargin=3em]
	\item 灵敏度指标数值越小,接收最小信号的能力越强
	\item 灵敏度指标数值越小,对与有用信号同时出现的干扰信号的响应越灵敏
	\item 灵敏度指标数值越大,对与有用信号同时出现的干扰信号的响应越灵敏
	\item 灵敏度指标数值越大,接收最小信号的能力越强
\end{enumerate}

\noindent\textbf{解说:}接收机灵敏度指标数值大小所反映的意义是\textbf{灵敏度指标数值越小,接收最小信号的能力越强}。\\\noindent\textbf{答案:}A%%???


\bigskip


\noindent\textbf{问题:}静噪灵敏度是指:

\begin{enumerate}[label=\Alph*), leftmargin=3em]
	\item 关闭静噪电路时所能接收到的最小射频信号的输入电平
	\item 能够使静噪电路退出静噪状态的射频信号最小输入电平
	\item 带有静噪功能的接收机关闭静噪功能时,按照灵敏度定义测得的灵敏度
	\item 带有静噪功能的接收机开启静噪功能时,按照灵敏度定义测得的灵敏度
\end{enumerate}

\noindent\textbf{解说:}静噪灵敏度是指\textbf{能够使静噪电路退出静噪状态的射频信号最小输入电平}。\\\noindent\textbf{答案:}B%%%???


\bigskip


\noindent\textbf{问题:}“衰减”和“衰落”是无线电通信技术中常用的名词。它们的含义分别是指:

\begin{enumerate}[label=\Alph*), leftmargin=3em]
	\item 衰减和衰落是一回事,指信号通过信道或电路后功率减少
	\item 衰减是指信号通过信道或电路后发生幅度随时间而起伏,衰落是指信号通过信道或电路后功率减少
	\item 衰减是指信号通过信道或电路后功率减少,衰落是指信号通过信道或电路后发生幅度随时间而起伏
	\item 衰减和衰落是一回事,指信号通过信道或电路后发生幅度随时间而起伏
\end{enumerate}

\noindent\textbf{解说:}\textbf{衰减是指信号通过信道或电路后功率减少,衰落是指信号通过信道或电路后发生幅度随时间而起伏}。\\\noindent\textbf{答案:}C%%%???


\bigskip


\noindent\textbf{问题:}无线电发信机在无调制情况下,在一个射频周期内供给天线馈线的平均功率称为:

\begin{enumerate}[label=\Alph*), leftmargin=3em]
	\item 载波功率
	\item 峰包功率
	\item 平均功率
	\item 无用功率
\end{enumerate}

\noindent\textbf{解说:}无线电发信机在无调制情况下,在一个射频周期内供给天线馈线的平均功率称为\textbf{载波功率}。\\\noindent\textbf{答案:}A%%???


\bigskip


\noindent\textbf{问题:}调频发射机在发射的语音信号上附加一个人耳听不到的低频音频,用来打开接收机的静噪。这一技术的常用名词是:

\begin{enumerate}[label=\Alph*), leftmargin=3em]
	\item 单音频脉冲
	\item CTCSS
	\item 载波静噪
	\item DTMF
\end{enumerate}

\noindent\textbf{解说:}\textbf{CTCSS}是调频发射机在发射的语音信号上附加一个人耳听不到的低频音频,用来打开接收机的静噪的技术。\\\noindent\textbf{答案:}B%%%????


\bigskip


\noindent\textbf{问题:}下列哪一项决定了FM信号的频偏?

\begin{enumerate}[label=\Alph*), leftmargin=3em]
	\item 同时取决于被调制信号的频率和幅度
	\item 取决于被调制信号与载波之间的相位角关系
	\item 取决于被调制信号的幅度
	\item 取决于被调制信号的频率
\end{enumerate}

\noindent\textbf{解说:}FM信号的频偏\textbf{取决于被调制信号的幅度}。\\\noindent\textbf{答案:}C%%%???


\bigskip


\noindent\textbf{问题:}下列哪一项术语表述了接收机区分不同信号的能力?

\begin{enumerate}[label=\Alph*), leftmargin=3em]
	\item 选择性
	\item 灵敏度
	\item 本底噪声
	\item 扫描速度
\end{enumerate}

\noindent\textbf{解说:}\textbf{选择性}表述了接收机区分不同信号的能力。\\\noindent\textbf{答案:}A%??


\bigskip


\noindent\textbf{问题:}接收机“过载”通常是指:

\begin{enumerate}[label=\Alph*), leftmargin=3em]
	\item 接收机消耗的电流太大了
	\item 由于接收机的音量调节得过大而导致的附加干扰
	\item 输入信号过于强大,以致导致机内产生附加干扰
	\item 接收机电源的电压太高了
\end{enumerate}

\noindent\textbf{解说:}接收机“过载”通常是指\textbf{输入信号过于强大,以致导致机内产生附加干扰}。\\\noindent\textbf{答案:}C%??


\bigskip


\noindent\textbf{问题:}以下哪种语音调制常被用于长距离弱信号的VHF或UHF联络?

\begin{enumerate}[label=\Alph*), leftmargin=3em]
	\item FM
	\item AM
	\item PM
	\item SSB
\end{enumerate}

\noindent\textbf{解说:}\textbf{SSB}语音调制常被用于长距离弱信号的VHF或UHF联络。\\\noindent\textbf{答案:}D%??


\bigskip


\noindent\textbf{问题:}下列哪种调制被VHF和UHF业余电台本地通信广泛使用?

\begin{enumerate}[label=\Alph*), leftmargin=3em]
	\item FM
	\item SSB
	\item AM
	\item PSK
\end{enumerate}


\noindent\textbf{解说:}\textbf{FM}调制被VHF和UHF业余电台本地通信广泛使用。\\\noindent\textbf{答案:}A%??


\bigskip


\noindent\textbf{问题:}在给业余收发信机供电的整流电源中,开关电源可以做得比变压器直接降压整流的线性电源轻巧是因为:

\begin{enumerate}[label=\Alph*), leftmargin=3em]
	\item 开关电源使用轻质铝材做散热器,不需要通过笨重的铁芯变压器散热
	\item 开关电路中半导体元器件的损耗小,可以使用轻巧的小规格元器件
	\item 开关电源使用了半导体器件,不需要变压器
	\item 开关电源中变压器的工作频率高得多,可以缩小磁性材料截面和减少线圈匝数
\end{enumerate}

\noindent\textbf{解说:}开关电源可以做得比变压器直接降压整流的线性电源轻巧是因为\textbf{开关电源中变压器的工作频率高得多,可以缩小磁性材料截面和减少线圈匝数}。\\\noindent\textbf{答案:}D%??


\bigskip


\noindent\textbf{问题:}由半波长偶极天线和馈电电缆构成的天馈系统,理想的工作状态是:

\begin{enumerate}[label=\Alph*), leftmargin=3em]
	\item 天线上只有驻波,馈线上只有行波
	\item 天线上只有行波,馈线上只有驻波
	\item 天线和馈线上都只有驻波
	\item 天线和馈线上都只有行波
\end{enumerate}

\noindent\textbf{解说:}由半波长偶极天线和馈电电缆构成的天馈系统,理想的工作状态是\textbf{天线上只有驻波,馈线上只有行波}。\\\noindent\textbf{答案:}A%??


\bigskip


\noindent\textbf{问题:}在零仰角附近具有主辐射瓣的垂直接地天线,其振子的长度应为:

\begin{enumerate}[label=\Alph*), leftmargin=3em]
	\item 1/4波长的偶数倍
	\item 1/4波长的任意整数倍
	\item 1/4波长的奇数倍
	\item 1/2波长的奇数倍
\end{enumerate}

\noindent\textbf{解说:}在零仰角附近具有主辐射瓣的垂直接地天线,其振子的长度应为\textbf{1/4波长的奇数倍}。\\\noindent\textbf{答案:}C%??


\bigskip


\noindent\textbf{问题:}在为业余电台选购射频电缆作为天线馈线时,最重要的两项电气参数是:

\begin{enumerate}[label=\Alph*), leftmargin=3em]
	\item 长度为1/4波长、终端分别为开路和短路时的电压驻波比
	\item 铜或铝导线的截面积和常温下的最大额定电流
	\item 导线绝缘层的耐压和最高额定环境温度
	\item 特性阻抗和工作频率下单位长度的传输功率损耗
\end{enumerate}

\noindent\textbf{解说:}在为业余电台选购射频电缆作为天线馈线时,最重要的两项电气参数是\textbf{特性阻抗和工作频率下单位长度的传输功率损耗}。\\\noindent\textbf{答案:}D%??


\bigskip


\noindent\textbf{问题:}通常把垂直偶极天线或者垂直接地天线称为“全向天线”,是因为:

\begin{enumerate}[label=\Alph*), leftmargin=3em]
	\item 它们在水平方向和垂直方向都有指向性
	\item 它们在空间没有指向性,向各个方向均匀辐射
	\item 它们在垂直方向没有指向性,但在立体空间有方向性
	\item 它们在水平方向没有指向性,但在立体空间有方向性
\end{enumerate}

\noindent\textbf{解说:}通常把垂直偶极天线或者垂直接地天线称为“全向天线”,是因为\textbf{它们在水平方向没有指向性,但在立体空间有方向性}。\\\noindent\textbf{答案:}D%??


\bigskip


\noindent\textbf{问题:}振子电气长度为1/4波长的垂直接地天线的最大辐射方向为:

\begin{enumerate}[label=\Alph*), leftmargin=3em]
	\item 在水平方向没有指向性,在垂直方向指向水平面
	\item 在水平方向呈现8字形方向特性
	\item 在垂直方向没有指向性,在水平方向指向前方
	\item 在空间没有指向性,向各个方向均匀辐射
\end{enumerate}

\noindent\textbf{解说:}振子电气长度为1/4波长的垂直接地天线的最大辐射方向为:\textbf{在水平方向没有指向性,在垂直方向指向水平面}。\\\noindent\textbf{答案:}A%??


\bigskip


\noindent\textbf{问题:}垂直接地天线(GP)的构造为电气长度为1/4波长的垂直振子加一个“接地”反射体,因其简单而被大量应用于手持和车载业余电台,但这种天线的实际工作情况往往与理论值相差较大,尤其在频率较低的频段。最常见的原因和改善办法是:

\begin{enumerate}[label=\Alph*), leftmargin=3em]
	\item 一般天线与电缆直接相连,匹配不良。在天线和电联之间加接“巴伦”即可解决问题
	\item 1/4波长垂直振子太短。在振子中串联加感线圈即可解决问题
	\item 1/4波长垂直振子太短。改成1/2波长即可解决问题
	\item 缺乏有效的接地反射体。GP天线必须有足够大的接地反射体来形成振子镜像,否则谐振频率和阻抗都将与理论值有显著偏差,应尽量用大面积金属体与天线的接地端直接连接
\end{enumerate}

\noindent\textbf{解说:}最常见的原因是:\textbf{缺乏有效的接地反射体。GP天线必须有足够大的接地反射体来形成振子镜像,否则谐振频率和阻抗都将与理论值有显著偏差},改善办法是:\textbf{应尽量用大面积金属体与天线的接地端直接连接}。\\\noindent\textbf{答案:}D


\bigskip


\noindent\textbf{问题:}天线增益是指:

\begin{enumerate}[label=\Alph*), leftmargin=3em]
	\item 天线辐射到空间的电磁波功率与输入到天线的总射频功率之比
	\item 天线在最大辐射方向上的辐射功率密度与相同条件下基准天线的辐射功率密度之比
	\item 输入到天线的总射频功率与天线辐射到空间的电磁波功率之比
	\item 天线在接收或发射信号时所具有的放大倍数
\end{enumerate}

\noindent\textbf{解说:}天线增益是指\textbf{天线在最大辐射方向上的辐射功率密度与相同条件下基准天线的辐射功率密度之比}。\\\noindent\textbf{答案:}B


\bigskip


\noindent\textbf{问题:}某商品天线说明书给出的增益指标以dBi为单位。其意义为:
\begin{enumerate}[label=\Alph*), leftmargin=3em]
	\item “相对于无方向性点源天线的增益”,即最大辐射方向上的辐射功率密度与理想点源天线的辐射功率密度之比
	\item “相对于半波长偶极子天线的增益”,即最大辐射方向上的辐射功率密度与半波长偶极振子的最大辐射功率密度之比
	\item “相对于垂直短天线的增益”,即以高度远小于波长的短垂直天线作为基准作为比较基准得到的天线增益
	\item “相对于1/4波长接地天线的增益”,即以1/4波长垂直接地天线作为比较基准得到的天线增益
\end{enumerate}

\noindent\textbf{解说:}dBi意义为\textbf{“相对于无方向性点源天线的增益”,即最大辐射方向上的辐射功率密度与理想点源天线的辐射功率密度之比}。\\\noindent\textbf{答案:}A


\bigskip


\noindent\textbf{问题:}某商品天线说明书给出的增益指标以dBd为单位。其意义为:

\begin{enumerate}[label=\Alph*), leftmargin=3em]
	\item “相对于1/4波长接地天线的增益”,即以1/4波长垂直接地天线作为比较基准得到的天线增益
	\item 这种表达没有说清楚计算增益所采用的比较基准,缺乏实际意义
	\item “相对于无方向性点源天线的增益”,即最大辐射方向上的辐射功率密度与理想点源天线的辐射功率密度之比
	\item “相对于半波长偶极子天线的增益”,即最大辐射方向上的辐射功率密度与半波长偶极振子的最大辐射功率密度之比
\end{enumerate}

\noindent\textbf{解说:}dBd意义为\textbf{“相对于半波长偶极子天线的增益”,即最大辐射方向上的辐射功率密度与半波长偶极振子的最大辐射功率密度之比}。\\\noindent\textbf{答案:}D


\bigskip


\noindent\textbf{问题:}某商品天线说明书给出的增益指标以dB为单位。其意义为:

\begin{enumerate}[label=\Alph*), leftmargin=3em]
	\item “相对于无方向性点源天线的增益”,即最大辐射方向上的辐射功率密度与理想点源天线的辐射功率密度之比
	\item 这种表达没有说清楚计算增益所采用的比较基准,缺乏实际意义
	\item “相对于垂直短天线天线的增益”,即以高度远小于波长的短垂直天线作为基准作为比较基准得到的天线增益
	\item “相对于半波长偶极子天线的增益”,即最大辐射方向上的辐射功率密度与半波长偶极振子的最大辐射功率密度之比
\end{enumerate}

\noindent\textbf{解说:这种表达没有说清楚计算增益所采用的比较基准,缺乏实际意义}。\\\noindent\textbf{答案:}B


\bigskip


\noindent\textbf{问题:}两款VHF垂直全向天线,用作发射。甲天线增益4.5dBd,乙天线增益为5.85dBi。它们在远处某接收天线中形成的信号功率差为:

\begin{enumerate}[label=\Alph*), leftmargin=3em]
	\item 甲信号比乙信号强1.35dB
	\item 乙信号比甲信号强3.5dB
	\item 甲信号比乙信号强0.8dB
	\item 乙信号比甲信号强1.35dB
\end{enumerate}

\noindent\textbf{解说:甲信号比乙信号强0.8dB}。\\\noindent\textbf{答案:}C%%%%%%%%%%%


\bigskip


\noindent\textbf{问题:}两款VHF垂直全向天线,用作发射。甲天线增益2.9dBd,乙天线增益为5.85dBi。它们在远处某接收天线中形成的信号功率差为:

\begin{enumerate}[label=\Alph*), leftmargin=3em]
	\item 乙信号比甲信号强2.95dB
	\item 甲信号比乙信号强2.95dB
	\item 乙信号比甲信号强0.8dB
	\item 乙信号比甲信号强7.1dB
\end{enumerate}

\noindent\textbf{解说:乙信号比甲信号强0.8dB}。\\\noindent\textbf{答案:}C%%%%%%%%%%%


\bigskip


\noindent\textbf{问题:}以dBd为单位的天线增益则是指:

\begin{enumerate}[label=\Alph*), leftmargin=3em]
	\item 最大辐射方向辐射功率密度与1/4波长垂直接地天线水平方向辐射功率密度之比的dB值
	\item 最大辐射方向辐射功率密度与半波长振子最大辐射方向辐射功率密度之比的dB值
	\item 最大辐射方向辐射功率密度与最小辐射方向辐射功率密度之比的dB值
	
	\item 最大辐射方向辐射功率密度与理想点源天线最大辐射方向辐射功率密度之比的dB值
\end{enumerate}

\noindent\textbf{解说:}以dBd为单位的天线增益是指\textbf{最大辐射方向辐射功率密度与半波长振子最大辐射方向辐射功率密度之比的dB值}。\\\noindent\textbf{答案:}B


\bigskip


\noindent\textbf{问题:}以dBi为单位的天线增益则是指:
\begin{enumerate}[label=\Alph*), leftmargin=3em]
	\item 最大辐射方向辐射功率密度与理想点源天线最大辐射方向辐射功率密度之比的dB值
	\item 最大辐射方向辐射功率密度与最小辐射方向辐射功率密度之比的dB值
	\item 最大辐射方向辐射功率密度与1/4波长垂直接地天线水平方向辐射功率密度之比的dB值
	\item 最大辐射方向辐射功率密度与半波长振子最大辐射方向辐射功率密度之比的dB值
\end{enumerate}

\noindent\textbf{解说:}以dBi为单位的天线增益是指\textbf{最大辐射方向辐射功率密度与理想点源天线最大辐射方向辐射功率密度之比的dB值}。\\\noindent\textbf{答案:}A


\bigskip


\noindent\textbf{问题:}安装哪个部件可以减少在音频同轴电缆屏蔽层外皮中的感应射频电流?

\begin{enumerate}[label=\Alph*), leftmargin=3em]
	\item 在电缆前端前置放大器
	\item 在电缆芯线中串联低通滤波器
	\item 在电缆芯线中串联带通滤波器	
	\item 在电缆外面套铁氧体磁环
\end{enumerate}

\noindent\textbf{解说:}\textbf{在电缆外面套铁氧体磁环}可以减少在音频同轴电缆屏蔽层外皮中的感应射频电流。\\\noindent\textbf{答案:}D


\bigskip


\noindent\textbf{问题:}射频同轴电缆在业余电台中的主要用处是?

\begin{enumerate}[label=\Alph*), leftmargin=3em]
	\item 将信号从TNC传送到电脑
	\item 用来遮蔽天线塔上的支杆,管子等柱状物体
	\item 将电能从汽车的电池输送到移动车载电台
	\item 将无线电信号从发射机传送到天线
\end{enumerate}

\noindent\textbf{解说:}射频同轴电缆在业余电台中的主要用处是\textbf{将无线电信号从发射机传送到天线}。\\\noindent\textbf{答案:}D


\bigskip


\noindent\textbf{问题:}假负载的主要作用是?

\begin{enumerate}[label=\Alph*), leftmargin=3em]
	\item 增强接收机的信噪比
	\item 防止发射机过调制
	\item 在测试设备时不让无线电信号真正地发射出去
	\item 改善天线的辐射效率
\end{enumerate}

\noindent\textbf{解说:}假负载的主要作用是\textbf{在测试设备时不让无线电信号真正地发射出去}。\\\noindent\textbf{答案:}C


\bigskip


\noindent\textbf{问题:}下列哪一个设备可以用来测定天线当前的谐振频率?

\begin{enumerate}[label=\Alph*), leftmargin=3em]
	\item 电子管电压表
	\item 天线分析仪
	\item 频率计数器
	\item 品质因数计
\end{enumerate}

\noindent\textbf{解说:}\textbf{天线分析仪}可以用来测定天线当前的谐振频率。\\\noindent\textbf{答案:}B


\bigskip


\noindent\textbf{问题:}在业余无线电通信中,“驻波比”通常用来衡量:

\begin{enumerate}[label=\Alph*), leftmargin=3em]
	\item 电台接地的质量
	\item 传输线的制造质量
	\item 负载与传输线的匹配质量
	\item 发射机的能效比
\end{enumerate}

\noindent\textbf{解说:}在业余无线电通信中,“驻波比”通常用来衡量\textbf{负载与传输线的匹配质量}。\\\noindent\textbf{答案:}C


\bigskip


\noindent\textbf{问题:}天线与馈线完美匹配时,在驻波表中显示的驻波比是?

\begin{enumerate}[label=\Alph*), leftmargin=3em]
	\item 0
	\item 2:1
	\item 1:1
	\item 1:3
\end{enumerate}

\noindent\textbf{解说:}当天线与馈线完美匹配时,在驻波表中显示的驻波比是\textbf{1:1}。\\\noindent\textbf{答案:}C


\bigskip


\noindent\textbf{问题:}馈线中的功率损耗会?

\begin{enumerate}[label=\Alph*), leftmargin=3em]
	\item 使驻波比增加
	\item 使你的信号失真
	\item 传回你的发射机并可能导致损害
	\item 变成热量
\end{enumerate}

\noindent\textbf{解说:}馈线中的功率损耗会\textbf{变成热量}。\\\noindent\textbf{答案:}D


\bigskip


\noindent\textbf{问题:}导致同轴电缆损害的最常见的原因是什么?

\begin{enumerate}[label=\Alph*), leftmargin=3em]
	\item 伽马射线
	\item 电缆受潮
	\item 速度因子超过了1
	\item 过载
\end{enumerate}

\noindent\textbf{解说:}导致同轴电缆损害的最常见的原因是\textbf{电缆受潮}。\\\noindent\textbf{答案:}B


\bigskip


\noindent\textbf{问题:}为什么要求同轴电缆的外皮能抵挡紫外线?

\begin{enumerate}[label=\Alph*), leftmargin=3em]
	
	\item 紫外线会使电缆外皮中的功率损耗加大
	\item 这样就可以减少谐波
	\item 紫外线能破坏电缆的外皮,使水分渗入
	\item 紫外线和射频信号会混合在一起导致干扰
\end{enumerate}

\noindent\textbf{解说:}导致同轴电缆损害的最常见的原因是电缆受潮,所以我们要求同轴电缆的外皮能抵挡紫外线,以尽量避免\textbf{电缆的外皮被紫外线破坏而使水分渗入}。\\\noindent\textbf{答案:}C%?


\bigskip


\noindent\textbf{问题:}和固体电介质同轴电缆相比,空气电介质同轴电缆的劣势是什么?

\begin{enumerate}[label=\Alph*), leftmargin=3em]
	\item 空气电介质电缆不能在冰点以下使用
	\item 空气电介质电缆的每米损耗更大
	\item 空气电介质电缆要采取特别的手段来防止水分进入电缆
	\item 空气电介质电缆不能用于VHF或UHF段
\end{enumerate}

\noindent\textbf{解说:}和固体电介质同轴电缆相比,空气电介质同轴电缆的劣势是\textbf{它要采取特别的手段来防止水分进入电缆}。\\\noindent\textbf{答案:}C%??


\bigskip


\noindent\textbf{问题:}下列关于直立天线的陈述,哪一项是正确的?

\begin{enumerate}[label=\Alph*), leftmargin=3em]
	\item 相位被反相了
	\item 该天线发射的电磁波磁场垂直于地面
	\item 该天线发射的电磁波电场垂直于地面
	\item 相位被倒转了
\end{enumerate}

\noindent\textbf{解说:}\textbf{直立天线发射的电磁波电场垂直于地面}。\\\noindent\textbf{答案:}C


\bigskip


\noindent\textbf{问题:}就大多数手持电台随机附送的“橡皮天线”来说,它的劣势是什么?

\begin{enumerate}[label=\Alph*), leftmargin=3em]
	\item 如果橡胶的一头损坏了,那么它很快就会解体
	\item 相对于全尺寸天线,它的发射和接收效率较低
	\item 它发射的是圆极化的信号
	\item 以上三项全部正确
\end{enumerate}

\noindent\textbf{解说:}就大多数手持电台随机附送的“橡皮天线”来说,它的劣势是\textbf{相对于全尺寸天线,它的发射和接收效率较低}。\\\noindent\textbf{答案:}B


\bigskip


\noindent\textbf{问题:}天线的增益是什么意思?

\begin{enumerate}[label=\Alph*), leftmargin=3em]
	\item 相对于参考天线,在某一方向上信号强度的增加
	\item 相对于参考天线,发射或接收时的阻抗增加
	\item 对于发射机的功率输入,天线附加的一部分额外功率
	\item 当发射机在更高的频率发射时,天线会额外地减少一部分功率
\end{enumerate}

\noindent\textbf{解说:}天线的增益是指\textbf{相对于参考天线,在某一方向上信号强度的增加}。\\\noindent\textbf{答案:}A


\bigskip


\noindent\textbf{问题:}为什么在使用同轴电缆连接天线时,最好有一个较低的驻波比?

\begin{enumerate}[label=\Alph*), leftmargin=3em]
	\item 能延长天线的使用寿命
	\item 使能量更有效率地传送,减少损耗
	\item 减少对电视的干扰
	\item 能延长发信机的使用寿命
\end{enumerate}

\noindent\textbf{解说:}馈线中的功率损耗会变成热量。在使用同轴电缆连接天线时,为\textbf{使能量更有效率地传送,减少损耗},最好有一个较低的驻波比。\\\noindent\textbf{答案:}B


\bigskip


\noindent\textbf{问题:}业余无线电普遍使用的同轴电缆的特性阻抗是?

\begin{enumerate}[label=\Alph*), leftmargin=3em]
	\item 8欧姆
	\item 50欧姆
	\item 600欧姆
	\item 12欧姆
\end{enumerate}

\noindent\textbf{解说:}业余无线电普遍使用的同轴电缆的特性阻抗是\textbf{50欧姆}。\\\noindent\textbf{答案:}B


\bigskip


\noindent\textbf{问题:}为什么同轴电缆在业余无线电界的使用相对于其他馈线来说更多?

\begin{enumerate}[label=\Alph*), leftmargin=3em]
	\item 因为它可以相对其它馈线传送更大的功率
	\item 因为它比其它任何一种馈线的损耗都低
	\item 因为它使用方便,与周围环境之间的相互影响小
	\item 因为它比其它任何一种馈线都便宜
\end{enumerate}

\noindent\textbf{解说:}\textbf{因为同轴电缆使用方便,与周围环境之间的相互影响小},所以在业余无线电界使用同轴电缆比较多。\\\noindent\textbf{答案:}C


\bigskip


\noindent\textbf{问题:}通过同轴电缆的信号频率越高,通常产生什么影响?

\begin{enumerate}[label=\Alph*), leftmargin=3em]
	\item 驻波比越高
	\item 特性阻抗越高
	\item 反射功率越高
	\item 损耗越高
\end{enumerate}

\noindent\textbf{解说:}通过同轴电缆的信号频率越高,产生的\textbf{损耗越高}。\\\noindent\textbf{答案:}D


\bigskip


\noindent\textbf{问题:}对于400MHz以上的信号,通常会使用的同轴电缆连接器是:

\begin{enumerate}[label=\Alph*), leftmargin=3em]
	\item N型连接器
	\item M型连接器
	\item DB-23型连接器
	\item RS-213型连接器
\end{enumerate}

\noindent\textbf{解说:}对于400MHz以上的信号,通常会使用\textbf{N型同轴电缆连接器}。\\\noindent\textbf{答案:}A


\bigskip


\noindent\textbf{问题:}下列哪一项有可能导致驻波比不稳定读数的?

\begin{enumerate}[label=\Alph*), leftmargin=3em]
	\item 天线与馈线的连接头接触不良
	\item 来自其他电台的干扰导致信号失真
	\item 发射机采用了频率调制
	\item 发射机过调制
\end{enumerate}

\noindent\textbf{解说:}\textbf{天线与馈线的连接头接触不良}有可能导致驻波比读数不稳定。\\\noindent\textbf{答案:}A


\bigskip


\noindent\textbf{问题:}下列馈线中,哪一种在VHF和UHF频段的损耗最小?

\begin{enumerate}[label=\Alph*), leftmargin=3em]
	\item 多芯不平衡电缆
	\item 75欧姆同轴软电缆
	\item 50欧姆同轴软电缆
	\item 空气介质同轴硬电缆
\end{enumerate}

\noindent\textbf{解说:}\textbf{空气介质同轴硬电缆}在VHF和UHF频段的损耗最小。\\\noindent\textbf{答案:}D


\bigskip


\noindent\textbf{问题:}无线电波在真空中的速度大致为:(“\texttt{x^m}”表示“x的m次方”)

\begin{enumerate}[label=\Alph*), leftmargin=3em]
	\item \texttt{3×10^7} 米/秒
	\item \texttt{6.88×10^6} 米/秒
	\item \texttt{3×10^8} 米/秒
	\item \texttt{3.1416×10^12} 米/秒
\end{enumerate}

\noindent\textbf{解说:}无线电波在真空中的传播速度和真空中的光速一样,约30万公里/小时,即\textbf{\texttt{3×10^8 }米/秒}。\\\noindent\textbf{答案:}C


\bigskip


\noindent\textbf{问题:}天线工程中计算振子长度时需要知道电波在天线中的传播速度。电波在天线导线中的传播速度大约是:

\begin{enumerate}[label=\Alph*), leftmargin=3em]
	\item 真空波速的0.95倍
	\item 与真空波速相同
	\item 真空波速的1.41倍
	\item 真空波速的1.05倍
\end{enumerate}

\noindent\textbf{解说:}电波在天线导线中的传播速度大约是\textbf{真空波速的0.95倍}。\\\noindent\textbf{答案:}A


\bigskip


\noindent\textbf{问题:}天线工程中计算馈线长度时经常需要知道电波在馈线中的传播速度。常用业余频段的电波在同轴电缆中的传播速度大约是:

\begin{enumerate}[label=\Alph*), leftmargin=3em]
	\item 真空波速的1.54倍
	\item 真空波速的0.65倍
	\item 与真空波速相同
	\item 真空波速的1.65倍
\end{enumerate}

\noindent\textbf{解说:}常用业余频段的电波在同轴电缆中的传播速度大约是\textbf{真空波速的0.65倍}。\\\noindent\textbf{答案:}B


\bigskip


\noindent\textbf{问题:}无线电波按传播方式可主要分为下列种类:

\begin{enumerate}[label=\Alph*), leftmargin=3em]
	\item 地面波、天波、空间波、散射波等
	\item 正弦波、方波、三角波等
	\item 长波、中波、短波、超短波、微波等
	\item 调幅波、调频波、调相波等
\end{enumerate}

\noindent\textbf{解说:}无线电波按传播方式可主要有\textbf{地面波、天波、空间波、散射波等}。\\\noindent\textbf{答案:}A


\bigskip


\noindent\textbf{问题:}无线电波在自由空间中的传播路径损耗遵循下列规律:

\begin{enumerate}[label=\Alph*), leftmargin=3em]
	\item 与距离的平方成正比,与频率无关
	\item 与距离的平方成正比,与频率的平方成反比
	\item 与距离的平方成正比,与频率的平方成正比
	\item 与距离成正比,与频率的平方成正比
\end{enumerate}

\noindent\textbf{解说:}无线电波在自由空间中的传播路径损耗\textbf{与距离的平方成正比,与频率的平方成正比}。\\\noindent\textbf{答案:}C


\bigskip


\noindent\textbf{问题:}地波是沿地面传播的无线电波,其衰减因子取决于:

\begin{enumerate}[label=\Alph*), leftmargin=3em]
	\item 电波频率、发射功率和天线增益
	\item 电波频率、太阳活动情况和地磁活动情况
	\item 天线高度、发射功率和调制方式
	\item 电波频率、地面导电率和传播距离
\end{enumerate}

\noindent\textbf{解说:}地波的衰减因子取决于\textbf{电波频率、地面导电率和传播距离}。\\\noindent\textbf{答案:}D

\bigskip


\noindent\textbf{问题:}决定超短波视距传播距离极限的主要因素是:

\begin{enumerate}[label=\Alph*), leftmargin=3em]
	\item 发射天线和接收天线离海平面的绝对高度值
	\item 发射天线和接收天线离地面的相对于波长的高度系数,即离地高度除以波长
	\item 发射天线和接收天线的增益
	\item 发射天线和接收天线离地面的相对高度值
\end{enumerate}

\noindent\textbf{解说:}决定超短波视距传播距离极限的主要因素是\textbf{发射天线和接收天线离地面的相对高度值}。\\\noindent\textbf{答案:}D


\bigskip


\noindent\textbf{问题:}多径传播对UHF波段或VHF波段数据通信的影响是:

\begin{enumerate}[label=\Alph*), leftmargin=3em]
	\item 可能使误码率增大
	\item 随着传播路径的增加,接收到的数据速率会线性地减小
	\item 随着传播路径的增加,数据通信速率会线性地增加
	\item 如果使用FM方式调制的话,不会观察到任何反常现象
\end{enumerate}

\noindent\textbf{解说:}多径传播可能使UHF波段或VHF波段数据通信的\textbf{误码率增大}。\\\noindent\textbf{答案:}A


\bigskip


\noindent\textbf{问题:}在一个周期内,电磁波走过一定的距离,这个距离叫做:

\begin{enumerate}[label=\Alph*), leftmargin=3em]
	\item 波长
	\item 波展
	\item 波速
	\item 波形
\end{enumerate}

\noindent\textbf{解说:}在一个周期内,电磁波走过的一定的距离叫做\textbf{波长}。\\\noindent\textbf{答案:}A


\bigskip


\noindent\textbf{问题:}无线电波的两个组成部分是?

\begin{enumerate}[label=\Alph*), leftmargin=3em]
	\item 电压和电流
	\item 电场和磁场
	\item 直流和交流
	\item 电离辐射和非电离辐射
\end{enumerate}

\noindent\textbf{解说:}无线电波由\textbf{电场和磁场}组成。\\\noindent\textbf{答案:}B


\bigskip


\noindent\textbf{问题:}无线电波的传播速度有多快?

\begin{enumerate}[label=\Alph*), leftmargin=3em]
	\item 和它的波长成反比例关系
	\item 和音速一样
	\item 和光速一样
	\item 速度随着频率的增加而增加
\end{enumerate}

\noindent\textbf{解说:}无线电波的传播速度\textbf{和光速一样}。\\\noindent\textbf{答案:}C


\bigskip


\noindent\textbf{问题:}无线电波的频率和它的波长有什么样的关系?

\begin{enumerate}[label=\Alph*), leftmargin=3em]
	\item 该电磁波的波长取决于信号所占的带宽
	\item 如果频率增加,则波长变短
	\item 如果频率增加,则波长变长
	\item 波长与频率之间没有固定的联系
\end{enumerate}

\noindent\textbf{解说:}无线电波的\textbf{频率如果增加,则波长变短}。\\\noindent\textbf{答案:}B


\bigskip


\noindent\textbf{问题:}在已知电磁波频率的情况下,下面哪一种方法可以用来计算它的波长?

\begin{enumerate}[label=\Alph*), leftmargin=3em]
	\item 使用频率的赫兹数(Hz)乘以300可以得到以米为单位的波长
	\item 使用300除以频率的兆赫数(MHz)可以得到以米为单位的波长
	\item 将频率的兆赫数(MHz)除以300可以得到以米为单位的波长
	\item 将频率的赫兹数(Hz)除以300可以得到以米为单位的波长
\end{enumerate}

\noindent\textbf{解说:}通过\textbf{使用300除以频率的兆赫数(MHz)可以得到以米为单位的波长}。\\\noindent\textbf{答案:}B


\bigskip


\noindent\textbf{问题:}在自由空间中,无线电波的速度约是多少?

\begin{enumerate}[label=\Alph*), leftmargin=3em]
	\item 300000000米/每秒
	\item 300000英里/每小时
	\item 3000千米/每秒
	\item 18600英里/每小时
\end{enumerate}

\noindent\textbf{解说:}无线电波在自由空间中的速度约为30万公里/每小时,即\textbf{300000000米/每秒}。\\\noindent\textbf{答案:}A


\bigskip


\noindent\textbf{问题:}如果你收到了一个从上千公里以外的距离传播过来的VHF信号,最可能的原因是:

\begin{enumerate}[label=\Alph*), leftmargin=3em]
	\item 信号被突发E电离层反射过来
	\item 信号经过了地下传输过来
	\item 信号被附近的雷雨区反射过来
	\item 信号经过了宇宙的反射
\end{enumerate}

\noindent\textbf{解说:}当\textbf{信号被突发E电离层反射过来}时,有可能使你能收到从上千公里以外的距离传播过来的VHF信号。\\\noindent\textbf{答案:}A


\bigskip


\noindent\textbf{问题:}下列哪一个缩写经常在无线电工程文章中用来代表各种类型的无线电波?

\begin{enumerate}[label=\Alph*), leftmargin=3em]
	\item AF
	\item VHF
	\item RF
	\item HF
\end{enumerate}

\noindent\textbf{解说:RF}是射频无线电波的缩写,经常在无线电工程文章中用来代表各种类型的无线电波。\\\noindent\textbf{答案:}C


\bigskip


\noindent\textbf{问题:}在空间中传播的电磁波,一般被叫做什么?

\begin{enumerate}[label=\Alph*), leftmargin=3em]
	\item 声波
	\item 压力波
	\item 重力波
	\item 无线电波
\end{enumerate}

\noindent\textbf{解说:}《中华人民共和国无线电频率划分规定》第1.1.4款规定:\textbf{无线电波}频率规定在3000GHz以下,不用人造波导而在空间传播的电磁波。\\\noindent\textbf{答案:}D


\bigskip


\noindent\textbf{问题:}收发信机面板上或设置菜单中的符号VOX代表什么功能?

\begin{enumerate}[label=\Alph*), leftmargin=3em]
	\item 发信机声控,接入后将根据对话筒有无语音输入的判别自动控制收发转换
	\item 自动天线调谐,对天线电路的电压驻波比进行检测并进行自动补偿,以维持最小驻波比
	\item 发信自动音量控制,对音频输入电平进行检测并反馈控制,以维持其在适当限度之内
	\item 发信自动电平控制,对射频输出电平进行检测并反馈控制,以维持其在适当限度之内
\end{enumerate}

\noindent\textbf{解说:}VOX代表\textbf{发信机声控,接入后将根据对话筒有无语音输入的判别自动控制收发转换}。\\\noindent\textbf{答案:}A


\bigskip


\noindent\textbf{问题:}收发信机中的PTT是指什么信号?

\begin{enumerate}[label=\Alph*), leftmargin=3em]
	\item 收信机前置放大器,在接收微弱信号时接入(此时某些技术指标可能低于额定值)
	\item 自动天线调谐,对天线电路的电压驻波比进行检测并进行自动补偿,以维持最小驻波比
	\item 发信语音压缩,对音频输入电平进行检测并反馈控制,以提升语音包络幅度较小的部分
	\item 按键发射,有信号(一般为对地接通)时发射机由等待转为发射
\end{enumerate}

\noindent\textbf{解说:}PTT代表\textbf{按键发射,有信号(一般为对地接通)时发射机由等待转为发射}。\\\noindent\textbf{答案:}D


\bigskip


\noindent\textbf{问题:}收发信机面板上或设置菜单中的符号SQL代表什么功能?

\begin{enumerate}[label=\Alph*), leftmargin=3em]
	\item 发信语音压缩,对音频输入电平进行检测并反馈控制,以提升语音包络幅度较小的部分
	\item 收信机前置放大器,在接收微弱信号时接入(此时某些技术指标可能低于额定值)
	\item 静噪控制,检测到接收信号低于一定电平时关断音频输出
	\item 自动天线调谐,对天线电路的电压驻波比进行检测并进行自动补偿,以维持最小驻波比
\end{enumerate}

\noindent\textbf{解说:}SQL代表\textbf{静噪控制,检测到接收信号低于一定电平时关断音频输出},可在没有信号的情况下关闭音频输出,使其不会输出噪声 。\\\noindent\textbf{答案:}C


\bigskip


\noindent\textbf{问题:}有些调频接收机的参数设置菜单有NFM和WFM两种选择。它们的含义是:

\begin{enumerate}[label=\Alph*), leftmargin=3em]
	\item NFM为窄带调频方式,适用于信道带宽25kHz/12.5kHz的通信信号;WFM为宽带调频方式,适用于接收信道带宽180kHz左右的广播信号
	\item NFM代表数字化语音方式,WFM代表模拟语音方式
	\item NFM为单频率守候方式,WFM为双频率守候方式
	\item NFM为调频通信本地方式(较低灵敏度),WFM为调频通信远程方式(最高灵敏度)
\end{enumerate}

\noindent\textbf{解说:NFM为窄带调频方式,适用于信道带宽25kHz/12.5kHz的通信信号;WFM为宽带调频方式,适用于接收信道带宽180kHz左右的广播信号}。\\\noindent\textbf{答案:}A


\bigskip


\noindent\textbf{问题:}某些对讲机具有发送DTMF码的功能。缩写DTMF指的是:

\begin{enumerate}[label=\Alph*), leftmargin=3em]
	\item 数字设备识别码,即在松开PTT按键时自动发送一串代表设备代号的二进制数据
	\item 亚音调静噪,即从67-250.3Hz的38个亚音调频率中选取一个作为选通信号,代表38种状态之一,接收机没有收到特定的选通信号时自动关闭音频输出	
	\item 自动静噪,即在接收机没有收到信号时自动关闭音频输出
	\item 双音多频编码,由8个音调频率中的两个频率组合成的控制信号,代表16种状态之一,用于遥控和传输数字等简单字符
\end{enumerate}

\noindent\textbf{解说:}DTMF指\textbf{双音多频编码,由8个音调频率中的两个频率组合成的控制信号,代表16种状态之一,用于遥控和传输数字等简单字符}。\\\noindent\textbf{答案:}D


\bigskip


\noindent\textbf{问题:}某些对讲机具有发送CTCSS码的功能。缩写CTCSS指的是:

\begin{enumerate}[label=\Alph*), leftmargin=3em]
	\item 双音多频编码,由8个音调频率中的两个频率组合成的控制信号,代表16种状态之一,用于遥控和传输数字等简单字符
	\item 亚音调静噪,即从67-250.3Hz的38个亚音调频率中选取一个作为选通信号,代表38种状态之一,接收机没有收到特定的选通信号时自动关闭音频输出
	\item 自动静噪,即在接收机没有收到信号时自动关闭音频输出
	\item 数字设备识别码,即在松开PTT按键时自动发送一串代表设备代号的二进制数据
\end{enumerate}

\noindent\textbf{解说:}CTCSS指\textbf{亚音调静噪,即从67-250.3Hz的38个亚音调频率中选取一个作为选通信号,代表38种状态之一,接收机没有收到特定的选通信号时自动关闭音频输出}。\\\noindent\textbf{答案:}B


\bigskip


\noindent\textbf{问题:}关于是否可以在FM话音通信时单凭接收机听到对方语音的音量大小来准确判断对方信号的强弱,正确答案及其理由是:

\begin{enumerate}[label=\Alph*), leftmargin=3em]
	\item 不能。因为信号越强,自动增益控制作用也越强,增益的急剧减小使声音反而被压低
	\item 不能。因为鉴频输出大小只取决于射频信号的频偏,而且正常信号的幅度会被限幅电路切齐到同样大小
	\item 能。调频信号越强,频偏也必然越大,解调后的声音也越大
	\item 能。最后的信号是接收到的射频信号经过放大处理得到的,当然信号强声音越大
\end{enumerate}

\noindent\textbf{解说:不能。因为鉴频输出大小只取决于射频信号的频偏,而且正常信号的幅度会被限幅电路切齐到同样大小}。\\\noindent\textbf{答案:}B


\bigskip


\noindent\textbf{问题:}用设置在NFM方式的对讲机接收WFM信号,其效果为:

\begin{enumerate}[label=\Alph*), leftmargin=3em]
	\item 可以正常听到信号,但声音的高音频部分衰减较大,缺乏高音
	\item 听不到信号,但接收到信号时调频噪声会变得寂静
	\item 可以听到信号,但当调制信号幅度较大、音调较高时会发生明显非线性失真
	\item 可以正常听到信号,但声音比较小
\end{enumerate}

\noindent\textbf{解说:}其效果为\textbf{可以听到信号,但当调制信号幅度较大、音调较高时会发生明显非线性失真}。\\\noindent\textbf{答案:}C


\bigskip


\noindent\textbf{问题:}用设置在WFM方式的对讲机接收NFM信号,其效果为:

\begin{enumerate}[label=\Alph*), leftmargin=3em]
	\item 听不到信号,但接收到信号时调频噪声会变得寂静
	\item 可以正常听到信号,但声音的高音频部分衰减较大,缺乏高音
	\item 可以听到信号,但当调制信号幅度较大、音调较高时会发生明显非线性失真
	\item 可以正常听到信号,但声音比较小
\end{enumerate}

\noindent\textbf{解说:}其效果为\textbf{可以正常听到信号,但声音比较小}。\\\noindent\textbf{答案:}D

\bigskip


\noindent\textbf{问题:}调频接收机没有接收到信号时,会输出强烈的噪声。关于这种噪声的描述是:

\begin{enumerate}[label=\Alph*), leftmargin=3em]
	\item 由天线接收到的背景噪声的随机幅度变化经放大形成,其大小与天线背景噪声电压的平方成正比
	\item 由天线接收到的背景噪声的随机幅度变化经放大形成,其大小与天线背景噪声电压成正比
	\item 由天线接收到的背景噪声的随机幅度变化经放大形成,其大小与天线背景噪声电压的平方根成正比
	\item 由天线背景噪声和机内电路噪声的随机频率变化经鉴频形成,其大小与天线接收到的背景噪声幅度无关
\end{enumerate}

\noindent\textbf{解说:噪声由天线背景噪声和机内电路噪声的随机频率变化经鉴频形成,其大小与天线接收到的背景噪声幅度无关}。\\\noindent\textbf{答案:}D%%%%%%%%%%%%

\bigskip


\noindent\textbf{问题:}如果业余中继台发射机被断断续续的干扰信号所启动,夹杂着不清楚的语音,根据覆盖区内其他业余电台的监听,确定中继台上行频率并没有电台工作。则:

\begin{enumerate}[label=\Alph*), leftmargin=3em]
	\item 肯定是中继台接收机受到了人为恶意干扰
	\item 可能是中继台发射机发生了寄生振荡
	\item 可能是中继台接收机发生了寄生振荡
	\item 可能是中继台附近的两个其他发射机的强信号在中继台上行频率造成了互调干扰
\end{enumerate}

\noindent\textbf{解说:可能是中继台附近的两个其他发射机的强信号在中继台上行频率造成了互调干扰}。\\\noindent\textbf{答案:}D%???


\bigskip


\noindent\textbf{问题:}业余电台在进行业余卫星通信时使用超过常规要求的发射功率,造成的结果以及对这种做法的态度是:

\begin{enumerate}[label=\Alph*), leftmargin=3em]
	\item 上行功率越大,转发的效果越好,通信范围越大;可提倡
	\item 上行功率太大造成浪费和电磁污染;不提倡
	\item 上行功率超过一定值对通信效果改善不大,但并无明显坏处;无所谓
	\item 过强的上行信号会使卫星转发器压低对其他信道的转发功率,严重影响别人通信;必须反对
\end{enumerate}

\noindent\textbf{解说:过强的上行信号会使卫星转发器压低对其他信道的转发功率,严重影响别人通信;必须反对}。\\\noindent\textbf{答案:}D%???

\bigskip


\noindent\textbf{问题:}即使在空旷平地,接收到的本地VHF/UHF信号强度也可能会随着接收位置的移动而发生变化,最主要的可能原因是:

\begin{enumerate}[label=\Alph*), leftmargin=3em]
	\item 移动过程中设备与大地之间的分布电容发生微小的变动
	\item 不同接收位置大地导电率有差异
	\item 直射和经地面反射等多条路径到达的电波相位不同,互相叠加或抵消造成衰落(多径效应)
	\item 收发位置之间的空气流动造成电波折射不均匀而飘移
\end{enumerate}

\noindent\textbf{解说:}即使在空旷平地,接收到的本地VHF/UHF信号强度也可能会随着接收位置的移动而发生变化,最主要的可能原因是:\textbf{直射和经地面反射等多条路径到达的电波相位不同,互相叠加或抵消造成衰落(多径效应)}。\\\noindent\textbf{答案:}C%%%%%%%%%



\bigskip


\noindent\textbf{问题:}在相距不远的两点接收同一个远方信号,信号强度发生很大差别,且差别随两点间距离的增大呈周期性变化。这是因为:

\begin{enumerate}[label=\Alph*), leftmargin=3em]
	\item 大气扰动影响
	\item 地磁影响影响
	\item 发射机到两接收点传播距离不同造成传播路径衰耗不同
	\item 多径传播,各路径到达的信号相位延迟不同而互相干涉
\end{enumerate}

\noindent\textbf{解说:}在相距不远的两点接收同一个远方信号,信号强度发生很大差别,且差别随两点间距离的增大呈周期性变化。这是因为:\textbf{多径传播,各路径到达的信号相位延迟不同而互相干涉}。\\\noindent\textbf{答案:}D%%%%%%%%%%%%%

\bigskip


\noindent\textbf{问题:}收发信机中的静噪控制的目的是什么?

\begin{enumerate}[label=\Alph*), leftmargin=3em]
	\item 可以进行自动增益控制
	\item 在没有信号的情况下,关闭音频输出,使其不会输出噪音%%%%在没有信号的情况下,关闭音频输出,使其不会输出噪音。
	\item 使接收机的输出音量调到最大
	\item 控制发射机的输出功率
\end{enumerate}

\noindent\textbf{解说:}收发信机中的静噪控制的目的是\textbf{在没有信号的情况下,关闭音频输出,使其不会输出噪音}。\\\noindent\textbf{答案:}B%%%%%%%%%%%%
\bigskip


\noindent\textbf{问题:}如果对方报告你的调频电台发射的信号听起来失真严重、可辨度差,可能的原因是:

\begin{enumerate}[label=\Alph*), leftmargin=3em]
	\item 三项都可能
	\item 电台的发射频率不准确
	\item 电台所处的位置不好
	\item 电台的电源电压不足
\end{enumerate}

\noindent\textbf{解说:}如果对方报告你的调频电台发射的信号听起来失真严重、可辨度差,可能的原因是:\textbf{(一)电台的电源电压不足;(二)电台所处的位置不好;(三)电台的发射频率不准确}。\\\noindent\textbf{答案:}A

\bigskip


\noindent\textbf{问题:}一些VHF/UHF业余无线电调频手持对讲机或车载台的设置菜单中有“全频偏”和“半频偏”的选择,其表示的意义是:

\begin{enumerate}[label=\Alph*), leftmargin=3em]
	\item 全频偏方式适用于收发异频的中继台通信,半频偏方式适用于同频对讲通信
	\item 全频偏方式下发射频率的误差比半频偏方式大一倍
	\item 分别表示信道间隔为25kHz或者12.5kHz
	\item 全频偏方式语音信号经过压缩,半频偏方式语音信号只压缩低音频分量
\end{enumerate}

\noindent\textbf{解说:}\textbf{“全频偏”表示信道间隔为25kHz,“半频偏”表示信道间隔为12.5kHz}。\\\noindent\textbf{答案:}C

\bigskip


\noindent\textbf{问题:}用电压表检查一节干电池两端电压,未使用时测得1.5伏左右,用旧后测得1.2伏左右。正确的说法是:

\begin{enumerate}[label=\Alph*), leftmargin=3em]
	\item 旧干电池的电动势为1.5伏
	\item 旧干电池的电动势为0
	\item 旧干电池的电动势为1.2伏
	\item 旧干电池的电动势为1.35伏
\end{enumerate}

\textbf{解说}:旧干电池的电动势始终不变,为\textbf{1.5伏}。其他三项错误。\\\noindent\textbf{答案:}A

\bigskip


\noindent\textbf{问题:}测量一个元件是否短路,应该使用:

\begin{enumerate}[label=\Alph*), leftmargin=3em]
	\item 万用电表的电压挡
	\item 万用电表的电流挡
	\item 频谱分析仪
	\item 万用电表的电阻挡
\end{enumerate}

\noindent\textbf{解说:}使用\textbf{万用电表的电阻挡}测量一个元件是否短路。\\\noindent\textbf{答案:}D

\bigskip


\noindent\textbf{问题:}大致判断一个干电池是否已经失效,应该使用:

\begin{enumerate}[label=\Alph*), leftmargin=3em]
	\item 音频信号发生器
	\item 万用电表的电流挡
	\item 万用电表的电压挡
	\item 万用电表的电阻挡
\end{enumerate}

\noindent\textbf{解说:}使用\textbf{万用电表的电压挡}判断一个干电池是否已经失效。\\\noindent\textbf{答案:}C

\bigskip


\noindent\textbf{问题:}通信或家用设备的劣质开关电源会造成对无线电接收机的电磁干扰,其源头主要是:

\begin{enumerate}[label=\Alph*), leftmargin=3em]
	\item 元器件接点不稳造成火花放电干扰
	\item 工频电源变压器漏磁感应
	\item 开关电源中整流电路的滤波电容容量不足
	\item 开关电路的谐波辐射
\end{enumerate}

\noindent\textbf{解说:}通信或家用设备的劣质开关电源会造成对无线电接收机的电磁干扰,其源头主要是\textbf{开关电路的谐波辐射}。\\\noindent\textbf{答案:}D

\bigskip


\noindent\textbf{问题:}家用微波炉一般的工作频带是:

\begin{enumerate}[label=\Alph*), leftmargin=3em]
	\item VHF(甚高频)
	\item UHF(特高频)
	\item SHF(超高频)
	\item EHF(极高频)
\end{enumerate}

\noindent\textbf{解说:}家用微波炉一般的工作频率是2.45Ghz,其工作频带位于\textbf{UHF(特高频)}。\\\noindent\textbf{答案:}B


\bigskip


\noindent\textbf{问题:}下列哪种设备可以用来代替普通的扬声器,可在嘈杂的环境中更好地抄收语音信号?

\begin{enumerate}[label=\Alph*), leftmargin=3em]
	\item 耳机
	\item 低通滤波器
	\item 吊杆话筒
	\item 视频显示器
\end{enumerate}

\noindent\textbf{解说:耳机}可以用来代替普通的扬声器,可在嘈杂的环境中更好地抄收语音信号。低通滤波器用来过滤去除高频信号,视频显示器用来显示图像,吊杆话筒用来拾音,其他三项不符合题意。\\\noindent\textbf{答案:}A

\bigskip


\noindent\textbf{问题:}移动车载电台通常使用的电源电压是?

\begin{enumerate}[label=\Alph*), leftmargin=3em]
	\item 约30伏特
	\item 约240伏特
	\item 约12伏特
	\item 约120伏特
\end{enumerate}

\noindent\textbf{解说:}移动车载电台通常使用的电源电压\textbf{约12伏特}。\\\noindent\textbf{答案:}C

\bigskip


\noindent\textbf{问题:}下列哪一项可以在电路电流严重过载时保护电路不受损坏?

\begin{enumerate}[label=\Alph*), leftmargin=3em]
	\item 熔断器
	\item 电感
	\item 电容
	\item 屏蔽层
\end{enumerate}

\noindent\textbf{解说:熔断器}可以在电路电流严重过载时保护电路不受损坏。\\\noindent\textbf{答案:}A
\bigskip


\noindent\textbf{问题:}一个新的干电池的标称电压是多少?

\begin{enumerate}[label=\Alph*), leftmargin=3em]
	\item 1.2伏
	\item 1.5伏
	\item 1.0伏
	\item 2.2伏
\end{enumerate}

\noindent\textbf{解说:}一个新的干电池的标称电压是\textbf{1.5伏}。一个充电电池充满电的电压是1.2伏。\\\noindent\textbf{答案:}B

\bigskip


\noindent\textbf{问题:}下列哪一种电池不能充电?

\begin{enumerate}[label=\Alph*), leftmargin=3em]
	\item 锂离子电池
	\item 碱性电池
	\item 镍镉电池
	\item 铅酸电池
\end{enumerate}

\noindent\textbf{解说:碱性电池}不能充电。镍镉电池、铅酸电池和锂离子电池可以充电。\\\noindent\textbf{答案:}B
\bigskip


\noindent\textbf{问题:}LED是什么的缩写?

\begin{enumerate}[label=\Alph*), leftmargin=3em]
	\item 阴极射线管
	\item 场效应管
	\item 发光二极管
	\item 液晶显示
\end{enumerate}

\noindent\textbf{解说:}LED是\textbf{发光二极管}的缩写。其余选项为:CRT为阴极射线管的缩写,FET为场效应管的缩写,LCD为液晶显示的缩写。\\\noindent\textbf{答案:}C

\bigskip


\noindent\textbf{问题:}LCD是什么的缩写?

\begin{enumerate}[label=\Alph*), leftmargin=3em]
	\item 液晶显示器
	\item 阴极射线管
	\item 发光二极管
	\item 场效应管
\end{enumerate}

\noindent\textbf{解说:}LCD是\textbf{液晶显示器}的缩写。其余选项为:LED是发光二极管的缩写,CRT为阴极射线管的缩写,FET为场效应管的缩写。\\\noindent\textbf{答案:}A

\bigskip


\noindent\textbf{问题:}下列哪一项设备可以把交变电流变成变化的直流?

\begin{enumerate}[label=\Alph*), leftmargin=3em]
	\item 整流器
	\item 反射器
	\item 放大器
	\item 变压器
\end{enumerate}

\noindent\textbf{解说:整流器}可以把交变电流变成变化的直流,变压器可以改变输出电压,放大器可以放大信号。\\\noindent\textbf{答案:}A


\bigskip


\noindent\textbf{问题:}下列部件中经常被用来把220伏的市电转换成更低电压交流电的是:

\begin{enumerate}[label=\Alph*), leftmargin=3em]
	\item 变压器
	\item 二极管
	\item 可变电容器
	\item 晶体管
\end{enumerate}

\noindent\textbf{解说:变压器}经常被用来把220伏的市电转换成更低电压交流电。其他三项都不能。\\\noindent\textbf{答案:}A

\bigskip


\noindent\textbf{问题:}下列器件中能够当做指示灯使用的是:

\begin{enumerate}[label=\Alph*), leftmargin=3em]
	\item LED
	\item 双极型晶体管
	\item 齐纳二极管
	\item FET
\end{enumerate}

\noindent\textbf{解说:LED}能够当做指示灯使用。其他三项都不能。\\\noindent\textbf{答案:}A

\bigskip


\noindent\textbf{问题:}如何把一个电压表正确地连接至电路?

\begin{enumerate}[label=\Alph*), leftmargin=3em]
	\item 与电路同相连接
	\item 并联至电路中
	\item 串联至电路中
	\item 正交至电路中
\end{enumerate}

\noindent\textbf{解说:}电压表要\textbf{并联至电路中}。\\\noindent\textbf{答案:}B

\bigskip


\noindent\textbf{问题:}电流表通常要怎样连接至电路?

\begin{enumerate}[label=\Alph*), leftmargin=3em]
	\item 并联至电路中
	\item 与电路同相连接
	\item 正交至电路中
	\item 串联至电路中
\end{enumerate}

\noindent\textbf{解说:}电流表要\textbf{串联至电路中}。\\\noindent\textbf{答案:}D


\bigskip


\noindent\textbf{问题:}哪一种设备被用来测量电阻?

\begin{enumerate}[label=\Alph*), leftmargin=3em]
	\item 天线分析仪
	\item 欧姆表
	\item 示波器
	\item 频谱分析仪
\end{enumerate}

\noindent\textbf{解说:欧姆表}用来测量电阻。\\\noindent\textbf{答案:}B


\bigskip


\noindent\textbf{问题:}下列哪一种行为有可能损坏万用表?

\begin{enumerate}[label=\Alph*), leftmargin=3em]
	\item 没有让待测量设备适当地预热
	\item 把万用表设在“毫安”挡放置了一夜
	\item 在大电压量程下测量一个非常小的电压
	\item 在电阻挡试图测量电压
\end{enumerate}

\noindent\textbf{解说:在电阻挡试图测量电压}有可能损坏万用表,因为当万用表设在电阻档时绝对不能通电。其他三个选项都不会损坏万用表。\\\noindent\textbf{答案:}D


\bigskip


\noindent\textbf{问题:}我们经常使用万用表测量以下哪些物理量?

\begin{enumerate}[label=\Alph*), leftmargin=3em]
	\item 电压和电阻
	\item 驻波比和射频功率
	\item 信号的强度和噪音
	\item 阻抗和电抗
\end{enumerate}

\noindent\textbf{解说:}我们经常使用万用表测量\textbf{电压和电阻}。\\\noindent\textbf{答案:}A

\bigskip


\noindent\textbf{问题:}下列哪一种方法可以用来定位无线电噪音源或者恶意干扰源?

\begin{enumerate}[label=\Alph*), leftmargin=3em]
	\item 回波定位
	\item 无线电测向
	\item 相位锁定
	\item 多普勒雷达
\end{enumerate}

\noindent\textbf{解说:}\textbf{无线电测向}可以用来定位无线电噪音源或者恶意干扰源。相位锁定是控制频率或相位的系统。多普勒雷达是利用多普勒效应来测量物体运动速度的雷达。回波定位是某些动物向周围环境发出声波,通过反射波得知物体的大小、方向、距离等。\\\noindent\textbf{答案:}B%%%???回声定位
