\chapter{无线电安全防护}

\textbf{问题:}防雷装置的作用是防止雷电危害。传统防雷装置的主要组成部分是:

\begin{enumerate}[label=\Alph*), leftmargin=3em]
	\item 接闪器(避雷针)、引下线、接地体
	\item 天线、限流器、地线
	\item 避雷针、高压指示灯、过流保护器
	\item 避雷针、过压保护器、熔丝
\end{enumerate}

\textbf{解说:}传统防雷装置主要由接闪器(避雷针)、引下线、接地体等部分组成。%?
\textbf{答案:}A

\textbf{问题:}防雷接地的作用是:

\begin{enumerate}[label=\Alph*), leftmargin=3em]
	\item 把接闪器引入的雷击电流有效地泄入大地
	\item 用接闪器感应到的雷电高压启动过压保护电路
	\item 有效地阻断接闪器引入的雷击电流使其不致流入大地
	\item 当接闪器引入雷击电流时迅速烧断熔丝,阻断其流动
\end{enumerate}

\textbf{解说:}防雷接地的作用是把接闪器引入的雷击电流有效地泄入大地。%?

\textbf{答案:}A

\textbf{问题:}关于对防雷接地基本要求的正确说法:

\begin{enumerate}[label=\Alph*), leftmargin=3em]
	\item 要有单独的接地体,接地电阻要小,接闪器到接地体之间的引下线应尽量短而粗
	\item 可以利用与埋地金属管线相连的自来水管作为接地体
	\item 交流电网的“零线”在配电系统中已经接地,因此可代替防雷接地体及其引下线
	\item 接闪器到接地体之间的引下线平时没有电流流过,采用直径0.5毫米的导线为好
\end{enumerate}

\textbf{解说:}防雷接地的基本要求是:要有单独的接地体,接地电阻要小,接闪器到接地体之间的引下线应尽量短而粗。%?

\textbf{答案:}A

\textbf{问题:}单支避雷针的保护范围大致有多大:

\begin{enumerate}[label=\Alph*), leftmargin=3em]
	\item 以避雷针为顶点的45度圆锥体以内空间
	\item 避雷针周围水平方圆30米内的任何物体
	\item 避雷针周围所有比避雷针低的空间
	\item 以避雷针为顶点、避雷针高度为半径的半球体以内空间
\end{enumerate}

\textbf{解说:}单支避雷针的保护范围大致为以避雷针为顶点的45度圆锥体以内空间。%?

\textbf{答案:}A

\textbf{问题:}安全电压是指不致使人直接致死或致残的电压。根据国家标准GB3805-83《安全电压》,一般环境条件下允许持续接触的“安全特低电压”为:

\begin{enumerate}[label=\Alph*), leftmargin=3em]
	\item 24V
	\item 48V
	\item 72V
	\item 6V
\end{enumerate}

\textbf{解说:}根据国家标准GB3805-83《安全电压》,一般环境条件下允许持续接触的“安全特低电压”上限为24V。%?

\textbf{答案:}A

\textbf{问题:}触及裸露的射频导线时,与触及相同电压的直流或50Hz交流导线相比,对人身安全危险的大致差别为:
\begin{enumerate}[label=\Alph*), leftmargin=3em]
	\item 致死危险性下降,但皮肤容易灼伤
	\item 更容易出现呼吸或心脏麻痹、更易致死
	\item 射频电流对人体没有危险
	\item 触及不同频率、相同电压的导线时,对人体的危害没有差别
\end{enumerate}
\textbf{解说:}触及裸露的射频导线时,与触及相同电压的直流或50Hz交流导线相比,对人身安全危险的大致差别是致死危险性下降,但皮肤容易灼伤。\\
\textbf{答案:}A

\textbf{问题:}如遇设备、电线或者电源引起失火,正确的处置为:
\begin{enumerate}[label=\Alph*), leftmargin=3em]
	\item 立即切断电源,使用二氧化碳灭火器灭火
	\item 立即切断电源,使用化学泡沫灭火器灭火
	\item 立即切断电源,用水灭火
	\item 迅速起身逃离火场
\end{enumerate}
\textbf{解说:}如遇设备、电线或者电源引起失火,正确的处置方法是立即切断电源,使用二氧化碳灭火器灭火。\\
\textbf{答案:}A

\textbf{问题:}必须带电检修由市电供电的无线电设备时,应做到:
\begin{enumerate}[label=\Alph*), leftmargin=3em]
	\item 双脚与地绝缘,单手操作,另一只手不触摸机壳等任何与电路设备有关的金属物品
	\item 双脚与地绝缘,单手操作,另一只手通过触摸机壳或水管等途径良好接地
	\item 只要设备外壳良好接地,双脚是否与地绝缘、双手是否同时操作都没有关系
	\item 只要确保设备外壳与地绝缘,双脚是否与地绝缘、双手是否同时操作都没有关系
\end{enumerate}
\textbf{解说:}必须带电检修由市电供电的无线电设备时,应做到:双脚与地绝缘,单手操作,另一只手不触摸机壳等任何与电路设备有关的金属物品。\\
\textbf{答案:}A

\textbf{问题:}两手分别接触电压有效值相同但频率不同的电路两端时,对人体生命安全威胁由大到小的排序为:
\begin{enumerate}[label=\Alph*), leftmargin=3em]
	\item 工频交流电、HF射频交流电、UHF射频交流电
	\item UHF射频交流电、HF射频交流电、工频交流电
	\item HF射频交流电、工频交流电、UHF射频交流电
	\item HF射频交流电、UHF射频交流电、工频交流电
\end{enumerate}
\textbf{解说:}当两手分别接触电压有效值相同但频率不同的电路两端时,对人体生命安全威胁由大到小的排序为:工频交流电、HF射频交流电、UHF射频交流电。\\
\textbf{答案:}A

\textbf{问题:}电路中的保险丝起到什么作用?
\begin{enumerate}[label=\Alph*), leftmargin=3em]
	\item 过载时切断电路
	\item 防止电源纹波损害电路
	\item 限制电流,防止触电
	\item 以上三项全部正确
\end{enumerate}
\textbf{解说:}电路中的保险丝起到过载时切断电路的作用。\\
\textbf{答案:}A

\textbf{问题:}为什么在需要安装5安培保险丝的地方安装一个20安培的保险丝是不可取的?
\begin{enumerate}[label=\Alph*), leftmargin=3em]
	\item 过大的电流可能导致火灾
	\item 电源的纹波会显著增大
	\item 因为20安培的保险丝电流更大,所以它更容易熔断
	\item 其他三项全部正确
\end{enumerate}
\textbf{解说:}电路中的保险丝起到过载时切断电路的作用。把较小熔断电流的保险丝擅自换成大熔断电流的保险丝是错误的,这样做有可能造成在过大的电流时不能切断电路,甚至可能导致火灾。\\
\textbf{答案:}A

\textbf{问题:}防止设备外壳带电危险的措施包括:
\begin{enumerate}[label=\Alph*), leftmargin=3em]
	\item 其他三项全部正确
	\item 将所有的交流供电设备全部连接至一个安全地线
	\item 安装漏电保护断路器
	\item 所有使用交流供电的设备的电源线都使用带有单独保护地线端的三线插头
\end{enumerate}
\textbf{解说:}防止设备外壳带电危险的措施包括:将所有的交流供电设备全部连接至一个安全地线、安装漏电保护断路器、所有使用交流供电的设备的电源线都使用带有单独保护地线端的三线插头等措施。因此其他三项正确。\\
\textbf{答案:}A

\textbf{问题:}在为同轴电缆馈线安装避雷器时应当注意什么?
\begin{enumerate}[label=\Alph*), leftmargin=3em]
	\item 将所有避雷器的地线接到同一个金属板上,然后将这个金属板接到室外的接地极
	\item 在每一个避雷器的接地线上安装开关,以防止射频过载损伤避雷器
	\item 将每一个避雷器单独引出接地线,并且将它们都连接至电台的地线
	\item 要在每一个避雷器处两端并联一个开关,以便在使用大功率输出的时候可以将避雷器旁路掉
\end{enumerate}
\textbf{解说:}在为同轴电缆馈线安装避雷器时应当注意要将所有避雷器的地线接到同一个金属板上,然后将这个金属板接到室外的接地极。\\
\textbf{答案:}A

\textbf{问题:}常规的12伏酸铅蓄电池通常有什么潜在的危险?
\begin{enumerate}[label=\Alph*), leftmargin=3em]
	\item 如果通风不良,有爆炸风险的气体会聚集
	\item 有高电压,存在触电的风险
	\item 它会释放臭氧,进而污染大气层
	\item 长时间不使用可能会引起自燃
\end{enumerate}
\textbf{解说:}如果通风不良,有爆炸风险的气体会聚集是常规的12伏酸铅蓄电池潜在的危险。\\
\textbf{答案:}A

\textbf{问题:}设备电源拔掉电源线以后,检修时还有什么安全风险?
\begin{enumerate}[label=\Alph*), leftmargin=3em]
	\item 充满高电压的电容器可能造成电击
	\item 地磁场可能在变压器中激起感应电流导致电源损坏
	\item 打开电源外壳可能引起保险丝烧断
	\item 静电可能损坏接地系统
\end{enumerate}
\textbf{解说:}检修时应注意,设备电源拔掉电源线以后,充满高电压的电容器可能造成电击。\\
\textbf{答案:}A

\textbf{问题:}自制一台由220伏交流供电的设备,推荐采用的安全措施是:
\begin{enumerate}[label=\Alph*), leftmargin=3em]
	\item 交流电源入口火线端串联安装保险丝
	\item 在交流供电入口处并联安装一个交流电压表
	\item 在交流供电入口处串联安装一个电感
	\item 在交流供电入口处并联安装一个电容
\end{enumerate}
\textbf{解说:}在自制由220伏交流供电的设备时,推荐采用的安全措施是在交流电源入口火线端串联安装保险丝。\\
\textbf{答案:}A