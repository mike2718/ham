\chapter{避免和解决射频干扰}


\textbf{问题:}假设中继台的收、发信机共用天线,上下行频率分别为F1和F2。要防止中继台发射机对接收机产生干扰,应该对中继台设备采取下列措施:
\begin{enumerate}[label=\Alph*), leftmargin=1.5cm]
	\item 在发信机与天线间串联中心频率为F1的带阻滤波器,在收信机与天线间串接中心频率为F2的带阻滤波器
	\item 在发信机与天线间串联中心频率为F2的带阻滤波器,在收信机与天线间串接中心频率为F2的带阻滤波器
	\item 在发信机与天线间串联中心频率为F1的带通滤波器,在收信机与天线间串接中心频率为F2的带通滤波器
\end{enumerate}
\textbf{解说:}当中继台的收、发信机共用天线,上下行频率分别为F1和F2,要防止中继台发射机对接收机产生干扰,应该\textbf{在中继台设备的发信机与天线间串联中心频率为F1的带阻滤波器,在收信机与天线间串接中心频率为F2的带阻滤波器}。\\
\textbf{答案:}A


\textbf{问题:}要防止业余HF发射机的杂散发射干扰天线附近的电话机,应该在电话机和电话线之间之间串联:
\begin{enumerate}[label=\Alph*), leftmargin=1.5cm]	
	\item 截止频率约为3kHz的带阻滤波器
	\item 截止频率约为3kHz的高通滤波器
	\item 截止频率不高于1MHz的低通滤波器
	\item 中心频率约为3kHz的带通滤波器
\end{enumerate}
\textbf{解说:}要防止业余HF发射机的杂散发射干扰天线附近的电话机,应该在电话机和电话线之间之间串联\textbf{截止频率不高于1MHz的低通滤波器}。\\
\textbf{答案:}C


\textbf{问题:}我国业余电台应该遵守的关于电磁辐射污染的具体管理规定文件为:
\begin{enumerate}[label=\Alph*), leftmargin=1.5cm]
	\item 美国FCC《射频电磁场人员暴露准则的测评方法》
	\item 国际非电离辐射防护委员会《限制时变电场和磁场暴露的导则》
	\item 国家标准《电磁辐射防护规定》
	\item 《业余电台管理办法》
\end{enumerate}
\textbf{解说:}我国业余电台应该遵守的关于电磁辐射污染的具体管理规定文件是\textbf{国家标准《电磁辐射防护规定》}。\\
\textbf{答案:}C


\textbf{问题:}按照我国国家标准《电磁辐射防护规定》,可以免于管理的电磁辐射体为 :
\begin{enumerate}[label=\Alph*), leftmargin=1.5cm]
	\item 所有业余电台
	\item 发射频率30MHz以下的所有业余电台
	\item 输出功率不大于15W的移动式无线电通信设备,以及等效辐射功率在0.1-3MHz不大于300W、在3MHz-300GHz不大于100瓦的辐射体
	\item 发射频率30MHz以上的所有业余电台
\end{enumerate}
\textbf{解说:}按照我国国家标准《电磁辐射防护规定》,可以免于管理的电磁辐射体(豁免水平)为\textbf{输出功率不大于15W的移动式无线电通信设备,以及等效辐射功率在0.1-3MHz不大于300W、在3MHz-300GHz不大于100瓦的辐射体}。\\
\textbf{答案:}C



\textbf{问题:}按照我国国家标准《电磁辐射防护规定》,凡其功率大于豁免水平(3MHz以上等效辐射功率100瓦)的一切电磁波辐射体的所有者,必须:
\begin{enumerate}[label=\Alph*), leftmargin=1.5cm]
	\item 向所在地区的环境保护部门申报、登记,并接受监督
	\item 向所在地区的环境保护部门缴纳环境保护费
	\item 向所在地区的无线电管理机构交纳环境电磁辐射监测报告
	\item 业余电台可不受环境保护部门的监督管理
\end{enumerate}
\textbf{解说:}按照我国国家标准《电磁辐射防护规定》,凡其功率大于豁免水平(3MHz以上等效辐射功率100瓦)的一切电磁波辐射体的所有者,必须\textbf{向所在地区的环境保护部门申报、登记,并接受监督}。\\
\textbf{答案:}A

\textbf{问题:}按照我国国家标准《电磁辐射防护规定》,负责对超过豁免水平(3MHz以上等效辐射功率100瓦)电磁波辐射体所在的场所以及周围环境的电磁辐射水平进行监测的是:
\begin{enumerate}[label=\Alph*), leftmargin=1.5cm]
	\item 所在地区业余无线电协会
	\item 所在地区无线电管理机构
	\item 其拥有者
	\item 所在地区环境保护部门
\end{enumerate}
\textbf{解说:}按照我国国家标准《电磁辐射防护规定》,\textbf{其拥有者}负责对超过豁免水平(3MHz以上等效辐射功率100瓦)电磁波辐射体所在的场所以及周围环境的电磁辐射水平进行监测。\\
\textbf{答案:}C


\textbf{问题:}按照我国国家标准《电磁辐射防护规定》,当监测到超过豁免水平的电磁辐射体使环境电磁辐射水平超过规定的限值时,必须:
\begin{enumerate}[label=\Alph*), leftmargin=1.5cm]
	\item 尽快采取措施降低辐射水平,同时向环境保护部门报告
	\item 立即到环境保护部门缴纳电磁污染治理费
	\item 立即向无线电管理机构报告
	\item 立即向当地城管机构报告
\end{enumerate}
\textbf{解说:}按照我国国家标准《电磁辐射防护规定》,当监测到超过豁免水平的电磁辐射体使环境电磁辐射水平超过规定的限值时,必须\textbf{尽快采取措施降低辐射水平,同时向环境保护部门报告}。\\
\textbf{答案:}A


\textbf{问题:}按照我国国家标准《电磁辐射防护规定》,对超过豁免水平的电磁辐射体的环境电磁辐射水平监测应在下列地点进行:
\begin{enumerate}[label=\Alph*), leftmargin=1.5cm]
	\item 在距辐射体天线2000米以内最大辐射方向上选点测量
	\item 在距辐射体天线2000米以内选择任意位置测量
	\item 在距辐射体天线100米以内最小辐射方向上选点测量
	\item 在发射机射频输出端口进行测量
\end{enumerate}
\textbf{解说:}对超过豁免水平的电磁辐射体的环境电磁辐射水平监测应\textbf{在距辐射体天线2000米以内最大辐射方向上选点测量}。\\
\textbf{答案:}A


\textbf{问题:}我国国家标准《电磁辐射防护规定》所规定的电磁辐射防护限值的公众照射基本限值,其基本计量方法是:
\begin{enumerate}[label=\Alph*), leftmargin=1.5cm]
	\item 任何时刻的瞬时辐射场强都应小于与频率有关的限值
	\item 任何时刻的瞬时辐射场强都应小于与频率无关的固定限值
	\item 一天24小时内任意6分钟内全身平均的比吸收率(SAR)应小于每公斤体重限值
	\item 任何时刻的全身平均的比吸收率(SAR)都应小于每公斤体重限值
\end{enumerate}
\textbf{解说:}我国国家标准《电磁辐射防护规定》所规定的电磁辐射防护限值的公众照射基本限值,其基本计量方法是:\textbf{一天24小时内任意6分钟内全身平均的比吸收率(SAR)应小于每公斤体重限值}。\\
\textbf{答案:}C



\textbf{问题:}我国国家标准《电磁辐射防护规定》规定电磁辐射公众照射导出限值中,对环境电磁辐射场强一天24小时内任意6分钟内的平均值要求最严格的频率范围为:
\begin{enumerate}[label=\Alph*), leftmargin=1.5cm]
	\item 30MHz- 3GHz
	\item 3MHz- 30MHz
	\item 1200MHz-30GHz
	\item 300kHz-30MHz
\end{enumerate}
\textbf{解说:}我国国家标准《电磁辐射防护规定》规定电磁辐射公众照射导出限值中,对环境电磁辐射场强一天24小时内任意6分钟内的平均值要求最严格的频率范围为\textbf{30MHz- 3GHz}。\\
\textbf{答案:}A


\textbf{问题:}移动车载台的直流电源负极应当接在哪里?
\begin{enumerate}[label=\Alph*), leftmargin=1.5cm]
	\item 连接在固定住电台的挂置架上
	\item 连接在天线座上
	\item 连接在电池的负极或发动机的接地带
	\item 可以连接在汽车的任意的金属部分
\end{enumerate}
\textbf{解说:}移动车载台的直流电源负极应当\textbf{连接在电池的负极或发动机的接地带}。\\
\textbf{答案:}C


\textbf{问题:}为什么电磁辐射防护规定国家标准中的照射限值随着频率的变化而不同?
\begin{enumerate}[label=\Alph*), leftmargin=1.5cm]
	\item 较低频率的无线电波不会穿透人体
	\item 人体会对某些特定频率的电磁波吸收量更大
	\item 较低频率的无线电波相对高频率的无线电波拥有更高的能量
	\item 在自然中高频电磁波不常见
\end{enumerate}
\textbf{解说:}因为\textbf{人体会对某些特定频率的电磁波吸收量更大}。\\
\textbf{答案:}B
