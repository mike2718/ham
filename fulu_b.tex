\chapter{附录}

\section{字母解释法(节录)}

\begin{tabular}{cc}%
\begin{tabular}[t]{|c|l|}
	\hline
	\textbf{字母} & \textbf{单词} \\
	\hline
	A & Alfa \\
	\hline
	B & Bravo \\
	\hline
	C & Charlie \\
	\hline
	D & Delta \\
	\hline
	E & Echo \\
	\hline
	F & Foxtrot \\
	\hline
	G & Golf \\
	\hline
	H & Hotel \\
	\hline
	I & India \\
	\hline
	J & Juliett \\
	\hline
	K & Kilo \\
	\hline
	L & Lima \\
	\hline
	M & Mike \\
	\hline
	N & November \\
	\hline
	O & Oscar \\
	\hline
	P & Papa \\
	\hline
	Q & Quebec \\
	\hline
	R & Romeo \\
	\hline
	S & Sierra \\
	\hline
	T & Tango \\
	\hline
	U & Uniform \\
	\hline
	V & Victor \\
	\hline
	W & Whiskey \\
	\hline
	X & X-ray \\
	\hline
	Y & Yankee \\
	\hline
	Z & Zulu \\
	\hline
\end{tabular} &
\begin{tabular}[t]{|c|l|}
	\hline
	\textbf{字母} & \textbf{单词} \\
	\hline
	0 & Zero \\
	\hline
	1 & One \\
	\hline
	2 & Two \\
	\hline
	3 & Three \\
	\hline
	4 & Four \\
	\hline
	5 & Five \\
	\hline
	6 & Six \\
	\hline
	7 & Seven \\
	\hline
	8 & Eight \\
	\hline
	9 & Nine \\
	\hline
\end{tabular} \tabularnewline
\end{tabular}

\newpage

\section{业余通信中常用的Q简语(节录)}

\begin{tabular}{|l|l|l|}
	\hline
	    & \textbf{提问} & \textbf{回答或建议} \\
	\hline
	QRL & 你正忙着吗 & 我正忙着 \\
	\hline
	QRM & 你遇到他台干扰吗 & 我遇到他台干扰 \\
	\hline
	QRN & 你遇到天电干扰吗 & 我遇到天电干扰 \\
	\hline
	QRO & 要我增加功率吗 & \\
	\hline
	QRP & 要我减小功率吗 & \\
	\hline
	QRQ & 要我加快发送速度吗 & 请加快发送速度 \\
	\hline
	QRS & 要我减慢发送速度吗 & 请减慢发送速度 \\
	\hline
	QRT & 要我停止发送吗 & 请停止发送 \\
	\hline
	QRU & 你和我还有事吗 & 我和你无事了 \\
	\hline
	QRV & 你是否已准备好 & 我已准备好 \\
	\hline
	QRZ & 谁在呼叫我 & \\
	\hline
	QSA & 我的信号强度如何 & 你的信号强度为 × 级(1-5 级) \\
	\hline
	QSB & 我的信号有衰落吗 & 你的信号有衰落 \\
	\hline
	QSD & 我发报的手法有毛病吗 & 你发报的手法有毛病 \\
	\hline
	\multirow{2}{1em}{QSK} & 能在你的信号间隙中接收吗 & 我在发射的信号间隙中接收 \\
	& (即 QSK 插入方式)    & (即 QSK 插入方式) \\
	\hline
	QSL & 你能给我收据(或 QSL 卡片)吗 & 我给你收据(QSL 卡片)、我已收妥 \\
	\hline
	QSO & 你能直接和 ××× 电台通信吗 & 我能直接和 ××× 电台通信 \\
	\hline
	QSP & 你能传信到 ××× 电台吗 & 我能传信到 ××× 电台 \\
	\hline
	\multirow{2}{1em}{QSX} & 你将在 nnnn KHz(或 MHz)频率 & 我将在 nnnn KHz(或 MHz)频率 \\
	    & 守听 ××× 电台吗 & 守听 ××× 电台 \\
	\hline
	QSY & 要我将频率改到 nnnn 频率吗 & 请将频率改到 nnnn 频率 \\
	\hline
	QTH & 你的电台位置在哪里 & 我的电台位置是 ×××× \\
	\hline
\end{tabular}

\newpage



\section{业余通信常用缩语表(节录)}

\newpage

\section{发射类别的表示方法(节录)}

\begin{tabular}{|c|c|}
	\hline
	\multicolumn{2}{|c|}{\textbf{主载波类型(第一个符号)}} \\
	\hline
	\textbf{符号} & \textbf{描述} \\
	\hline
	A & 双边带 \\
	\hline
	F & 调频 \\
	\hline
	G & 调相 \\
	\hline
	J & 单边带、抑制载波 \\
	\hline
\end{tabular}

\bigskip

\begin{tabular}{|c|c|}
	\hline
	\multicolumn{2}{|c|}{\textbf{调制主载波的信号的性质(第二个符号)}} \\
	\hline
	\textbf{符号} & \textbf{描述} \\
	\hline
	1 & 不用调制副载波但包含量化或数字信息的单个通路 \\
	\hline
	2 & 利用调制副载波且包含量化或数字信息的单个通路 \\
	\hline
	3 & 包含模拟信息的单个通路 \\
	\hline
\end{tabular}

\bigskip

\begin{tabular}{|c|c|}
	\hline
	\multicolumn{2}{|c|}{\textbf{被发送信息类型(第三个符号)}} \\
	\hline
	\textbf{符号} & \textbf{描述} \\
	\hline
	A & 电报──用于人工收听 \\
	\hline
	B & 电报──用于自动接收 \\
	\hline
	E & 电话(包括声音广播) \\
	\hline
	F & 电视(视频) \\
	\hline
\end{tabular}

\newpage

\section{无线电频段和波段表(节录)}

\begin{tabular}{|c|c|c|c|c|}
	\hline
	\textbf{频段名称} & \textbf{缩写} & \textbf{频率范围} & \textbf{波长范围} & \textbf{波段名称} \\
	%\hline
	%反正不考
	%极低频 & ELF & 3–30 Hz & 100,000–10,000 km & 极长波 \\
	%\hline
	%超低频 & SLF & 30–300 Hz & 10,000–1,000 km & 超长波 \\
	%\hline
	%特低频 & ULF & 300–3,000 Hz & 1,000–100 km & 特长波 \\
	%\hline
	%甚低频 & VLF & 3–30 kHz & 100–10 km & 甚长波 \\
	\hline
	低频 & LF & 30–300 kHz & 10–1 km & 长波 \\
	\hline
	中频 & MF & 300–3,000 kHz & 1,000–100 m & 中波 \\
	\hline
	高频 & HF & 3–30 MHz & 100–10 m & 短波 \\
	\hline
	甚高频 & VHF & 30–300 MHz & 10–1 m & 米波 \\
	\hline
	特高频 & UHF & 300–3,000 MHz & 1–0.1 m & 分米波 \\
	\hline
	超高频 & SHF & 3–30 GHz & 100–10 mm & 厘米波 \\
	\hline
	极高频 & EHF & 30–300 GHz & 10–1 mm & 毫米波 \\
	\hline
	%反正不考
	%至高频 & THF & 300–3,000 GHz & 1–0.1 mm & 丝米波 \\
	%\hline
\end{tabular}

\newpage






\section{业余无线电频率划分表(节录)}


\begin{tabular}{|c|c|c|}
	\hline
	\textbf{波段名称} & \textbf{频率范围} & \textbf{频率范围} \\
	\hline
	160米 & 1800–2000 kHz & 1.8-2 MHz \\
	\hline
	80米 & 3.5–4.0 MHz & 3500–4000 kHz \\
	\hline
	40米 & 7.0–7.3 MHz &  \\
	\hline
	20米 & 14.0–14.35 MHz & \\
	\hline
	15米 & 21–21.45 MHz & \\
	\hline
	10米 & 28–29.7 MHz & \\
	\hline
	6米 & 50–54 MHz & \\
	\hline
	2米 & 144–148 MHz & \\
	\hline
	0.7米 & 420–450 MHz & \\
	\hline
\end{tabular}

\newpage

\section{计算公式}

直流电路欧姆定律:
$$\mbox{电流}I(A)=\frac{\mbox{电压}U(V)}{\mbox{电阻值}R(\Omega)}$$

峰值电压的计算公式:
$$\mbox{峰值}(V)=\frac{\mbox{峰--峰值}}{2}$$

正弦交流电压的有效值计算公式:
$$\mbox{交流的有效值}(V)=\frac{\mbox{峰值}}{\sqrt{2}}=\mbox{峰值}\times0.707$$

功率增益的计算公式:
$$\mbox{功率增益}=10 \log_{10} \left( {\frac{P_{ \mbox{输出} }}{P_{ \mbox{输入} }}}\right)\ \mathrm{dB}$$

电压增益的计算公式:
$$\mbox{电压增益}=20 \log \left( {\frac{V_{ \mbox{输出} }}{V_{ \mbox{输入} }}} \right)\ \mathrm{dB}$$

%电流增益的计算公式:
%$$\mbox{电流增益}=20 \log \left( {\frac{I_{ \mbox{输出} }}{I_{ \mbox{输入} }}} \right)\ \mathrm{dB}$$

绝对增益与相对增益的关系:
$$\mbox{绝对增益} = \mbox{相对增益} + 2.15 \ \mathrm{dB}$$

\newpage


\section{以10为底的简易对数表}

说明:强烈建议背诵$\log_{10} 1.0$、$\log 2.0$、……$\log 10.0$等第0列的10个数值到小数点后三位,考试闭卷,无法使用计算器,如果不背诵,将无法计算。

\begin{longtable}[c]{|c|c|}
\hline
\textbf{N} & \textbf{0} \\
\hline
\endfirsthead
\hline
\textbf{N} & \textbf{0} \\
\endhead
\textbf{1.0} & .0000 \\ \hline
\textbf{2.0} & .3010 \\ \hline
\textbf{3.0} & .4771 \\ \hline
\textbf{4.0} & .6021 \\ \hline
\textbf{5.0} & .6990 \\ \hline
\textbf{6.0} & .7782 \\ \hline
\textbf{7.0} & .8451 \\ \hline
\textbf{8.0} & .9031 \\ \hline
\textbf{9.0} & .9542 \\ \hline
\textbf{10.0} & 1.0000 \\ \hline
\end{longtable}


%\begin{longtable}[c]{|c|c|c|c|c|c|c|c|c|c|c|}
%\hline
%$\log_{10}$ & \textbf{0} & \textbf{1} & \textbf{2} & \textbf{3} & \textbf{4}
% & \textbf{5} & \textbf{6} & \textbf{7} & \textbf{8} & \textbf{9} \\
%\hline
%\endfirsthead
%\hline
%\textbf{$\log$} & \textbf{0} & \textbf{1} & \textbf{2} & \textbf{3} & \textbf{4} & \textbf{5} & \textbf{6} & \textbf{7} & \textbf{8} & \textbf{9} \\
%\endhead
%1.0 & \textbf{.0000} & .0043 & .0086 & .0128 & .0170 & .0212 & .0253 & .0294 & .0334 & .0374 \\
%\hline
%1.1 & .0414 & .0453 & .0492 & .0531 & .0569 & .0607 & .0645 & .0682 & .0719 & .0755 \\
%\hline
%1.2 & .0792 & .0828 & .0864 & .0899 & .0934 & .0969 & .1004 & .1038 & .1072 & .1106 \\
%\hline
%1.3 & .1139 & .1173 & .1206 & .1239 & .1271 & .1303 & .1335 & .1367 & .1399 & .1430 \\
%\hline
%1.4 & .1461 & .1492 & .1523 & .1553 & .1584 & .1614 & .1644 & .1673 & .1703 & .1732 \\
%\hline
%1.5 & .1761 & .1790 & .1818 & .1847 & .1875 & .1903 & .1931 & .1959 & .1987 & .2014 \\
%\hline
%1.6 & .2041 & .2068 & .2095 & .2122 & .2148 & .2175 & .2201 & .2227 & .2253 & .2279 \\
%\hline
%1.7 & .2304 & .2330 & .2355 & .2380 & .2405 & .2430 & .2455 & .2480 & .2504 & .2529 \\
%\hline
%1.8 & .2553 & .2577 & .2601 & .2625 & .2648 & .2672 & .2695 & .2718 & .2742 & .2765 \\
%\hline
%1.9 & .2788 & .2810 & .2833 & .2856 & .2878 & .2900 & .2923 & .2945 & .2967 & .2989 \\
%\hline
%2.0 & \textbf{.3010} & .3032 & .3054 & .3075 & .3096 & .3118 & .3139 & .3160 & .3181 & .3201 \\
%\hline
%2.1 & .3222 & .3243 & .3263 & .3284 & .3304 & .3324 & .3345 & .3365 & .3385 & .3404 \\
%\hline
%2.2 & .3424 & .3444 & .3464 & .3483 & .3502 & .3522 & .3541 & .3560 & .3579 & .3598 \\
%\hline
%2.3 & .3617 & .3636 & .3655 & .3674 & .3692 & .3711 & .3729 & .3747 & .3766 & .3784 \\
%\hline
%2.4 & .3802 & .3820 & .3838 & .3856 & .3874 & .3892 & .3909 & .3927 & .3945 & .3962 \\
%\hline
%2.5 & .3979 & .3997 & .4014 & .4031 & .4048 & .4065 & .4082 & .4099 & .4116 & .4133 \\
%\hline
%2.6 & .4150 & .4166 & .4183 & .4200 & .4216 & .4232 & .4249 & .4265 & .4281 & .4298 \\
%\hline
%2.7 & .4314 & .4330 & .4346 & .4362 & .4378 & .4393 & .4409 & .4425 & .4440 & .4456 \\
%\hline
%2.8 & .4472 & .4487 & .4502 & .4518 & .4533 & .4548 & .4564 & .4579 & .4594 & .4609 \\
%\hline
%2.9 & .4624 & .4639 & .4654 & .4669 & .4683 & .4698 & .4713 & .4728 & .4742 & .4757 \\
%\hline
%3.0 & .4771 & .4786 & .4800 & .4814 & .4829 & .4843 & .4857 & .4871 & .4886 & .4900 \\
%\hline
%3.1 & .4914 & .4928 & .4942 & .4955 & .4969 & .4983 & .4997 & .5011 & .5024 & .5038 \\
%\hline
%3.2 & .5051 & .5065 & .5079 & .5092 & .5105 & .5119 & .5132 & .5145 & .5159 & .5172 \\
%\hline
%3.3 & .5185 & .5198 & .5211 & .5224 & .5237 & .5250 & .5263 & .5276 & .5289 & .5302 \\
%\hline
%3.4 & .5315 & .5328 & .5340 & .5353 & .5366 & .5378 & .5391 & .5403 & .5416 & .5428 \\
%\hline
%3.5 & .5441 & .5453 & .5465 & .5478 & .5490 & .5502 & .5514 & .5527 & .5539 & .5551 \\
%\hline
%3.6 & .5563 & .5575 & .5587 & .5599 & .5611 & .5623 & .5635 & .5647 & .5658 & .5670 \\
%\hline
%3.7 & .5682 & .5694 & .5705 & .5717 & .5729 & .5740 & .5752 & .5763 & .5775 & .5786 \\
%\hline
%3.8 & .5798 & .5809 & .5821 & .5832 & .5843 & .5855 & .5866 & .5877 & .5888 & .5899 \\
%\hline
%3.9 & .5911 & .5922 & .5933 & .5944 & .5955 & .5966 & .5977 & .5988 & .5999 & .6010 \\
%\hline
%4.0 & .6021 & .6031 & .6042 & .6053 & .6064 & .6075 & .6085 & .6096 & .6107 & .6117 \\
%\hline
%4.1 & .6128 & .6138 & .6149 & .6160 & .6170 & .6180 & .6191 & .6201 & .6212 & .6222 \\
%\hline
%4.2 & .6232 & .6243 & .6253 & .6263 & .6274 & .6284 & .6294 & .6304 & .6314 & .6325 \\
%\hline
%4.3 & .6335 & .6345 & .6355 & .6365 & .6375 & .6385 & .6395 & .6405 & .6415 & .6425 \\
%\hline
%4.4 & .6435 & .6444 & .6454 & .6464 & .6474 & .6484 & .6493 & .6503 & .6513 & .6522 \\
%\hline
%4.5 & .6532 & .6542 & .6551 & .6561 & .6571 & .6580 & .6590 & .6599 & .6609 & .6618 \\
%\hline
%4.6 & .6628 & .6637 & .6646 & .6656 & .6665 & .6675 & .6684 & .6693 & .6702 & .6712 \\
%\hline
%4.7 & .6721 & .6730 & .6739 & .6749 & .6758 & .6767 & .6776 & .6785 & .6794 & .6803 \\
%\hline
%4.8 & .6812 & .6821 & .6830 & .6839 & .6848 & .6857 & .6866 & .6875 & .6884 & .6893 \\
%\hline
%4.9 & .6902 & .6911 & .6920 & .6928 & .6937 & .6946 & .6955 & .6964 & .6972 & .6981 \\
%\hline
%5.0 & \textbf{.6990} & .6998 & .7007 & .7016 & .7024 & .7033 & .7042 & .7050 & .7059 & .7067 \\
%\hline
%5.1 & .7076 & .7084 & .7093 & .7101 & .7110 & .7118 & .7126 & .7135 & .7143 & .7152 \\
%\hline
%5.2 & .7160 & .7168 & .7177 & .7185 & .7193 & .7202 & .7210 & .7218 & .7226 & .7235 \\
%\hline
%5.3 & .7243 & .7251 & .7259 & .7267 & .7275 & .7284 & .7292 & .7300 & .7308 & .7316 \\
%\hline
%5.4 & .7324 & .7332 & .7340 & .7348 & .7356 & .7364 & .7372 & .7380 & .7388 & .7396 \\
%\hline
%5.5 & .7404 & .7412 & .7419 & .7427 & .7435 & .7443 & .7451 & .7459 & .7466 & .7474 \\
%\hline
%5.6 & .7482 & .7490 & .7497 & .7505 & .7513 & .7520 & .7528 & .7536 & .7543 & .7551 \\
%\hline
%5.7 & .7559 & .7566 & .7574 & .7582 & .7589 & .7597 & .7604 & .7612 & .7619 & .7627 \\
%\hline
%5.8 & .7634 & .7642 & .7649 & .7657 & .7664 & .7672 & .7679 & .7686 & .7694 & .7701 \\
%\hline
%5.9 & .7709 & .7716 & .7723 & .7731 & .7738 & .7745 & .7752 & .7760 & .7767 & .7774 \\
%\hline
%6.0 & .7782 & .7789 & .7796 & .7803 & .7810 & .7818 & .7825 & .7832 & .7839 & .7846 \\
%\hline
%6.1 & .7853 & .7860 & .7868 & .7875 & .7882 & .7889 & .7896 & .7903 & .7910 & .7917 \\
%\hline
%6.2 & .7924 & .7931 & .7938 & .7945 & .7952 & .7959 & .7966 & .7973 & .7980 & .7987 \\
%\hline
%6.3 & .7993 & .8000 & .8007 & .8014 & .8021 & .8028 & .8035 & .8041 & .8048 & .8055 \\
%\hline
%6.4 & .8062 & .8069 & .8075 & .8082 & .8089 & .8096 & .8102 & .8109 & .8116 & .8122 \\
%\hline
%6.5 & .8129 & .8136 & .8142 & .8149 & .8156 & .8162 & .8169 & .8176 & .8182 & .8189 \\
%\hline
%6.6 & .8195 & .8202 & .8209 & .8215 & .8222 & .8228 & .8235 & .8241 & .8248 & .8254 \\
%\hline
%6.7 & .8261 & .8267 & .8274 & .8280 & .8287 & .8293 & .8299 & .8306 & .8312 & .8319 \\
%\hline
%6.8 & .8325 & .8331 & .8338 & .8344 & .8351 & .8357 & .8363 & .8370 & .8376 & .8382 \\
%\hline
%6.9 & .8388 & .8395 & .8401 & .8407 & .8414 & .8420 & .8426 & .8432 & .8439 & .8445 \\
%\hline
%7.0 & \textbf{.8451} & .8457 & .8463 & .8470 & .8476 & .8482 & .8488 & .8494 & .8500 & .8506 \\
%\hline
%7.1 & .8513 & .8519 & .8525 & .8531 & .8537 & .8543 & .8549 & .8555 & .8561 & .8567 \\
%\hline
%7.2 & .8573 & .8579 & .8585 & .8591 & .8597 & .8603 & .8609 & .8615 & .8621 & .8627 \\
%\hline
%7.3 & .8633 & .8639 & .8645 & .8651 & .8657 & .8663 & .8669 & .8675 & .8681 & .8686 \\
%\hline
%7.4 & .8692 & .8698 & .8704 & .8710 & .8716 & .8722 & .8727 & .8733 & .8739 & .8745 \\
%\hline
%7.5 & .8751 & .8756 & .8762 & .8768 & .8774 & .8779 & .8785 & .8791 & .8797 & .8802 \\
%\hline
%7.6 & .8808 & .8814 & .8820 & .8825 & .8831 & .8837 & .8842 & .8848 & .8854 & .8859 \\
%\hline
%7.7 & .8865 & .8871 & .8876 & .8882 & .8887 & .8893 & .8899 & .8904 & .8910 & .8915 \\
%\hline
%7.8 & .8921 & .8927 & .8932 & .8938 & .8943 & .8949 & .8954 & .8960 & .8965 & .8971 \\
%\hline
%7.9 & .8976 & .8982 & .8987 & .8993 & .8998 & .9004 & .9009 & .9015 & .9020 & .9025 \\
%\hline
%8.0 & .9031 & .9036 & .9042 & .9047 & .9053 & .9058 & .9063 & .9069 & .9074 & .9079 \\
%\hline
%8.1 & .9085 & .9090 & .9096 & .9101 & .9106 & .9112 & .9117 & .9122 & .9128 & .9133 \\
%\hline
%8.2 & .9138 & .9143 & .9149 & .9154 & .9159 & .9165 & .9170 & .9175 & .9180 & .9186 \\
%\hline
%8.3 & .9191 & .9196 & .9201 & .9206 & .9212 & .9217 & .9222 & .9227 & .9232 & .9238 \\
%\hline
%8.4 & .9243 & .9248 & .9253 & .9258 & .9263 & .9269 & .9274 & .9279 & .9284 & .9289 \\
%\hline
%8.5 & .9294 & .9299 & .9304 & .9309 & .9315 & .9320 & .9325 & .9330 & .9335 & .9340 \\
%\hline
%8.6 & .9345 & .9350 & .9355 & .9360 & .9365 & .9370 & .9375 & .9380 & .9385 & .9390 \\
%\hline
%8.7 & .9395 & .9400 & .9405 & .9410 & .9415 & .9420 & .9425 & .9430 & .9435 & .9440 \\
%\hline
%8.8 & .9445 & .9450 & .9455 & .9460 & .9465 & .9469 & .9474 & .9479 & .9484 & .9489 \\
%\hline
%8.9 & .9494 & .9499 & .9504 & .9509 & .9513 & .9518 & .9523 & .9528 & .9533 & .9538 \\
%\hline
%9.0 & .9542 & .9547 & .9552 & .9557 & .9562 & .9566 & .9571 & .9576 & .9581 & .9586 \\
%\hline
%9.1 & .9590 & .9595 & .9600 & .9605 & .9609 & .9614 & .9619 & .9624 & .9628 & .9633 \\
%\hline
%9.2 & .9638 & .9643 & .9647 & .9652 & .9657 & .9661 & .9666 & .9671 & .9675 & .9680 \\
%\hline
%9.3 & .9685 & .9689 & .9694 & .9699 & .9703 & .9708 & .9713 & .9717 & .9722 & .9727 \\
%\hline
%9.4 & .9731 & .9736 & .9741 & .9745 & .9750 & .9754 & .9759 & .9763 & .9768 & .9773 \\
%\hline
%9.5 & .9777 & .9782 & .9786 & .9791 & .9795 & .9800 & .9805 & .9809 & .9814 & .9818 \\
%\hline
%9.6 & .9823 & .9827 & .9832 & .9836 & .9841 & .9845 & .9850 & .9854 & .9859 & .9863 \\
%\hline
%9.7 & .9868 & .9872 & .9877 & .9881 & .9886 & .9890 & .9894 & .9899 & .9903 & .9908 \\
%\hline
%9.8 & .9912 & .9917 & .9921 & .9926 & .9930 & .9934 & .9939 & .9943 & .9948 & .9952 \\
%\hline
%9.9 & .9956 & .9961 & .9965 & .9969 & .9974 & .9978 & .9983 & .9987 & .9991 & .9996 \\
%\hline
%\end{longtable}

\newpage

\section{逻辑门的真值表}


\begin{tabular}{cc}%
	\begin{tabular}{|c|c|c|}
		\multicolumn{3}{c}{\textbf{与门(AND)}} \\
		\hline
		\multicolumn{2}{|c|}{\textbf{输入}} & \textbf{输出} \\
		\hline
		A & B & A AND B \\
		\hline
		0 & 0 & 0 \\
		\hline
		0 & 1 & 0 \\
		\hline
		1 & 0 & 0 \\
		\hline
		1 & 1 & 1 \\
		\hline
	\end{tabular} &
	\begin{tabular}{|c|c|c|}
		\multicolumn{3}{c}{\textbf{与非门(NAND)}} \\
		\hline
		\multicolumn{2}{|c|}{\textbf{输入}} & \textbf{输出} \\
		\hline
		A & B & A NAND B \\
		\hline
		0 & 0 & 1 \\
		\hline
		0 & 1 & 1 \\
		\hline
		1 & 0 & 1 \\
		\hline
		1 & 1 & 0 \\
		\hline
	\end{tabular} \tabularnewline
\end{tabular}

\bigskip

\begin{tabular}{cc}%
	\begin{tabular}{|c|c|c|}
		\multicolumn{3}{c}{\textbf{或门(OR)}} \\
		\hline
		\multicolumn{2}{|c|}{\textbf{输入}} & \textbf{输出} \\
		\hline
		A & B & A OR B \\
		\hline
		0 & 0 & 0 \\
		\hline
		0 & 1 & 1 \\
		\hline
		1 & 0 & 1 \\
		\hline
		1 & 1 & 1 \\
		\hline
	\end{tabular} &
	\begin{tabular}{|c|c|c|}
		\multicolumn{3}{c}{\textbf{或非门(NOR)}} \\
		\hline
		\multicolumn{2}{|c|}{\textbf{输入}} & \textbf{输出} \\
		\hline
		A & B & A NOR B \\
		\hline
		0 & 0 & 1 \\
		\hline
		0 & 1 & 0 \\
		\hline
		1 & 0 & 0 \\
		\hline
		1 & 1 & 0 \\
		\hline
	\end{tabular}  \tabularnewline
\end{tabular}

\bigskip

\begin{tabular}{cc}%
	\begin{tabular}{|c|c|c|}
		\multicolumn{3}{c}{\textbf{异或门(XOR)}} \\
		\hline
		\multicolumn{2}{|c|}{\textbf{输入}} & \textbf{输出} \\
		\hline
		A & B & A XOR B \\
		\hline
		0 & 0 & 0 \\
		\hline
		0 & 1 & 1 \\
		\hline
		1 & 0 & 1 \\
		\hline
		1 & 1 & 0 \\
		\hline
	\end{tabular} &
	\begin{tabular}{|c|c|c|}
		\multicolumn{3}{c}{\textbf{异或非门(NXOR)}} \\
		\hline
		\multicolumn{2}{|c|}{\textbf{输入}} & \textbf{输出} \\
		\hline
		A & B & A XNOR B \\
		\hline
		0 & 0 & 1 \\
		\hline
		0 & 1 & 0 \\
		\hline
		1 & 0 & 0 \\
		\hline
		1 & 1 & 1 \\
		\hline
	\end{tabular} \tabularnewline
\end{tabular}

\newpage

\section{计算数字逻辑电路对应的输出信号的C程序}

\lstinputlisting[language=C]{calc.c}

\newpage

\section{有用网址}

\begin{longtable}{|p{8cm}|p{8cm}|}
	\hline
	\textbf{名称} & \textbf{网址} \\
	\hline
	业余无线电台操作技术能力验证考核报名及信息管理系统 & \url{http://114.115.246.55:8091/CRAC/crac/index.html} \\
	\hline
	中国无线电协会业余无线电分会 & \url{http://www.crac.org.cn} \\
	\hline
\end{longtable}
