\chapter{附录}

\section{字母解释法(节录)}

\begin{tabular}{cc}%
\begin{tabular}[t]{|c|l|}
	\hline
	\textbf{字母} & \textbf{单词} \\
	\hline
	A & Alfa \\
	\hline
	B & Bravo \\
	\hline
	C & Charlie \\
	\hline
	D & Delta \\
	\hline
	E & Echo \\
	\hline
	F & Foxtrot \\
	\hline
	G & Golf \\
	\hline
	H & Hotel \\
	\hline
	I & India \\
	\hline
	J & Juliett \\
	\hline
	K & Kilo \\
	\hline
	L & Lima \\
	\hline
	M & Mike \\
	\hline
	N & November \\
	\hline
	O & Oscar \\
	\hline
	P & Papa \\
	\hline
	Q & Quebec \\
	\hline
	R & Romeo \\
	\hline
	S & Sierra \\
	\hline
	T & Tango \\
	\hline
	U & Uniform \\
	\hline
	V & Victor \\
	\hline
	W & Whiskey \\
	\hline
	X & X-ray \\
	\hline
	Y & Yankee \\
	\hline
	Z & Zulu \\
	\hline
\end{tabular} &
\begin{tabular}[t]{|c|l|}
	\hline
	\textbf{字母} & \textbf{单词} \\
	\hline
	0 & Zero \\
	\hline
	1 & One \\
	\hline
	2 & Two \\
	\hline
	3 & Three \\
	\hline
	4 & Four \\
	\hline
	5 & Five \\
	\hline
	6 & Six \\
	\hline
	7 & Seven \\
	\hline
	8 & Eight \\
	\hline
	9 & Nine \\
	\hline
\end{tabular} \tabularnewline
\end{tabular}

\newpage

\section{业余通信中常用的Q简语(节录)}

\begin{longtable}{|l|l|l|}
	\hline
	 & \textbf{提问} & \textbf{回答或建议} \\
	\hline
	QRL & 你正忙着吗 & 我正忙着 \\
	\hline
	QRM & 你遇到他台干扰吗 & 我遇到他台干扰 \\
	\hline
	QRN & 你遇到天电干扰吗 & 我遇到天电干扰 \\
	\hline
	QRO & 要我增加功率吗 & \\
	\hline
	QRP & 要我减小功率吗 & \\
	\hline
	QRQ & 要我加快发送速度吗 & 请加快发送速度 \\
	\hline
	QRS & 要我减慢发送速度吗 & 请减慢发送速度 \\
	\hline
	QRT & 要我停止发送吗 & 请停止发送 \\
	\hline
	QRU & 你和我还有事吗 & 我和你无事了 \\
	\hline
	QRV & 你是否已准备好 & 我已准备好 \\
	\hline
	QRZ & 谁在呼叫我 & \\
	\hline
	QSA & 我的信号强度如何 & 你的信号强度为 × 级(1-5 级) \\
	\hline
	QSB & 我的信号有衰落吗 & 你的信号有衰落 \\
	\hline
	QSD & 我发报的手法有毛病吗 & 你发报的手法有毛病 \\
	\hline
	\multirow{2}{1em}{QSK} & 能在你的信号间隙中接收吗 & 我在发射的信号间隙中接收 \\
	& (即 QSK 插入方式)    & (即 QSK 插入方式) \\
	\hline
	QSL & 你能给我收据(或 QSL 卡片)吗 & 我给你收据(QSL 卡片)、我已收妥 \\
	\hline
	QSO & 你能直接和 ××× 电台通信吗 & 我能直接和 ××× 电台通信 \\
	\hline
	QSP & 你能传信到 ××× 电台吗 & 我能传信到 ××× 电台 \\
	\hline
	\multirow{2}{1em}{QSX} & 你将在 nnnn KHz(或 MHz)频率 & 我将在 nnnn KHz(或 MHz)频率 \\
	    & 守听 ××× 电台吗 & 守听 ××× 电台 \\
	\hline
	QSY & 要我将频率改到 nnnn 频率吗 & 请将频率改到 nnnn 频率 \\
	\hline
	QTH & 你的电台位置在哪里 & 我的电台位置是 ×××× \\
	\hline
\end{longtable}

\newpage



\section{业余通信常用缩语表(节录)}

说明:黑体字的缩语为考到的重要缩语。

\begin{longtable}[l]{llll}
& \textbf{缩语} & \textbf{原词} & \textbf{含义} \\
& \endfirsthead
& \textbf{缩语} & \textbf{原词} & \textbf{含义} \\
& \endhead
A & \textbf{ABT} & About & 大约 \\
& AC & Alternating current & 交流 \\
& \textbf{ADR} & Address & 地址 \\
& \textbf{ADDR} & Address & 地址 \\
& \textbf{AGC} & Automatic gain control & 收信机自动增益控制 \\
& \textbf{AGN} & Again & 再、再来一次 \\
& \textbf{AHR} & Another & 另一个 \\
& \textbf{ALC} & Automatic level control & 发信自动电平控制 \\
& \textbf{AM} & Amplitude modulation & 幅度调制 \\
& \textbf{ANT} & Antenna & 天线 \\
& \textbf{ARDF} & Amateur radio direction finding & 业余无线电测向 \\
& $\overline{\mathbf{A}\mathbf{S}}$ & & 请稍等 \\
& \textbf{AS} & & 亚洲、如同 \\
& ASK & Amplitude-shift keying & 移幅键控 \\
& \textbf{AT} & Antenna tuner & 自动天线调谐 \\
& \textbf{ATT} & Attenuated / Attenuator & 衰减/收信机输入衰减器 \\
& \textbf{ATU} & Antenna tuning unit & 天线调谐器 \\
B & \textbf{BALUN} & balanced to unbalanced / balancing unit & 平衡和不平衡/平衡单元 \\
& \textbf{BEAM} & Beam antenna & 定向天线 \\
& \textbf{BEST} & & 最好的 \\
& \textbf{BJT} & Beijing Time & 北京时间 \\
& \textbf{BK} & Break & 插入、打断 \\
& BREAK & & 我还没有说完 \\
& BREAK BREAK & & 我还没有说完 \\
& Break in & & 能在你的信号间隙中接收吗 \\
& \textbf{BURO} & QSL Bureau & QSL卡片管理局 \\
C & \textbf{C} & & 遇到、见面 \\
& \textbf{CFM} & Confirm & 确认 \\
& \textbf{CHEERIO} & & 再会、祝贺 \\
& \textbf{CL} & Close, call & 关闭(或呼叫) \\
& \textbf{CLG} & Calling & 呼叫 \\
& \textbf{CLS} & Call sign & 呼号 \\
& \textbf{CPI} & Copy & 抄收 \\
& CQ & & 普遍呼叫 \\
& \textbf{CW} & Continuous wave & 等幅电报 \\
& \textbf{CTCSS} & Continuous Tone-Coded Squelch System & 亚音调静噪 \\
D & \textbf{DATE} & & 日期 \\
& dB & decibel & 增益(单位) \\
& DC & Direct current & 直流 \\
& DR & Dear & 亲爱的 \\
& DTMF & Dual-tone multi-frequency signaling & 双音多频编码 \\
& \textbf{DP} & Dipole antenna & 偶级天线 \\
E & \textbf{EHF} & Extremely high frequency & 极高频 \\
& \textbf{EL} & & 单元(常用于天线振子) \\
& \textbf{ELE} & & 单元(常用于天线振子) \\
& \textbf{ELS} & & 单元(常用于天线振子) \\
& \textbf{ES} & & 和 \\
F & FAX & Facsimile & 传真 \\
& \textbf{FB} & Fine business & 很好的 \\
& \textbf{FER} & & 为了,对于 \\
& \textbf{FINE} & & 好的,精细的 \\
& \textbf{FM} & Frequency modulation & 频率调制、调频 \\
& \textbf{FR} & & 为了,对于 \\
& \textbf{FREQ} & Frequency & 频率 \\
& \textbf{FSK} & Frequency-shift keying & 移频键控 \\
G & GA & Go ahead & 继续、请过来 \\
& \textbf{GA} & Good afternoon & 下午好 \\
& \textbf{GB} & Goodbye & 再见 \\
& \textbf{GE} & Good evening & 晚上好 \\
& GHz & gigahertz & 吉赫兹(单位) \\
& \textbf{GL} & Good luck & 好运气 \\
& \textbf{GLD} & Glad & 高兴 \\
& \textbf{GM} & Good morning & 早晨好 \\
& \textbf{GMT} & Greenwich Mean Time & 格林威治时间 \\ %格林尼治标准时间
& \textbf{GN} & Good night & 晚安 \\
& \textbf{GND} & Ground & 地线、地面 \\
& \textbf{GP} & Ground-plane antenna & 垂直接地天线 \\
H & HF & High frequency & 高频、即SW(Shortwave) \\
& \textbf{HPE} & Hope & 希望 \\
& \textbf{HPI} & Happy & 幸福 \\
& \textbf{HPY} & Happy & 幸福 \\
& \textbf{HR} & Here & 这里 \\
& \textbf{HR} & Hear & 听到 \\
& HST & High-speed telegraphy & 快速收发报 \\
& \textbf{HW} & How & 怎样、如何 \\
& Hz & hertz & 赫兹(单位) \\
I & \textbf{ITA1} & Baudot code & 博多码、ITA2的前身 \\
& \textbf{ITA2} & & 国际2号电报码 \\
K & kHz & kilohertz & 千赫兹(单位) \\
& KP & & 收听 \\
L & \textbf{LCD} & Liquid-crystal display & 液晶显示器 \\
& \textbf{LED} & Light-emitting diode & 发光二极管 \\
M & MHz & megahertz & 兆赫兹(单位) \\
& \textbf{MNI} & Many & 很多 \\
& \textbf{MNY} & Many & 很多 \\
& \textbf{MODE} & & 方式 \\
& \textbf{MTRS} & Meters & 米 \\
& MUF & Maximum usable frequency & 最高可用频率 \\
N & \textbf{NAME} & & 名字 \\
& \textbf{NB} & Noise blanker & 抑噪 \\
& NFM & Narrowband FM & 窄带调频 \\
& \textbf{NICE} & & 良好的 \\
& \textbf{NW} & Now & 现在 \\
O & \textbf{OM} & Old man & 老朋友 \\%朋友
& \textbf{OP} & Operator & 操作员 \\
& \textbf{OPR} & Operator & 操作员 \\
& OVER & & 我的传输已结束,期待回复 \\
P & \textbf{PM} & Phase modulation & 相位调制 \\
& \textbf{P O BOX} & Post office box & 邮政信箱 \\
& \textbf{PRE} & Preamplifier & 收信机前置放大器 \\
& \textbf{PROC} & Processor & 发信语音压缩 \\
& \textbf{PSK} & Phase-shift keying & 相位偏移调制 \\
& PSK31 & Phase Shift Keying, 31 Baud & 移相键控、31波特 \\
& \textbf{PTT} & Push-to-talk & 按键发射 \\
& \textbf{PWR} & Power & 功率 \\
Q & QPSK31 & Phase Shift Keying, 31 Baud & 移相键控、31波特 \\
R & \textbf{RCVR} & Receiver & 收信机 \\
& \textbf{RF} & Radio frequency & 无线电波 \\
& \textbf{RIG} & Equipment & 电台设备 \\
& \textbf{RIT} & Receiver incremental tuning & 接收增量调谐 \\
& \textbf{RMKS} & Remarks & 备注、注释 \\
& \textbf{ROGER} & & 回答起始语,相当于“明白”, \\
& &  & 仅在已完全抄收对方刚才发送的信息时使用 \\
& \textbf{RPRT} & Report & 报告 \\
& RST & Readability, Strength, Tone & 可辨度、信号强度、音调系统 \\
& \textbf{RTTY} & Radioteletype & \\
& \textbf{RX} & Receiver & 收信机 \\
S & \textbf{SASE} & & 写好收信人地址的信封 \\
& \textbf{SHF} & Super high frequency & 超高频 \\
& \textbf{SID} & Sudden ionospheric disturbance & 突发电离层扰动 \\
& $\overline{\mathbf{S}\mathbf{K}}$ & & 结束通信 \\
& \textbf{SSB} & Single-sideband modulation & 单边带调制 \\
& \textbf{SSN} & Sunspot number & 太阳黑子平均数 \\
& \textbf{SSTV} & Slow-scan television & 慢扫描电视 \\
& STANDBY & & 等待你的呼叫 \\
& \textbf{SQL} & Squelch & 静噪 \\
T & THIS IS & & 此消息发自以下xxxx电台 \\
& TIME & & 时间 \\
& \textbf{TX} & Transmitter & 发信机 \\
U & \textbf{UHF} & Ultra high frequency & 特高频 \\
& UTC & Coordinated Universal Time & 协调世界时 \\
V & \textbf{VER} & Vertical antenna & 垂直天线 \\
& \textbf{VHF} & Very high frequency & 甚高频 \\
& \textbf{VOX} & Voice-operated switch & 发信机声控 \\
W & WFM & Wideband FM & 宽带调频 \\
& \textbf{WX} & Weather & 天气 \\
X & \textbf{XIT} & Transmitter Incremental Tuning & 发射增量调谐 \\
& \textbf{XMTR} & Transmitter & 发信机 \\
& \textbf{XCVR} & Transceiver & 收发信机 \\
Y & \textbf{YAGI} & Yagi–Uda antenna & 八木天线 \\
& \textbf{73} & & 向对方的致意、美好的祝愿 \\
& \textbf{88} & & 向对方异性操作员的致意、美好的祝愿 \\
\end{longtable}

\newpage



\section{术语表}

\begin{longtable}[l]{ll}
	\textbf{术语} & \textbf{含义} \\
	Class of emission & 发射类别 \\
	Frequency-shift telegraphy & 频移电报技术\\
	GB & 中国强制性国家标准前缀 \\
	GB/T & 中国推荐性国家标准前缀 \\
	Necessary bandwidth & 必要带宽 \\
\end{longtable}

\newpage



\section{发射类别的表示方法(节录)}

\begin{tabular}{|c|c|}
	\hline
	\multicolumn{2}{|c|}{\textbf{主载波类型(第一个符号)}} \\
	\hline
	\textbf{符号} & \textbf{描述} \\
	\hline
	A & 双边带 \\
	\hline
	F & 调频 \\
	\hline
	G & 调相 \\
	\hline
	J & 单边带、抑制载波 \\
	\hline
\end{tabular}

\bigskip

\begin{tabular}{|c|c|}
	\hline
	\multicolumn{2}{|c|}{\textbf{调制主载波的信号的性质(第二个符号)}} \\
	\hline
	\textbf{符号} & \textbf{描述} \\
	\hline
	1 & 不用调制副载波但包含量化或数字信息的单个通路 \\
	\hline
	2 & 利用调制副载波且包含量化或数字信息的单个通路 \\
	\hline
	3 & 包含模拟信息的单个通路 \\
	\hline
\end{tabular}

\bigskip

\begin{tabular}{|c|c|}
	\hline
	\multicolumn{2}{|c|}{\textbf{被发送信息类型(第三个符号)}} \\
	\hline
	\textbf{符号} & \textbf{描述} \\
	\hline
	A & 电报──用于人工收听 \\
	\hline
	B & 电报──用于自动接收 \\
	\hline
	E & 电话(包括声音广播) \\
	\hline
	F & 电视(视频) \\
	\hline
\end{tabular}

\newpage

\section{无线电频段和波段表(节录)}

\begin{tabular}{|c|c|c|c|c|}
	\hline
	\textbf{频段名称} & \textbf{缩写} & \textbf{频率范围} & \textbf{波长范围} & \textbf{波段名称} \\
	%\hline
	%反正不考
	%极低频 & ELF & 3–30 Hz & 100,000–10,000 km & 极长波 \\
	%\hline
	%超低频 & SLF & 30–300 Hz & 10,000–1,000 km & 超长波 \\
	%\hline
	%特低频 & ULF & 300–3,000 Hz & 1,000–100 km & 特长波 \\
	%\hline
	%甚低频 & VLF & 3–30 kHz & 100–10 km & 甚长波 \\
	\hline
	低频 & LF & 30–300 kHz & 10–1 km & 长波 \\
	\hline
	中频 & MF & 300–3,000 kHz & 1,000–100 m & 中波 \\
	\hline
	高频 & HF & 3–30 MHz & 100–10 m & 短波 \\
	\hline
	甚高频 & VHF & 30–300 MHz & 10–1 m & 米波 \\
	\hline
	特高频 & UHF & 300–3,000 MHz & 1–0.1 m & 分米波 \\
	\hline
	超高频 & SHF & 3–30 GHz & 100–10 mm & 厘米波 \\
	\hline
	极高频 & EHF & 30–300 GHz & 10–1 mm & 毫米波 \\
	\hline
	%反正不考
	%至高频 & THF & 300–3,000 GHz & 1–0.1 mm & 丝米波 \\
	%\hline
\end{tabular}

\newpage






\section{业余无线电频率划分表(节录)}


\begin{tabular}{|c|c|c|}
	\hline
	\textbf{波段名称} & \textbf{频率范围} & \textbf{频率范围} \\
	\hline
	160米 & 1800–2000 kHz & 1.8-2 MHz \\
	\hline
	80米 & 3.5–4.0 MHz & 3500–4000 kHz \\
	\hline
	40米 & 7.0–7.3 MHz &  \\
	\hline
	20米 & 14.0–14.35 MHz & \\
	\hline
	15米 & 21–21.45 MHz & \\
	\hline
	10米 & 28–29.7 MHz & \\
	\hline
	6米 & 50–54 MHz & \\
	\hline
	2米 & 144–148 MHz & \\
	\hline
	0.7米 & 420–450 MHz & \\
	\hline
\end{tabular}

\newpage

\section{计算公式}

直流电路欧姆定律:
$$\mbox{电流}I(A)=\frac{\mbox{电压}U(V)}{\mbox{电阻值}R(\Omega)}$$

峰值电压的计算公式:
$$\mbox{峰值}(V)=\frac{\mbox{峰--峰值}}{2}$$

正弦交流电压的有效值计算公式:
$$\mbox{交流的有效值}(V)=\frac{\mbox{峰值}}{\sqrt{2}}=\mbox{峰值}\times0.707$$

功率增益的计算公式:
$$\mbox{功率增益}=10 \log_{10} \left( {\frac{P_{ \mbox{输出} }}{P_{ \mbox{输入} }}}\right)\ \mathrm{dB}$$

电压增益的计算公式:
$$\mbox{电压增益}=20 \log \left( {\frac{V_{ \mbox{输出} }}{V_{ \mbox{输入} }}} \right)\ \mathrm{dB}$$

%电流增益的计算公式:
%$$\mbox{电流增益}=20 \log \left( {\frac{I_{ \mbox{输出} }}{I_{ \mbox{输入} }}} \right)\ \mathrm{dB}$$

绝对增益与相对增益的关系:
$$\mbox{绝对增益} = \mbox{相对增益} + 2.15 \ \mathrm{dB}$$

两数的乘积的对数等于两数的对数之和:
$$\log xy=\log x+\log y$$

两数的商的对数等于两数的对数之差:
$$\log\frac{x}{y}=\log x-\log y$$

某数的p次幂的对数等于该数的对数的p倍:
$$\log x^p =p\log x$$

\newpage


\section{以10为底的常用对数表(节录)}

说明:强烈建议背诵$\log_{10} 1.0$、$\log 2.0$、……$\log 10.0$等第0列的10个数值到小数点后三位,考试闭卷,无法使用计算器,如果不背诵,将无法计算。

\begin{longtable}[c]{|c|c|}
\hline
\textbf{N} & \textbf{0} \\
\hline
\endfirsthead
\hline
\textbf{N} & \textbf{0} \\
\endhead
\textbf{1.0} & .0000 \\ \hline
\textbf{2.0} & .3010 \\ \hline
\textbf{3.0} & .4771 \\ \hline
\textbf{4.0} & .6021 \\ \hline
\textbf{5.0} & .6990 \\ \hline
\textbf{6.0} & .7782 \\ \hline
\textbf{7.0} & .8451 \\ \hline
\textbf{8.0} & .9031 \\ \hline
\textbf{9.0} & .9542 \\ \hline
\textbf{10.0} & 1.0000 \\ \hline
\end{longtable}

\newpage

\section{逻辑门的真值表}


\begin{tabular}{cc}%
	\begin{tabular}{|c|c|c|}
		\multicolumn{3}{c}{\textbf{与门(AND)}} \\
		\hline
		\multicolumn{2}{|c|}{\textbf{输入}} & \textbf{输出} \\
		\hline
		A & B & A AND B \\
		\hline
		0 & 0 & 0 \\
		\hline
		0 & 1 & 0 \\
		\hline
		1 & 0 & 0 \\
		\hline
		1 & 1 & 1 \\
		\hline
	\end{tabular} &
	\begin{tabular}{|c|c|c|}
		\multicolumn{3}{c}{\textbf{与非门(NAND)}} \\
		\hline
		\multicolumn{2}{|c|}{\textbf{输入}} & \textbf{输出} \\
		\hline
		A & B & A NAND B \\
		\hline
		0 & 0 & 1 \\
		\hline
		0 & 1 & 1 \\
		\hline
		1 & 0 & 1 \\
		\hline
		1 & 1 & 0 \\
		\hline
	\end{tabular} \tabularnewline
\end{tabular}

\bigskip

\begin{tabular}{cc}%
	\begin{tabular}{|c|c|c|}
		\multicolumn{3}{c}{\textbf{或门(OR)}} \\
		\hline
		\multicolumn{2}{|c|}{\textbf{输入}} & \textbf{输出} \\
		\hline
		A & B & A OR B \\
		\hline
		0 & 0 & 0 \\
		\hline
		0 & 1 & 1 \\
		\hline
		1 & 0 & 1 \\
		\hline
		1 & 1 & 1 \\
		\hline
	\end{tabular} &
	\begin{tabular}{|c|c|c|}
		\multicolumn{3}{c}{\textbf{或非门(NOR)}} \\
		\hline
		\multicolumn{2}{|c|}{\textbf{输入}} & \textbf{输出} \\
		\hline
		A & B & A NOR B \\
		\hline
		0 & 0 & 1 \\
		\hline
		0 & 1 & 0 \\
		\hline
		1 & 0 & 0 \\
		\hline
		1 & 1 & 0 \\
		\hline
	\end{tabular}  \tabularnewline
\end{tabular}

\bigskip

\begin{tabular}{cc}%
	\begin{tabular}{|c|c|c|}
		\multicolumn{3}{c}{\textbf{异或门(XOR)}} \\
		\hline
		\multicolumn{2}{|c|}{\textbf{输入}} & \textbf{输出} \\
		\hline
		A & B & A XOR B \\
		\hline
		0 & 0 & 0 \\
		\hline
		0 & 1 & 1 \\
		\hline
		1 & 0 & 1 \\
		\hline
		1 & 1 & 0 \\
		\hline
	\end{tabular} &
	\begin{tabular}{|c|c|c|}
		\multicolumn{3}{c}{\textbf{异或非门(NXOR)}} \\
		\hline
		\multicolumn{2}{|c|}{\textbf{输入}} & \textbf{输出} \\
		\hline
		A & B & A XNOR B \\
		\hline
		0 & 0 & 1 \\
		\hline
		0 & 1 & 0 \\
		\hline
		1 & 0 & 0 \\
		\hline
		1 & 1 & 1 \\
		\hline
	\end{tabular} \tabularnewline
\end{tabular}

\newpage

\section{计算逻辑门电路输出信号题型答案的C程序}

\lstinputlisting[language=C]{jisuan.c}

\newpage



\section{将呼号转换成字母解释法的单词组合的C程序}

\lstinputlisting[language=C]{icao.c}

\newpage



\section{业余无线电台呼号所属分区信息查询C程序}

\lstinputlisting[language=C]{hhcx.c}

\newpage

\section{有用网址}

\begin{longtable}{|p{8cm}|p{8cm}|}
	\hline
	\textbf{名称} & \textbf{网址} \\
	\hline
	业余无线电台操作技术能力验证考核报名及信息管理系统 & \url{http://114.115.246.55:8091/CRAC/crac/index.html} \\
	\hline
	中国无线电协会业余无线电分会 & \url{http://www.crac.org.cn} \\
	\hline
\end{longtable}
