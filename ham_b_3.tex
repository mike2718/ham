\chapter{无线电系统理论}




\noindent\textbf{问题:}收发信机面板上或设置菜单中的符号VOX代表什么功能?
\begin{enumerate}[label=\Alph*), leftmargin=3em]
\item 发信机声控,接入后将根据对话筒有无语音输入的判别自动控制收发转换
\item 发信自动电平控制,对射频输出电平进行检测并反馈控制,以维持其在适当限度之内
\item 发信自动音量控制,对音频输入电平进行检测并反馈控制,以维持其在适当限度之内
\item 自动天线调谐,对天线电路的电压驻波比进行检测并进行自动补偿,以维持最小驻波比
\end{enumerate}
\noindent\textbf{解说:}VOX代表\textbf{发信机声控,接入后将根据对话筒有无语音输入的判别自动控制收发转换}。\\\noindent\textbf{答案:}A



\bigskip


\noindent\textbf{问题:}收发信机中的PTT是指什么信号?
\begin{enumerate}[label=\Alph*), leftmargin=3em]
\item 按键发射,有信号(一般为对地接通)时发射机由等待转为发射
\item 发信语音压缩,对音频输入电平进行检测并反馈控制,以提升语音包络幅度较小的部分
\item 自动天线调谐,对天线电路的电压驻波比进行检测并进行自动补偿,以维持最小驻波比
\item 收信机前置放大器,在接收微弱信号时接入(此时某些技术指标可能低于额定值)
\end{enumerate}
\noindent\textbf{解说:}PTT代表\textbf{按键发射,有信号(一般为对地接通)时发射机由等待转为发射}。\\\noindent\textbf{答案:}A



\bigskip


\noindent\textbf{问题:}收发信机面板上或设置菜单中的符号SQL代表什么功能?
\begin{enumerate}[label=\Alph*), leftmargin=3em]
\item 收信机前置放大器,在接收微弱信号时接入(此时某些技术指标可能低于额定值)
\item 发信语音压缩,对音频输入电平进行检测并反馈控制,以提升语音包络幅度较小的部分
\item 自动天线调谐,对天线电路的电压驻波比进行检测并进行自动补偿,以维持最小驻波比
\item 静噪控制,检测到接收信号低于一定电平时关断音频输出
\end{enumerate}
\noindent\textbf{解说:}SQL代表\textbf{静噪控制,检测到接收信号低于一定电平时关断音频输出}。\\\noindent\textbf{答案:}D



\bigskip


\noindent\textbf{问题:}有些调频接收机的参数设置菜单有NFM和WFM两种选择。它们的含义是:
\begin{enumerate}[label=\Alph*), leftmargin=3em]
\item NFM为窄带调频方式,适用于信道带宽25kHz/12.5kHz的通信信号;WFM为宽带调频方式,适用于接收信道带宽180kHz左右的广播信号
\item NFM为调频通信本地方式(较低灵敏度),WFM为调频通信远程方式(最高灵敏度)
\item NFM代表数字化语音方式,WFM代表模拟语音方式
\item NFM为单频率守候方式,WFM为双频率守候方式
\end{enumerate}
\noindent\textbf{解说:NFM为窄带调频方式,适用于信道带宽25kHz/12.5kHz的通信信号;WFM为宽带调频方式,适用于接收信道带宽180kHz左右的广播信号}。\\\noindent\textbf{答案:}A



\bigskip


\noindent\textbf{问题:}某些对讲机具有发送DTMF码的功能。缩写DTMF指的是:
\begin{enumerate}[label=\Alph*), leftmargin=3em]
\item 数字设备识别码,即在松开PTT按键时自动发送一串代表设备代号的二进制数据
\item 自动静噪,即在接收机没有收到信号时自动关闭音频输出
\item 双音多频编码,由8个音调频率中的两个频率组合成的控制信号,代表16种状态之一,用于遥控和传输数字等简单字符
\item 亚音调静噪,即从67-250.3Hz的38个亚音调频率中选取一个作为选通信号,代表38种状态之一,接收机没有收到特定的选通信号时自动关闭音频输出
\end{enumerate}
\noindent\textbf{解说:}DTMF指\textbf{双音多频编码,由8个音调频率中的两个频率组合成的控制信号,代表16种状态之一,用于遥控和传输数字等简单字符}。\\\noindent\textbf{答案:}C



\bigskip


\noindent\textbf{问题:}某些对讲机具有发送CTCSS码的功能。缩写CTCSS指的是:
\begin{enumerate}[label=\Alph*), leftmargin=3em]
\item 自动静噪,即在接收机没有收到信号时自动关闭音频输出
\item 亚音调静噪,即从67-250.3Hz的38个亚音调频率中选取一个作为选通信号,代表38种状态之一,接收机没有收到特定的选通信号时自动关闭音频输出
\item 双音多频编码,由8个音调频率中的两个频率组合成的控制信号,代表16种状态之一,用于遥控和传输数字等简单字符
\item 数字设备识别码,即在松开PTT按键时自动发送一串代表设备代号的二进制数据
\end{enumerate}
\noindent\textbf{解说:}CTCSS指\textbf{亚音调静噪,即从67-250.3Hz的38个亚音调频率中选取一个作为选通信号,代表38种状态之一,接收机没有收到特定的选通信号时自动关闭音频输出}。\\\noindent\textbf{答案:}B



\bigskip


\noindent\textbf{问题:}关于是否可以在FM话音通信时单凭接收机听到对方语音的音量大小来准确判断对方信号的强弱,正确答案及其理由是:
\begin{enumerate}[label=\Alph*), leftmargin=3em]
\item 不能。因为信号越强,自动增益控制作用也越强,增益的急剧减小使声音反而被压低
\item 能。调频信号越强,频偏也必然越大,解调后的声音也越大
\item 不能。因为鉴频输出大小只取决于射频信号的频偏,而且正常信号的幅度会被限幅电路切齐到同样大小
\item 能。最后的信号是接收到的射频信号经过放大处理得到的,当然信号强声音越大
\end{enumerate}
\noindent\textbf{解说:}正确答案为\textbf{不能}。理由是\textbf{因为鉴频输出大小只取决于射频信号的频偏,而且正常信号的幅度会被限幅电路切齐到同样大小}。\\\noindent\textbf{答案:}C


\bigskip


\noindent\textbf{问题:}用设置在NFM方式的对讲机接收WFM信号,其效果为:
\begin{enumerate}[label=\Alph*), leftmargin=3em]
\item 可以正常听到信号,但声音比较小
\item 可以听到信号,但当调制信号幅度较大、音调较高时会发生明显非线性失真
\item 听不到信号,但接收到信号时调频噪声会变得寂静
\item 可以正常听到信号,但声音的高音频部分衰减较大,缺乏高音
\end{enumerate}
\noindent\textbf{解说:}用设置在NFM方式的对讲机接收WFM信号,其效果为\textbf{可以听到信号,但当调制信号幅度较大、音调较高时会发生明显非线性失真}。\\\noindent\textbf{答案:}B



\bigskip


\noindent\textbf{问题:}用设置在WFM方式的对讲机接收NFM信号,其效果为:
\begin{enumerate}[label=\Alph*), leftmargin=3em]
\item 可以正常听到信号,但声音比较小
\item 可以听到信号,但当调制信号幅度较大、音调较高时会发生明显非线性失真
\item 听不到信号,但接收到信号时调频噪声会变得寂静
\item 可以正常听到信号,但声音的高音频部分衰减较大,缺乏高音
\end{enumerate}
\noindent\textbf{解说:}用设置在WFM方式的对讲机接收NFM信号,其效果为\textbf{可以正常听到信号,但声音比较小}A



\bigskip


\noindent\textbf{问题:}调频接收机没有接收到信号时,会输出强烈的噪声。关于这种噪声的描述是:
\begin{enumerate}[label=\Alph*), leftmargin=3em]
\item 由天线背景噪声和机内电路噪声的随机频率变化经鉴频形成,其大小与天线接收到的背景噪声幅度无关
\item 由天线接收到的背景噪声的随机幅度变化经放大形成,其大小与天线背景噪声电压的平方根成正比
\item 由天线接收到的背景噪声的随机幅度变化经放大形成,其大小与天线背景噪声电压的平方成正比
\item 由天线接收到的背景噪声的随机幅度变化经放大形成,其大小与天线背景噪声电压成正比
\end{enumerate}
\noindent\textbf{解说:}这种噪声\textbf{由天线背景噪声和机内电路噪声的随机频率变化经鉴频形成,其大小与天线接收到的背景噪声幅度无关}。\\\noindent\textbf{答案:}A


\bigskip


\noindent\textbf{问题:}如果业余中继台发射机被断断续续的干扰信号所启动,夹杂着不清楚的语音,根据覆盖区内其他业余电台的监听,确定中继台上行频率并没有电台工作。则:
\begin{enumerate}[label=\Alph*), leftmargin=3em]
\item 可能是中继台发射机发生了寄生振荡
\item 肯定是中继台接收机受到了人为恶意干扰
\item 可能是中继台接收机发生了寄生振荡
\item 可能是中继台附近的两个其他发射机的强信号在中继台上行频率造成了互调干扰
\end{enumerate}
\noindent\textbf{解说:可能是中继台附近的两个其他发射机的强信号在中继台上行频率造成了互调干扰}。\\\noindent\textbf{答案:}D



\bigskip


\noindent\textbf{问题:}业余电台在进行业余卫星通信时使用超过常规要求的发射功率,造成的结果以及对这种做法的态度是:
\begin{enumerate}[label=\Alph*), leftmargin=3em]
\item 上行功率太大造成浪费和电磁污染;不提倡
\item 上行功率超过一定值对通信效果改善不大,但并无明显坏处;无所谓
\item 上行功率越大,转发的效果越好,通信范围越大;可提倡
\item 过强的上行信号会使卫星转发器压低对其他信道的转发功率,严重影响别人通信;必须反对
\end{enumerate}
\noindent\textbf{解说:}造成的结果是\textbf{过强的上行信号会使卫星转发器压低对其他信道的转发功率,严重影响别人通信};对这种做法的态度是\textbf{必须反对}。\\\noindent\textbf{答案:}D



\bigskip


\noindent\textbf{问题:}即使在空旷平地,接收到的本地VHF/UHF信号强度也可能会随着接收位置的移动而发生变化,最主要的可能原因是:
\begin{enumerate}[label=\Alph*), leftmargin=3em]
\item 移动过程中设备与大地之间的分布电容发生微小的变动
\item 收发位置之间的空气流动造成电波折射不均匀而飘移
\item 直射和经地面反射等多条路径到达的电波相位不同,互相叠加或抵消造成衰落(多径效应)
\item 不同接收位置大地导电率有差异
\end{enumerate}
\noindent\textbf{解说:}最主要的可能原因是:\textbf{直射和经地面反射等多条路径到达的电波相位不同,互相叠加或抵消造成衰落(多径效应)}。\\\noindent\textbf{答案:}C




\bigskip


\noindent\textbf{问题:}在相距不远的两点接收同一个远方信号,信号强度发生很大差别,且差别随两点间距离的增大呈周期性变化。这是因为:
\begin{enumerate}[label=\Alph*), leftmargin=3em]
\item 多径传播,各路径到达的信号相位延迟不同而互相干涉
\item 大气扰动影响
\item 发射机到两接收点传播距离不同造成传播路径衰耗不同
\item 地磁影响影响
\end{enumerate}
\noindent\textbf{解说:}这是因为\textbf{多径传播,各路径到达的信号相位延迟不同而互相干涉}。\\\noindent\textbf{答案:}A



\bigskip


\noindent\textbf{问题:}下列哪种设备可以用来代替普通的扬声器,可在嘈杂的环境中更好地抄收语音信号?
\begin{enumerate}[label=\Alph*), leftmargin=3em]
\item 视频显示器
\item 耳机
\item 低通滤波器
\item 吊杆话筒
\end{enumerate}
\noindent\textbf{解说:耳机}可以用来代替普通的扬声器,可在嘈杂的环境中更好地抄收语音信号。其他选项的用途为:低通滤波器用来过滤去除高频信号,视频显示器用来显示图像,吊杆话筒用来拾音。\\\noindent\textbf{答案:}B



\bigskip


\noindent\textbf{问题:}收发信机中的静噪控制的目的是什么?
\begin{enumerate}[label=\Alph*), leftmargin=3em]
\item 可以进行自动增益控制
\item 使接收机的输出音量调到最大
\item 控制发射机的输出功率
\item 在没有信号的情况下,关闭音频输出,使其不会输出噪音%在没有信号的情况下,关闭音频输出,使其不会输出噪音。
\end{enumerate}
\noindent\textbf{解说:}收发信机中的静噪控制的目的是\textbf{在没有信号的情况下,关闭音频输出,使其不会输出噪音}。\\\noindent\textbf{答案:}D




\bigskip


\noindent\textbf{问题:}如果对方报告你的调频电台发射的信号听起来失真严重、可辨度差,可能的原因是:
\begin{enumerate}[label=\Alph*), leftmargin=3em]
\item 三项都可能
\item 电台所处的位置不好
\item 电台的电源电压不足
\item 电台的发射频率不准确
\end{enumerate}
\noindent\textbf{解说:}如果对方报告你的调频电台发射的信号听起来失真严重、可辨度差,可能的原因是:
\begin{enumerate}[label=, leftmargin=0.8em]
\item (一)\textbf{电台的电源电压不足};
\item (二)\textbf{电台所处的位置不好};
\item (三)\textbf{电台的发射频率不准确}。
\end{enumerate}
\noindent\textbf{答案:}A



\bigskip


\noindent\textbf{问题:}一些VHF/UHF业余无线电调频手持对讲机或车载台的设置菜单中有“全频偏”和“半频偏”的选择,其表示的意义是:
\begin{enumerate}[label=\Alph*), leftmargin=3em]
\item 分别表示信道间隔为25kHz或者12.5kHz
\item 全频偏方式适用于收发异频的中继台通信,半频偏方式适用于同频对讲通信
\item 全频偏方式语音信号经过压缩,半频偏方式语音信号只压缩低音频分量
\item 全频偏方式下发射频率的误差比半频偏方式大一倍
\end{enumerate}
\noindent\textbf{解说:}\textbf{“全频偏”表示信道间隔为25kHz,“半频偏”表示信道间隔为12.5kHz}。\\\noindent\textbf{答案:}A



\bigskip


\noindent\textbf{问题:}发现有业余电台的发射操作技巧不够规范,但还不至于造成严重的干扰和影响,正确的做法是:
\begin{enumerate}[label=\Alph*), leftmargin=3em]
\item 立即报告无线电管理机构进行干涉
\item 立即在频率上当面加以指出和纠正
\item 通过电话、邮件等方式提出善意的改进建议
\item 立即报告当地业余无线电协会,由其总部电台到频率上进行纠察
\end{enumerate}
\noindent\textbf{解说:}正确的做法是\textbf{通过电话、邮件等方式提出善意的改进建议}。\\\noindent\textbf{答案:}C


\bigskip


\noindent\textbf{问题:}接收机设置项目中缩写“NB”和“SQL”的中文简称和作用是:
\begin{enumerate}[label=\Alph*), leftmargin=3em]
\item NB为“抑噪”,切除高于平均信号的大幅度突发脉冲噪声;SQL为“静噪”,信噪比达不到一定水平时自动关闭音频输出
\item NB和SQL都是指“静噪”,切除高于平均信号的大幅度突发脉冲噪声
\item NB和SQL都是指“抑噪”,收不到有用信号时自动关断背景噪声
\item NB和SQL都是指“静噪”,收不到带有预期的特定控制信号时自动关断音频输出
\end{enumerate}
\noindent\textbf{解说:}\textbf{NB为“抑噪”,切除高于平均信号的大幅度突发脉冲噪声;SQL为“静噪”,信噪比达不到一定水平时自动关闭音频输出}。\\\noindent\textbf{答案:}A


\bigskip


\noindent\textbf{问题:}单边带发信机语音压缩的作用是:
\begin{enumerate}[label=\Alph*), leftmargin=3em]
\item 压低语音信号的低频分量,提升高频分量,以增加信号的带宽,使高音更加细腻
\item 压缩信号所占用的频谱宽度,提高无线电频谱的利用率
\item 压低较强语音信号的幅度、提升较弱信号的幅度,以改善小幅度语音在接收端的信噪比
\item 压低较弱语音信号的幅度、提升较强信号的幅度,以增加语音的动态范围和抑扬顿挫感
\end{enumerate}
\noindent\textbf{解说:}单边带发信机语音压缩的作用是\textbf{压低较强语音信号的幅度、提升较弱信号的幅度,以改善小幅度语音在接收端的信噪比}。\\\noindent\textbf{答案:}C




\bigskip


\noindent\textbf{问题:}收发信机面板上的符号ALC代表什么功能?
\begin{enumerate}[label=\Alph*), leftmargin=3em]
\item 自动天线调谐,对天线电路的电压驻波比进行检测并进行自动补偿,以维持最小驻波比
\item 发信自动电平控制,对射频输出电平进行检测并反馈控制,以维持其在适当限度之内
\item 发信自动音量控制,对音频输入电平进行检测并反馈控制,以维持其在适当限度之内
\item 自动频率控制,对发射频率的漂移进行检测并反馈控制,以维持准确的工作频率
\end{enumerate}
\noindent\textbf{解说:}ALC代表\textbf{发信自动电平控制,对射频输出电平进行检测并反馈控制,以维持其在适当限度之内}。\\\noindent\textbf{答案:}B



\bigskip


\noindent\textbf{问题:}收发信机面板上的符号AT代表什么功能?
\begin{enumerate}[label=\Alph*), leftmargin=3em]
\item 发信自动电平控制,对射频输出电平进行检测并反馈控制,以维持其在适当限度之内
\item 自动频率控制,对发射频率的漂移进行检测并反馈控制,以维持准确的工作频率
\item 发信自动音量控制,对音频输入电平进行检测并反馈控制,以维持其在适当限度之内
\item 自动天线调谐,对天线电路的电压驻波比进行检测并进行自动补偿,以维持最小驻波比
\end{enumerate}
\noindent\textbf{解说:}AT代表\textbf{自动天线调谐,对天线电路的电压驻波比进行检测并进行自动补偿,以维持最小驻波比}。\\\noindent\textbf{答案:}D




\bigskip


\noindent\textbf{问题:}收发信机面板上的符号ATT代表什么功能?
\begin{enumerate}[label=\Alph*), leftmargin=3em]
\item 发信自动电平控制,对射频输出电平进行检测并反馈控制,以维持其在适当限度之内
\item 发信自动音量控制,对音频输入电平进行检测并反馈控制,以维持其在适当限度之内
\item 收信机输入衰减器,在接收大信号时接入,使信号不致过大而使前级电路过载
\item 自动天线调谐,对天线电路的电压驻波比进行检测并进行自动补偿,以维持最小驻波比
\end{enumerate}
\noindent\textbf{解说:}ATT代表\textbf{收信机输入衰减器,在接收大信号时接入,使信号不致过大而使前级电路过载}。\\\noindent\textbf{答案:}C




\bigskip


\noindent\textbf{问题:}收发信机面板上的符号AGC代表什么功能?
\begin{enumerate}[label=\Alph*), leftmargin=3em]
\item 发信自动电平控制,对射频输出电平进行检测并反馈控制,以维持其在适当限度之内
\item 收信自动音量控制,对音频输出电平进行检测并反馈控制,以维持其在适当限度之内
\item 自动天线调谐,对天线电路的电压驻波比进行检测并进行自动补偿,以维持最小驻波比
\item 收信机自动增益控制,对中频级信号电平进行检测并反馈控制,防止电路过载
\end{enumerate}
\noindent\textbf{解说:}AGC代表\textbf{收信机自动增益控制,对中频级信号电平进行检测并反馈控制,防止电路过载}。\\\noindent\textbf{答案:}D



\bigskip


\noindent\textbf{问题:}收发信机面板上的符号PRE代表什么功能?
\begin{enumerate}[label=\Alph*), leftmargin=3em]
\item 发信语音压缩,对音频输入电平进行检测并反馈控制,以提升语音包络幅度较小的部分
\item 自动天线调谐,对天线电路的电压驻波比进行检测并进行自动补偿,以维持最小驻波比
\item 发信自动电平控制,对射频输出电平进行检测并反馈控制,以维持其在适当限度之内
\item 收信机前置放大器,在接收微弱信号时接入(此时某些技术指标可能低于额定值)
\end{enumerate}
\noindent\textbf{解说:}PRE代表\textbf{收信机前置放大器,在接收微弱信号时接入(此时某些技术指标可能低于额定值)}。\\\noindent\textbf{答案:}D


\bigskip


\noindent\textbf{问题:}收发信机面板上的符号PROC代表什么功能?
\begin{enumerate}[label=\Alph*), leftmargin=3em]
\item 自动天线调谐,对天线电路的电压驻波比进行检测并进行自动补偿,以维持最小驻波比
\item 发信自动电平控制,对射频输出电平进行检测并反馈控制,以维持其在适当限度之内
\item 发信语音压缩,对音频输入电平进行检测并反馈控制,以提升语音包络幅度较小的部分
\item 收信机前置放大器,在接收微弱信号时接入(此时某些技术指标可能低于额定值)
\end{enumerate}
\noindent\textbf{解说:}PROC代表\textbf{发信语音压缩,对音频输入电平进行检测并反馈控制,以提升语音包络幅度较小的部分}。\\\noindent\textbf{答案:}C



\bigskip


\noindent\textbf{问题:}业余收发信机面饭上RIT的中文名称和代表的意义是:
\begin{enumerate}[label=\Alph*), leftmargin=3em]
\item 清除信道频率存贮器的所有数据
\item 接收增量调谐,在接收频率的主调谐不变的基础上,对接收频率进行附加微调
\item 发射增量调谐,在发射频率的主调谐不变的基础上,对发射频率进行附加微调
\item 异频收发,接收和发射使用互相独立的频率
\end{enumerate}
\noindent\textbf{解说:}RIT的中文名称是\textbf{接收增量调谐},其代表的意义是\textbf{在接收频率的主调谐不变的基础上,对接收频率进行附加微调}。\\\noindent\textbf{答案:}B




\bigskip


\noindent\textbf{问题:}业余收发信机面饭上XIT的中文名称和代表的意义是:
\begin{enumerate}[label=\Alph*), leftmargin=3em]
\item 异频收发,接收和发射使用互相独立的频率
\item 清除信道频率存贮器的所有数据
\item 发射增量调谐,在发射频率的主调谐不变的基础上,对发射频率进行附加微调
\item 接收增量调谐,在接收频率的主调谐不变的基础上,对接收频率进行附加微调
\end{enumerate}
\noindent\textbf{解说:}XIT的中文名称是\textbf{发射增量调谐},其代表的意义是\textbf{在发射频率的主调谐不变的基础上,对发射频率进行附加微调}。\\\noindent\textbf{答案:}C




\bigskip


\noindent\textbf{问题:}收听射频/中频增益和音频增益分开控制的通信接收机时,较好的设置方法是:
\begin{enumerate}[label=\Alph*), leftmargin=3em]
\item 信号特弱时尽量把音频增益开到最大,信号特强时尽量把射频/中频增益开到最大,然后从低到高调整另一个增益以得到适当的音量
\item 任何情况下都应将音频增益放在中间位置,然后从低到高调整射频/中频增益以得到适当的音量
\item 任何情况下都应将射频/中频增益放在中间位置,然后从低到高调整音频增益以得到适当的音量
\item 信号特弱时尽量把射频/中频增益开到最大,信号特强时尽量把音频增益开到最大,然后从低到高调整另一个增益以得到适当的音量
\end{enumerate}
\noindent\textbf{解说:}较好的设置方法是\textbf{信号特弱时尽量把射频/中频增益开到最大,信号特强时尽量把音频增益开到最大,然后从低到高调整另一个增益以得到适当的音量}。\\\noindent\textbf{答案:}D



\bigskip


\noindent\textbf{问题:}如果短波业余电台附近环境有不可避免的强烈噪声源影响接收微弱信号,合理的做法是:
\begin{enumerate}[label=\Alph*), leftmargin=3em]
\item 尽量只呼叫和回答能听到的电台,必须发起CQ呼叫时应降低功率
\item 发起CQ呼叫时应增大功率,以便压倒环境噪声
\item 发起正常满功率CQ呼叫,并应尽量调小接收机射频增益
\item 发起正常满功率CQ呼叫,并设置好接收机的NB、AGC等功能
\end{enumerate}
\noindent\textbf{解说:}合理的做法是\textbf{尽量只呼叫和回答能听到的电台,必须发起CQ呼叫时应降低功率}。\\\noindent\textbf{答案:}A




\bigskip


\noindent\textbf{问题:}某业余通信接收机的中频滤波器带宽有100Hz、400Hz、2.7kHz和6kHz几挡选择。如果要为接收SSB、AM、PSK31和CW方式的信号分别从中选择合适的挡位,应该依次为:
\begin{enumerate}[label=\Alph*), leftmargin=3em]
\item 2.7kHz、6kHz、100Hz、400Hz
\item 2.7kHz、400Hz、6kHz、100Hz
\item 2.7kHz、100Hz、6kHz、400Hz
\item 6kHz、2.7kHz、400Hz、100Hz
\end{enumerate}
\noindent\textbf{解说:}应该依次为\textbf{2.7kHz、6kHz、100Hz、400Hz}。\\\noindent\textbf{答案:}A




\bigskip


\noindent\textbf{问题:}单边带发信机中发信自动电平控制ALC的主要作用是:
\begin{enumerate}[label=\Alph*), leftmargin=3em]
\item 实现天线电路阻抗的自动匹配
\item 防止过驱动带来的调制失真
\item 防止话筒过于灵敏造成的背景噪音
\item 改善发信频率的稳定度
\end{enumerate}
\noindent\textbf{解说:}ALC的主要作用是\textbf{防止过驱动带来的调制失真}。\\\noindent\textbf{答案:}B




\bigskip


\noindent\textbf{问题:}业余电台发射单边带语音信号中,语音虽然基本正常,但操作员周围噪杂的声音很响,应该:
\begin{enumerate}[label=\Alph*), leftmargin=3em]
\item 调低发射机的话筒增益
\item 调低发射机的射频输出功率
\item 重新调整发射机天线电路的匹配
\item 重新调整发信机的自动电平控制(ALC)
\end{enumerate}
\noindent\textbf{解说:}应该\textbf{调低发射机的话筒增益}。\\\noindent\textbf{答案:}A




\bigskip


\noindent\textbf{问题:}如果将发射机的话筒增益设置得过高会导致什么问题?
\begin{enumerate}[label=\Alph*), leftmargin=3em]
\item 发射机的频率会变得不稳定
\item 发射机的输出功率将会特别高
\item 发射机发射的信号可能会失真
\item 驻波比会增加
\end{enumerate}
\noindent\textbf{解说:}如果将发射机的话筒增益设置得过高会导致\textbf{发射机发射的信号可能会失真}。\\\noindent\textbf{答案:}C



\bigskip


\noindent\textbf{问题:}电台的下列哪一项控制功能可以使听起来音调偏高或偏低的SSB语音信号变得正常?
\begin{enumerate}[label=\Alph*), leftmargin=3em]
\item RIT功能
\item 带宽选择
\item 自动增益控制或限幅器
\item 哑音静噪
\end{enumerate}
\noindent\textbf{解说:}\textbf{RIT功能}可以使听起来音调偏高或偏低的SSB语音信号变得正常。\\\noindent\textbf{答案:}A




\bigskip


\noindent\textbf{问题:}下述通信不属于电信(telecommunication)范畴:
\begin{enumerate}[label=\Alph*), leftmargin=3em]
\item 光通信
\item 有线通信
\item 无线电通信
\item 邮政通信
\end{enumerate}
\noindent\textbf{解说:}\textbf{邮政通信}不属于电信范畴。\\\noindent\textbf{答案:}D



\bigskip


\noindent\textbf{问题:}下列情况会产生减幅波辐射:
\begin{enumerate}[label=\Alph*), leftmargin=3em]
\item 对讲机按键发射
\item 电路接触点打火
\item 医用高频加热器泄漏
\item 电视机本振泄漏
\end{enumerate}
\noindent\textbf{解说:}电路接触点打火\textbf{会产生减幅波辐射}。\\\noindent\textbf{答案:}B




\bigskip


\noindent\textbf{问题:}在业余无线电中,计算莫尔斯电码的WPM速度时采用的信号时值标准(以一个“点”信号的时间为比较基准)为,点信号、划信号、字符内点划信号的间隔、字符之间的间隔、单词(组)之间的间隔分别为:
\begin{enumerate}[label=\Alph*), leftmargin=3em]
\item 1、3、1、3、5
\item 1、5、2、3、3
\item 1、5、1、5、7
\item 1、3、1、3、7
\end{enumerate}
\noindent\textbf{解说:}点信号、划信号、字符内点划信号的间隔、字符之间的间隔、单词(组)之间的间隔分别为:\textbf{1、3、1、3、7}。\\\noindent\textbf{答案:}D




\bigskip


\noindent\textbf{问题:}无线电原理经常用到缩写VFO,它代表:
\begin{enumerate}[label=\Alph*), leftmargin=3em]
\item 压控振荡器
\item 可变频率振荡器
\item 可变频率石英振荡器
\item 石英晶体元件
\end{enumerate}
\noindent\textbf{解说:}VFO代表\textbf{可变频率振荡器}。\\\noindent\textbf{答案:}B



\bigskip


\noindent\textbf{问题:}无线电原理经常用到缩写XTAL,它代表:
\begin{enumerate}[label=\Alph*), leftmargin=3em]
\item 石英晶体元件
\item 晶体振荡器
\item 压控振荡器
\item 可变频率振荡器
\end{enumerate}
\noindent\textbf{解说:}XTAL代表\textbf{石英晶体元件}。\\\noindent\textbf{答案:}A



\bigskip


\noindent\textbf{问题:}业余无线电慢扫描电视传送的是:
\begin{enumerate}[label=\Alph*), leftmargin=3em]
\item 逐行扫描的活动图像
\item 交叉扫描的活动图像
\item 逐行扫描的静止图像
\item 交叉扫描的静止图像
\end{enumerate}
\noindent\textbf{解说:}业余无线电慢扫描电视传送的是\textbf{逐行扫描的静止图像}。\\\noindent\textbf{答案:}C



\bigskip


\noindent\textbf{问题:}地球同步(geosynchronous)卫星是指:
\begin{enumerate}[label=\Alph*), leftmargin=3em]
\item 运行周期等于地球自转周期的地球卫星
\item 所经过地点的地方时基本相同的卫星
\item 其轨道平面通过地球北极和南极地区的卫星
\item 瞬时轨道平面与太阳始终保持固定取向的卫星
\end{enumerate}
\noindent\textbf{解说:}地球同步卫星是指\textbf{运行周期等于地球自转周期的地球卫星}。\\\noindent\textbf{答案:}A




\bigskip


\noindent\textbf{问题:}太阳同步(轨道)(Sun-synchronousorbit)卫星是指:
\begin{enumerate}[label=\Alph*), leftmargin=3em]
\item 运行周期等于地球自转周期的地球卫星
\item 对地球保持大致相对静止的卫星
\item 瞬时轨道平面与太阳始终保持固定取向的卫星
\item 圆形及顺行轨道位于地球赤道平面上,并对地球保持相对静止的卫星
\end{enumerate}
\noindent\textbf{解说:}太阳同步(轨道)卫星是指\textbf{瞬时轨道平面与太阳始终保持固定取向的卫星}。\\\noindent\textbf{答案:}C



\bigskip


\noindent\textbf{问题:}地球(geostationary)静止卫星是指:
\begin{enumerate}[label=\Alph*), leftmargin=3em]
\item 其轨道平面通过地球北极和南极地区的卫星
\item 所有的地球同步卫星
\item 对地球保持大致相对静止的卫星
\item 所经过地点的地方时基本相同的卫星
\end{enumerate}
\noindent\textbf{解说:}地球静止卫星是指\textbf{对地球保持大致相对静止的卫星}。\\\noindent\textbf{答案:}C




\bigskip


\noindent\textbf{问题:}卫星的周期是指:
\begin{enumerate}[label=\Alph*), leftmargin=3em]
\item 卫星两次正好从地面某一点的正上方通过的间隔时间
\item 卫星随地球绕太阳一周所需的时间
\item 卫星沿轨道绕地球一周所需的时间
\item 卫星绕质心自旋一周所需的时间
\end{enumerate}
\noindent\textbf{解说:}卫星的周期是指\textbf{卫星沿轨道绕地球一周所需的时间}。\\\noindent\textbf{答案:}C



\bigskip


\noindent\textbf{问题:}业余低轨卫星的转发器覆盖范围有限。利用这类卫星进行全球性业余无线电通信的解决方法是:
\begin{enumerate}[label=\Alph*), leftmargin=3em]
\item 由地面站进行地面中继
\item 由卫星对上行数据进行存贮和转发
\item 地面业余电台换用更大功率的发射机
\item 增加地面业余电台的天线高度
\end{enumerate}
\noindent\textbf{解说:}解决方法是\textbf{由卫星对上行数据进行存贮和转发}。\\\noindent\textbf{答案:}B




\bigskip


\noindent\textbf{问题:}我国发射的第一颗业余卫星的发射年份、名称、国际OSCAR系列号和转发器模式分别为:
\begin{enumerate}[label=\Alph*), leftmargin=3em]
\item 2010年,希望一号(XW-1),HO68,V/U(J)
\item 2008年,希望一号(XW-1),无OSCAR编号,B
\item 2008年,希望一号(XW-1),HO68,V/U(J)
\item 2009年,希望一号(XW-1),HO68,V/U(J)
\end{enumerate}
\noindent\textbf{解说:}我国发射的第一颗业余卫星的发射年份、名称、国际OSCAR系列号和转发器模式分别为\textbf{2009年,希望一号(XW-1),HO68,V/U(J)}。\\\noindent\textbf{答案:}D




\bigskip


\noindent\textbf{问题:}根据数据串行通信收发两端的时钟只需要在传送一个字节的时间内保持同步还是需要在传送一整块数据的时间内保持同步,可以分为“异步”和“同步”两种方式。下列业余无线电数字通信方式中属于异步方式的例子是:
\begin{enumerate}[label=\Alph*), leftmargin=3em]
\item PACKET
\item PACTOR-II
\item QPSK31
\item RTTY
\end{enumerate}
\noindent\textbf{解说:}\textbf{RTTY}属于异步方式。\\\noindent\textbf{答案:}D



\bigskip


\noindent\textbf{问题:}根据数据串行通信收发两端的时钟只需要在传送一个字节的时间内保持同步还是需要在传送一整块数据的时间内保持同步,可以分为“异步”和“同步”两种方式。下列业余无线电数字通信方式中不属于同步方式的例子是:
\begin{enumerate}[label=\Alph*), leftmargin=3em]
\item PACKET
\item RTTY
\item BPSK31
\item AMTOR-FEC
\end{enumerate}
\noindent\textbf{解说:}\textbf{RTTY}不属于同步方式。\\\noindent\textbf{答案:}B



\bigskip


\noindent\textbf{问题:}在进行串行异步数字通信时,双方需要设置相同的波特率,数据位数,校验位数,停止位数。RTTY最常用的设置为:
\begin{enumerate}[label=\Alph*), leftmargin=3em]
\item 50(或45.45),5,N,1
\item 50(或45.45),8,2,3
\item 2295,2125,170,5
\item 31.25,7,170,0.3
\end{enumerate}
\noindent\textbf{解说:}RTTY最常用的设置为\textbf{50(或45.45),5,N,1}。\\\noindent\textbf{答案:}A



\bigskip


\noindent\textbf{问题:}PACKET是业余无线电爱好者利用X.25数据分组通信协议开发的业余无线电通信方式,用于HF频段、VHF频段和卫星通信时通常采用的信号速率分别为:
\begin{enumerate}[label=\Alph*), leftmargin=3em]
\item 300波特、1200波特、9600波特
\item 300波特、2400波特、19200波特
\item 600波特、1200波特、2400波特
\item 1200波特、9600波特、19200波特
\end{enumerate}
\noindent\textbf{解说:}用于HF频段、VHF频段和卫星通信时通常采用的信号速率分别为\textbf{300波特、1200波特、9600波特}。\\\noindent\textbf{答案:}A



\bigskip


\noindent\textbf{问题:}国际2号电报码(ITA2)的俗称、在业余无线电通信中应用场合及其与计算机常用的数据交换代码相比的主要特点是:
\begin{enumerate}[label=\Alph*), leftmargin=3em]
\item GB2312编码,用于传输汉字,由两个字节连用代表一个汉字
\item ASCII码,用于PSK31通信,每字节仅包含7位二进制数据
\item 博多码(Baudot code),用于RTTY通信,每字节仅包含5位二进制数据
\item 莫尔斯电码,用于CW通信,每个字符包含的信号数量不等(非对称码)
\end{enumerate}
\noindent\textbf{解说:}国际2号电报码(ITA2)的俗称是\textbf{博多码(Baudot code)}、在业余无线电通信中应用场合及其与计算机常用的数据交换代码相比的主要特点是\textbf{用于RTTY通信,每字节仅包含5位二进制数据}。\\\noindent\textbf{答案:}C



\bigskip


\noindent\textbf{问题:}亚音调静噪(CTCSS)是指附加在发射端信号中的一个亚音频控制音调。这个信号的频率范围大致是:
\begin{enumerate}[label=\Alph*), leftmargin=3em]
\item 220Hz - 2503Hz
\item 16Hz - 20kHz
\item 67Hz - 250.3Hz
\item 16kHz - 20kHz
\end{enumerate}
\noindent\textbf{解说:}这个信号的频率范围大致是\textbf{67Hz - 250.3Hz}。\\\noindent\textbf{答案:}C



\bigskip


\noindent\textbf{问题:}在超外差式收信机电路中,信号通道的有用信号频率比本振频率低(或者高)一个中频频率。但比本振频率高(或者低)一个中频频率的信号也可能窜入信号通道,称为“镜像频率干扰”或“镜频干扰”。某VHF对讲机使用说明书的技术指标部分给出了第一中频(IF)为45.05MHz,但没有更多的资料。由此可推测当接收145.00MHz信号时下述频率之一的强信号可能造成镜频干扰:
\begin{enumerate}[label=\Alph*), leftmargin=3em]
\item 90.10MHz或.180.20MHz
\item 190.05MHz或99.95MHz
\item 235.10MHz或54.90MHz
\item 45.05MHz或90.10MHz
\end{enumerate}
\noindent\textbf{解说:}由此可推测当接收145.00MHz信号时,\textbf{235.10MHz或54.90MHz}之一的强信号可能造成镜频干扰。\\\noindent\textbf{答案:}C




\bigskip


\noindent\textbf{问题:}在超外差式收信机电路中,信号通道的有用信号频率比本振频率低(或者高)一个中频频率。但比本振频率高(或者低)一个中频频率的信号也可能窜入信号通道,称为“镜像频率干扰”或“镜频干扰”。某对讲机使用说明书的技术指标部分给出了在NFM方式时第一中频(IF)为47.25MHz,但没有更多的资料。由此可推测当接收145.00MHz信号时下述频率之一的强信号可能造成镜频干扰:%:
\begin{enumerate}[label=\Alph*), leftmargin=3em]
\item 50.50MHz或101.00MHz
\item 151.50MHz或.202.00MHz
\item 192.25MHz或97.75MHz
\item 239.50MHz或50.50MHz
\end{enumerate}
\noindent\textbf{解说:}由此可推测当接收145.00MHz信号时,\textbf{239.50MHz或50.50MHz}之一的强信号可能造成镜频干扰。\\\noindent\textbf{答案:}D



\bigskip


\noindent\textbf{问题:}在超外差式收信机电路中,信号通道的有用信号频率比本振频率低(或者高)一个中频频率。但比本振频率高(或者低)一个中频频率的信号也可能窜入信号通道,称为“镜像频率干扰”或“镜频干扰”。某对讲机使用说明书的技术指标部分给出了接收NFM信号时第一中频(IF)为47.25MHz,但没有更多的资料。由此可推测当接收435.00MHz信号时下述频率之一的强信号可能造成镜频干扰:
\begin{enumerate}[label=\Alph*), leftmargin=3em]
\item 47.25MHz或94.50MHz
\item 141.70MHz或.236.25MHz
\item 387.75MHz或482.25MHz
\item 340.50MHz或529.50MHz
\end{enumerate}
\noindent\textbf{解说:}由此可推测当接收435.00MHz信号时\textbf{340.50MHz或529.50MHz}之一的强信号可能造成镜频干扰。\\\noindent\textbf{答案:}D



\bigskip


\noindent\textbf{问题:}在超外差式收信机电路中,信号通道的有用信号频率比本振频率低(或者高)一个中频频率。但比本振频率高(或者低)一个中频频率的信号也可能窜入信号通道,称为“镜像频率干扰”或“镜频干扰”。某UHF对讲机的使用说明书技术指标部分给出了第一中频(IF)为58.525MHz,但没有更多的资料。由此可推测当接收435.00MHz信号时下述频率之一的强信号可能造成镜频干扰:
\begin{enumerate}[label=\Alph*), leftmargin=3em]
\item 58.525MHz或117.05MHz
\item 376.475MHz或493.525MHz
\item 234.10.05MHz或.468.20MHz
\item 317.95MHz或552.05MHz
\end{enumerate}
\noindent\textbf{解说:}由此可推测当接收435.00MHz信号时\textbf{317.95MHz或552.05MHz}之一的强信号可能造成镜频干扰。\\\noindent\textbf{答案:}D




\bigskip


\noindent\textbf{问题:}现代超外差式业余收发信机面板上常设有选择中频滤波器带宽的控制部件。这些中频滤波器负责抑制的干扰种类为:
\begin{enumerate}[label=\Alph*), leftmargin=3em]
\item 中频频率干扰
\item 邻近频率干扰
\item 镜像频率干扰
\item 突发脉冲干扰
\end{enumerate}
\noindent\textbf{解说:}这些中频滤波器负责抑制的干扰种类为\textbf{邻近频率干扰}。\\\noindent\textbf{答案:}B




\bigskip


\noindent\textbf{问题:}超外差式业余收发信机中负责抑制镜像频率干扰的部件是:
\begin{enumerate}[label=\Alph*), leftmargin=3em]
\item 中频放大级中的限幅电路
\item 变频级之后的中频滤波器
\item 变频级之前的预选滤波器
\item 带有音调控制的音频滤波器
\end{enumerate}
\noindent\textbf{解说:}。超外差式业余收发信机中负责抑制镜像频率干扰的部件是\textbf{变频级之前的预选滤波器}。\\\noindent\textbf{答案:}C



\bigskip


\noindent\textbf{问题:}很多具有静噪功能的FM通信接收机在对方松开话筒PTT键后,会听到一声很明显的“嘶啦”或“喀拉”噪声拖尾,其原因是:
\begin{enumerate}[label=\Alph*), leftmargin=3em]
\item 此类电路根据鉴频输出中的强高音频噪声分量判断电台信号是否消失,从而关断音频输出。该项检测需占用一定时间,造成静噪的延迟,短时间漏出鉴频噪声
\item 该噪声由发射台话筒PTT接点跳动造成,发射到接收端
\item 该噪声是发射设备有目的地发射,作为结束发射的一种标志
\item 接收机自动增益电路的时间常数造成
\end{enumerate}
\noindent\textbf{解说:}。其原因是\textbf{此类电路根据鉴频输出中的强高音频噪声分量判断电台信号是否消失,从而关断音频输出。该项检测需占用一定时间,造成静噪的延迟,短时间漏出鉴频噪声}。\\\noindent\textbf{答案:}A



\bigskip


\noindent\textbf{问题:}能够确定直流电路中任何一个两端元件工作状况的基本参数包括:
\begin{enumerate}[label=\Alph*), leftmargin=3em]
\item 波长、电压、电容量
\item 频率、电荷量、电场强度
\item 驻波比、功率、阻抗
\item 通过电流、两端电压、电阻
\end{enumerate}
\noindent\textbf{解说:}能够确定直流电路中任何一个两端元件工作状况的基本参数包括\textbf{通过电流、两端电压、电阻}。\\\noindent\textbf{答案:}D



\bigskip


\noindent\textbf{问题:}物理量“电流”描述的是:
\begin{enumerate}[label=\Alph*), leftmargin=3em]
\item 电源所能供出的最大的电荷数量
\item 单位时间内流过电路的电荷数量
\item 流过电路的累计电荷数量
\item 电荷在电路导体内的运动速度
\end{enumerate}
\noindent\textbf{解说:}物理量“电流”描述的是\textbf{单位时间内流过电路的电荷数量}。\\\noindent\textbf{答案:}B



\bigskip


\noindent\textbf{问题:}物理量“电压”描述的是:
\begin{enumerate}[label=\Alph*), leftmargin=3em]
\item 电源把其它形式的能量转化为电能的能力
\item 电源加在电路两端驱动电子流动的力量大小
\item 单位时间内流过电路的电荷数量
\item 电源所能供出的最大的电荷数量
\end{enumerate}
\noindent\textbf{解说:}物理量“电压”描述的是\textbf{电源加在电路两端驱动电子流动的力量大小}。\\\noindent\textbf{答案:}B


\bigskip


\noindent\textbf{问题:}物理量“电动势”描述的是:
\begin{enumerate}[label=\Alph*), leftmargin=3em]
\item 单位时间内流过电路的电荷数量
\item 电源加在电路两端驱动电子流动的力量大小
\item 电源所能供出的最大的电荷数量
\item 电源把其它形式的能量转化为电能的能力
\end{enumerate}
\noindent\textbf{解说:}物理量“电动势”描述的是\textbf{电源把其它形式的能量转化为电能的能力}。\\\noindent\textbf{答案:}D



\bigskip


\noindent\textbf{问题:}物理量“电阻”描述的是:
\begin{enumerate}[label=\Alph*), leftmargin=3em]
\item 电路阻断电流所需要的过度时间
\item 电流克服电路阻力的能力大小
\item 电路对电流的阻碍力大小
\item 电路阻力所消耗的能量多少
\end{enumerate}
\noindent\textbf{解说:}物理量“电阻”描述的是\textbf{电路对电流的阻碍力大小}。\\\noindent\textbf{答案:}C



\bigskip


\noindent\textbf{问题:}物理量“功率”描述的是:
\begin{enumerate}[label=\Alph*), leftmargin=3em]
\item 电源总共能够做的功
\item 电源所能供出的最大的电荷数量
\item 负载总共消耗的能量
\item 单位时间里电所能够做的功
\end{enumerate}
\noindent\textbf{解说:}物理量“功率”描述的是\textbf{单位时间里电所能够做的功}。\\\noindent\textbf{答案:}D



\bigskip


\noindent\textbf{问题:}正弦交流电压或电流的峰值(peak value)是指:(”x^m”表示“x的m次方”)
\begin{enumerate}[label=\Alph*), leftmargin=3em]
\item 一个周期内瞬时值的平均乘以值2^(1/2)
\item 负半周最大幅度与正半周最大幅度的差值的二次方
\item 从零点算起的最大值
\item 负半周最大幅度与正半周最大幅度的平均值
\end{enumerate}
\noindent\textbf{解说:}正弦交流电压或电流的峰值是指\textbf{从零点算起的最大值}。\\\noindent\textbf{答案:}C


\bigskip


\noindent\textbf{问题:}正弦交流电压或电流的峰值峰-峰值(peak to peak)是指:
\begin{enumerate}[label=\Alph*), leftmargin=3em]
\item 负半周最大幅度与正半周最大幅度的差值的二次方
\item 从零点算起的最大值
\item 负半周最大幅度与正半周最大幅度的差值的平方根
\item 从负半周峰值到正半周峰值之间的差
\end{enumerate}
\noindent\textbf{解说:}正弦交流电压或电流的峰值峰-峰值是指\textbf{从负半周峰值到正半周峰值之间的差}。\\\noindent\textbf{答案:}D



\bigskip


\noindent\textbf{问题:}任意交流电压的有效值是指:(”x^m”表示“x的m次方”)
\begin{enumerate}[label=\Alph*), leftmargin=3em]
\item 最终转换成应用所需的有用能量的那部分电压值
\item 电压的峰值除以2^(1/2)
\item 电压的平均值乘以2^(1/2)
\item 在同一电阻上可以转换出与该交流电压效果相同的热量的直流电压
\end{enumerate}
\noindent\textbf{解说:}任意交流电压的有效值是指\textbf{在同一电阻上可以转换出与该交流电压效果相同的热量的直流电压}。\\\noindent\textbf{答案:}D



\bigskip


\noindent\textbf{问题:}把两个幅度相等、相位相差360°的正弦电压信号源相串联,得到的是:
\begin{enumerate}[label=\Alph*), leftmargin=3em]
\item 幅度为单个信号源的2倍、相位与原信号源相同的正弦电压
\item 幅度与单个信号源的相同、相位与原信号源相差180°的正弦电压
\item 幅度与单个信号源的相同、频率比原信号高一倍的正弦电压
\item 电压为0
\end{enumerate}
\noindent\textbf{解说:}得到的是\textbf{幅度为单个信号源的2倍、相位与原信号源相同的正弦电压}。\\\noindent\textbf{答案:}A



\bigskip


\noindent\textbf{问题:}把两个幅度相等、相位相差180°的正弦电压信号源相串联,得到的是:
\begin{enumerate}[label=\Alph*), leftmargin=3em]
\item 幅度与单个信号源的相同、频率比原信号高一倍的正弦电压
\item 幅度为单个信号源的2倍、相位与原信号源相同的正弦电压
\item 幅度与单个信号源的相同、相位与原信号源相差90°的正弦电压
\item 电压为0
\end{enumerate}
\noindent\textbf{解说:}得到的是\textbf{电压为0}。\\\noindent\textbf{答案:}D



\bigskip


\noindent\textbf{问题:}把两个幅度相等、相位相差90°的正弦电压信号源相串联,得到的是:
\begin{enumerate}[label=\Alph*), leftmargin=3em]
\item 幅度与单个信号源的相同、频率比原信号高一倍的正弦电压
\item 幅度为单个信号源的2倍、相位与原信号源相同的正弦电压
\item 幅度与单个信号源的相同、相位与原信号源相差45°的正弦电压
\item 幅度为单个信号源的1.41倍、相位与原信号源相差45°的正弦电压
\end{enumerate}
\noindent\textbf{解说:}得到的是\textbf{幅度为单个信号源的1.41倍、相位与原信号源相差45°的正弦电压}。\\\noindent\textbf{答案:}D



\bigskip


\noindent\textbf{问题:}将一个电阻为R的负载接到电压为U的电源上。负载中的电流I和负载消耗的功率P分别为:(U、I、R、P的单位分别为伏特、安培、欧姆、瓦特,”x^m”表示“x的m次方”)
\begin{enumerate}[label=\Alph*), leftmargin=3em]
\item I = U×R,P =U^2×R
\item I = U×R,P =U/R
\item I = R/U,P =U×R
\item I = U/R,P = U^2/R
\end{enumerate}
\noindent\textbf{解说:}负载中的电流I和负载消耗的功率P分别为:\textbf{I = U/R,P = U^2/R}。\\\noindent\textbf{答案:}D




\bigskip


\noindent\textbf{问题:}一个电阻为R的负载中流过的电流为I。其两端的电压U所消耗的功率P分别为:(U、I、R、P的单位分别为伏特、安培、欧姆、瓦特,”x^m”表示“x的m次方”)
\begin{enumerate}[label=\Alph*), leftmargin=3em]
\item U = I + R,P =I ×R
\item U = I / R,P = I^2/R
\item U = R / I,P = R / I^2
\item U = I×R,P = I^2×R
\end{enumerate}
\noindent\textbf{解说:}其两端的电压U所消耗的功率P分别为:\textbf{U = I×R,P = I^2×R}。\\\noindent\textbf{答案:}D


\bigskip


\noindent\textbf{问题:}一个电阻负载两端电压为U,流过的电流为I。它的电阻R和所消耗的功率P分别为:(U、I、R、P的单位分别为伏特、安培、欧姆、瓦特,”x^m”表示“x的m次方”)
\begin{enumerate}[label=\Alph*), leftmargin=3em]
\item R = I / U,P = U^2/I
\item R = U/I,P = U×I
\item R = U×I,P = U/ I^2
\item R = U^2/I,P = U^2×I
\end{enumerate}
\noindent\textbf{解说:}它的电阻R和所消耗的功率P分别为:\textbf{R = U/I,P = U×I}。\\\noindent\textbf{答案:}B




\bigskip


\noindent\textbf{问题:}一个电阻负载两端电压为U,所消耗的功率为P。流过负载的电流I和负载的电阻R分别为:(U、I、R、P的单位分别为伏特、安培、欧姆、瓦特,”x^m”表示“x的m次方”)
\begin{enumerate}[label=\Alph*), leftmargin=3em]
\item I = P×U,R = P/U
\item I = P/U,R = U^2/P
\item I = U/P,R = P/U
\item I = P/ U^2,R = P×U
\end{enumerate}
\noindent\textbf{解说:}流过负载的电流I和负载的电阻R分别为:\textbf{I = P/U,R = U^2/P}。\\\noindent\textbf{答案:}B




\bigskip


\noindent\textbf{问题:}有阻值分别为R1和R2的两个负载,其中R1的电阻值是R2的N倍,把它们并联后接到电源上,则:(”x^m”表示“x的m次方”)
\begin{enumerate}[label=\Alph*), leftmargin=3em]
\item 流过R1的电流是R2的1/N,R1消耗的功率是R2的1/N
\item 流过R1的电流与R2的相同,R1消耗的功率是R2的N倍
\item 流过R1的电流是R2的N倍,R1消耗的功率是R2的N^2倍
\item 流过R1的电流与R2的相同,R1消耗的功率是R2的1/N^2
\end{enumerate}
\noindent\textbf{解说:}\textbf{流过R1的电流是R2的1/N,R1消耗的功率是R2的1/N}。\\\noindent\textbf{答案:}A




\bigskip


\noindent\textbf{问题:}有阻值分别为R1和R2的两个负载,其中R1的电阻值是R2的N倍,把它们并联后接到电源上,则:(”x^m”表示“x的m次方”)
\begin{enumerate}[label=\Alph*), leftmargin=3em]
\item R1两端的电压是R2的N倍,R1消耗的功率是R2的N^2倍
\item R1两端的电压与R2的相同,R1消耗的功率是R2的N^2倍
\item R1两端的电压是R2的1/N,R1消耗的功率是R2的1/N^2
\item R1两端的电压与R2的相同,R1消耗的功率是R2的1/N
\end{enumerate}
\noindent\textbf{解说:}\textbf{R1两端的电压与R2的相同,R1消耗的功率是R2的1/N}。\\\noindent\textbf{答案:}D




\bigskip


\noindent\textbf{问题:}有阻值分别为R1和R2的两个负载,其中R1的电阻值是R2的N倍,把它们串联后接到电源上,则:(”x^m”表示“x的m次方”)
\begin{enumerate}[label=\Alph*), leftmargin=3em]
\item 流过R1的电流与R2的相同,R1消耗的功率是R2的N倍
\item 流过R1的电流是R2的1/N,R1消耗的功率是R2的1/N^2
\item 流过R1的电流与R2的相同,R1消耗的功率是R2的1/N
\item 流过R1的电流是R2的N倍,R1消耗的功率是R2的N^2倍
\end{enumerate}
\noindent\textbf{解说:}\textbf{流过R1的电流与R2的相同,R1消耗的功率是R2的N倍}。\\\noindent\textbf{答案:}A



\bigskip


\noindent\textbf{问题:}有阻值分别为R1和R2的两个负载,其中R1的电阻值是R2的N倍,把它们串联后接到电源上,则:(”x^m”表示“x的m次方”)
\begin{enumerate}[label=\Alph*), leftmargin=3em]
\item R1两端的电压是R2的N倍,R1消耗的功率是R2的N倍
\item R1两端的电压与R2的相同,R1消耗的功率是R2的N^2倍
\item R1两端的电压是R2的1/N,R1消耗的功率是R2的1/N^2
\item R1两端的电压与R2的相同,R1消耗的功率是R2的1/N
\end{enumerate}
\noindent\textbf{解说:}\textbf{R1两端的电压是R2的N倍,R1消耗的功率是R2的N倍}。\\\noindent\textbf{答案:}A



\bigskip


\noindent\textbf{问题:}已知A、B两个设备的工作电压相同,A的耗电功率是B的N倍。则:(”x^m”表示“x的m次方”)
\begin{enumerate}[label=\Alph*), leftmargin=3em]
\item A的工作电流是B的N倍
\item A的工作电流是B的N^(1/2)倍
\item A的工作电流是B的1/N倍
\item A的工作电流是B的N^2倍
\end{enumerate}
\noindent\textbf{解说:}\textbf{A的工作电流是B的N倍}。\\\noindent\textbf{答案:}A



\bigskip


\noindent\textbf{问题:}已知A、B两个设备的工作电压相同,A的额定电流是B的N倍。则:(”x^m”表示“x的m次方”)
\begin{enumerate}[label=\Alph*), leftmargin=3em]
\item A的耗电功率是B的N^(1/2)倍
\item A的耗电功率是B的N倍
\item A的耗电功率是B的N^2倍
\item A的耗电功率是B的1/N倍
\end{enumerate}
\noindent\textbf{解说:}\textbf{A的耗电功率是B的N倍}。\\\noindent\textbf{答案:}B


\bigskip


\noindent\textbf{问题:}将N个相同的电阻负载串联后接到电源上。与每个负载单独接到电源相比:(”x^m”表示“x的m次方”)
\begin{enumerate}[label=\Alph*), leftmargin=3em]
\item 串联后流过每个电阻的电流减少到1/N,每个电阻的耗电功率减少到1/N^2
\item 串联后流过每个电阻的电流减少到1/N,每个电阻的耗电功率减少到1/N
\item 串联后流过每个电阻的电流不变,每个电阻的耗电功率减少到1/N
\item 串联后流过每个电阻的电流增加到N倍,每个电阻的耗电功率增加到N^2倍
\end{enumerate}
\noindent\textbf{解说:}与每个负载单独接到电源相比:\textbf{串联后流过每个电阻的电流减少到1/N,每个电阻的耗电功率减少到1/N^2}。\\\noindent\textbf{答案:}A




\bigskip


\noindent\textbf{问题:}将N个相同的电阻负载串联后接到电源上。与每个负载单独接到电源相比:(”x^m”表示“x的m次方”)
\begin{enumerate}[label=\Alph*), leftmargin=3em]
\item 串联后每个电阻两端的电压减少到1/N,每个电阻的耗电功率减少到1/ N^2
\item 串联后每个电阻两端的电压不变,每个电阻的耗电功率减少到1/N
\item 串联后每个电阻两端的电压增加到N倍,每个电阻的耗电功率增加到N^2倍
\item 串联后每个电阻两端的电压减少到1/N,每个电阻的耗电功率减少到1/N
\end{enumerate}
\noindent\textbf{解说:}与每个负载单独接到电源相比:\textbf{串联后每个电阻两端的电压减少到1/N,每个电阻的耗电功率减少到1/ N^2}。\\\noindent\textbf{答案:}A




\bigskip


\noindent\textbf{问题:}将N个相同的电阻负载并联后接到电源上。与每个负载单独接到电源相比:(”x^m”表示“x的m次方”)
\begin{enumerate}[label=\Alph*), leftmargin=3em]
\item 并联后流过每个电阻的电流增加到N倍,两个电阻的总耗电功率增加到N^2倍
\item 并联后流过每个电阻的电流减少到1/N,两个电阻的总耗电功率减少到1/N^2
\item 并联后流过每个电阻的电流不变,所有电阻的总耗电功率增加到N^2倍
\item 并联后流过每个电阻的电流不变,所有电阻的总耗电功率增加到N倍
\end{enumerate}
\noindent\textbf{解说:}与每个负载单独接到电源相比:\textbf{并联后流过每个电阻的电流不变,所有电阻的总耗电功率增加到N倍}。\\\noindent\textbf{答案:}D



\bigskip


\noindent\textbf{问题:}将N个相同的电阻负载并联后接到电源上。与每个负载单独接到电源相比:(”x^m”表示“x的m次方”)
\begin{enumerate}[label=\Alph*), leftmargin=3em]
\item 并联后每个电阻两端的电压不变,所有电阻的总耗电功率增加到N^2倍
\item 并联后每个电阻两端的电压增加到N倍,两个电阻的总耗电功率增加到N^2倍
\item 并联后每个电阻两端的电压不变,所有电阻的总耗电功率增加到N倍
\item 并联后每个电阻两端的电压减少到1/N,两个电阻的总耗电功率减少到1/N^2
\end{enumerate}
\noindent\textbf{解说:}与每个负载单独接到电源相比:\textbf{并联后每个电阻两端的电压不变,所有电阻的总耗电功率增加到N倍}。\\\noindent\textbf{答案:}C



\bigskip


\noindent\textbf{问题:}一个电阻负载,如果将其两端的工作电压提高百分之N,则:(”x^m”表示“x的m次方”)
\begin{enumerate}[label=\Alph*), leftmargin=3em]
\item 耗电量增加到原来的[百分之(100+N)]^2
\item 耗电量比原来的增加百分之N
\item 耗电量增加到原来的百分之(100+N)
\item 耗电量比原来的增加[百分之N]^2
\end{enumerate}
\noindent\textbf{解说:}\textbf{耗电量增加到原来的[百分之(100+N)]^2}。\\\noindent\textbf{答案:}A



\bigskip


\noindent\textbf{问题:}一个电阻负载,如果将其两端的工作电压降低百分之N,则:(”x^m”表示“x的m次方”)
\begin{enumerate}[label=\Alph*), leftmargin=3em]
\item 耗电量比原来的减少[百分之N]^2
\item 耗电量比原来的减少百分之N
\item 耗电量减少到原来的[百分之(100-N)]^2
\item 耗电量减少到原来的百分之(100-N)
\end{enumerate}
\noindent\textbf{解说:}\textbf{耗电量减少到原来的[百分之(100-N)]^2}。\\\noindent\textbf{答案:}C




\bigskip


\noindent\textbf{问题:}分别用电压为120V的蓄电池组和电压最大值为120V的交流变压器驱动同样的电阻负载,哪一个电阻每分钟发出的热量多?
\begin{enumerate}[label=\Alph*), leftmargin=3em]
\item 蓄电池驱动的电阻所发的热量是交流变压器上的电阻的1.4倍左右
\item 两个电源驱动的电阻发热相同
\item 蓄电池驱动的电阻所发的热量是交流变压器上的电阻的0.7倍左右
\item 蓄电池驱动的电阻所发的热量是交流变压器上的电阻的2倍左右
\end{enumerate}
\noindent\textbf{解说:}\textbf{蓄电池驱动的电阻所发的热量是交流变压器上的电阻的2倍左右}。\\\noindent\textbf{答案:}D



\bigskip


\noindent\textbf{问题:}分别用电压为120V的蓄电池组和电压有效值为120V的交流变压器驱动同样的电阻负载,哪一个电阻每分钟发出的热量多?
\begin{enumerate}[label=\Alph*), leftmargin=3em]
\item 蓄电池驱动的电阻所发的热量是交流变压器上的电阻的2倍左右
\item 蓄电池驱动的电阻所发的热量是交流变压器上的电阻的1.4倍左右
\item 蓄电池驱动的电阻所发的热量是交流变压器上的电阻的0.7倍左右
\item 两个电源驱动的电阻发热相同
\end{enumerate}
\noindent\textbf{解说:}\textbf{两个电源驱动的电阻发热相同}。\\\noindent\textbf{答案:}D



\bigskip


\noindent\textbf{问题:}分别用电压为120V的蓄电池组和电压有效值为120V的交流变压器串联二极管后驱动同样的电阻负载,哪一个电阻每分钟发出的热量多?(忽略二极管的电压降)
\begin{enumerate}[label=\Alph*), leftmargin=3em]
\item 蓄电池驱动的电阻所发的热量是交流变压器上的电阻的2倍左右
\item 蓄电池驱动的电阻所发的热量是交流变压器上的电阻的1.4倍左右
\item 蓄电池驱动的电阻所发的热量是交流变压器上的电阻的0.7倍左右
\item 两个电源驱动的电阻发热相同
\end{enumerate}
\noindent\textbf{解说:}\textbf{蓄电池驱动的电阻所发的热量是交流变压器上的电阻的2倍左右}。\\\noindent\textbf{答案:}A



\bigskip


\noindent\textbf{问题:}分别用电压为120V的蓄电池组和电压最大值为120V的交流变压器经过带电容滤波的全波整流电路驱动同样的电阻负载,哪一个电阻每分钟发出的热量多?(忽略整流器的电压降)
\begin{enumerate}[label=\Alph*), leftmargin=3em]
\item 蓄电池驱动的电阻所发的热量是交流变压器上的电阻的1.4倍左右
\item 蓄电池驱动的电阻所发的热量是交流变压器上的电阻的2倍左右
\item 蓄电池驱动的电阻所发的热量是交流变压器上的电阻的0.7倍左右
\item 两个电源驱动的电阻发热相同
\end{enumerate}
\noindent\textbf{解说:}\textbf{两个电源驱动的电阻发热相同}。\\\noindent\textbf{答案:}D



\bigskip


\noindent\textbf{问题:}分别用电压有效值为120V、频率为50Hz的交流电源和电压有效值为120V、频率为10kHz的方波电源驱动同样的电阻负载,哪一个电阻每分钟发出的热量多?
\begin{enumerate}[label=\Alph*), leftmargin=3em]
\item 10kHz电路电阻所发的热量是50Hz电路电阻的1/5倍左右
\item 10kHz电路电阻所发的热量是50Hz电路电阻的200倍左右
\item 两个电源驱动的电阻发热相同
\item 10kHz电路电阻所发的热量是50Hz电路电阻的5倍左右
\end{enumerate}
\noindent\textbf{解说:}\textbf{两个电源驱动的电阻发热相同}。\\\noindent\textbf{答案:}C



\bigskip


\noindent\textbf{问题:}在业余收发信机的常见元器件中,标有耐压指标的是:
\begin{enumerate}[label=\Alph*), leftmargin=3em]
\item 熔丝
\item 电容
\item 电阻
\item 电感
\end{enumerate}
\noindent\textbf{解说:}在业余收发信机的常见元器件中,标有耐压指标的是\textbf{电容}。\\\noindent\textbf{答案:}B

\bigskip


\noindent\textbf{问题:}在业余收发信机的常见元器件中,以额定耗散功率指标分类的是:
\begin{enumerate}[label=\Alph*), leftmargin=3em]
\item 电容
\item 电感
\item 电阻
\item 熔丝
\end{enumerate}
\noindent\textbf{解说:}在业余收发信机的常见元器件中,以额定耗散功率指标分类的是\textbf{电阻}。\\\noindent\textbf{答案:}C

\bigskip


\noindent\textbf{问题:}在业余收发信机的常见元器件中,标有额定电流指标的是:
\begin{enumerate}[label=\Alph*), leftmargin=3em]
\item 熔丝
\item 电阻
\item 电感
\item 电容
\end{enumerate}
\noindent\textbf{解说:}在业余收发信机的常见元器件中,标有额定电流指标的是\textbf{熔丝}。\\\noindent\textbf{答案:}A


\bigskip


\noindent\textbf{问题:}在业余收发信机电路中,经常用于隔直流或者给交流信号提供旁路的元件是:
\begin{enumerate}[label=\Alph*), leftmargin=3em]
\item 电感
\item 半导体开关器件
\item 电阻
\item 电容
\end{enumerate}
\noindent\textbf{解说:}在业余收发信机电路中,经常用于隔直流或者给交流信号提供旁路的元件是\textbf{电容}。\\\noindent\textbf{答案:}D

\bigskip


\noindent\textbf{问题:}在业余收发信机电路中,经常用谐振回路来筛选一定频率的信号。组成谐振回路的主要元器件是:
\begin{enumerate}[label=\Alph*), leftmargin=3em]
\item 半导体三极管和电阻的组合
\item 电阻和电容的组合
\item 电容和电感的组合
\item 电感和电阻的组合
\end{enumerate}
\noindent\textbf{解说:}在业余收发信机电路中,经常用谐振回路来筛选一定频率的信号。组成谐振回路的主要元器件是\textbf{电容和电感的组合}。\\\noindent\textbf{答案:}C



\bigskip


\noindent\textbf{问题:}在电容器两端施加一定幅度的正弦交流电压。流过电容器的电流幅度:
\begin{enumerate}[label=\Alph*), leftmargin=3em]
\item 与电压和电容量都成反比
\item 与电容量成正比,与电压成反比
\item 与电压和电容量都成正比
\item 与电压成正比,与电容量成反比
\end{enumerate}
\noindent\textbf{解说:}在电容器两端施加一定幅度的正弦交流电压。流过电容器的电流幅度\textbf{与电压和电容量都成正比}。\\\noindent\textbf{答案:}C


\bigskip


\noindent\textbf{问题:}在线圈两端施加一定幅度的正弦交流电压。流过线圈的电流幅度:
\begin{enumerate}[label=\Alph*), leftmargin=3em]
\item 与电感量成正比,与电压成反比
\item 与电压和电感量都成反比
\item 与电压和电感量都成正比
\item 与电压成正比,与电感量成反比
\end{enumerate}
\noindent\textbf{解说:}在线圈两端施加一定幅度的正弦交流电压。流过线圈的电流幅度\textbf{与电压成正比,与电感量成反比}。\\\noindent\textbf{答案:}D

\bigskip


\noindent\textbf{问题:}构成振荡器的必备元素是:
\begin{enumerate}[label=\Alph*), leftmargin=3em]
\item 任意放大器、LC或晶体谐振电路
\item LC或晶体谐振电路、正反馈电路
\item 放大倍数大于1的放大器、正反馈电路
\item 放大倍数大于1的放大器、负反馈电路
\end{enumerate}
\noindent\textbf{解说:}构成振荡器的必备元素是\textbf{放大倍数大于1的放大器、正反馈电路}。\\\noindent\textbf{答案:}C

\bigskip


\noindent\textbf{问题:}在直流电路中,用来阻碍电流流动的元件是?
\begin{enumerate}[label=\Alph*), leftmargin=3em]
\item 电感
\item 变压器
\item 电阻
\item 电压表
\end{enumerate}
\noindent\textbf{解说:}在直流电路中,用来阻碍电流流动的元件是\textbf{电阻}。\\\noindent\textbf{答案:}C

\bigskip


\noindent\textbf{问题:}下列哪一个元件经常用来实现音量调节的功能?
\begin{enumerate}[label=\Alph*), leftmargin=3em]
\item 变压器
\item 定值电阻
\item 功率电阻
\item 电位器
\end{enumerate}
\noindent\textbf{解说:}\textbf{电位器}经常用来实现音量调节的功能。\\\noindent\textbf{答案:}D


\bigskip


\noindent\textbf{问题:}电位器控制什么电学物理量?
\begin{enumerate}[label=\Alph*), leftmargin=3em]
\item 场强
\item 电容
\item 电阻
\item 电感
\end{enumerate}
\noindent\textbf{解说:}电位器控制\textbf{电阻}。\\\noindent\textbf{答案:}C

\bigskip


\noindent\textbf{问题:}哪一种电子元件由两个或多个使用绝缘材料分离开的片状导体组成?
\begin{enumerate}[label=\Alph*), leftmargin=3em]
\item 电容
\item 电阻
\item 电位器
\item 振荡器
\end{enumerate}
\noindent\textbf{解说:}\textbf{电容}由两个或多个使用绝缘材料分离开的片状导体组成。\\\noindent\textbf{答案:}A


\bigskip


\noindent\textbf{问题:}哪一种电子元件一般由线圈组成?
\begin{enumerate}[label=\Alph*), leftmargin=3em]
\item 二极管
\item 电容
\item 电感
\item 开关
\end{enumerate}
\noindent\textbf{解说:}\textbf{电感}一般由线圈组成。\\\noindent\textbf{答案:}C



\bigskip


\noindent\textbf{问题:}哪一种电子元件用来接通或切断电路?
\begin{enumerate}[label=\Alph*), leftmargin=3em]
\item 开关
\item 电感
\item 可变电阻
\item 齐纳二极管
\end{enumerate}
\noindent\textbf{解说:}\textbf{开关}用来接通或切断电路。\\\noindent\textbf{答案:}A


\bigskip


\noindent\textbf{问题:}一个充满电的镍镉电池的标称电压是多少?
\begin{enumerate}[label=\Alph*), leftmargin=3em]
\item 2.2伏
\item 1.5伏
\item 1.0伏
\item 1.2伏
\end{enumerate}
\noindent\textbf{解说:}一个充满电的镍镉电池的标称电压是\textbf{1.2伏}。\\\noindent\textbf{答案:}D



\bigskip


\noindent\textbf{问题:}下列哪一种元器件可以用一个较小的电流来控制较大的电流?
\begin{enumerate}[label=\Alph*), leftmargin=3em]
\item 电感
\item 电阻
\item 晶体管
\item 电容
\end{enumerate}
\noindent\textbf{解说:}\textbf{晶体管}可以用一个较小的电流来控制较大的电流。\\\noindent\textbf{答案:}C



\bigskip


\noindent\textbf{问题:}下列哪一种元器件只允许单方向的电流流动?
\begin{enumerate}[label=\Alph*), leftmargin=3em]
\item 电阻
\item 稳压元件
\item 熔断器
\item 二极管
\end{enumerate}
\noindent\textbf{解说:}\textbf{二极管}只允许单方向的电流流动。\\\noindent\textbf{答案:}D




\bigskip


\noindent\textbf{问题:}下列哪一种元器件既可以当作电子开关又可以当作放大器使用?
\begin{enumerate}[label=\Alph*), leftmargin=3em]
\item 电压表
\item 电位器
\item 单刀双掷开关
\item 晶体管
\end{enumerate}
\noindent\textbf{解说:}\textbf{晶体管}既可以当作电子开关又可以当作放大器使用。\\\noindent\textbf{答案:}D



\bigskip


\noindent\textbf{问题:}下列哪一种元器件可以放大信号?
\begin{enumerate}[label=\Alph*), leftmargin=3em]
\item 多芯电池
\item 可变电阻
\item 晶体管
\item 电解电容
\end{enumerate}
\noindent\textbf{解说:}\textbf{晶体管}可以放大信号。\\\noindent\textbf{答案:}C


\bigskip


\noindent\textbf{问题:}继电器的功能可以描述为:
\begin{enumerate}[label=\Alph*), leftmargin=3em]
\item 由电流控制的放大器
\item 一个光学传感器
\item 由电磁铁控制的开关
\item 无线电转发设备
\end{enumerate}
\noindent\textbf{解说:}继电器的功能为\textbf{由电磁铁控制的开关}。\\\noindent\textbf{答案:}C


\bigskip


\noindent\textbf{问题:}下列哪一项和电感一起使用,可以制作一个谐振电路?
\begin{enumerate}[label=\Alph*), leftmargin=3em]
\item 电位器
\item 齐纳二极管
\item 电阻
\item 电容
\end{enumerate}
\noindent\textbf{解说:}\textbf{电容}和电感一起使用,可以制作一个谐振电路。\\\noindent\textbf{答案:}D



\bigskip


\noindent\textbf{问题:}集成电路是指:
\begin{enumerate}[label=\Alph*), leftmargin=3em]
\item 将一个电路的大量元器件集合于一个单晶片上所制成的器件
\item 多极继电器
\item 把多个电阻或电容元件堆积在一起
\item 变压器
\end{enumerate}
\noindent\textbf{解说:}集成电路是\textbf{将一个电路的大量元器件集合于一个单晶片上所制成的器件}。\\\noindent\textbf{答案:}A



\bigskip


\noindent\textbf{问题:}附图中的电路元器件符号代表的是:

\begin{circuitikz}[]
	\ctikzset{grounds/scale=3, ground=european}
	\draw (0,0) node[rground]{};
\end{circuitikz}

\begin{enumerate}[label=\Alph*), leftmargin=3em]
\item 二极管
\item 接地
\item 电阻
\item 天线
\end{enumerate}%LK0497.jpg
\noindent\textbf{解说:}附图中的电路元器件符号代表的是\textbf{接地}。\\\noindent\textbf{答案:}B






\bigskip


\noindent\textbf{问题:}附图中的电路元器件符号代表的是:

\begin{circuitikz}[european]
	\draw (0,-2) -- (0,.75);
	\draw (-1,.75) -- (0,-.5) -- (1,.75);
\end{circuitikz}


\begin{enumerate}[label=\Alph*), leftmargin=3em]
\item 接地
\item 二极管
\item 天线
\item 电阻
\end{enumerate}%LK0498.jpg
\noindent\textbf{解说:}附图中的电路元器件符号代表的是\textbf{接地}。\\\noindent\textbf{答案:}C


\bigskip


\noindent\textbf{问题:}附图中的电路元器件符号代表的是:

\begin{circuitikz}[]
	\ctikzset{misc/scale=1.5}
	\draw (0,0) to [fuse]     (2,0) node [right] {};
\end{circuitikz}


\begin{enumerate}[label=\Alph*), leftmargin=3em]
\item 二极管
\item 电阻
\item 电容
\item 熔断器
\end{enumerate}%LK0499.jpg
\noindent\textbf{解说:}附图中的电路元器件符号代表的是\textbf{熔断器}。\\\noindent\textbf{答案:}D


\bigskip


\noindent\textbf{问题:}附图中的电路元器件符号代表的是:

\begin{circuitikz}[european]
	\ctikzset{capacitors/scale=1.5, ccapacitorshape=european}
	\draw (0,0) to [C](0,1.5);
\end{circuitikz}

\begin{enumerate}[label=\Alph*), leftmargin=3em]
\item 二极管
\item 熔断器
\item 电容器
\item 电阻
\end{enumerate}%LK0500.jpg
\noindent\textbf{解说:}附图中的电路元器件符号代表的是\textbf{电容器}。\\\noindent\textbf{答案:}C


\bigskip


\noindent\textbf{问题:}附图中的电路元器件符号代表的是:

\begin{circuitikz}[]
	\ctikzset{resistors/scale=1, resistor=european}
	\draw (0,0) to [R](2,0);
\end{circuitikz}

\begin{enumerate}[label=\Alph*), leftmargin=3em]
\item 电阻
\item 电容器
\item 压电晶体
\item 熔断器
\end{enumerate}%LK0501.jpg
\noindent\textbf{解说:}附图中的电路元器件符号代表的是\textbf{电阻}。\\\noindent\textbf{答案:}A


\bigskip


\noindent\textbf{问题:}附图中的电路元器件符号代表的是:

\begin{circuitikz}[]
	\ctikzset{diodes/scale=1.5, emptydiodeshape=europe}
	\draw (0,1.5) to [D-](0,0);
\end{circuitikz}

\begin{enumerate}[label=\Alph*), leftmargin=3em]
\item 线圈
\item 电容器
\item 电阻
\item 二极管
\end{enumerate}%LK0502.jpg
\noindent\textbf{解说:}附图中的电路元器件符号代表的是\textbf{二极管}。\\\noindent\textbf{答案:}D


\bigskip


\noindent\textbf{问题:}附图中的电路元器件符号代表的是:

\begin{circuitikz}[]
	\ctikzset{inductors/scale=1.5, inductor=american}
	\draw (0,0) to[L] ++(1.7,0);
\end{circuitikz}

\begin{enumerate}[label=\Alph*), leftmargin=3em]
\item 线圈
\item 电阻
\item 二极管
\item 电容器
\end{enumerate}%LK0503.jpg
\noindent\textbf{解说:}附图中的电路元器件符号代表的是\textbf{线圈}。\\\noindent\textbf{答案:}A



\bigskip


\noindent\textbf{问题:}附图中的电路元器件符号代表的是:

\begin{circuitikz}[]
	\ctikzset{batteries/scale=1.5, battery2shape=europe}
	\draw (1.5,0) to [battery2] (0,0);
\end{circuitikz}

\begin{enumerate}[label=\Alph*), leftmargin=3em]
\item 线圈
\item 二极管
\item 电池
\item 电阻
\end{enumerate}%LK0504.jpg
\noindent\textbf{解说:}附图中的电路元器件符号代表的是\textbf{电池}。\\\noindent\textbf{答案:}C


\bigskip


\noindent\textbf{问题:}附图中的电路元器件符号代表的是:

\begin{circuitikz}[]
	\ctikzset{capacitors/scale=1.5, piezoelectricshape=europe}
	\draw (0,0)  to [PZ] (0,1.5);
\end{circuitikz}


\begin{enumerate}[label=\Alph*), leftmargin=3em]
\item 电池
\item 压电晶体
\item 电阻
\item 二极管
\end{enumerate}%LK0505.jpg
\noindent\textbf{解说:}附图中的电路元器件符号代表的是\textbf{压电晶体}。\\\noindent\textbf{答案:}B


\bigskip


\noindent\textbf{问题:}附图中的电路元器件符号代表的是:

\begin{circuitikz}[]
	\ctikzset{diodes/scale=1.5, emptyzdiodeshape=europe}
	\draw (0,1.5)  to [zD-, mirror] (0,0);
\end{circuitikz}

\begin{enumerate}[label=\Alph*), leftmargin=3em]
\item 压电晶体
\item 电阻
\item 发光二极管
\item 稳压二极管
\end{enumerate}%LK0506.jpg
\noindent\textbf{解说:}附图中的电路元器件符号代表的是\textbf{稳压二极管}。\\\noindent\textbf{答案:}D



\bigskip


\noindent\textbf{问题:}附图中的电路元器件符号代表的是:

\begin{circuitikz}[]
	\ctikzset{diodes/scale=1.5, emptylediodeshape=europe}
	\draw (0,1.5)  to [leD-, mirror] (0,0);
\end{circuitikz}
%%% 光的方向不对

\begin{enumerate}[label=\Alph*), leftmargin=3em]
\item 发光二极管
\item 电阻
\item 压电晶体
\item 稳压二极管
\end{enumerate}%LK0507.jpg
\noindent\textbf{解说:}附图中的电路元器件符号代表的是\textbf{发光二极管}。\\\noindent\textbf{答案:}A
%%%???



\bigskip


\noindent\textbf{问题:}附图中的电路元器件符号代表的是:

\begin{circuitikz}[]
	\ctikzset{transistors/scale=1.5}
	\draw (0,0) node[pnp]{};
\end{circuitikz}

\begin{enumerate}[label=\Alph*), leftmargin=3em]
\item PNP双极型半导体三极管
\item NPN双极型半导体三极管
\item 绝缘栅场效应半导体三极管
\item 结型场效应半导体三极管
\end{enumerate}%LK0508.jpg
\noindent\textbf{解说:}附图中的电路元器件符号代表的是\textbf{PNP双极型半导体三极管}。\\\noindent\textbf{答案:}A


\bigskip


\noindent\textbf{问题:}附图中的电路元器件符号代表的是:
	
\begin{circuitikz}[]
	\ctikzset{transistors/scale=1.5}
	\draw (0,0) node[npn]{};
\end{circuitikz}

\begin{enumerate}[label=\Alph*), leftmargin=3em]
\item NPN双极型半导体三极管
\item 结型场效应半导体三极管
\item 绝缘栅场效应半导体三极管
\item PNP双极型半导体三极管
\end{enumerate}%LK0509.jpg
\noindent\textbf{解说:}附图中的电路元器件符号代表的是\textbf{NPN双极型半导体三极管}。\\\noindent\textbf{答案:}A


\bigskip


\noindent\textbf{问题:}附图中的电路元器件符号代表的是:

\begin{circuitikz}[]
	\ctikzset{transistors/scale=1.5}
	\draw (0,0) node[njfet]{};
\end{circuitikz}

\begin{enumerate}[label=\Alph*), leftmargin=3em]
\item 结型场效应半导体三极管
\item 绝缘栅场效应半导体三极管
\item NPN双极型半导体三极管
\item PNP双极型半导体三极管
\end{enumerate}%LK0510.jpg
\noindent\textbf{解说:}附图中的电路元器件符号代表的是\textbf{结型场效应半导体三极管}。\\\noindent\textbf{答案:}A


\bigskip


\noindent\textbf{问题:}附图中的电路元器件符号代表的是:

\begin{circuitikz}[]
	\ctikzset{transistors/scale=1.5}
	\draw (0,0) node[pigfetebulk]{};
\end{circuitikz}

\begin{enumerate}[label=\Alph*), leftmargin=3em]
\item 结型场效应半导体三极管
\item 绝缘栅场效应半导体三极管
\item NPN双极型半导体三极管
\item PNP双极型半导体三极管
\end{enumerate}%LK0511.jpg
\noindent\textbf{解说:}附图中的电路元器件符号代表的是\textbf{绝缘栅场效应半导体三极管}。\\\noindent\textbf{答案:}B



\bigskip


\noindent\textbf{问题:}将电阻R和电容C串联后突然接到直流电压U上,电容C两端的电压会:
\begin{enumerate}[label=\Alph*), leftmargin=3em]
\item 从U按指数规律逐渐减小到0
\item 从0按指数规律逐渐增加到U
\item 从0突然跳到U,然后再按指数规律逐渐减小到0
\item 从U突然跳到0,然后再按指数规律逐渐增大到U
\end{enumerate}
\noindent\textbf{解说:}将电阻R和电容C串联后突然接到直流电压U上,电容C两端的电压会\textbf{从0按指数规律逐渐增加到U}。\\\noindent\textbf{答案:}B


\bigskip


\noindent\textbf{问题:}将电阻R和电容C串联后突然接到直流电压U上,电阻R两端的电压会:
\begin{enumerate}[label=\Alph*), leftmargin=3em]
\item 从U突然跳到0,然后再按直线规律逐渐减小到U
\item 从U按直线规律逐渐减小到0
\item 从0按指数规律逐渐增加到U
\item 从0突然跳到U,然后再按指数规律逐渐减小到0
\end{enumerate}
\noindent\textbf{解说:}将电阻R和电容C串联后突然接到直流电压U上,电阻R两端的电压会\textbf{从0突然跳到U,然后再按指数规律逐渐减小到0}。\\\noindent\textbf{答案:}D



\bigskip


\noindent\textbf{问题:}将电阻R和电容C串联后突然接到直流电压U上,流过电阻R的电流会:
\begin{enumerate}[label=\Alph*), leftmargin=3em]
\item 从0按指数规律逐渐增加到U/R
\item 从U/R突然跳到0并保持
\item 从0突然跳到U/R并保持
\item 从0突然跳到U/R,然后再按指数规律逐渐减小到0
\end{enumerate}
\noindent\textbf{解说:}将电阻R和电容C串联后突然接到直流电压U上,流过电阻R的电流会\textbf{从0突然跳到U/R,然后再按指数规律逐渐减小到0}。\\\noindent\textbf{答案:}D


\bigskip


\noindent\textbf{问题:}将电阻R和电容C串联后突然接到直流电压U上,流过电容C的电流会:
\begin{enumerate}[label=\Alph*), leftmargin=3em]
\item 从U/R突然跳到0并保持
\item 从0突然跳到U/R,然后再按指数规律逐渐减小到0
\item 从0突然跳到U/R并保持
\item 从0按指数规律逐渐增加到U/R
\end{enumerate}
\noindent\textbf{解说:}将电阻R和电容C串联后突然接到直流电压U上,流过电容C的电流会\textbf{从0突然跳到U/R,然后再按指数规律逐渐减小到0}。\\\noindent\textbf{答案:}B



\bigskip


\noindent\textbf{问题:}电阻R和电容C并联后接在电压为U的直流电源上。突然断开电源,电容C两端的电压会:
\begin{enumerate}[label=\Alph*), leftmargin=3em]
\item 从0按指数规律逐渐增加到U
\item 从U按指数规律逐渐减小到0
\item 从0突然跳到U并保持
\item 从U突然跳到0并保持
\end{enumerate}
\noindent\textbf{解说:}电阻R和电容C并联后接在电压为U的直流电源上。突然断开电源,电容C两端的电压会\textbf{从U按指数规律逐渐减小到0}。\\\noindent\textbf{答案:}B



\bigskip


\noindent\textbf{问题:}电阻R和电容C并联后接在电压为U的直流电源上。突然断开电源,电阻R两端的电压会:
\begin{enumerate}[label=\Alph*), leftmargin=3em]
\item 从U按指数规律逐渐减小到0
\item 从0按指数规律逐渐增加到U
\item 从U突然跳到0并保持
\item 从0突然跳到U并保持
\end{enumerate}
\noindent\textbf{解说:}电阻R和电容C并联后接在电压为U的直流电源上。突然断开电源,电阻R两端的电压会\textbf{从U按指数规律逐渐减小到0}。\\\noindent\textbf{答案:}A


\bigskip


\noindent\textbf{问题:}电阻R和电容C并联后接在电压为U的直流电源上。突然断开电源,流过电阻R的电流会:
\begin{enumerate}[label=\Alph*), leftmargin=3em]
\item 从0突然跳到U/R,然后再按指数规律逐渐减小到0
\item 从U/R按指数规律逐渐减小到0
\item 从U突然跳到0,然后再按指数规律逐渐增大到U/R
\item 从0按指数规律逐渐增加到U/R
\end{enumerate}
\noindent\textbf{解说:}电阻R和电容C并联后接在电压为U的直流电源上。突然断开电源,流过电阻R的电流会\textbf{从U/R按指数规律逐渐减小到0}。\\\noindent\textbf{答案:}B




\bigskip


\noindent\textbf{问题:}电阻R和电容C并联后接在电压为U的直流电源上。突然断开电源,流过电容C的电流会:
\begin{enumerate}[label=\Alph*), leftmargin=3em]
\item 从U突然跳到0,然后再按指数规律逐渐增大到U/R
\item 从0突然跳到U/R,然后再按指数规律逐渐减小到0
\item 从0突然跳到U/R并保持
\item 从0按指数规律逐渐增加到U/R
\end{enumerate}
\noindent\textbf{解说:}电阻R和电容C并联后接在电压为U的直流电源上。突然断开电源,流过电容C的电流会\textbf{从0突然跳到U/R,然后再按指数规律逐渐减小到0}。\\\noindent\textbf{答案:}B



\bigskip


\noindent\textbf{问题:}将电阻R和电感L串联后突然接到直流电压U上,电感L两端的电压会:
\begin{enumerate}[label=\Alph*), leftmargin=3em]
\item 从U突然跳到0并保持
\item 从0突然跳到U,然后再按指数规律逐渐减小到0
\item 从0按指数规律逐渐增加到U
\item 从U按指数规律逐渐减小到0
\end{enumerate}
\noindent\textbf{解说:}将电阻R和电感L串联后突然接到直流电压U上,电感L两端的电压会\textbf{从0突然跳到U,然后再按指数规律逐渐减小到0}。\\\noindent\textbf{答案:}B



\bigskip


\noindent\textbf{问题:}将电阻R和电感L串联后突然接到直流电压U上,电阻R两端的电压会:
\begin{enumerate}[label=\Alph*), leftmargin=3em]
\item 从U按指数规律逐渐减小到0
\item 从0按指数规律逐渐增加到U
\item 从0突然跳到U,然后再按指数规律逐渐减小到0
\item 从U突然跳到0并保持
\end{enumerate}
\noindent\textbf{解说:}将电阻R和电感L串联后突然接到直流电压U上,电阻R两端的电压会\textbf{从0按指数规律逐渐增加到U}。\\\noindent\textbf{答案:}B




\bigskip


\noindent\textbf{问题:}将电阻R和电感L串联后突然接到直流电压U上,流过电阻R的电流会:
\begin{enumerate}[label=\Alph*), leftmargin=3em]
\item 从0按指数规律逐渐增加到U/R
\item 从U/R按指数规律逐渐减小到0
\item 从0突然跳到U/R,然后再按指数规律逐渐减小到0
\item 从0突然跳到U/R并保持U/R
\end{enumerate}
\noindent\textbf{解说:}将电阻R和电感L串联后突然接到直流电压U上,流过电阻R的电流会\textbf{从0按指数规律逐渐增加到U/R}。\\\noindent\textbf{答案:}A




\bigskip


\noindent\textbf{问题:}将电阻R和电感L串联后突然接到直流电压U上,流过电感L的电流会:
\begin{enumerate}[label=\Alph*), leftmargin=3em]
\item 从0突然跳到U/R,然后再按指数规律逐渐减小到0
\item 从U/R按指数规律逐渐减小到0
\item 从0突然跳到U/R并保持
\item 从0按指数规律逐渐增加到U/R
\end{enumerate}
\noindent\textbf{解说:}将电阻R和电感L串联后突然接到直流电压U上,流过电感L的电流会\textbf{从0按指数规律逐渐增加到U/R}。\\\noindent\textbf{答案:}D




\bigskip


\noindent\textbf{问题:}电阻R和电感L并联后接在电流为I的直流电路中。突然断开电路,电感L两端的电压会:
\begin{enumerate}[label=\Alph*), leftmargin=3em]
\item 保持在0
\item 保持在I*R
\item 从I*R按指数规律逐渐减小到0
\item 从0按指数规律逐渐增加到I*R
\end{enumerate}
\noindent\textbf{解说:}电阻R和电感L并联后接在电流为I的直流电路中。突然断开电路,电感L两端的电压会\textbf{从I*R按指数规律逐渐减小到0}。\\\noindent\textbf{答案:}C




\bigskip


\noindent\textbf{问题:}用一个电压为4.2伏的低电压电池和一堆无源电子元件做电路实验,但连接电路时感觉手不慎被高电压击了一下。可能产生这个高电压的元件是:
\begin{enumerate}[label=\Alph*), leftmargin=3em]
\item 大电流高反压二极管
\item 电源变压器的绕组
\item 额定功率为50瓦的大电阻
\item 电解电容器
\end{enumerate}
\noindent\textbf{解说:}可能产生这个高电压的元件是\textbf{电源变压器的绕组}。\\\noindent\textbf{答案:}B

\bigskip


\noindent\textbf{问题:}用SSB接收机的天线引线靠近一个晶体管LC振荡器电路板,接收其信号。振荡器电路接通电源后,发现收到的信号音调会从低到高或者从高到低变化。这主要因为:
\begin{enumerate}[label=\Alph*), leftmargin=3em]
\item 随着射频能量泄露,电路的输出功率下降
\item 元器件通电发热,引起相关LC参数变化,造成谐振频率漂移
\item 接收机的声音能量反馈到电路板引起
\item 半导体晶体管处于老化过程
\end{enumerate}
\noindent\textbf{解说:}这主要因为\textbf{元器件通电发热,引起相关LC参数变化,造成谐振频率漂移}。\\\noindent\textbf{答案:}B



\bigskip


\noindent\textbf{问题:}在无线电电路中常用于产生基准频率的元件中,按频率稳定度由低到高的排列为:
\begin{enumerate}[label=\Alph*), leftmargin=3em]
\item LC回路,RC定时电路,陶瓷谐振器,石英声表面波元件,石英晶体谐振器
\item RC定时电路,LC回路,陶瓷谐振器,石英声表面波元件,石英晶体谐振器
\item RC定时电路,陶瓷谐振器,LC回路,石英晶体谐振器,石英声表面波元件
\item RC定时电路,LC回路,石英声表面波元件,石英晶体谐振器,陶瓷谐振器
\end{enumerate}
\noindent\textbf{解说:}在无线电电路中常用于产生基准频率的元件中,按频率稳定度由低到高的排列为:\textbf{RC定时电路,LC回路,陶瓷谐振器,石英声表面波元件,石英晶体谐振器}。\\\noindent\textbf{答案:}B



\bigskip


\noindent\textbf{问题:}在无线电技术中,通常把放大器分为A、B、C、D等类别,这种分类是依据:
\begin{enumerate}[label=\Alph*), leftmargin=3em]
\item 放大器件的工作点所处的范围
\item 放大器件的最高工作频率
\item 放大器件的最大输出功率
\item 放大器件的质量等级
\end{enumerate}
\noindent\textbf{解说:}在无线电技术中,通常把放大器分为A、B、C、D等类别,这种分类是依据:\textbf{放大器件的工作点所处的范围}。\\\noindent\textbf{答案:}A



\bigskip


\noindent\textbf{问题:}根据放大器的工作状态,通常把放大器分为A、B、C、D等类别。A类放大器是指:
\begin{enumerate}[label=\Alph*), leftmargin=3em]
\item 放大器件在半个信号周期内工作点处于线性区、另半个信号周期内处于截止区的放大器
\item 放大器件在整个信号周期内始终工作在线性区的放大器
\item 放大器件在半个信号周期内处于截止区,另半个周期处于饱和区的放大器
\item 放大器件在半个信号周期内处于截止区,另半个周期的部分时间候处于饱和区的放大器
\end{enumerate}
\noindent\textbf{解说:}A类放大器是指\textbf{放大器件在整个信号周期内始终工作在线性区的放大器}。\\\noindent\textbf{答案:}B


\bigskip


\noindent\textbf{问题:}根据放大器的工作状态,通常把放大器分为A、B、C、D等类别。B类放大器是指:
\begin{enumerate}[label=\Alph*), leftmargin=3em]
\item 放大器件在半个信号周期内处于截止区,另半个周期处于饱和区的放大器
\item 放大器件在整个信号周期内始终工作在线性区的放大器
\item 放大器件在半个信号周期内处于截止区,另半个周期的部分时间候处于饱和区的放大器
\item 放大器件在半个信号周期内工作点处于线性区、另半个信号周期内处于截止区的放大器
\end{enumerate}
\noindent\textbf{解说:}B类放大器是指\textbf{放大器件在半个信号周期内工作点处于线性区、另半个信号周期内处于截止区的放大器}。\\\noindent\textbf{答案:}D





\bigskip


\noindent\textbf{问题:}根据放大器的工作状态,通常把放大器分为A、B、C、D等类别。C类放大器是指:
\begin{enumerate}[label=\Alph*), leftmargin=3em]
\item 放大器件在整个信号周期内始终工作在线性区的放大器
\item 放大器件在多于半个信号周期的时间内处于截止区,另半个周期的部分时间候处于线性区的放大器
\item 放大器件在半个信号周期内处于截止区,另半个周期处于饱和区的放大器
\item 放大器件在半个信号周期内工作点处于线性区、另半个信号周期内处于截止区的放大器
\end{enumerate}
\noindent\textbf{解说:}C类放大器是指\textbf{放大器件在多于半个信号周期的时间内处于截止区,另半个周期的部分时间候处于线性区的放大器}。\\\noindent\textbf{答案:}B





\bigskip


\noindent\textbf{问题:}根据放大器的工作状态,通常把放大器分为A、B、C、D等类别。D类放大器是指:
\begin{enumerate}[label=\Alph*), leftmargin=3em]
\item 放大器件在整个信号周期内始终工作在线性区的放大器
\item 放大器件在半个信号周期内处于截止区,另半个周期处于饱和区的放大器
\item 放大器件在半个信号周期内处于截止区,另半个周期的部分时间候处于饱和区的放大器
\item 放大器件在半个信号周期内工作点处于线性区、另半个信号周期内处于截止区的放大器
\end{enumerate}
\noindent\textbf{解说:}D类放大器是指\textbf{放大器件在半个信号周期内处于截止区,另半个周期处于饱和区的放大器}。\\\noindent\textbf{答案:}B





\bigskip


\noindent\textbf{问题:}A、B、C、D四类放大器按输出波形失真由小到大的排列顺序是:
\begin{enumerate}[label=\Alph*), leftmargin=3em]
\item B、A、D、C
\item D、C、B、A
\item A、B、C、D
\item A、C、B、D
\end{enumerate}
\noindent\textbf{解说:}A、B、C、D四类放大器按输出波形失真由小到大的排列顺序是:\textbf{A、B、C、D}。\\\noindent\textbf{答案:}C





\bigskip


\noindent\textbf{问题:}A、B、C、D四类放大器用作射频功率放大时,按电源效率由高到低的排列顺序是:
\begin{enumerate}[label=\Alph*), leftmargin=3em]
\item B、A、D、C
\item D、C、B、A
\item D、C、A、B
\item A、B、C、D
\end{enumerate}
\noindent\textbf{解说:}A、B、C、D四类放大器用作射频功率放大时,按电源效率由高到低的排列顺序是:\textbf{D、C、B、A}。\\\noindent\textbf{答案:}B





\bigskip


\noindent\textbf{问题:}A、B、C、D四类放大器中,适宜于做小信号放大器的是:
\begin{enumerate}[label=\Alph*), leftmargin=3em]
\item C
\item D
\item B
\item A
\end{enumerate}
\noindent\textbf{解说:}A、B、C、D四类放大器中,适宜于做小信号放大器的是:\textbf{A}类放大器。\\\noindent\textbf{答案:}D






\bigskip


\noindent\textbf{问题:}A、B、C、D四类放大器中,属于大信号放大器的全部类别有:
\begin{enumerate}[label=\Alph*), leftmargin=3em]
\item A、B、C、D
\item B、C、D
\item A、C、D
\item C、D
\end{enumerate}
\noindent\textbf{解说:}A、B、C、D四类放大器中,属于大信号放大器的全部类别有:\textbf{B、C、D}。\\\noindent\textbf{答案:}B





\bigskip


\noindent\textbf{问题:}很多业余电台的末级和末前级射频输出放大器中采用两个并联的输出半导体功率管,这是为了:
\begin{enumerate}[label=\Alph*), leftmargin=3em]
\item 双管并联,得到双倍的器件耐压,减少损坏几率
\item 构成推挽电路,减小输出波形的失真
\item 双管并联,使每个功率管的失真互相补偿,减少失真,降低杂散发射
\item 双管并联,得到双倍的输出电流和输出功率
\end{enumerate}
\noindent\textbf{解说:}很多业余电台的末级和末前级射频输出放大器中采用两个并联的输出半导体功率管,这是为了\textbf{双管并联,得到双倍的输出电流和输出功率}。\\\noindent\textbf{答案:}D





\bigskip


\noindent\textbf{问题:}很多现代无线电设备的音频功率放大电路采用两个串联的输出半导体功率管,分别负责信号正、负半周的放大。这种电路的通用名称和作用是:
\begin{enumerate}[label=\Alph*), leftmargin=3em]
\item 双管串联电路,得到双倍的输出电流和输出功率
\item 双管串联电路,得到较高的输出阻抗以改善与负载的阻抗匹配
\item 推挽放大电路,实现极小静态工作点下的高电源效率的线性功率放大
\item 双管串联电路,得到较高的输入阻抗以改善与推动级之间的阻抗匹配
\end{enumerate}
\noindent\textbf{解说:}很多现代无线电设备的音频功率放大电路采用两个串联的输出半导体功率管,分别负责信号正、负半周的放大。这种电路的通用名称和作用是\textbf{推挽放大电路,实现极小静态工作点下的高电源效率的线性功率放大}。\\\noindent\textbf{答案:}C






\bigskip


\noindent\textbf{问题:}放大器的负反馈是指这样的电路:
\begin{enumerate}[label=\Alph*), leftmargin=3em]
\item 将放大器输入信号的一部分直通到放大器的输出端,起到加强输出信号的作用
\item 将放大器输出信号的一部分回输到放大器的输入端,起到抵消输入信号的作用
\item 将放大器输入信号的一部分直通到放大器的输出端,起到抵消输出信号的作用
\item 将放大器输出信号的一部分回输到放大器的输入端,起到加强输入信号的作用

\end{enumerate}
\noindent\textbf{解说:}放大器的负反馈是指这样的电路:\textbf{将放大器输出信号的一部分回输到放大器的输入端,起到抵消输入信号的作用}。\\\noindent\textbf{答案:}B

%%%???

\bigskip


\noindent\textbf{问题:}很多现代业余无线电收发信机的本机振荡电路采用了直接数字频率合成(DDS)方式。它的主要特点是:
\begin{enumerate}[label=\Alph*), leftmargin=3em]
\item 直接产生纯净的正弦波信号,不需要采用任何滤波器
\item 采用同样的频率源振荡器时频率稳定度优于锁相环频率合成方式
\item 电路结构简洁,无锁相捕捉范围限制,不产生相位噪声,跳换频率快
\item 与锁相环频率合成方式相比,可以使用速度较低的数字元器件
\end{enumerate}
\noindent\textbf{解说:}很多现代业余无线电收发信机的本机振荡电路采用了直接数字频率合成(DDS)方式。它的主要特点是\textbf{电路结构简洁,无锁相捕捉范围限制,不产生相位噪声,跳换频率快}。\\\noindent\textbf{答案:}C

%%%???

\bigskip


\noindent\textbf{问题:}频移电报技术(frequency-shift telegraphy)是指:电报信号控制载波频率在预定的范围之内变化的调频电报技术。下述业余通信使用的是移频电报技术:
\begin{enumerate}[label=\Alph*), leftmargin=3em]
\item CW
\item PSK31
\item SSTV
\item RTTY
\end{enumerate}
\noindent\textbf{解说:}\textbf{RTTY}使用的是移频电报技术。\\\noindent\textbf{答案:}D



\bigskip


\noindent\textbf{问题:}无线电通信选择不同调制方式的主要考虑因素是:
\begin{enumerate}[label=\Alph*), leftmargin=3em]
\item 有利于提高接收机选择性、尽量提高话筒灵敏度、防止产生谐波干扰
\item 信息在传递过程中的保真度、信号的抗干扰能力、尽量节省无线电频谱资源
\item 防止与附近发射机产生三阶互调、尽量降低本振相位噪声、采用高中频方案
\item 改善天线阻抗匹配、尽量提高发射频率稳定度、尽量减少杂散发射
\end{enumerate}
\noindent\textbf{解说:}无线电通信选择不同调制方式的主要考虑因素是\textbf{信息在传递过程中的保真度、信号的抗干扰能力、尽量节省无线电频谱资源}。\\\noindent\textbf{答案:}B



\bigskip


\noindent\textbf{问题:}接收机解调部件的作用是:
\begin{enumerate}[label=\Alph*), leftmargin=3em]
\item 从接收到的已调制射频信号中分离出原始信号
\item 对接收到的射频信号进行选频放大
\item 从接收到的已调制射频信号中提取出载频分量
\item 对接收到的射频信号进行宽带线性放大
\end{enumerate}
\noindent\textbf{解说:}接收机解调部件的作用是\textbf{从接收到的已调制射频信号中分离出原始信号}。\\\noindent\textbf{答案:}A



\bigskip


\noindent\textbf{问题:}选择解调部件的主要应考因素是:
\begin{enumerate}[label=\Alph*), leftmargin=3em]
\item 尽量提升已调制信号中的载频分量
\item 尽量忠实地还原原始信号
\item 尽量补偿接收到的射频信号的频率偏移
\item 尽量对已调制信号加以放大
\end{enumerate}
\noindent\textbf{解说:}选择解调部件的主要应考因素是\textbf{尽量忠实地还原原始信号}。\\\noindent\textbf{答案:}B




\bigskip


\noindent\textbf{问题:}在HF业余频段的数据通信段中,用收信机的SSB挡听到一个由两种音调交替组成的信号,这个信号的调制方式最可能属于下述种类:
\begin{enumerate}[label=\Alph*), leftmargin=3em]
\item SSTV或FAX
\item ASK
\item FSK
\item PSK
\end{enumerate}
\noindent\textbf{解说:}这个信号的调制方式最可能属于\textbf{FSK}。\\\noindent\textbf{答案:}C




\bigskip


\noindent\textbf{问题:}在HF业余频段的数据通信段中,用收信机的SSB挡听到一个音调不变但又似乎不断颤动的信号,这个信号的调制方式最可能属于下述种类:
\begin{enumerate}[label=\Alph*), leftmargin=3em]
\item SSTV或FAX
\item PSK
\item FSK
\item ASK
\end{enumerate}
\noindent\textbf{解说:}这个信号的调制方式最可能属于\textbf{PSK}。\\\noindent\textbf{答案:}B




\bigskip


\noindent\textbf{问题:}用收信机的SSB挡在业余频段中,听到一个音调大致以约为几分之一秒的重复周期连续变化、并夹有一种规律的“笃、笃”声的信号,这个信号的调制方式最可能属于下述种类:
\begin{enumerate}[label=\Alph*), leftmargin=3em]
\item PSK
\item SSTV或FAX
\item FSK
\item ASK
\end{enumerate}
\noindent\textbf{解说:}这个信号的调制方式最可能属于\textbf{SSTV或FAX}。\\\noindent\textbf{答案:}B



\bigskip


\noindent\textbf{问题:}对于给定的SSB发射设备,决定其输出信号实际占用带宽的因素是:
\begin{enumerate}[label=\Alph*), leftmargin=3em]
\item 所传输信号的最高频率越高,射频输出占用带宽越宽,但与其幅度和带宽无关
\item 所传输信号的带宽越宽,射频输出占用带宽越宽,但与其幅度和最高频率无关
\item 所传输信号的幅度越大,射频输出占用带宽越宽,但与其频率和带宽无关
\item 射频输出实际占用带宽为由电路决定的固定值,通信常用的是2.7kHz
\end{enumerate}
\noindent\textbf{解说:}对于给定的SSB发射设备,决定其输出信号实际占用带宽的因素是\textbf{所传输信号的带宽越宽,射频输出占用带宽越宽,但与其幅度和最高频率无关}。\\\noindent\textbf{答案:}B




\bigskip


\noindent\textbf{问题:}业余SSTV通信和有些模拟ATV采用调频方式而不是广播电视图像的调幅方式,主要原因是: 
\begin{enumerate}[label=\Alph*), leftmargin=3em]
\item 业余电台信号较弱,调频解调可以更好地抗拒叠加在信号上的外界噪声所引起的幅度变化
\item 便于与调频通话方式所使用的设备兼容
\item 调频收信设备的灵敏度比调幅的高一个数量级
\item 调频方式占用的频带比调幅方式窄
\end{enumerate}
\noindent\textbf{解说:}业余SSTV通信和有些模拟ATV采用调频方式而不是广播电视图像的调幅方式,主要原因是\textbf{业余电台信号较弱,调频解调可以更好地抗拒叠加在信号上的外界噪声所引起的幅度变化}。\\\noindent\textbf{答案:}A




\bigskip


\noindent\textbf{问题:}如果发射机在不同工作模式时最大射频输出功率相同,无语音调制时,实际射频输出由大到小的排序为:
\begin{enumerate}[label=\Alph*), leftmargin=3em]
\item FM,AM,SSB
\item SSB,AM,FM
\item SSB,FM,AM
\item AM,SSB,FM
\end{enumerate}
\noindent\textbf{解说:}实际射频输出由大到小的排序为:\textbf{FM,AM,SSB}。\\\noindent\textbf{答案:}A




\bigskip


\noindent\textbf{问题:}下列调制得到的信号幅度恒定不变:
\begin{enumerate}[label=\Alph*), leftmargin=3em]
\item 移频键控FSK
\item 幅度调制AM
\item 幅度键控调制ASK
\item 单边带幅度调制SSB
\end{enumerate}
\noindent\textbf{解说:}\textbf{移频键控FSK}调制得到的信号幅度恒定不变。\\\noindent\textbf{答案:}A




\bigskip


\noindent\textbf{问题:}下列调制得到的信号幅度恒定不变:
\begin{enumerate}[label=\Alph*), leftmargin=3em]
\item 幅度调制AM
\item 单边带幅度调制SSB
\item 幅度键控调制ASK
\item 频率调制FM
\end{enumerate}
\noindent\textbf{解说:}\textbf{频率调制FM}调制得到的信号幅度恒定不变。\\\noindent\textbf{答案:}D




\bigskip


\noindent\textbf{问题:}下列调制得到的信号幅度恒定不变:
\begin{enumerate}[label=\Alph*), leftmargin=3em]
\item 幅度键控调制ASK
\item 相位调制PM
\item 单边带幅度调制SSB
\item 幅度调制AM
\end{enumerate}
\noindent\textbf{解说:}\textbf{相位调制PM}调制得到的信号幅度恒定不变。\\\noindent\textbf{答案:}B



\bigskip


\noindent\textbf{问题:}下列调制得到的信号幅度恒定不变:
\begin{enumerate}[label=\Alph*), leftmargin=3em]
\item 单边带幅度调制SSB
\item 幅度调制AM
\item 移相键控调制PSK
\item 幅度键控调制ASK
\end{enumerate}
\noindent\textbf{解说:}\textbf{移相键控调制PSK}调制得到的信号幅度恒定不变。\\\noindent\textbf{答案:}C




\bigskip


\noindent\textbf{问题:}下列调制得到的信号周期恒定不变:
\begin{enumerate}[label=\Alph*), leftmargin=3em]
\item 相位调制PM
\item 频率键控调制FSK
\item 频率调制FM
\item 单边带幅度调制SSB
\end{enumerate}
\noindent\textbf{解说:}\textbf{相位调制PM}调制得到的信号周期恒定不变。\\\noindent\textbf{答案:}A




\bigskip


\noindent\textbf{问题:}下列调制得到的信号中载频分量幅度恒定不变:
\begin{enumerate}[label=\Alph*), leftmargin=3em]
\item 载波抑制单边带幅度调制SSB
\item 幅度调制AM
\item 频率键控调制FSK
\item 频率调制FM
\end{enumerate}
\noindent\textbf{解说:}\textbf{幅度调制AM}调制得到的信号中载频分量幅度恒定不变。\\\noindent\textbf{答案:}B




\bigskip


\noindent\textbf{问题:}信号经过下列调制后,频带宽度可能会大于原有值:
\begin{enumerate}[label=\Alph*), leftmargin=3em]
\item 单边带幅度调制SSB
\item 幅度调制AM
\item 频率调制FM
\item 幅度键控调制ASK
\end{enumerate}
\noindent\textbf{解说:}信号经过\textbf{频率调制FM}后,频带宽度可能会大于原有值。\\\noindent\textbf{答案:}C



\bigskip


\noindent\textbf{问题:}什么是PSK31?
\begin{enumerate}[label=\Alph*), leftmargin=3em]
\item 一种高速率的数据通信模式
\item 一种压缩数字电视信号的方法
\item 一种低速率的数据通信模式
\item 一种可以减少噪音对FM信号干扰的方法
\end{enumerate}
\noindent\textbf{解说:}PSK31是\textbf{一种低速率的数据通信模式}。\\\noindent\textbf{答案:}C




\bigskip


\noindent\textbf{问题:}谐振回路的通带宽度BW是指:
\begin{enumerate}[label=\Alph*), leftmargin=3em]
\item 回路两端电压信号幅度从中心频率衰减30\%时上、下限频率的间距
\item 回路两端电压信号幅度从中心频率衰减95\%时上、下限频率的间距
\item 回路两端电压信号幅度从中心频率衰减80\%时上、下限频率的间距
\item 回路两端电压信号幅度从中心频率衰减3dB时上、下限频率的间距
\end{enumerate}
\noindent\textbf{解说:}谐振回路的通带宽度BW是指\textbf{回路两端电压信号幅度从中心频率衰减3dB时上、下限频率的间距}。\\\noindent\textbf{答案:}D



\bigskip


\noindent\textbf{问题:}滤波器的“截止频率”是指:
\begin{enumerate}[label=\Alph*), leftmargin=3em]
\item 输出频率特性曲线从通带的0dB变化到-3dB的频率
\item 低于该频率的信号将被滤波器完全切除
\item 高于该频率的信号将被滤波器完全切除
\item 高于该频率的信号将会在滤波器中发生非线性失真
\end{enumerate}
\noindent\textbf{解说:}滤波器的“截止频率”是指\textbf{输出频率特性曲线从通带的0dB变化到-3dB的频率}。\\\noindent\textbf{答案:}A



\bigskip


\noindent\textbf{问题:}滤波器的“3dB带宽”是指:
\begin{enumerate}[label=\Alph*), leftmargin=3em]
\item 滤波器维持3dB增益的频率比范围宽度
\item 输出信号相对于输入信号衰减3dB以上(含3dB)的频率范围宽度
\item 输出信号相对于输入信号衰减3dB以下(含3dB)的频率范围宽度
\item 输出频率特性曲线从通带的0dB变化到-3dB的频率之间的宽度
\end{enumerate}
\noindent\textbf{解说:}滤波器的“3dB带宽”是指\textbf{输出频率特性曲线从通带的0dB变化到-3dB的频率之间的宽度}。\\\noindent\textbf{答案:}D



\bigskip


\noindent\textbf{问题:}220V.AC/13.8V.DC通信开关电源的一般工作过程是:
\begin{enumerate}[label=\Alph*), leftmargin=3em]
\item 由变压器将交流输入变为低压交流,由半导体开关电路变成超音频脉冲电流,经整流滤波为低压直流
\item 将交流输入整流滤波为高压直流,由半导体开关电路变成高压脉冲电流,由变压器变成低压脉冲,整流滤波为低压直流
\item 将交流输入整流滤波为高压直流,由变压器变成低压脉冲,由半导体开关电路变成低压直流,滤波后输出
\item 由大功率半导体三极管将交流输入变为高压直流,由专用集成电路变成超音频脉冲电流,经整流滤波为低压直流
\end{enumerate}
\noindent\textbf{解说:}220V.AC/13.8V.DC通信开关电源的一般工作过程是\textbf{将交流输入整流滤波为高压直流,由半导体开关电路变成高压脉冲电流,由变压器变成低压脉冲,整流滤波为低压直流}。\\\noindent\textbf{答案:}B





\bigskip


\noindent\textbf{问题:}具有两个输入的与门(AND)是最简单的数字逻辑电路之一。如果两个输入信号的组合分别为00、01、10、11,对应的输出信号应为:
\begin{enumerate}[label=\Alph*), leftmargin=3em]
\item 1、0、1、0
\item 0、0、0、1
\item 0、1、1、1
\item 0、1、0、1
\end{enumerate}
\noindent\textbf{解说:}对应的输出信号应为\textbf{0、0、0、1}。\\\noindent\textbf{答案:}B





\bigskip


\noindent\textbf{问题:}具有两个输入的或门(OR)是最简单的数字逻辑电路之一。如果两个输入信号组合分别为00、01、10、11,对应的输出信号应为:
\begin{enumerate}[label=\Alph*), leftmargin=3em]
\item 1、0、1、0
\item 0、1、1、0
\item 0、1、0、1
\item 0、1、1、1
\end{enumerate}
\noindent\textbf{解说:}对应的输出信号应为\textbf{0、1、1、1}。\\\noindent\textbf{答案:}D



\bigskip


\noindent\textbf{问题:}具有两个输入的异或门(XOR)是最简单的数字逻辑电路之一。如果两个输入信号组合分别为00、01、10、11,对应的输出信号应为:
\begin{enumerate}[label=\Alph*), leftmargin=3em]
\item 0、1、1、0
\item 0、1、1、1
\item 0、1、0、1
\item 1、0、1、0
\end{enumerate}
\noindent\textbf{解说:}对应的输出信号应为\textbf{0、1、1、0}。\\\noindent\textbf{答案:}A




\bigskip


\noindent\textbf{问题:}具有两个输入的与非门(NAND)是最简单的数字逻辑电路之一。如果两个输入信号组合分别为00、01、10、11,对应的输出信号应为:
\begin{enumerate}[label=\Alph*), leftmargin=3em]
\item 1、0、1、0
\item 1、1、1、0
\item 0、1、0、1
\item 0、1、1、1
\end{enumerate}
\noindent\textbf{解说:}对应的输出信号应为\textbf{1、1、1、0}。\\\noindent\textbf{答案:}B






\bigskip


\noindent\textbf{问题:}具有两个输入的或非门(NOR)是最简单的数字逻辑电路之一。如果两个输入信号组合分别为00、01、10、11,对应的输出信号应为:
\begin{enumerate}[label=\Alph*), leftmargin=3em]
\item 0、1、1、1
\item 1、0、0、0
\item 1、0、1、0
\item 0、1、0、1
\end{enumerate}
\noindent\textbf{解说:}对应的输出信号应为\textbf{1、0、0、0}。\\\noindent\textbf{答案:}B





\bigskip


\noindent\textbf{问题:}具有两个输入的异或非门(NXOR)是最简单的数字逻辑电路之一。如果两个输入信号组合分别为00、01、10、11,对应的输出信号应为:
\begin{enumerate}[label=\Alph*), leftmargin=3em]
\item 0、1、1、1
\item 1、0、0、1
\item 1、1、0、0
\item 0、1、1、0
\end{enumerate}
\noindent\textbf{解说:}对应的输出信号应为\textbf{1、0、0、1}。\\\noindent\textbf{答案:}B




\bigskip


\noindent\textbf{问题:}一个重复频率为F的非正弦周期信号的频谱包含有:
\begin{enumerate}[label=\Alph*), leftmargin=3em]
\item 无穷多个连续的频率分量	
\item 频率为F以外的无穷多个频率分量
\item 频率为F的整数倍的无穷多个频率分量
\item 频率为F的一个频率分量
\end{enumerate}
\noindent\textbf{解说:}一个重复频率为F的非正弦周期信号的频谱包含有\textbf{频率为F的整数倍的无穷多个频率分量}。\\\noindent\textbf{答案:}C




\bigskip


\noindent\textbf{问题:}“频率失真”是指电路的输出信号波形与输入信号相比,发生了下列变化:
\begin{enumerate}[label=\Alph*), leftmargin=3em]
\item 信号的幅度发生了改变
\item 产生了新的频率分量
\item 不同频率分量的相位延迟差发生了改变
\item 各频率分量的比例发生了改变
\end{enumerate}
\noindent\textbf{解说:}“频率失真”是指电路的输出信号波形与输入信号相比,\textbf{各频率分量的比例发生了改变}。\\\noindent\textbf{答案:}D

%%%???



\bigskip


\noindent\textbf{问题:}“非线性失真”是指电路的输出信号波形与输入信号相比,发生了下列变化:
\begin{enumerate}[label=\Alph*), leftmargin=3em]
\item 产生了新的频率分量
\item 不同频率分量的相位延迟差发生了改变
\item 各频率分量的比例发生了改变
\item 信号的幅度发生了改变
\end{enumerate}
\noindent\textbf{解说:}“非线性失真”是指电路的输出信号波形与输入信号相比,\textbf{产生了新的频率分量}。\\\noindent\textbf{答案:}A

%%%???

\bigskip


\noindent\textbf{问题:}“相位失真”是指电路的输出信号波形与输入信号相比,发生了下列变化:
\begin{enumerate}[label=\Alph*), leftmargin=3em]
\item 信号的幅度发生了改变
\item 不同频率分量的相位延迟差发生了改变
\item 各频率分量的比例发生了改变
\item 产生了新的频率分量
\end{enumerate}
\noindent\textbf{解说:}“相位失真”是指电路的输出信号波形与输入信号相比,\textbf{不同频率分量的相位延迟差发生了改变}。\\\noindent\textbf{答案:}B



\bigskip


\noindent\textbf{问题:}在射频电路分析中,能产生信号频率以外的新频率分量的元器件属于有源元器件,可能成为形成干扰的重要环节。下列元器件中属于有源元器件的有:
\begin{enumerate}[label=\Alph*), leftmargin=3em]
\item 半导体二极管
\item 电解电容器
\item 宽带变压器
\item 碱性干电池
\end{enumerate}
\noindent\textbf{解说:}下列元器件中属于有源元器件的有\textbf{半导体二极管}。\\\noindent\textbf{答案:}A



\bigskip


\noindent\textbf{问题:}在射频电路分析中,能产生信号频率以外的新频率分量的元器件属于有源元器件,可能成为形成干扰的重要环节。下列元器件中属于有源元器件的有:
\begin{enumerate}[label=\Alph*), leftmargin=3em]
\item 电阻假负载
\item 可调电感器
\item 晶体滤波器
\item 半导体三极管
\end{enumerate}
\noindent\textbf{解说:}下列元器件中属于有源元器件的有\textbf{半导体三极管}。\\\noindent\textbf{答案:}D



\bigskip


\noindent\textbf{问题:}频率为f1、f2的两个正弦交流信号流过一个非线性元件,会发生“混频”。混频产物中属于三阶互调的干扰信号的频率是:
\begin{enumerate}[label=\Alph*), leftmargin=3em]
\item 4f1$\pm$f2、5f1$\pm$2f2、6f1$\pm$3f2……
\item f1$\pm$f2、2f1$\pm$2f2、3f1$\pm$3f2
\item 2f1$\pm$f2、2f2$\pm$f1
\item 2f1、3f1、2f2、3f2
\end{enumerate}
\noindent\textbf{解说:}混频产物中属于三阶互调的干扰信号的频率是\textbf{2f1$\pm$f2、2f2$\pm$f1}。\\\noindent\textbf{答案:}C



\bigskip


\noindent\textbf{问题:}“差拍”现象是指:
\begin{enumerate}[label=\Alph*), leftmargin=3em]
\item 两个不同频率信号经过非线性电路得到频率为两者之差的新频率信号
\item 凡是接收机收到的连续音频叫声都叫做“差拍”
\item 一个单频率信号经过非线性电路得到一系列谐波信号,相邻信号之间的频率差等于单频信号的频率。这样一族谐波信号的集合总称“差拍”
\item 发射机用以调制载波的连续音频信号
\end{enumerate}
\noindent\textbf{解说:}“差拍”现象是指\textbf{两个不同频率信号经过非线性电路得到频率为两者之差的新频率信号}。\\\noindent\textbf{答案:}A





\bigskip


\noindent\textbf{问题:}一个CW电报信号在频谱仪上显示为:
\begin{enumerate}[label=\Alph*), leftmargin=3em]
\item 两条水平直线
\item 一条正弦波曲线
\item 一条固定的直线
\item 一条闪动的垂直线
\end{enumerate}
\noindent\textbf{解说:}一个CW电报信号在频谱仪上显示为\textbf{一条闪动的垂直线}。\\\noindent\textbf{答案:}D




\bigskip


\noindent\textbf{问题:}一个RTTY信号在频谱仪上显示为:
\begin{enumerate}[label=\Alph*), leftmargin=3em]
\item 两条闪动的垂直线
\item 一条正弦波曲线
\item 一条垂直线
\item 一条复杂的周期性曲线
\end{enumerate}
\noindent\textbf{解说:}一个RTTY信号在频谱仪上显示为\textbf{两条闪动的垂直线}。\\\noindent\textbf{答案:}A



\bigskip


\noindent\textbf{问题:}一个SSB话音信号在频谱仪上显示为:
\begin{enumerate}[label=\Alph*), leftmargin=3em]
\item 一条复杂的周期性曲线
\item 一组随语音出现和变化的非对称垂直线
\item 一条固定的直线
\item 一条随语音闪烁的直线
\end{enumerate}
\noindent\textbf{解说:}一个SSB话音信号在频谱仪上显示为\textbf{一组随语音出现和变化的非对称垂直线}。\\\noindent\textbf{答案:}B




\bigskip


\noindent\textbf{问题:}一个AM话音信号在频谱仪上显示为:
\begin{enumerate}[label=\Alph*), leftmargin=3em]
\item 多条固定的直线
\item 一条随语音闪烁的直线
\item 一条固定的垂直线,左右伴随一组对称的随语音出现和变化的垂直线
\item 一条复杂的周期性曲线
\end{enumerate}
\noindent\textbf{解说:}一个AM话音信号在频谱仪上显示为\textbf{一条固定的垂直线,左右伴随一组对称的随语音出现和变化的垂直线}。\\\noindent\textbf{答案:}C



\bigskip


\noindent\textbf{问题:}下列几种图表中,最容易用来表达和解释LC振荡器温度漂移程度的是:
\begin{enumerate}[label=\Alph*), leftmargin=3em]
\item 相位矢量图
\item 波形图
\item 频谱瀑布图
\item 频谱图
\end{enumerate}
\noindent\textbf{解说:}最容易用来表达和解释LC振荡器温度漂移程度的是\textbf{频谱瀑布图}。\\\noindent\textbf{答案:}C



\bigskip


\noindent\textbf{问题:}正弦交流信号通过下列电路时会产生高次谐波:
\begin{enumerate}[label=\Alph*), leftmargin=3em]
\item 调谐在信号的整倍频上的谐振电路
\item 二极管整流器或三极管开关放大器
\item 复杂电容电感网络
\item 复杂电阻网络
\end{enumerate}
\noindent\textbf{解说:}正弦交流信号通过\textbf{二极管整流器或三极管开关放大器}时会产生高次谐波。\\\noindent\textbf{答案:}B



\bigskip


\noindent\textbf{问题:}下列发射模式中拥有最窄带宽的是:
\begin{enumerate}[label=\Alph*), leftmargin=3em]
\item 调频话音
\item CW
\item 单边带话音
\item 慢扫描电视
\end{enumerate}
\noindent\textbf{解说:}\textbf{CW}发射模式拥有最窄带宽。\\\noindent\textbf{答案:}B



\bigskip


\noindent\textbf{问题:}无线电发射设备参数和业余无线电原理书籍中经常出现缩写为ppm的度量单位。其中文含义和最经常的用处是:
\begin{enumerate}[label=\Alph*), leftmargin=3em]
\item “每分钟点数”,常用于描述太阳黑子的发生频度
\item “每分钟点数”,常用于描述随机干扰的出现次数
\item “百万分比”,常用于描述频率的相对稳定度
\item “每分钟平均脉冲数”,常用于描述单个脉冲的出现次数
\end{enumerate}
\noindent\textbf{解说:}\textbf{“百万分比”,常用于描述频率的相对稳定度}。\\\noindent\textbf{答案:}C

%%%???


\bigskip


\noindent\textbf{问题:}甲乙两种业余无线电台设备资料列出接收机灵敏度指标分别为0.1μV和0.15μV。正确的推论为:
\begin{enumerate}[label=\Alph*), leftmargin=3em]
\item 可以推断甲机接收微弱信号的能力比乙机的高,因为可以接收的信号更微弱
\item 凭此指标还无法比较两者接收微弱信号的能力,因没有给出测量灵敏度时的输出信号质量条件
\item 可以推断甲机接收微弱信号的能力比乙机的低,因为灵敏度数值比较小
\item 可以推断甲机承受强信号的能力比乙机的低,因为灵敏度数值比较小
\end{enumerate}
\noindent\textbf{解说:}正确的推论为:\textbf{凭此指标还无法比较两者接收微弱信号的能力,因没有给出测量灵敏度时的输出信号质量条件}。\\\noindent\textbf{答案:}B

%%%???


\bigskip


\noindent\textbf{问题:}决定接收机抗拒与工作频率相距较远的强信号干扰的主要选择性指标是:
\begin{enumerate}[label=\Alph*), leftmargin=3em]
\item 前端带宽
\item 带内波动和信道带宽
\item 信道带宽、信道选择性和信道滤波器特性矩形系数
\item 镜像抑制比
\end{enumerate}

\bigskip


\noindent\textbf{问题:}决定接收机抗拒与工作频率相距两倍于中频的频率上强信号干扰的主要选择性指标是:
\begin{enumerate}[label=\Alph*), leftmargin=3em]
\item 镜像抑制比
\item 前端带宽
\item 信道带宽、信道选择性和信道滤波器特性矩形系数
\item 带内波动和信道带宽
\end{enumerate}
\noindent\textbf{解说:}决定接收机抗拒与工作频率相距两倍于中频的频率上强信号干扰的主要选择性指标是\textbf{镜像抑制比}。\\\noindent\textbf{答案:}A



\bigskip


\noindent\textbf{问题:}制约现代无线电接收机灵敏度的主要因素是:
\begin{enumerate}[label=\Alph*), leftmargin=3em]
\item 电源噪声
\item 放大电路的稳定性
\item 放大电路的增益
\item 机内噪声
\end{enumerate}
\noindent\textbf{解说:}制约现代无线电接收机灵敏度的主要因素是\textbf{机内噪声}。\\\noindent\textbf{答案:}D


\bigskip


\noindent\textbf{问题:}业余通信接收机大多带有接收信号强度指示。VHF/UHF频段的最小刻度S1对应于输入信号功率电平-141dBm(50Ω输入电压0.02μV)标为S1,而HF频段的S1则对应于输入信号功率电平-121dBm(0.2μV)。这是因为:
\begin{enumerate}[label=\Alph*), leftmargin=3em]
\item V/UHF频段较寂静而HF频段外界背景噪声电平较高,前者可感知的最小信号电平比后者低约20dB
\item HF业余电台功率一般比较大,VHF/UHF电台功率比较小,因此需要不同的刻度标准
\item 由于电路技术的原因,HF频段接收机的灵敏度只能做到比VHF/UHF频段低大约20dB
\item HF业余电台主要用于DX通信,VHF/UHF电台主要做本地通信,因此需要不同的刻度标准
\end{enumerate}
\noindent\textbf{解说:}因为\textbf{V/UHF频段较寂静而HF频段外界背景噪声电平较高,前者可感知的最小信号电平比后者低约20dB}。\\\noindent\textbf{答案:}A




\bigskip


\noindent\textbf{问题:}业余调频中继台发射机只要被上行信号正常启动,就会一直继续发射载波,上行信号消失不能使其停止。可能的原因是:
\begin{enumerate}[label=\Alph*), leftmargin=3em]
\item 中继台上下行隔离不良,中继台发射的载波窜入中继台接收机造成自锁
\item 中继台发射机电源电压不稳
\item 中继台接收机电源电压不稳
\item 肯定受到人为恶意干扰
\end{enumerate}
\noindent\textbf{解说:}可能的原因是\textbf{中继台上下行隔离不良,中继台发射的载波窜入中继台接收机造成自锁}。\\\noindent\textbf{答案:}A



\bigskip


\noindent\textbf{问题:}业余调频中继台发射机被上行信号正常启动,但上行信号消失后经常会继续发射一段或长或短的时间并夹杂有一些不清楚的语音。可能的原因是:
\begin{enumerate}[label=\Alph*), leftmargin=3em]
\item 中继台接收机电源电压不稳
\item 中继台下行信号与附近的其他通信发射机形成对中继台上行频率的互调干扰
\item 肯定受到人为恶意干扰
\item 中继台发射机电源电压不稳
\end{enumerate}
\noindent\textbf{解说:}可能的原因是\textbf{中继台下行信号与附近的其他通信发射机形成对中继台上行频率的互调干扰}。\\\noindent\textbf{答案:}B




\bigskip


\noindent\textbf{问题:}业余中继台上下行共用一副天线时,需要在接收机、发信机和天线之间插入一个:
\begin{enumerate}[label=\Alph*), leftmargin=3em]
\item 收发信转换开关
\item 环形器(circulator)
\item 双工器(duplexer)
\item 功率分配器(power divider)
\end{enumerate}
\noindent\textbf{解说:}业余中继台上下行共用一副天线时,需要在接收机、发信机和天线之间插入一个\textbf{双工器(duplexer)}。\\\noindent\textbf{答案:}C



\bigskip


\noindent\textbf{问题:}业余中继台上下行共用一副天线时,需要在接收机、发信机和天线之间插入一个双工器,其基本构造和作用为:
\begin{enumerate}[label=\Alph*), leftmargin=3em]
\item 一组滤波器,阻止中继台发射信号反馈进入中继台接收机
\item 一个匹配网络,使天线、中继发射机、中继接收机三者之间都满足阻抗匹配条件
\item 一个环形器,使信号只能沿中继发射机-天线-中继接收机的方向前进
\item 一组半导体开关,在中继台发射时关断中继台接收机
\end{enumerate}
\noindent\textbf{解说:}双工器的基本构造和作用为\textbf{一组滤波器,阻止中继台发射信号反馈进入中继台接收机}。\\\noindent\textbf{答案:}A




\bigskip


\noindent\textbf{问题:}无线电接收机的灵敏度是指:
\begin{enumerate}[label=\Alph*), leftmargin=3em]
\item 接收机正常工作所需的最小输入信号强度
\item 接收机正常工作所需的最大输入信号强度
\item 接收机正常工作所需的最小电源功率
\item 接收机正常工作所需的最大电源电压
\end{enumerate}
\noindent\textbf{解说:}无线电接收机的灵敏度是指\textbf{接收机正常工作所需的最小输入信号强度}。\\\noindent\textbf{答案:}A



\bigskip


\noindent\textbf{问题:}一部业余无线电台,工作电压直流13.8伏,FM发射方式的射频输出载波功率为N瓦,电源效率约80\%。发射时的工作电流约为:
\begin{enumerate}[label=\Alph*), leftmargin=3em]
\item 13.8$\times$N(安)
\item 0.058$\times$N(安)
\item 0.091$\times$N(安)
\item 13.8/N$\times$80\%(安)
\end{enumerate}
\noindent\textbf{解说:}发射时的工作电流约为\textbf{0.091$\times$N(安)}。\\\noindent\textbf{答案:}C




\bigskip


\noindent\textbf{问题:}一部业余无线电台,工作电压交流220伏,FM发射方式的射频输出载波功率为N瓦,电源效率约80\%。发射时的工作电流约为:
\begin{enumerate}[label=\Alph*), leftmargin=3em]
\item 0.0036$\times$N(安)
\item 200/N$\times$80\%(安)
\item 220$\times$N(安)
\item 0.0057$\times$N(安)
\end{enumerate}
\noindent\textbf{解说:}发射时的工作电流约为\textbf{0.0057$\times$N(安)}。\\\noindent\textbf{答案:}D




\bigskip


\noindent\textbf{问题:}一部业余无线电台,FM发射方式的射频输出载波功率为N瓦,电源效率约80\%。通话时每发射10秒钟的电源消耗约为:
\begin{enumerate}[label=\Alph*), leftmargin=3em]
\item 0.0768 / N(千瓦小时)%0.0768 /N(千瓦小时)
\item 220 / N(千瓦小时)
\item 0.0022$\times$N(千瓦小时)
\item 0.0000035$\times$N(千瓦小时)
\end{enumerate}
\noindent\textbf{解说:}通话时每发射10秒钟的电源消耗约为\textbf{0.0000035$\times$N(千瓦小时)}。\\\noindent\textbf{答案:}D





\bigskip


\noindent\textbf{问题:}一部业余无线电台,FM发射方式的射频输出载波功率为10瓦,电源效率约80\%。连续发话10秒钟,在此期间发射到空间的平均功率:
\begin{enumerate}[label=\Alph*), leftmargin=3em]
\item 约为12.5瓦
\item 约为8瓦
\item 约为10瓦
\item 肯定高于10瓦
\end{enumerate}
\noindent\textbf{解说:}连续发话10秒钟,在此期间发射到空间的平均功率\textbf{约为10瓦}。\\\noindent\textbf{答案:}C




\bigskip


\noindent\textbf{问题:}一部业余无线电台,CW发射方式的射频输出载波功率为10瓦,电源效率约80\%。连续发报10秒钟,在此期间发射到空间的平均功率:
\begin{enumerate}[label=\Alph*), leftmargin=3em]
\item 显著低于10瓦
\item 约为12.5瓦
\item 约为8瓦
\item 约为10瓦
\end{enumerate}
\noindent\textbf{解说:}连续发报10秒钟,在此期间发射到空间的平均功率\textbf{显著低于10瓦}。\\\noindent\textbf{答案:}A




\bigskip


\noindent\textbf{问题:}从能量转换的观点,“匹配”是指:
\begin{enumerate}[label=\Alph*), leftmargin=3em]
\item 选择电路参数,使负载能够得到最高实际输出功率的状态
\item 选择电路参数,使负载能够得到最高实际输出电流的状态
\item 选择电路参数,使电源内阻能够达到最小功率损耗的状态
\item 选择电路参数,使负载能够得到最高实际输出电压的状态
\end{enumerate}
\noindent\textbf{解说:}从能量转换的观点,“匹配”是指\textbf{选择电路参数,使负载能够得到最高实际输出功率的状态}。\\\noindent\textbf{答案:}A





\bigskip


\noindent\textbf{问题:}信号源和负载达到阻抗“匹配”时,信号源内阻损耗的功率为:
\begin{enumerate}[label=\Alph*), leftmargin=3em]
\item 与负载得到的输出功率相等
\item 损耗功率为零
\item 与负载得到的输出功率相比可以忽略不计
\item 是负载得到的输出功率的一半
\end{enumerate}
\noindent\textbf{解说:}信号源和负载达到阻抗“匹配”时,信号源内阻损耗的功率\textbf{与负载得到的输出功率相等}。\\\noindent\textbf{答案:}A



\bigskip


\noindent\textbf{问题:}一个放大器具有20dB的信号增益,其意义是:
\begin{enumerate}[label=\Alph*), leftmargin=3em]
\item 放大器把相当于输入信号的100倍的能量从电源转移到了输出负载
\item 放大器把相输入信号的能量放大了100倍
\item 放大器把相输入信号的能量放大了99倍
\item 放大器产生了当于输入信号的100倍的能量,供给了输出负载
\end{enumerate}
\noindent\textbf{解说:}一个放大器具有20dB的信号增益,其意义是\textbf{放大器把相当于输入信号的100倍的能量从电源转移到了输出负载}。\\\noindent\textbf{答案:}A



\bigskip


\noindent\textbf{问题:}射频信号通过某电路时产生了20dB的损耗。这部分被损耗的能量:
\begin{enumerate}[label=\Alph*), leftmargin=3em]
\item 返回了信号源
\item 一部分在电路中消失了,一部分返回了信号源
\item 在电路中消失了
\item 在电路中被转化为热能等其他形式,或者通过电磁辐射等转移到了其他地方
\end{enumerate}
\noindent\textbf{解说:}这部分被损耗的能量\textbf{在电路中被转化为热能等其他形式,或者通过电磁辐射等转移到了其他地方}。\\\noindent\textbf{答案:}D



\bigskip


\noindent\textbf{问题:}某电路输出信号功率是输入信号功率的100倍,该电路的增益为:
\begin{enumerate}[label=\Alph*), leftmargin=3em]
\item 100dB
\item 1dB
\item 10dB
\item 20dB
\end{enumerate}
\noindent\textbf{解说:}该电路的增益为\textbf{20dB}。\\\noindent\textbf{答案:}D



\bigskip


\noindent\textbf{问题:}某电路输出信号功率是输入信号功率的100万倍,该电路的增益为:
\begin{enumerate}[label=\Alph*), leftmargin=3em]
\item 100万dB
\item 60dB
\item 100dB
\item 99万dB
\end{enumerate}
\noindent\textbf{解说:}该电路的增益为\textbf{60dB}。\\\noindent\textbf{答案:}B




\bigskip


\noindent\textbf{问题:}某电路输出信号功率是输入信号功率的5倍,该电路的增益约为:
\begin{enumerate}[label=\Alph*), leftmargin=3em]
\item 7dB
\item 3.5dB
\item 5dB
\item 14dB
\end{enumerate}
\noindent\textbf{解说:}该电路的增益约为\textbf{7dB}。\\\noindent\textbf{答案:}A




\bigskip


\noindent\textbf{问题:}某电路输出信号功率是输入信号功率的2倍,该电路的增益约为:
\begin{enumerate}[label=\Alph*), leftmargin=3em]
\item 3dB
\item 0.5dB
\item 1dB
\item 2dB
\end{enumerate}
\noindent\textbf{解说:}该电路的增益约为\textbf{3dB}。\\\noindent\textbf{答案:}A




\bigskip


\noindent\textbf{问题:}某电路输出信号电压是输入信号电压的100倍,该电路的增益为:
\begin{enumerate}[label=\Alph*), leftmargin=3em]
\item 100dB
\item 20dB
\item 40dB
\item 10dB
\end{enumerate}
\noindent\textbf{解说:}该电路的增益为\textbf{40dB}。\\\noindent\textbf{答案:}C




\bigskip


\noindent\textbf{问题:}某电路输出信号电压是输入信号电压的1万倍,该电路的增益为:(”x^m”表示“x的m次方”)
\begin{enumerate}[label=\Alph*), leftmargin=3em]
\item 80dB
\item 10,000 dB
\item 9,999 dB
\item 10^4dB
\end{enumerate}
\noindent\textbf{解说:}该电路的增益为\textbf{80dB}。\\\noindent\textbf{答案:}A



\bigskip


\noindent\textbf{问题:}某电路输出信号电压是输入信号电压的10倍,该电路的增益约为:
\begin{enumerate}[label=\Alph*), leftmargin=3em]
\item 20dB
\item 14dB
\item 15dB
\item 7dB
\end{enumerate}
\noindent\textbf{解说:}该电路的增益约为\textbf{20dB}。\\\noindent\textbf{答案:}A



\bigskip


\noindent\textbf{问题:}某电路输出信号电压是输入信号电压的2倍,该电路的增益约为:
\begin{enumerate}[label=\Alph*), leftmargin=3em]
\item 0.5dB
\item 2dB
\item 6dB
\item 4dB
\end{enumerate}
\noindent\textbf{解说:}该电路的增益约为\textbf{6dB}。\\\noindent\textbf{答案:}C





\bigskip


\noindent\textbf{问题:}某电路输出信号功率是输入信号功率的1/100,该电路的增益为:
\begin{enumerate}[label=\Alph*), leftmargin=3em]
\item 100dB
\item -20dB
\item -100dB
\item -10dB
\end{enumerate}
\noindent\textbf{解说:}该电路的增益为\textbf{-20dB}。\\\noindent\textbf{答案:}B




\bigskip


\noindent\textbf{问题:}某电路输出信号功率是输入信号功率的百万分之一,该电路的增益为:
\begin{enumerate}[label=\Alph*), leftmargin=3em]
\item -100dB
\item 990000dB
\item -60dB
\item -1000000dB
\end{enumerate}
\noindent\textbf{解说:}该电路的增益为\textbf{-60dB}。\\\noindent\textbf{答案:}C




\bigskip


\noindent\textbf{问题:}某电路输出信号功率是输入信号功率的1/5,该电路的增益约为:
\begin{enumerate}[label=\Alph*), leftmargin=3em]
\item 3.5dB
\item -5dB
\item -14dB
\item -7dB
\end{enumerate}
\noindent\textbf{解说:}该电路的增益约为\textbf{-7dB}。\\\noindent\textbf{答案:}D





\bigskip


\noindent\textbf{问题:}某电路输出信号功率是输入信号功率的1/2,该电路的增益约为:
\begin{enumerate}[label=\Alph*), leftmargin=3em]
\item -1dB
\item -3dB
\item 0.5dB
\item -2dB
\end{enumerate}
\noindent\textbf{解说:}该电路的增益约为\textbf{-3dB}。\\\noindent\textbf{答案:}B




\bigskip


\noindent\textbf{问题:}某电路输出信号电压是输入信号电压的1/100,该电路的增益为:
\begin{enumerate}[label=\Alph*), leftmargin=3em]
\item -40dB
\item -20dB
\item -10dB
\item -100dB
\end{enumerate}
\noindent\textbf{解说:}该电路的增益为\textbf{-40dB}。\\\noindent\textbf{答案:}A



\bigskip


\noindent\textbf{问题:}某电路输出信号电压是输入信号电压的万分之一,该电路的增益为:(”x^m”表示“x的m次方”)
\begin{enumerate}[label=\Alph*), leftmargin=3em]
\item -10,000 dB
\item 1/10,000 dB
\item 10^-4dB
\item -80dB
\end{enumerate}
\noindent\textbf{解说:}该电路的增益为\textbf{-80dB}。\\\noindent\textbf{答案:}D



\bigskip


\noindent\textbf{问题:}某电路输出信号电压是输入信号电压的1/10倍,该电路的增益约为:
\begin{enumerate}[label=\Alph*), leftmargin=3em]
\item -7dB
\item -14dB
\item 0.143dB
\item -20dB
\end{enumerate}
\noindent\textbf{解说:}该电路的增益约为\textbf{-20dB}。\\\noindent\textbf{答案:}D



\bigskip


\noindent\textbf{问题:}某电路输出信号电压是输入信号电压的1/2,该电路的增益约为:
\begin{enumerate}[label=\Alph*), leftmargin=3em]
\item -0.5dB
\item -2dB
\item 0.5dB
\item -6dB
\end{enumerate}
\noindent\textbf{解说:}该电路的增益约为\textbf{-6dB}。\\\noindent\textbf{答案:}D



\bigskip


\noindent\textbf{问题:}信号依次通过增益分别为 x dB、y dB和 z dB的三个电路,总增益为:
\begin{enumerate}[label=\Alph*), leftmargin=3em]
\item (x × y × z)dB
\item (x + y + z)dB
\item (x × y × z)倍
\item (x + y + z)倍
\end{enumerate}
\noindent\textbf{解说:}总增益为\textbf{(x + y + z)dB}。\\\noindent\textbf{答案:}B




\bigskip


\noindent\textbf{问题:}信号依次通过增益分别为 x dB、y dB和 z dB的三个电路,总增益为:(”x^m”表示“x的m次方”)
\begin{enumerate}[label=\Alph*), leftmargin=3em]
\item 10^(x × y × z) 倍
\item (x + y + z)倍
\item (x × y × z)倍
\item 10^((x + y + z)/ 10) 倍%10^((x + y + z)/10) 倍
\end{enumerate}
\noindent\textbf{解说:}总增益为\textbf{10^((x + y + z)/ 10) 倍}。\\\noindent\textbf{答案:}D



\bigskip


\noindent\textbf{问题:}接收机的接收信号强度表每两档的信号强度相差6dB。接收某电台信号,发射功率为20dBW时读数为S9。该台减小发射功率后,接收机读数变为S4。此时该台的发射功率约为(以W为单位):
\begin{enumerate}[label=\Alph*), leftmargin=3em]
\item 1.73W
\item 10.24W
\item 0.098W
\item 0.156W
\end{enumerate}
\noindent\textbf{解说:}此时该台的发射功率约为\textbf{0.098W}。\\\noindent\textbf{答案:}C





\bigskip


\noindent\textbf{问题:}接收机的接收信号强度表每两档的信号强度相差6dB。接收某电台信号,发射功率为10dBW时读数为S8。该台减小发射功率后,接收机读数变为S5。此时该台的发射功率约为(以W为单位):
\begin{enumerate}[label=\Alph*), leftmargin=3em]
\item 0.098W
\item 10.24W
\item 1.73W
\item 0.156W
\end{enumerate}
\noindent\textbf{解说:}此时该台的发射功率约为\textbf{0.156W}。\\\noindent\textbf{答案:}D





\bigskip


\noindent\textbf{问题:}功率为0 dBm的射频信号通过增益为 23 dB的电路后,输出功率为:
\begin{enumerate}[label=\Alph*), leftmargin=3em]
\item 0.23W
\item 23W
\item 0.2W
\item 23mW
\end{enumerate}
\noindent\textbf{解说:}输出功率为\textbf{0.2W}。\\\noindent\textbf{答案:}C



\bigskip


\noindent\textbf{问题:}功率为0dBμ的射频信号通过增益为 36 dB的电路后,输出功率为:
\begin{enumerate}[label=\Alph*), leftmargin=3em]
\item 360μW
\item 4mW
\item 3.6W
\item 36mW
\end{enumerate}
\noindent\textbf{解说:}输出功率为\textbf{4mW}。\\\noindent\textbf{答案:}B


\bigskip


\noindent\textbf{问题:}功率为0 dBW的射频信号通过增益为 -36 dB的电路后,输出功率为:
\begin{enumerate}[label=\Alph*), leftmargin=3em]
\item 25mW
\item 0.25 mW
\item 360μW
\item 3.6mW
\end{enumerate}
\noindent\textbf{解说:}输出功率为\textbf{0.25 mW}。\\\noindent\textbf{答案:}B



\bigskip


\noindent\textbf{问题:}功率为0 dBW的射频信号通过衰减量为 40 dB的衰减器后,输出功率为:
\begin{enumerate}[label=\Alph*), leftmargin=3em]
\item 140μW
\item 100μW
\item 0.40W%0.40 W
\item 40mW
\end{enumerate}
\noindent\textbf{解说:}输出功率为\textbf{100μW}。\\\noindent\textbf{答案:}B




\bigskip


\noindent\textbf{问题:}某业余电台以100瓦功率发射时,对方接收机的信号强度指示为S8。现双方天线不变,将发射功率降到25瓦,对方接收的信号强度指示将变为:【提示:收信机信号强度指示S1至S9每档相差6dB】
\begin{enumerate}[label=\Alph*), leftmargin=3em]
\item S7
\item S6
\item S4
\item S5
\end{enumerate}
\noindent\textbf{解说:}对方接收的信号强度指示将变为\textbf{S7}。\\\noindent\textbf{答案:}A



\bigskip


\noindent\textbf{问题:}某业余电台以80瓦功率发射时,对方接收机的信号强度指示为S8。现双方天线不变,将发射功率降为5瓦QRP,对方接收的信号强度指示将变为:【提示:收信机信号强度指示S1至S9每档相差6dB】
\begin{enumerate}[label=\Alph*), leftmargin=3em]
\item S6
\item S2
\item S4
\item S7
\end{enumerate}
\noindent\textbf{解说:}对方接收的信号强度指示将变为\textbf{S6}。\\\noindent\textbf{答案:}A



\bigskip


\noindent\textbf{问题:}在特定方向上具有主辐射瓣的水平偶极天线,其振子的总长度应为:
\begin{enumerate}[label=\Alph*), leftmargin=3em]
\item 1/2波长的奇数倍
\item 1/4波长的奇数倍
\item 1/2波长的偶数倍
\item 1/2波长的任意整数倍
\end{enumerate}
\noindent\textbf{解说:}在特定方向上具有主辐射瓣的水平偶极天线,其振子的总长度应为\textbf{1/2波长的奇数倍}。\\\noindent\textbf{答案:}A



\bigskip


\noindent\textbf{问题:}偶极天线与工作频率发生谐振的充分和必要条件是:
\begin{enumerate}[label=\Alph*), leftmargin=3em]
\item 两臂总电气长度为工作波长的整数倍
\item 两臂总电气长度为1/2工作波长的奇数倍
\item 两臂总电气长度为1/2工作波长的整数倍
\item 两臂总电气长度为1/4工作波长的整数倍
\end{enumerate}
\noindent\textbf{解说:}偶极天线与工作频率发生谐振的充分和必要条件是\textbf{两臂总电气长度为1/2工作波长的整数倍}。\\\noindent\textbf{答案:}C



\bigskip


\noindent\textbf{问题:}偶极天线两臂总长度选取下列电气长度时,在垂直于天线轴线方向的增益达到峰值:
\begin{enumerate}[label=\Alph*), leftmargin=3em]
\item 1/2工作波长的整数倍
\item 1/2工作波长的偶数倍
\item 1/4工作波长的奇数倍
\item 1/2工作波长的奇数倍
\end{enumerate}
\noindent\textbf{解说:}偶极天线两臂总长度选取下列电气长度时,在垂直于天线轴线方向的增益达到峰值\textbf{1/2工作波长的奇数倍}。\\\noindent\textbf{答案:}D



\bigskip


\noindent\textbf{问题:}制作工作频率为f(单位:兆赫兹)的某相控天线阵列需要长度为1/4波长的同轴电缆。其大致长度(单位:米)为: 
\begin{enumerate}[label=\Alph*), leftmargin=3em]
\item 71.3 / f
\item 48.8 / f
\item 149.8 / f
\item 75 / f
\end{enumerate}
\noindent\textbf{解说:}其大致长度(单位:米)为\textbf{48.8 / f}。\\\noindent\textbf{答案:}B

%%%???

\bigskip


\noindent\textbf{问题:}制作工作频率为f(单位:兆赫兹)的半波长偶极天线。每个振子的大致长度(单位:米)为:
\begin{enumerate}[label=\Alph*), leftmargin=3em]
\item 71.3 / f
\item 48.8 / f
\item 150 / f
\item 142.6 / f
\end{enumerate}
\noindent\textbf{解说:}制作工作频率为f(单位:兆赫兹)的半波长偶极天线。每个振子的大致长度(单位:米)为\textbf{71.3 / f}。\\\noindent\textbf{答案:}A

%%%???

\bigskip


\noindent\textbf{问题:}南北走向的水平极化偶极天线,中点馈电,通过特性阻抗为50欧的电缆连接到输入/输出阻抗为50欧的收发信机,通信对象在东西方向。选择天线长度的原则是:
\begin{enumerate}[label=\Alph*), leftmargin=3em]
\item 天线臂长为四分之一波长的奇数倍时,通信效果肯定最好
\item 天线臂长为四分之一波长的偶数倍时,通信效果肯定最好
\item 只要天线与工作频率谐振,通信效果一定好
\item 只要电压驻波比达到最低,通信效果一定好
\end{enumerate}
\noindent\textbf{解说:}选择天线长度的原则是\textbf{天线臂长为四分之一波长的奇数倍时,通信效果肯定最好}。\\\noindent\textbf{答案:}A



\bigskip


\noindent\textbf{问题:}对称半波振子每一臂的长度为波长的:
\begin{enumerate}[label=\Alph*), leftmargin=3em]
\item 4倍
\item 1/2倍
\item 1/4倍
\item 2倍
\end{enumerate}
\noindent\textbf{解说:}对称半波振子每一臂的长度为波长的\textbf{1/4倍}。\\\noindent\textbf{答案:}C

\bigskip


\noindent\textbf{问题:}对一个偶极子天线怎么做,才能让它的谐振频率升高一些?
\begin{enumerate}[label=\Alph*), leftmargin=3em]
\item 将振子截短一些
\item 在振子的两端加上电容帽
\item 将振子加长一些
\item 在振子某部位串联一个线圈
\end{enumerate}
\noindent\textbf{解说:}\textbf{将振子截短一些}能让一个偶极子天线的谐振频率升高一些。\\\noindent\textbf{答案:}A



\bigskip


\noindent\textbf{问题:}甲偶极天线的增益为6.15dBi,乙偶极天线的增益为1dBd。当两副天线按同样条件架设、用同样功率驱动时、在它们最大发射方向的同一远方地点接收时,两天线产生的信号功率的关系为:
\begin{enumerate}[label=\Alph*), leftmargin=3em]
\item 甲天线的信号功率为乙天线的1/2
\item 甲天线的信号功率为乙天线的两倍
\item 甲天线的信号功率为乙天线的5.15倍
\item 甲天线的信号功率为乙天线的6.15倍
\end{enumerate}
\noindent\textbf{解说:}两天线产生的信号功率的关系为\textbf{甲天线的信号功率为乙天线的两倍}。\\\noindent\textbf{答案:}B


\bigskip


\noindent\textbf{问题:}甲天线的增益为0dBd,乙天线的增益为2dBi。当两副天线按同样条件架设、用同样功率驱动时、在它们最大发射方向的同一远方地点接收并比较收到的信号功率强度,正确的说法为:
\begin{enumerate}[label=\Alph*), leftmargin=3em]
\item 甲、乙天线的效果实际相同
\item 甲天线效果为零,不能工作,乙天线效果比甲天线好2倍%。
\item 甲天线的效果与半波长偶极天线相当,乙天线发射的信号强度比甲天线好2dB%。
\item 甲天线的效果与半波长偶极天线相当,乙天线比甲天线略差%。
\end{enumerate}
\noindent\textbf{解说:}正确的说法为\textbf{甲天线的效果与半波长偶极天线相当,乙天线比甲天线略差}。\\\noindent\textbf{答案:}D



\bigskip


\noindent\textbf{问题:}业余条件测试天线增益的典型方法如图。用场强表或接收机接收设置在远处同一地点、最大辐射方向朝向自己的半波偶极天线(上)和待测天线(下)。调整送到两副天线的射频功率Po和P,使接收到的场强相同。待测天线的增益dBd值为:
\begin{enumerate}[label=\Alph*), leftmargin=3em]
\item 10 lg(P/Po)
\item P - Po
\item 10 lg(P-Po)
\item 10 lg(Po/P)
\end{enumerate}
\noindent\textbf{解说:}待测天线的增益dBd值为\textbf{10 lg(Po/P)}。\\\noindent\textbf{答案:}D


%LK0927.jpg
%%%???

\bigskip


\noindent\textbf{问题:}业余条件测试天线增益的典型方法如图。用场强表或接收机接收设置在远处同一地点、最大辐射方向朝向自己的半波偶极天线(上)和待测天线(下)。调整送到两副天线的射频功率Po和P,使接收到的场强相同。待测天线的增益dBi值为:
\begin{enumerate}[label=\Alph*), leftmargin=3em]
\item 20 lg(P/Po) + 2.15
\item 10 lg(Po/P) + 2.15
\item 10 lg(P-Po) + 2.15
\item 10 lg(P/Po) + 2.15
\end{enumerate}
\noindent\textbf{解说:}待测天线的增益dBi值为\textbf{10 lg(Po/P) + 2.15}。\\\noindent\textbf{答案:}B


%LK0928.jpg

\bigskip


\noindent\textbf{问题:}某业余电台使用半波长偶极天线发射时,对方接收机的信号强度指示为S4。现发射功率不变,发信端改用增益为 8.15 dBi的八木天线(最大辐射方向不变),对方接收的信号强度指示将变为:【提示:收信机信号强度指示S1至S9每档相差6dB】
\begin{enumerate}[label=\Alph*), leftmargin=3em]
\item S7
\item S5
\item S8
\item S6
\end{enumerate}
\noindent\textbf{解说:}对方接收的信号强度指示将变为\textbf{S5}。\\\noindent\textbf{答案:}B



\bigskip


\noindent\textbf{问题:}某业余电台使用半波长偶极天线发射时,对方接收机的信号强度指示为S4。现发射功率不变,发信端改用增益为 12 dBd的八木天线(最大辐射方向不变),对方接收的信号强度指示将变为:【提示:收信机信号强度指示S1至S9每档相差6dB】
\begin{enumerate}[label=\Alph*), leftmargin=3em]
\item S6
\item S5
\item S8
\item S7
\end{enumerate}
\noindent\textbf{解说:}对方接收的信号强度指示将变为\textbf{S6}。\\\noindent\textbf{答案:}A



\bigskip


\noindent\textbf{问题:}某业余电台使用半波长偶极天线发射时,对方亦使用半波长偶极天线接收,接收机的信号强度指示为S4。现发射功率不变,收发双方都改用增益为 8.15 dBi的八木天线(最大辐射方向不变),对方接收的信号强度指示将变为:【提示:收信机信号强度指示S1至S9每档相差6dB】
\begin{enumerate}[label=\Alph*), leftmargin=3em]
\item S4
\item S6
\item S5
\item S7
\end{enumerate}
\noindent\textbf{解说:}对方接收的信号强度指示将变为\textbf{S6}。\\\noindent\textbf{答案:}B


\bigskip


\noindent\textbf{问题:}甲、乙业余电台相距2000公里,均使用1/2波长水平偶极天线,正在HF频段进行稳定的通信。现其中一方改用1/2波长垂直偶极天线,改变前后的通信效果的比较将是:
\begin{enumerate}[label=\Alph*), leftmargin=3em]
\item 通信效果的变化不确定,取决于当时天波反射途中极化方向的旋转情况
\item 通信效果变好
\item 通信效果没有改变
\item 通信效果变差
\end{enumerate}
\noindent\textbf{解说:}改变前后的通信效果的比较将是\textbf{通信效果的变化不确定,取决于当时天波反射途中极化方向的旋转情况}。\\\noindent\textbf{答案:}A



\bigskip


\noindent\textbf{问题:}垂直偶极天线所发射的无线电波的极化方式为:
\begin{enumerate}[label=\Alph*), leftmargin=3em]
\item 右旋圆极化波
\item 水平极化波
\item 垂直极化波
\item 左旋圆极化波
\end{enumerate}
\noindent\textbf{解说:}垂直偶极天线所发射的无线电波的极化方式为\textbf{垂直极化波}。\\\noindent\textbf{答案:}C



\bigskip


\noindent\textbf{问题:}水平偶极天线所发射的无线电波的极化方式为:
\begin{enumerate}[label=\Alph*), leftmargin=3em]
\item 左旋圆极化波
\item 右旋圆极化波
\item 垂直极化波
\item 水平极化波
\end{enumerate}
\noindent\textbf{解说:}水平偶极天线所发射的无线电波的极化方式为\textbf{水平极化波}。\\\noindent\textbf{答案:}D



\bigskip


\noindent\textbf{问题:}假设收发天线均采用半波长偶极天线。在依靠电离层反射的远距离通信中,发射天线和接收天线的最佳极化方式为:
\begin{enumerate}[label=\Alph*), leftmargin=3em]
\item 收发天线的极化方向都平行于两台之间的连线
\item 不确定,根据具体传播情况而经常变化
\item 收发天线都处于垂直于两台连线的平面内收发天线极化方向互相垂直
\item 收发天线都处于垂直于两台连线的平面内并且极化方向互相一致
\end{enumerate}
\noindent\textbf{解说:}在依靠电离层反射的远距离通信中,发射天线和接收天线的最佳极化方式为\textbf{不确定,根据具体传播情况而经常变化}。\\\noindent\textbf{答案:}B




\bigskip


\noindent\textbf{问题:}短波水平偶极类天线(如偶极天线和八木天线等)的发射仰角主要由下列因素决定:
\begin{enumerate}[label=\Alph*), leftmargin=3em]
\item 由天线的辐射和大地的反射叠加造成,仰角高低与天线离地高度与波长的比值有关
\item 由八木天线主梁所指的方向决定
\item 由天线振子的长度所决定
\item 由天线振子导体所指的方向决定
\end{enumerate}
\noindent\textbf{解说:}短波水平偶极类天线(如偶极天线和八木天线等)的发射仰角主要由下列因素决定\textbf{由天线的辐射和大地的反射叠加造成,仰角高低与天线离地高度与波长的比值有关}。\\\noindent\textbf{答案:}A



\bigskip


\noindent\textbf{问题:}架设短波天线时,天线发射仰角的大致选择原则是:
\begin{enumerate}[label=\Alph*), leftmargin=3em]
\item 近距离通信选择低发射仰角,远距离通信选择高发射仰角
\item 远距离通信选择低发射仰角,近距离通信选择高发射仰角
\item 较低频率通信选择低发射仰角,较高频率通信选择高发射仰角
\item 近处开阔时选择低发射仰角,近处有建筑物时选择高发射仰角
\end{enumerate}
\noindent\textbf{解说:}架设短波天线时,天线发射仰角的大致选择原则是\textbf{远距离通信选择低发射仰角,近距离通信选择高发射仰角}。\\\noindent\textbf{答案:}B




\bigskip


\noindent\textbf{问题:}架设短波天线时,天线高度的大致选择原则是:
\begin{enumerate}[label=\Alph*), leftmargin=3em]
\item 近处有建筑物时选择较低的高度,近处开阔时选择较高的高度
\item 较低频率通信选择较高的高度,较高频率通信选择较低的高度
\item 远距离通信选择较高的高度,近距离通信选择较低的高度
\item 远距离通信选择较低的高度,近距离通信选择较高的高度
\end{enumerate}
\noindent\textbf{解说:}架设短波天线时,天线高度的大致选择原则是\textbf{远距离通信选择较高的高度,近距离通信选择较低的高度}。\\\noindent\textbf{答案:}C



\bigskip


\noindent\textbf{问题:}在针对特定对象的DX通信中,计算天线最佳发射仰角的基本方法是:
\begin{enumerate}[label=\Alph*), leftmargin=3em]
\item 根据通信对象所在的方位、通信方向上障碍物所遮挡的仰角、本台周围的大地导电率、实际工作频率,找公式计算
\item 根据所使用电离层的大致高度、通信对象的大致距离、电波在传播途经中经电离层反射的次数,用简单几何方法计算
\item 根据通信对象所在的方位、地球半径、对方天线高度、实际工作频率、太阳平均黑子数,查表计算
\item 根据通信双方的发射功率、天线极化方向、通信方向上障碍物所遮挡的仰角、太阳10.7 cm 射电流量,找公式计算
\end{enumerate}
\noindent\textbf{解说:}在针对特定对象的DX通信中,计算天线最佳发射仰角的基本方法是\textbf{根据所使用电离层的大致高度、通信对象的大致距离、电波在传播途经中经电离层反射的次数,用简单几何方法计算}。\\\noindent\textbf{答案:}B



\bigskip


\noindent\textbf{问题:}通过目视判断全尺寸八木天线发射方向的办法是:
\begin{enumerate}[label=\Alph*), leftmargin=3em]
\item 比主振子短者为引向振子,比主振子长者为反射振子,反射振子朝向最大辐射方向
\item 比主振子短者为引向振子,比主振子长者为反射振子,引向振子朝向最大辐射方向
\item 主振子两端所指方向为最大辐射方向
\item 比主振子长者为引向振子,比主振子短者为反射振子,引向振子朝向最大辐射方向
\end{enumerate}
\noindent\textbf{解说:}通过目视判断全尺寸八木天线发射方向的办法是\textbf{比主振子短者为引向振子,比主振子长者为反射振子,引向振子朝向最大辐射方向}。\\\noindent\textbf{答案:}B

\bigskip


\noindent\textbf{问题:}北京的水平极化半波长偶极天线,通信对象为纽约的业余电台。按电波的最短传输途径考虑,天线的最佳走向应大致为:
\begin{enumerate}[label=\Alph*), leftmargin=3em]
\item 西偏南30度-东偏北30度
\item 东偏南30度-西偏北30度
\item 东-西
\item 南-北
\end{enumerate}
\noindent\textbf{解说:}按电波的最短传输途径考虑,天线的最佳走向应大致为\textbf{东-西}。\\\noindent\textbf{答案:}C



\bigskip


\noindent\textbf{问题:}在导电良好的地面上,决定短波天线辐射仰角的主要参数是:
\begin{enumerate}[label=\Alph*), leftmargin=3em]
\item 天线离地面的相对于波长的高度,即离地高度除以波长
\item 天线的绝对高度,与波长无关
\item 天线离海平面的绝对高度
\item 天线导线或者八木天线主梁与地面之间的夹角
\end{enumerate}

\bigskip


\noindent\textbf{问题:}什么是八木天线?
\begin{enumerate}[label=\Alph*), leftmargin=3em]
\item 一种可以将接收到的信号反向的天线
\item 一种可以集中聚集某一方向信号的天线
\item 一种由日本八木秀次发明的全向天线
\item 用八根木头制作的天线
\end{enumerate}
\noindent\textbf{解说:}八木天线是\textbf{一种可以集中聚集某一方向信号的天线}。\\\noindent\textbf{答案:}B




\bigskip


\noindent\textbf{问题:}在自由空间中的半波偶极子天线,哪个方向的辐射强度最大?
\begin{enumerate}[label=\Alph*), leftmargin=3em]
\item 沿着馈线的方向
\item 垂直于导体的方向
\item 各个方向强度相同
\item 沿着导体的方向
\end{enumerate}
\noindent\textbf{解说:}在自由空间中的半波偶极子天线,\textbf{垂直于导体的方向}方向的辐射强度最大。\\\noindent\textbf{答案:}B



\bigskip


\noindent\textbf{问题:}同样材料、同样直径、同样长度的实心铜线和空心铜管,在交流电路中的发热损耗情况为:
\begin{enumerate}[label=\Alph*), leftmargin=3em]
\item 在低频率下实心铜线损耗较大,在高频率下两者损耗一样
\item 在各种频率下两者耗差都一样
\item 在低频率下实心铜线损耗较小,在高频率下两者损耗一样
\item 在不同频率两者的发热损耗大小不好说,取决于具体散热条件
\end{enumerate}
\noindent\textbf{解说:}同样材料、同样直径、同样长度的实心铜线和空心铜管,在交流电路中的发热损耗情况为\textbf{在低频率下实心铜线损耗较小,在高频率下两者损耗一样}。\\\noindent\textbf{答案:}C

\bigskip


\noindent\textbf{问题:}把实心导线接到频率为数十兆赫兹的高频率射频电路中,则会有下列现象:
\begin{enumerate}[label=\Alph*), leftmargin=3em]
\item 表层的电流沿导线方向流动,内层电流形成螺旋状涡流
\item 导线外层和内层都有电流,但两者方向相反
\item 电流集中在导线表层,导线内部没有电流
\item 导线截面各处的电流密度均匀分布
\end{enumerate}
\noindent\textbf{解说:}把实心导线接到频率为数十兆赫兹的高频率射频电路中,则会有下列现象\textbf{电流集中在导线表层,导线内部没有电流}。\\\noindent\textbf{答案:}C


\bigskip


\noindent\textbf{问题:}工作在高频率下的射频部件积灰或受潮后,即使没有漏电,也可能因绝缘物体的物理变化而带来额外的:
\begin{enumerate}[label=\Alph*), leftmargin=3em]
\item 介质损耗
\item 磁滞损耗
\item 磁阻损耗
\item 涡流损耗
\end{enumerate}
\noindent\textbf{解说:}工作在高频率下的射频部件积灰或受潮后,即使没有漏电,也可能因绝缘物体的物理变化而带来额外的\textbf{介质损耗}。\\\noindent\textbf{答案:}A



\bigskip


\noindent\textbf{问题:}天线和馈线之间经常接一个俗称“巴伦(BALUN)”的部件。“巴伦”的由来是:
\begin{enumerate}[label=\Alph*), leftmargin=3em]
\item 著名天线阻抗匹配理论家的名字
\item 发明平衡不平衡转换器的人的名字
\item 宽带阻抗变压器的英文缩写
\item 平衡和不平衡两个英文字头的组合
\end{enumerate}
\noindent\textbf{解说:}“巴伦”的由来是\textbf{平衡和不平衡两个英文字头的组合}。\\\noindent\textbf{答案:}D




\bigskip


\noindent\textbf{问题:}天线和馈线之间经常接一个俗称“巴伦(BALUN)”的部件。它的主要功能是:
\begin{enumerate}[label=\Alph*), leftmargin=3em]
\item 展宽天线的工作频带
\item 降低天线的驻波比
\item 实现天线和馈线之间的自动阻抗匹配
\item 在平衡电路和不平衡电路之间传递射频能量,并阻断两者之间的任何寄生耦合
\end{enumerate}
\noindent\textbf{解说:}“巴伦”的主要功能是\textbf{在平衡电路和不平衡电路之间传递射频能量,并阻断两者之间的任何寄生耦合}。\\\noindent\textbf{答案:}D




\bigskip


\noindent\textbf{问题:}同轴电缆的绝缘介质相同时,影响特性阻抗的因素是:
\begin{enumerate}[label=\Alph*), leftmargin=3em]
\item 外导体内径越大,特性阻抗越高
\item 内导体外径越大,特性阻抗越高
\item 外导体内径和内导体外径的比越大,特性阻抗越高
\item 电缆的长度越长,特性阻抗越高
\end{enumerate}
\noindent\textbf{解说:}同轴电缆的绝缘介质相同时,影响特性阻抗的因素是\textbf{外导体内径和内导体外径的比越大,特性阻抗越高}。\\\noindent\textbf{答案:}C



\bigskip


\noindent\textbf{问题:}天线调谐器(俗称“天调”)的作用是:
\begin{enumerate}[label=\Alph*), leftmargin=3em]
\item 对天馈系统进行优化调谐,使整个系统的辐射功率获得一个附加的增益
\item 对不匹配的天馈系统进行补偿,使整个天馈系统的传输和辐射效率达到匹配天线的水平
\item 对不匹配的天馈系统进行补偿,虽然不能改善馈线的损耗,但能使天线本身的辐射效率达到匹配天线的水平
\item 补偿不匹配系统,向收发信机提供谐振的、阻抗匹配的负载,但不能改善天线本身的辐射效率
\end{enumerate}
\noindent\textbf{解说:}天线调谐器(俗称“天调”)的作用是\textbf{补偿不匹配系统,向收发信机提供谐振的、阻抗匹配的负载,但不能改善天线本身的辐射效率}。\\\noindent\textbf{答案:}D



\bigskip


\noindent\textbf{问题:}天线通过50欧同轴馈线与输出阻抗为50欧的收发信机相连接,并打算在天线电路中串入天线调谐器和通过式驻波功率计来监测和补偿天线的失配。理论上最理想的连接顺序为:
\begin{enumerate}[label=\Alph*), leftmargin=3em]
\item 天线-驻波功率计-天线调谐器-馈线-收发信机
\item 天线-天线调谐器-驻波功率计-馈线-收发信机
\item 天线-馈线-天线调谐器-驻波功率计-收发信机
\item 天线-天线调谐器-馈线-驻波功率计-收发信机
\end{enumerate}
\noindent\textbf{解说:}理论上最理想的连接顺序为\textbf{天线-天线调谐器-驻波功率计-馈线-收发信机}。\\\noindent\textbf{答案:}B




\bigskip


\noindent\textbf{问题:}塔上的天线通过50欧同轴馈线与输出阻抗为50欧的收发信机相连接,在天线电路中串入天线调谐器ATU和通过式驻波功率计M来监测和补偿天线的失配。有四种方案:1、ATU和M均在塔顶,2、ATU和M均在塔底,3、ATU在塔底、M在机房,4、ATU和M均在机房。当ATU调到最佳状态时,各方案按天线系统发射效率由高到低的排序为:
\begin{enumerate}[label=\Alph*), leftmargin=3em]
\item 方案4最好,方案2、3其次,方案1最差
\item 方案1最好,方案2、3其次,方案4最差
\item 方案3最好,方案4其次,方案2再其次,方案1最差
\item 方案2最好,方案1其次,方案4再其次,方案3最差
\end{enumerate}
\noindent\textbf{解说:}当ATU调到最佳状态时,各方案按天线系统发射效率由高到低的排序为\textbf{方案1最好,方案2、3其次,方案4最差}。\\\noindent\textbf{答案:}B

%%%???
%LK0938.jpg

\bigskip


\noindent\textbf{问题:}下列哪一种导体最适合射频接地使用?
\begin{enumerate}[label=\Alph*), leftmargin=3em]
\item 双绞线
\item 圆形多股线
\item 镀银软铜丝编织扁带
\item 圆形铜包钢单股线
\end{enumerate}
\noindent\textbf{解说:}\textbf{镀银软铜丝编织扁带}最适合射频接地使用。\\\noindent\textbf{答案:}C



\bigskip


\noindent\textbf{问题:}如果在驻波表上读到了4:1,这意味着?
\begin{enumerate}[label=\Alph*), leftmargin=3em]
\item 良好的阻抗匹配
\item 天线有4dB的损失
\item 阻抗匹配得不好
\item 天线的增益是4dB
\end{enumerate}
\noindent\textbf{解说:}如果在驻波表上读到了4:1,这意味着\textbf{阻抗匹配得不好}。\\\noindent\textbf{答案:}C



\bigskip


\noindent\textbf{问题:}收发信机天线调谐器(天调)的作用是什么?
\begin{enumerate}[label=\Alph*), leftmargin=3em]
\item 它可以使一部天线既能作为发射天线,又能作为接收天线
\item 它能够依照发射机当前工作的波段自动连接合适的天线
\item 它将发射机的输出阻抗和天线的输入阻抗进行良好的匹配
\item 它帮助接收机在遇到弱信号时能自动调准频率
\end{enumerate}
\noindent\textbf{解说:}收发信机天线调谐器(天调)的作用是\textbf{它将发射机的输出阻抗和天线的输入阻抗进行良好的匹配}。\\\noindent\textbf{答案:}C



\bigskip


\noindent\textbf{问题:}影响短波电离层传播的主要因素有:
\begin{enumerate}[label=\Alph*), leftmargin=3em]
\item 当地天气
\item 太阳黑子活动、太阳耀斑活动和地磁活动
\item 发射机功率
\item 地表温度
\end{enumerate}
\noindent\textbf{解说:}影响短波电离层传播的主要因素有\textbf{太阳黑子活动、太阳耀斑活动和地磁活动}。\\\noindent\textbf{答案:}B



\bigskip


\noindent\textbf{问题:}影响短波电离层传播的主要因素有:
\begin{enumerate}[label=\Alph*), leftmargin=3em]
\item 当地地面气压
\item 接收天线增益
\item 季节和昼夜
\item 发射机功率
\end{enumerate}
\noindent\textbf{解说:}影响短波电离层传播的主要因素有\textbf{季节和昼夜}。\\\noindent\textbf{答案:}C



\bigskip


\noindent\textbf{问题:}影响短波电离层传播的主要因素有:
\begin{enumerate}[label=\Alph*), leftmargin=3em]
\item 对流层气压
\item 发射机功率
\item 工作频率和通信距离
\item 高空云量
\end{enumerate}
\noindent\textbf{解说:}影响短波电离层传播的主要因素有\textbf{工作频率和通信距离}。\\\noindent\textbf{答案:}C



\bigskip


\noindent\textbf{问题:}对短波电离层传播发生主要影响的各电离层按高度自低到高分别称为:
\begin{enumerate}[label=\Alph*), leftmargin=3em]
\item D、E、F1、F2
\item F2、F1、E2、E1
\item A、B、C、D
\item F、E2、E1、D
\end{enumerate}
\noindent\textbf{解说:}对短波电离层传播发生主要影响的各电离层按高度自低到高分别称为\textbf{D、E、F1、F2}。\\\noindent\textbf{答案:}A



\bigskip


\noindent\textbf{问题:}各电离层对短波电离层传播所起的主要影响为:
\begin{enumerate}[label=\Alph*), leftmargin=3em]
\item D、E层可反射电波,F1、F2层不能反射但衰减电波
\item F1、E层可反射电波,D、F2层不能反射但衰减电波
\item F2、D层可反射电波,E、F1层不能反射但衰减电波
\item F2、F1、E层可反射电波,D层不能反射但衰减电波
\end{enumerate}
\noindent\textbf{解说:}\textbf{F2、F1、E层可反射电波,D层不能反射但衰减电波}。\\\noindent\textbf{答案:}D



\bigskip


\noindent\textbf{问题:}太阳黑子活动的平均周期约为:
\begin{enumerate}[label=\Alph*), leftmargin=3em]
\item 38年
\item 6.7年
\item 11.2年
\item 5.7年
\end{enumerate}
\noindent\textbf{解说:}太阳黑子活动的平均周期约为\textbf{11.2年}。\\\noindent\textbf{答案:}C



\bigskip


\noindent\textbf{问题:}太阳黑子活动的强弱用“太阳黑子平均数(SSN)”来描述。一般说来:
\begin{enumerate}[label=\Alph*), leftmargin=3em]
\item 只有在发生太阳耀斑的情况下SSN 才会影响短波远程通信效果
\item SSN与短波远程通信效果无直接关系
\item SSN小,有利于短波远程通信
\item SSN大,有利于短波远程通信
\end{enumerate}
\noindent\textbf{解说:}一般说来\textbf{SSN大,有利于短波远程通信}。\\\noindent\textbf{答案:}D


\bigskip


\noindent\textbf{问题:}当最高可用频率(MUF)为20MHz时,具有较大DX通联机会的业余频段是:
\begin{enumerate}[label=\Alph*), leftmargin=3em]
\item 18MHz
\item 21MHz
\item 24MHz
\item 14MHz
\end{enumerate}
\noindent\textbf{解说:}当最高可用频率(MUF)为20MHz时,具有较大DX通联机会的业余频段是\textbf{18MHz}。\\\noindent\textbf{答案:}A



\bigskip


\noindent\textbf{问题:}太阳耀斑引起的电离层扰动(SID)对短波通信的影响是:
\begin{enumerate}[label=\Alph*), leftmargin=3em]
\item 地球黑夜一面受到的影响会超过白昼面
\item 卫星通信受到的影响会超过地面直射波通信
\item 高纬度的传播路径影响会超过低纬度
\item 低频率受到的影响超过高频率
\end{enumerate}
\noindent\textbf{解说:}太阳耀斑引起的电离层扰动(SID)对短波通信的影响是\textbf{低频率受到的影响超过高频率}。\\\noindent\textbf{答案:}D



\bigskip


\noindent\textbf{问题:}HF频段远距离通信主要依靠下列传播方式:
\begin{enumerate}[label=\Alph*), leftmargin=3em]
\item 电离层反射
\item 中继台网转发
\item 对流层散射传播
\item 业余卫星转发
\end{enumerate}
\noindent\textbf{解说:}HF频段远距离通信主要依靠\textbf{电离层反射}传播。\\\noindent\textbf{答案:}A


\bigskip


\noindent\textbf{问题:}“静寂区”或者“越距”是指:
\begin{enumerate}[label=\Alph*), leftmargin=3em]
\item HF频段天波和地波都传播不到的中间区域
\item VHF/UHF频段超过视距电波传播不到的区域
\item VHF/UHF频段视距范围内但受障碍物阻挡电波传播不到的区域
\item 卫星通信中覆盖区以外电波传播不到的区域
\end{enumerate}
\noindent\textbf{解说:}“静寂区”或者“越距”是指\textbf{HF频段天波和地波都传播不到的中间区域}。\\\noindent\textbf{答案:}A



\bigskip


\noindent\textbf{问题:}大气层中的哪一部分使得无线电信号可以在全世界范围内传播?
\begin{enumerate}[label=\Alph*), leftmargin=3em]
\item 对流层
\item 电离层
\item 磁层
\item 平流层
\end{enumerate}
\noindent\textbf{解说:}大气层中的\textbf{电离层}使得无线电信号可以在全世界范围内传播。\\\noindent\textbf{答案:}B



\bigskip