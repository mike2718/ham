\chapter*{编著者的话}

无线电资源是全人类共同的财产。提到无线电,我们再熟悉不过的是日常生活中的手机和Wi-Fi,在军事上,人们利用无线电控制导弹、飞机,在救险活动上,人们利用无线电辅助实施灾害时的救援,在业余无线电领域,爱好者们互相通信,以提高技能,同时学习新知识。

笔者原本对业余无线电一无所知,因为在精通业余无线电的朋友的怂恿下,于2013年左右,笔者才逐渐开始对业余无线电感兴趣。笔者于2017年在日本横滨留学期间,通过考试考取了日本的三级业余无线电操作员执照,建立了第一个属于自己的业余无线电台,获得了人生中的第一个呼号——\texttt{JJ1}前缀的呼号,开始了持证业余无线电爱好者的旅途。归国之后,笔者于2021年获得了\texttt{BG7}前缀的呼号。

在归国后,笔者通过了国内业余无线电操作证考试,于2020年和2021年先后获得了国内A类和B类的业余电台操作证书。在业余电台操作证书考试应试学习过程中,笔者深深感到,国内现有的操作证考试应试书籍对于很多小学生读者来说,缺乏细致的解释,题目里的术语艰涩难懂,计算题不知道如何计算,用这些书籍学习的读者,想必难以通过操作证考试。在这样的背景下,笔者萌生了撰写一本老少皆能读懂的操作证考试的应试书籍的想法。

本书在写作过程中,为了让业余无线电知识几乎完全不了解的初学者也能读懂,笔者经过了反复的推敲,尽可能的把复杂的业余无线电知识简单易懂地展现给读者们。本书在解题的过程中,适当地介绍相关的术语,并把重点难点用加粗的字体标出,方便应试者快速记忆概念、理解计算方法。

% 希望本书能帮助您顺利通过考试。
% 关我啥事

% 咋的,往后就编不出来了?
