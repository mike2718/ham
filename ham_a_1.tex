\chapter{无线电法律}

\textbf{问题:}我国现行法律体系中专门针对无线电管理的最高法律文件及其立法机关是:

\begin{enumerate}[label=\Alph*), leftmargin=1cm]
	\item 中华人民共和国业余无线电台管理办法,工业和信息化部
	\item 中华人民共和国无线电管理办法,工业和信息化部
	\item 中华人民共和国电信条例,国务院
	\item 中华人民共和国无线电管理条例,国务院和中央军委
\end{enumerate}

\textbf{解说:}我国专门针对无线电管理的最高法律文件为《中华人民共和国无线电管理条例》,由中华人民共和国国务院和中华人民共和国中央军事委员会发布。

\textbf{答案:}D

\textbf{问题:}我国现行法律体系中专门针对业余无线电台管理的最高法律文件及其立法机关是:

\begin{enumerate}[label=\Alph*), leftmargin=1cm]
	\item 业余无线电台管理办法,工业和信息化部
	\item 个人业余无线电台管理暂行办法,国家体委和国家无委
	\item 业余无线电台管理暂行规定,国家体委和国家无委
	\item 中华人民共和国电信条例,国务院
\end{enumerate}

\textbf{解说:}我国专门针对无线电管理的最高法律文件为《中华人民共和国无线电管理条例》,其立法机关为中华人民共和国国务院和中华人民共和国中央军事委员会。

\textbf{答案:}D

\textbf{问题:}我国的无线电主管部门是:

\begin{enumerate}[label=\Alph*), leftmargin=1cm]
	\item 各级无线电管理机构
	\item 各级体育管理机构
	\item 各地业余无线电协会
	\item 各地电信管理局
\end{enumerate}

\textbf{解说:}根据《业余无线电台管理办法》第三条:\textbf{国家无线电管理机构和省、自治区、直辖市无线电管理机构}依法对业余无线电台实施监督管理。国家无线电管理机构和地方无线电管理机构统称无线电管理机构。\\\textbf{答案:}A

\textbf{问题:}我国依法负责对业余无线电台实施监督管理的机构是

\begin{enumerate}[label=\Alph*), leftmargin=1cm]
	\item 国家无线电管理机构和地方无线电管理机构
	\item 在国家或地方民政部门注册的业余无线电协会
	\item 国家体育管理机构和地方体育管理机构
	\item 国家和地方公安部门
\end{enumerate}

\textbf{解说:}根据《业余无线电台管理办法》第三条:国家无线电管理机构和省、自治区、直辖市无线电管理机构依法对业余无线电台实施监督管理。\textbf{国家无线电管理机构和地方无线电管理机构}统称无线电管理机构。\\\textbf{答案:}A

\textbf{问题:}《业余无线电台管理办法》所说的“地方无线电管理机构”指的是:

\begin{enumerate}[label=\Alph*), leftmargin=1cm]
	\item 省、自治区、直辖市无线电管理机构
	\item 地方业余无线电协会或者类似组织机构
	\item 地市县(区)及以下各级无线电管理机构
	\item 各地方与无线电设备生产销售和无线电应用有关的行政管理机构
\end{enumerate}

\textbf{解说:}根据《业余无线电台管理办法》第三条:国家无线电管理机构和\textbf{省、自治区、直辖市无线电管理机构}(以下简称地方无线电管理机构)依法对业余无线电台实施监督管理。\\\textbf{答案:}A

\textbf{问题:}国家鼓励和支持业余无线电台开展下列活动:

\begin{enumerate}[label=\Alph*), leftmargin=1cm]
	\item 无线电通信技术研究、普及活动以及突发重大自然灾害等紧急情况下的应急通信活动
	\item 休闲娱乐性交谈
	\item 机动车辆行车服务性通信活动
	\item 作为日常公益活动的通信工具
\end{enumerate}

\textbf{解说:}根据《业余无线电台管理办法》第四条:国家鼓励和支持\textbf{业余无线电通信技术的研究、普及和突发重大自然灾害等紧急情况下的应急无线电通信活动。}\\\textbf{答案:}A

\textbf{问题:}关于业余电台管理的正确说法是:

\begin{enumerate}[label=\Alph*), leftmargin=1cm]
	\item 依法设置的业余无线电台受国家法律保护
	\item 业余无线电爱好者的一切行为都受国家法律保护
	\item 通过法律手段限制业余无线电台的设置
	\item 在业余电台与其他业务电台遇到干扰纠纷时无条件优先保护其他业务电台
\end{enumerate}

\textbf{解说:}根据《业余无线电台管理办法》第四条:\textbf{依法设置的业余无线电台受国家法律保护。}\\\textbf{答案:}A

\textbf{问题:}无线电频率的使用必须得到各级无线电管理机构的批准,基本依据是“无线电频谱资源属于国家所有”,出自于下列法律:

\begin{enumerate}[label=\Alph*), leftmargin=1cm]
	\item 中华人民共和国物权法
	\item 中华人民共和国民法通则
	\item 中华人民共和国刑法
	\item 中华人民共和国电信法
\end{enumerate}

\textbf{解说:}根据《\textbf{中华人民共和国物权法}》第五十条:无线电频谱资源属于国家所有。\\\textbf{答案:}A

\textbf{问题:}我国对无线电管理术语“业余业务”、“卫星业余业务”和“业余无线电台”做出具体定义的法规文件是:
\begin{enumerate}[label=\Alph*), leftmargin=1cm]
	\item 中华人民共和国无线电频率划分规定
	\item 中华人民共和国无线电管理条例
	\item 中华人民共和国电信条例
	\item 无线电台执照管理规定
\end{enumerate}
\textbf{解说:《中华人民共和国无线电频率划分规定》}1.3.39、1.3.40和1.4.38分别定义了“业余业务”、“卫星业余业务”和“业余无线电台”。\\\textbf{答案:}A

\textbf{问题:}业余电台的法定用途为:
\begin{enumerate}[label=\Alph*), leftmargin=1cm]
	\item 供业余无线电爱好者进行自我训练、相互通信和技术研究
	\item 供公民在业余时间进行与个人生活事务有关的通信
	\item 供公民在业余时间进行休闲娱乐
	\item 供私家车主或者相应组织作为行车安全保障和途中消遣工具
\end{enumerate}
\textbf{解说:}根据《业余无线电台管理办法》第二十八条:业余无线电台供其设置人、使用人用于\textbf{相互通信、技术研究和自我训练。}\\\textbf{答案:}A

\textbf{问题:}无线电业余业务是供业余无线电爱好者作下列用途的无线电通信业务:
\begin{enumerate}[label=\Alph*), leftmargin=1cm]
	\item 自我训练、相互通信和技术研究
	\item 救灾抢险、车队联络和技术学习
	\item 娱乐休闲、报告路况和公益服务
	\item 技术教学、民兵训练和公益通信
\end{enumerate}
\textbf{解说:}《中华人民共和国无线电频率划分规定》1.3.39对业余无线电业务的定义:供业余无线电爱好者进行\textbf{自我训练、相互通信和技术研究}的无线电通信业务。\\\textbf{答案:}A

\textbf{问题:}业余无线电台供下列人群设置和使用:
\begin{enumerate}[label=\Alph*), leftmargin=1cm]
	\item 业余无线电爱好者,即经正式批准的、对无线电技术有兴趣的人,其兴趣纯系个人爱好而不涉及谋取利润
	\item 业余无线电爱好者,即任何对无线电技术有兴趣的人
	\item 对无线电技术不感兴趣,但希望把业余无线电台用于业余消遣的公民
	\item 对用无线电台解决日常通信有实际需求的任何公民、社团和单位
\end{enumerate}
\textbf{解说:}《中华人民共和国无线电频率划分规定》1.3.39对业余无线电爱好者的定义:\textbf{业余无线电爱好者系指经正式批准的、对无线电技术有兴趣的人,其兴趣纯系个人爱好而不涉及谋取利润。}\\\textbf{答案:}A

\textbf{问题:}“我不是业余无线电爱好者,申请设置业余电台只是为了行车方便,不需要遵守业余无线电的规范”。这种说法:
\begin{enumerate}[label=\Alph*), leftmargin=1cm]
	\item 是错误的,也是不具备“熟悉无线电管理规定”设台条件的表现
	\item 有一定道理,既然行车通信有需求,法规管理应该迎合个人需求
	\item 有一定道理,只要是遵守规定,业余电台也可以为非业余无线电爱好者所用
	\item 很难说对错,业余电台的定义可以因人而异
\end{enumerate}
\textbf{解说:}根据《业余无线电台管理办法》第十八条:使用业余无线电台,应当具备下列条件:(一)\textbf{熟悉无线电管理规定。}申请设置业余电台需要遵守无线电管理规定并取得相应的操作技术能力,因此题目这种说法是错误的。\\\textbf{答案:}A

\textbf{问题:}个人提出设置使用业余无线电台申请,就是表示自己对无线电技术发生了兴趣,确认了自己在有关业余无线电台活动中的身份是:
\begin{enumerate}[label=\Alph*), leftmargin=1cm]
	\item 业余无线电爱好者,但可以是正在起步的初学者
	\item 汽车俱乐部会员,但不是业余无线电爱好者
	\item 旅游爱好者,但不是业余无线电爱好者
	\item 其他职业人员,但不是业余无线电爱好者
\end{enumerate}
\textbf{解说:}《中华人民共和国无线电频率划分规定》1.3.39对业余无线电爱好者的定义:\textbf{业余无线电爱好者系指经正式批准的、对无线电技术有兴趣的人,其兴趣纯系个人爱好而不涉及谋取利润。}\\\textbf{答案:}A

\textbf{问题:}符合业余无线电爱好者基本条件的人群是:
\begin{enumerate}[label=\Alph*), leftmargin=1cm]
	\item 对无线电技术有兴趣并经无线电管理机构批准设置使用业余无线电台的人
	\item 任何对无线电技术有兴趣的公民
	\item 对无线电技术有兴趣并加入业余无线电协会的人
	\item 拥有较高无线电技术水平并加入业余无线电协会的人
\end{enumerate}
\textbf{解说:}《中华人民共和国无线电频率划分规定》1.3.39对业余无线电爱好者的定义:\textbf{业余无线电爱好者系指经正式批准的、对无线电技术有兴趣的人,其兴趣纯系个人爱好而不涉及谋取利润。}\\\textbf{答案:}A

\textbf{问题:}不同类别业余无线电台的主要区别在于:
\begin{enumerate}[label=\Alph*), leftmargin=1cm]
	\item 允许发射的频率范围和最大发射功率
	\item 所用业余无线电台设备的功能和价格
	\item 设置和操作人员的业余无线电知识和技术水平
	\item 所用业余无线电台的天线的高度和长度
\end{enumerate}
\textbf{解说:}不同类别业余无线电台的主要区别在于\textbf{允许发射的频率范围和最大发射功率。}\\\textbf{答案:}A

\textbf{问题:}A类业余无线电台允许发射的发射频率为:
\begin{enumerate}[label=\Alph*), leftmargin=1cm]
	\item 30-3000MHz范围内的各业余业务和卫星业余业务频段
	\item 各VHF和UHF频段
	\item 各业余业务和卫星业余业务频段
	\item 所有VHF和UHF频段
\end{enumerate}
\textbf{解说:}根据《工信部无201343号通知》:A类业余无线电台可以在\textbf{30-3000MHz范围内的各业余业务和卫星业余业务频段内}发射工作,且最大发射功率不大于25瓦。\\\textbf{答案:}A

\textbf{问题:}A类业余无线电台允许发射的最大发射功率为不大于:
\begin{enumerate}[label=\Alph*), leftmargin=1cm]
	\item 25瓦
	\item 100瓦
	\item 30MHz以下业余频段不大于100瓦,30MHz以上业余频段不大于25瓦
	\item 30MHz以上业余频段不大于100瓦,30MHz以下业余频段不大于25瓦
\end{enumerate}
\textbf{解说:}根据《工信部无201343号通知》:A类业余无线电台可以在30-1000MHz范围内的各业余业务和卫星业余业务频段内发射工作,且最大发射功率不大于\textbf{25瓦}。\\\textbf{答案:}A

\textbf{问题:}个人申请设置具有发信功能的业余无线电台的年龄条件是:
\begin{enumerate}[label=\Alph*), leftmargin=1cm]
	\item 年满十八周岁
	\item 年满十六周岁
	\item 年满十四周岁
	\item 具备《业余无线电台操作证书》者申请设置业余无线电台不受年龄限制
\end{enumerate}
\textbf{解说:}根据《业余无线电台管理办法》第六条:个人申请设置具有发信功能的业余无线电台的,应当\textbf{年满十八周岁}。\\\textbf{答案:}A

\textbf{问题:}独立操作具有发信功能业余无线电台的年龄条件是:
\begin{enumerate}[label=\Alph*), leftmargin=1cm]
	\item 具备《业余无线电台操作证书》者操作业余无线电台不受年龄限制
	\item 年满十六周岁
	\item 年满十四周岁
	\item 年满十八周岁
\end{enumerate}
\textbf{解说:}根据《工信部无201343号通知》,各类操作证书可作为设置、使用该类业余无线电台的操作技术能力证明,取得证书者年龄不受限制,故选A。\\\textbf{答案:}A

\textbf{问题:}申请设置业余无线电台应当具备的条件有:
\begin{enumerate}[label=\Alph*), leftmargin=1cm]
	\item 熟悉无线电管理规定、具备国家规定的操作技术能力、发射设备符合国家技术标准、法律和行政法规规定的其他条件
	\item 加入指定协会、具备当地无线电管理机构规定的操作技术能力、发射设备符合国家技术标准、法律和行政法规规定的其他条件
	\item 熟悉无线电管理规定、具备国家规定的操作技术能力、发射设备符合国家技术标准、当地无线电管理机构委托的受理机构设置的其他条件
	\item 熟悉无线电管理规定、具备当地无线电管理机构委托的考试机构设置的操作技术能力标准、发射设备符合国家技术标准、法律和行政法规规定的其他条件
\end{enumerate}
\textbf{解说:}根据《业余无线电台管理办法》第六条:申请设置业余无线电台,应当具备下列条
件:\textbf{(一)熟悉无线电管理规定;(二)具备国家无线电管理机构规定的操作技术能力;(三)无线电发射设备符合国家相关技术标准;(四)法律、行政法规规定的其他条件。}\\\textbf{答案:}A

\textbf{问题:}使用业余无线电台应当具备的条件有:
\begin{enumerate}[label=\Alph*), leftmargin=1cm]
	\item 熟悉无线电管理规定、具备国家规定的操作技术能力并取得相应操作技术能力证明
	\item 使用具有发信功能的业余无线电台的,应当年满十八周岁
	\item 具备国家或地方无线电管理机构核发的业余无线电台执照
	\item 熟悉无线电管理规定、实际上具备国家规定的操作技术能力但不必需取得相应的证明
\end{enumerate}
\textbf{解说:}根据《业余无线电台管理办法》第十八条:\textbf{使用业余无线电台,应当具备下列条件:(一)熟悉无线电管理规定;(二)具备国家无线电管理机构规定的操作技术能力,取得相应操作技术能力证明。}\\\textbf{答案:}A

%%%%%%%%%%%%%%%%%%%%%%%%%% 有疑问 %%%%%%%%%%%%%%%%%%%%%%%%%%%%%%%
\textbf{问题:}负责组织A类和B类业余无线电台所需操作技术能力的验证的机构是:
\begin{enumerate}[label=\Alph*), leftmargin=1cm]
	\item 地方无线电管理机构(或其委托单位)
	\item 国家无线电管理机构和地方无线电管理机构(或其委托单位)
	\item 地方教育、体育机构及其相关民间组织
	\item 地方业余无线电协会
\end{enumerate}
\textbf{解说:}根据《工信部无201343号通知》:\textbf{各省、自治区、直辖市无线电管理机构或其委托的机构}负责组织对A类和B类业余无线电台所需操作技术能力进行验证,以及核发验证合格证明等工作。故选A。\\\textbf{答案:}A
%%%%%%%%%%%%%%%%%%%%%%%%%% 有疑问 %%%%%%%%%%%%%%%%%%%%%%%%%%%%%%%

\textbf{问题:}2013年1月1日以后新获得的各类业余无线电台操作技术能力证明文件是:
\begin{enumerate}[label=\Alph*), leftmargin=1cm]
	\item 中国无线电协会颁发的“业余无线电台操作证书”
	\item 中国无线电运动协会颁发的“业余无线电台操作证书”
	\item 地方无线电协会或者其他业余无线电民间组织颁发的“业余无线电台操作证书”
	\item 地方无线电协会或者其他业余无线电民间组织出具的盖有公章的证明信件
\end{enumerate}
\textbf{解说:}根据《国无协[2013]1号》第一项:业余无线电台操作技术能力的验证考核通过闭卷考试等形式进行。验证合格证明为《\textbf{中国无线电协会业余电台操作证书}》(下简称《操作证书》),由\textbf{中国无线电协会}统一印制和编号。\\\textbf{答案:}A

\textbf{问题:}申请设置和使用业余无线电台的条件所规定的“具备国家规定的操作技术能力”,其标志为:
\begin{enumerate}[label=\Alph*), leftmargin=1cm]
	\item 取得相应操作技术能力证明,即中国无线电协会颁发的业余无线电台操作证书
	\item 取得相应操作技术能力证明,即地方业余无线电协会所颁发的业余无线电台操作证书
	\item 业余无线电爱好者公认具备国家规定的操作技术能力的人,可以免交相应的操作技术能力证明
	\item 境外无线电管理机构颁发或者开具的业余无线电台执照或证书也可以作为具备国家规定的操作技术能力的证明
\end{enumerate}
\textbf{解说:}《业余无线电台管理办法》第十八条:使用业余无线电台,应当具备下列条件:(一)熟悉无线电管理规定;(二)具备国家无线电管理机构规定的操作技术能力。\textbf{取得相应操作技术能力证明}。《国无协[2013]1号》第一项:业余无线电台操作技术能力的验证考核通过闭卷考试等形式进行。验证合格证明为《\textbf{中国无线电协会业余电台操作证书}》。\\\textbf{答案:}A

\textbf{问题:}合法设置业余电台的必要步骤是:
\begin{enumerate}[label=\Alph*), leftmargin=1cm]
	\item 按《业余无线电台管理办法》的规定办理设置审批手续,并取得业余电台执照
	\item 加入指定的业余无线电民间组织,并按其章程规定的办法办理申请手续
	\item 经过业余无线电协会或无线电运动协会同意
	\item 经过所在单位或居委会批准
\end{enumerate}
\textbf{解说:}《业余无线电台管理办法》第四条:\textbf{设置业余无线电台,应当按照本办法的规定办理审批手续,取得业余无线电台执照}。\\\textbf{答案:}A

\textbf{问题:}按照《业余电台管理办法》规定,申请设置使用配备有多台业余无线电发射设备的业余无线电台,应该:
\begin{enumerate}[label=\Alph*), leftmargin=1cm]
	\item 视为一个业余电台,指配一个电台呼号,但所有设备均应经过核定并将参数载入电台执照
	\item 视为一个业余电台,指配一个电台呼号,其中只需有一台设备加以核定并将参数载入电台执照
	\item 每台设备视为一个业余电台,各指配一个电台呼号,并都应经过核定并将参数载入电台执照
	\item 视为一个业余电台,指配一个电台呼号,每个频段选择一台设备加以核定并将参数载入电台执照
\end{enumerate}
\textbf{解说:}《业余无线电台管理办法》第三十二条:\textbf{核发业余无线电台执照的无线电管理机构已经为设置人指配业余无线电台呼号的,不另行为其指配其他业余无线电台呼号}。第十四条:\textbf{业余无线电台执照应当载明所核定的技术参数和发射设备等信息}。\\\textbf{答案:}A

\textbf{问题:}个人申请设置业余无线电台应当提交的书面材料为:
\begin{enumerate}[label=\Alph*), leftmargin=1cm]
	\item 两种表格,身份证和操作证书的原件、复印件
	\item 三种表格,身份证的原件、复印件
	\item 一种表格,身份证和操作证书的原件、复印件
	\item 本人写的申请书,身份证和操作证书的原件、复印件
\end{enumerate}
\textbf{解说:}《业余无线电台管理办法》第七条:申请设置业余无线电台,应当向设台地地方无线电管理机构提交下列书面材料:(一)《\textbf{业余无线电台设置(变更)申请表}》;(二)《\textbf{业余无线电台技术资料申报表}》;(三)\textbf{个人身份证明}或者设台单位证明材料的原件、复印件。(四)\textbf{具备相应操作技术能力证明材料的原件、复印件}。\\\textbf{答案:}A

\textbf{问题:}申请设置下列业余无线电台时应在《业余无线电台设置(变更)申请表》 的“台站种类”选择“特殊”类:
\begin{enumerate}[label=\Alph*), leftmargin=1cm]
	\item 中继台、信标台、空间台
	\item 移动操作的车载台
	\item 用于业余卫星通信的地面业余无线电台
	\item 需要到外地移动操作的手持台
\end{enumerate}
\textbf{解说:}《工信部无(2013]43号》第九项:《业余无线电台设置(变更)申请表》和《业余无线电台技术资料申报表》中特殊台站是指\textbf{业余信标台、空间业余无线电台、特殊实验电台等}特殊业余无线电台。\\\textbf{答案:}A

%%%%%%% 疑问?
\textbf{问题:}申请设置信标台、空间台和技术参数需要超出管理办法规定的特殊业余电台的办法为:
\begin{enumerate}[label=\Alph*), leftmargin=1cm]
	\item 在《业余无线电台设置(变更)申请表》 的“台站种类”选择“特殊”类,由地方无线电管理机构受理和初审后交国家无线电管理机构审批
	\item 先按设置一般业余电台的办法申请,然后再到本地无线电管理机构办理变更执照核定内容
	\item 按照设置一般业余电台的办法申请即可,然后根据需要操作就可以
	\item 必须由地方业余无线电协会作为申请单位,经本地无线电管理机构办理批准设台
\end{enumerate}
\textbf{解说:}《工信部无(2013]43号》第七项:\textbf{业余无线电台为试验特殊通信技术、需要临时超过本文件规定的发射功率等限值使用的,设台人或设台单位需事先向所在地无线电管理机构提出申请,并按照书面批准的时间、地点等限定条件使用}。\\\textbf{答案:}A

%%%%%%% 疑问?
\textbf{问题:}负责受理设置业余无线电台申请的机构为:
\begin{enumerate}[label=\Alph*), leftmargin=1cm]
	\item 设台地地方无线电管理机构或其正式委托的代理受理服务机构
	\item 中国无线电协会
	\item 国家无线电管理机构
	\item 任何地方业余无线电民间组织
\end{enumerate}
\textbf{解说:}《业余无线电台管理办法》第七条:申请设置业余无线电台,应当向\textbf{设台地地方无线电管理机构}提交下列书面材料:(一)《业余无线电台设置(变更)申请表》;(二)《业余无线电台技术资料申报表》;(三)个人身份证明或者设台单位证明材料的原
件、复印件。\\\textbf{答案:}A

%%%%%%% 疑问?
\textbf{问题:}设置在省、自治区、直辖市范围内通信的业余无线电台,审批机构为:
\begin{enumerate}[label=\Alph*), leftmargin=1cm]
	\item 设台地的地方无线电管理机构
	\item 设台地的地方无线电民间机构
	\item 设台地的地方无线电管理机构所委托的其他单位
	\item 国家无线电管理机构
\end{enumerate}
\textbf{解说:}《工信部无(2013]43号》第六项:委托\textbf{设台地所属省、自治区、直辖市无线电管理机构}负责审批除特殊业余无线电台以外的、通信范围涉及两个以上的省(自治区、直辖市)或者涉及境外的业余无线电台的设置,核发业余无线电台执照。\\\textbf{答案:}A

\textbf{问题:}设置通信范围涉及两个以上的省、自治区、直辖市或者涉及境外的一般业余无线电台,审批机构是下列中:
\begin{enumerate}[label=\Alph*), leftmargin=1cm]
	\item 国家无线电管理机构或其委托的设台地的地方无线电管理机构
	\item 设台地地方无线电管理机构
	\item 国家无线电管理机构委托的设台地地方无线电民间组织
	\item 设台地的地方无线电民间组织
\end{enumerate}

\textbf{问题:}按照在省、自治区、直辖市范围内通信所申请设置的业余无线电台,如想要将通信范围扩大至涉及两个以上的省、自治区、直辖市或者涉及境外,或者要到设台地以外进行异地发射操作,须办理下列手续:
\begin{enumerate}[label=\Alph*), leftmargin=1cm]
	\item 事先向核发执照的无线电管理机构申请办理变更手续,按相关流程经国家无线电管理机构或其委托的设台地的地方无线电管理机构批准后,换发业余无线电台执照
	\item 反正已经有了电台执照,可先扩大操作起来,等执照有效期届满时再申请办理变更手续,换发业余无线电台执照
	\item 只要不会被发现,可以不申请办理变更手续,悄悄越限操作
	\item 反正已经有了电台执照,只需向核发执照的无线电管理机构通报变更情况即可,不必申请办理变更和换发执照
\end{enumerate}

\textbf{问题:}业余无线电台执照有效期届满后需要继续使用的,应当在下列期限内向核发执照的无线电管理机构申请办理延续手续:
\begin{enumerate}[label=\Alph*), leftmargin=1cm]
	\item 有效期届满一个月前
	\item 有效期届满二十天前
	\item 有效期届满一个月之内
	\item 有效期届满三个月之内
\end{enumerate}
\textbf{解说:}《业余无线电台管理办法》第十三条:业余无线电台执照有效期届满后需要继续使用的,应当在\textbf{有效期届满前三十日以前}向核发执照的无线电管理机构申请办理延续手续。\\\textbf{答案:}A %%%%% 法律记载的是30日,题目是一个月,题目出的有问题

\textbf{问题:}因改进或调整业余发射设备使业余无线电台的技术参数超出其业余无线电台执照所核定的范围时,应当办理下列手续:
\begin{enumerate}[label=\Alph*), leftmargin=1cm]
	\item 及时向核发执照的无线电管理机构申请办理变更手续,换发业余无线电台执照
	\item 等执照有效期届满时向核发执照的无线电管理机构申请办理变更手续,换发业余无线电台执照
	\item 只要设备型号和产品序列号没有改变,不必申请办理变更手续
	\item 只需及时向核发执照的无线电管理机构通报变更情况,进行备案即可
\end{enumerate}
\textbf{解说:}《业余无线电台管理办法》第十四条:业余无线电台的技术参数不得超出其业余无线电台执照所核定的范围。需要变更业余无线电台执照核定内容的,应当\textbf{向核发执照的无线电管理机构申请办理变更手续,换发业余无线电台执照}。\\\textbf{答案:}A

\textbf{问题:}终止使用业余无线电台的,应当向下列机构申请注销执照:
\begin{enumerate}[label=\Alph*), leftmargin=1cm]
	\item 核发业余无线电台执照的无线电管理机构
	\item 国家无线电管理机构
	\item 受国家无线电管理机构委托的地方业余无线电民间组织
	\item 受国家无线电管理机构委托的全国性业余无线电民间组织
\end{enumerate}
\textbf{解说:}《业余无线电台管理办法》第十五条:终止使用业余无线电台的,应当向\textbf{核发业余无线电台执照的无线电管理机构}申请注销执照。\\\textbf{答案:}A

\textbf{问题:}经地方无线电管理机构批准设置的业余无线电台,设台地迁入其他省、自治区或者直辖市时,应办理的手续为:
\begin{enumerate}[label=\Alph*), leftmargin=1cm]
	\item 先到原核发执照的无线电管理机构办理申请注销原业余无线电台,再到迁入地的地方无线电管理机构办理申请设置业余无线电台的手续
	\item 持原电台执照直接到迁入地的地方无线电管理机构申请办理变更手续
	\item 持原电台执照直接到原核发执照的无线电管理机构申请办理变更手续
	\item 不需要办理任何手续
\end{enumerate}
\textbf{解说:}《《业余无线电台管理办法》第十五条:\textbf{终止使用业余无线电台的,应当向核发业余无线电台执照的无线电管理机构申请注销执照}。《业余无线电台管理办法》第十四条:\textbf{需要变更业余无线电台执照核定内容的,应当向核发执照的无线电管理机构申请办理变更手续,换发业余无线电台执照}。\\\textbf{答案:}A%%%%%疑问

\textbf{问题:}经国家无线电管理机构批准设置的业余无线电台,设台地迁入其他省、自治区或者直辖市时,应办理的手续为:
\begin{enumerate}[label=\Alph*), leftmargin=1cm]
	\item 先到原核发执照的无线电管理机构申请办理注销手续,缴回原电台执照,领取国家无线电管理机构已批准设台的证明,凭证明到迁入地的地方无线电管理机构完成申请变更手续,领取新电台执照
	\item 持原电台执照直接到迁入地的地方无线电管理机构申请办理变更手续
	\item 持原电台执照直接到原核发执照的无线电管理机构申请办理变更手续
	\item 不需要办理任何手续
\end{enumerate}
\textbf{解说:}《《业余无线电台管理办法》第十五条:\textbf{终止使用业余无线电台的,应当向核发业余无线电台执照的无线电管理机构申请注销执照}。《业余无线电台管理办法》第十四条:\textbf{需要变更业余无线电台执照核定内容的,应当向核发执照的无线电管理机构申请办理变更手续,换发业余无线电台执照}。\\\textbf{答案:}A%%%%%疑问

\textbf{问题:}业余无线电台专用无线电发射设备的重要特征是:
\begin{enumerate}[label=\Alph*), leftmargin=1cm]
	\item 发射频率不得超出业余频段
	\item 发射频率必须覆盖所有业余频段
	\item 发射方式必须包含调频
	\item 必须具有数字对讲方式
\end{enumerate}
\textbf{解说:}《业余无线电台管理办法》第十四条:\textbf{业余无线电台的技术参数不得超出其业余无线电台执照所核定的范围}。\\\textbf{答案:}A%%%%%疑问

\textbf{问题:}业余无线电发射设备的下列指标必须符合国家的相关规定:
\begin{enumerate}[label=\Alph*), leftmargin=1cm]
	\item 频率容限和杂散域发射功率
	\item 频率调制频偏和调制度
	\item 频率容限和带外发射
	\item 指配频带和必要带宽
\end{enumerate}
\textbf{解说:}《工信部无[2013]43号》第十项:受理设置业余无线电台申请的无线电理机构负责按照《中华人民共和国无线电频率划分规定》中有关投定,对申请使用的自制、改装、拼装的业余无线电发射设备的\textbf{频率容限和杂散发射}等射频指标进行检验。\\\textbf{答案:}A


\textbf{问题:}业余无线电台使用的发射设备必须符合下列条件:
\begin{enumerate}[label=\Alph*), leftmargin=1cm]
	\item 商品设备应当具备《无线电发射设备型号核准证》,自制、改装、拼装设备应通过国家相关技术标准的检测
	\item 必须具备《无线电发射设备型号核准证》
	\item 商品设备应当具备《无线电发射设备型号核准证》,自制、改装、拼装设备不受限制
	\item 国产商品设备应当具备《无线电发射设备型号核准证》,国外商品设备符合国际流行技术标准即可
\end{enumerate}
\textbf{解说:}《业余无线电台管理办法》第十一条:\textbf{业余无线电台无线电发射设备应当依法取得《中华人民共和国无线电发射设备型号核准证》。申请人可以使用符合国家相关技术标准的自制、改装、拼装的无线电发射设备办理审批手续。}\\\textbf{答案:}A


\textbf{问题:}对业余无线电台专用无线电发射设备进行型号核准的依据为:
\begin{enumerate}[label=\Alph*), leftmargin=1cm]
	\item 国家《无线电频率划分规定》中有关无线电发射设备技术指标的规定
	\item 地方无线电管理机构制订的技术标准
	\item 经国家认证的检测单位所制订的技术标准
	\item 国家关于专业无线电通信发射设备的技术标准
\end{enumerate}
\textbf{解说:}《业余无线电台管理办法》第十一条:对业余无线电台专用无线电发射设备进行型号核准,应当以\textbf{《中华人民共和国无线电频率划分规定》中有关无线电发射设备技术指标的规定}为依据。\\\textbf{答案:}A

\textbf{问题:}业余无线电台专用无线电发射设备的发射频率必须满足的条件是:
\begin{enumerate}[label=\Alph*), leftmargin=1cm]
	\item 发射频率不能超越业余业务或者卫星业余业务频段
	\item 发射频率包含所有业余业务或者卫星业余业务频段
	\item 发射频率包含至少一个业余业务或者卫星业余业务频段
	\item 发射频率可以在业余频段和非业余频段之间选择
\end{enumerate}
\textbf{解说:}《业余无线电台管理办法》第十一条:业余无线电台专用无线电发射设备不得用于其他无线电业务,其\textbf{发射频率应当在业余业务或者卫星业余业务频段内}。其他三项错误。\\\textbf{答案:}A

\textbf{问题:}业余电台的无线电发射设备应符国家规定的下列主要技术指标:
\begin{enumerate}[label=\Alph*), leftmargin=1cm]
	\item 符合频率容限、符合杂散发射最大允许功率电平
	\item 杂散发射不低于最大允许功率电平、电源电压及频率符合国家电网标准、采用标准天线阻抗
	\item 杂散发射不低于最大允许功率电平、频率漂移不低于频率容限、电源利用效率满足节能要求
	\item 工作频率范围足够宽、杂散发射不低于最大允许功率电平、带宽大于允许最低值
\end{enumerate}
\textbf{解说:}业余电台的无线电发射设备应符国家规定的下列主要技术指标:\textbf{符合频率容限、符合杂散发射最大允许功率电平}。\\\textbf{答案:}A%%%%%%%%%%%%

\textbf{问题:}频率容限是发射设备的重要指标,通常用下述单位来表示:
\begin{enumerate}[label=\Alph*), leftmargin=1cm]
	\item 百万分之几(或者赫兹)
	\item dB
	\item 瓦
	\item 百分之几(或者兆赫)
\end{enumerate}
\textbf{解说:}《中华人民共和国无线电频率划分规定》1.6.17中的定义:频率容限是发射所占频段的中心频率偏离指配频率,或发射的特征频率偏离参考频率的最大容许偏差。频率容限以\textbf{百万分之几或以若干赫兹}表示。\\\textbf{答案:}A

\textbf{问题:}杂散域发射功率是发射设备的重要指标,通常用下述单位来表示:
\begin{enumerate}[label=\Alph*), leftmargin=1cm]
	\item 绝对功率dBm、低于载波发射功率的分贝值dBc、低于PEP发射功率的相对值dB
	\item 绝对功率(瓦)
	\item 百分之几
	\item 千赫(或者赫芝)
\end{enumerate}
\textbf{解说:}《中华人民共和国无线电频率划分规定》:杂散域发射功率通常用发射机连接天馈线的输出端的杂散发射频率的峰包功率或平均功率表示,其参考测量带宽主要取决于发射机的无线业务种类。“dBc”是指相对于未调制的载波发射功率的分贝值。“PEP”是指供给天线传输线的峰包功率。\\\textbf{答案:}A%%%%%%%%%%%%%%%

\textbf{问题:}辐射(radiation)是指任何源的能量流以无线电波的形式向外发出。正确的说法是:
\begin{enumerate}[label=\Alph*), leftmargin=1cm]
	\item 闪电产生的电磁波干扰是一种辐射
	\item 沿电源线窜入接收机的差模干扰是一种辐射
	\item 射频电路中变压器内磁芯里的磁场是一种辐射
	\item 射频电路中电容器内极板间的电场是一种辐射
\end{enumerate}
\textbf{解说:闪电产生的电磁波干扰是一种辐射}。其他三项错误。\\\textbf{答案:}A%%%%%%%%%%%%

\textbf{问题:}发射(emission)是指:由无线电发信电台产生的辐射或辐射产物。正确的说法是:
\begin{enumerate}[label=\Alph*), leftmargin=1cm]
	\item 业余电台向周围发送的杂散产物是一种发射
	\item 无线电接收机本地振荡器辐射的能量是一种发射
	\item 医用高频电疗机向周围发送的无线电波能量是一种发射
	\item 闪电产生的电磁波干扰是一种发射
\end{enumerate}

\textbf{问题:}杂散发射是指必要带宽之外的一个或多个频率的发射,其发射电平可降低而不致影响相应信息的传输。一台发射机,工作频率为145.000MHz,但在435.000MHz的频率上也有发射。这种发射属于:
\begin{enumerate}[label=\Alph*), leftmargin=1cm]
	\item 杂散发射
	\item 带外发射
	\item 谐波发射
	\item 带内发射
\end{enumerate}

\textbf{问题:}业余无线电专用发射设备必须满足的主要技术指标要求包括:
\begin{enumerate}[label=\Alph*), leftmargin=1cm]
	\item 频率容限和杂散辐射不超过限值,发射频率不超出国家规定的业余频率
	\item 频率容限不低于限值,杂散辐射不超过限值,发射频率不超出国家规定的业余频率
	\item 频率容限和杂散辐射不超过限值,发射频率包括国家规定的业余频率
	\item 发射功率不低于功率限额,输出阻抗符合工业标准
\end{enumerate}

\textbf{问题:}业余无线电台使用的频率应当符合下述规定:
\begin{enumerate}[label=\Alph*), leftmargin=1cm]
	\item 《中华人民共和国无线电频率划分规定》
	\item ITU《无线电规则》第IV节“频率划分表”
	\item IARU三区“频率规划”
	\item 一般业余无线电书籍所叙述的频率
\end{enumerate}

\textbf{问题:}业余无线电台在业余业务、卫星业余业务作为次要业务使用的频率或者与其他主要业务共同使用的频率上发射操作时,应当注意:
\begin{enumerate}[label=\Alph*), leftmargin=1cm]
	\item 遵守无线电管理机构对该频率的使用规定
	\item 首先守听频率是否已由其他业务电台占用,如听不到,即可按照先来先用的原则放心使用
	\item 只要遵守了《中华人民共和国无线电频率划分规定》的有关规定即可放心使用
	\item 可以任意使用,但在遇到其他业务电台使用时要主动避让
\end{enumerate}

\textbf{问题:}关于业余频率的使用,正确的叙述是:
\begin{enumerate}[label=\Alph*), leftmargin=1cm]
	\item 业余无线电台在无线电管理机构核准其使用的频段内,享有平等的频率使用权
	\item 任何业余无线电台在任何频段都享有平等的频率使用权
	\item 业余无线电台在无线电管理机构核准其使用的频段内,不同类别的业余电台享有不同优先程度的频率使用权
	\item 依法成立的地方业余无线电民间组织的业余电台,在其常用的台网频率上享有比其他个人设置的业余电台优先的使用权
\end{enumerate}

\textbf{问题:}在无线电管理中,由国家将某个特定的频带列入频率划分表,规定该频带可在指定的条件下供业余业务或者卫星业余业务使用,这个过程称为:
\begin{enumerate}[label=\Alph*), leftmargin=1cm]
	\item 划分
	\item 分配
	\item 指配
	\item 授权
\end{enumerate}
\textbf{解说:}《中华人民共和国无线电频率划分规定》1.2.1对(频段的)划分定义:将某个特定的频段列入频率划分表,规定该频段可在指定的条件下供一种或多种地面或空间无线电通信业务或射电天文业务使用。\\
\textbf{答案:}A

\textbf{问题:}在无线电管理中,将无线电频率或频道规定由一个或多个部门,在指定的区域内供地面或空间无线电通信业务在指定条件下使用,这个过程称为:
\begin{enumerate}[label=\Alph*), leftmargin=1cm]
	\item 分配
	\item 划分
	\item 指配
	\item 授权
\end{enumerate}
\textbf{解说:}《中华人民共和国无线电频率划分规定》1.2.2对(无线电频率或无线电频道的)分配定义:将无线电频率或频道规定由一个或多个部门,在指定的区域内供地面或空间无线电通信业务在指定条件下使用。\\
\textbf{答案:}A

\textbf{问题:}在无线电管理中,将无线电频率或频道批准给具体的业余无线电台在规定条件下使用,这个过程称为:
\begin{enumerate}[label=\Alph*), leftmargin=1cm]
	\item 指配
	\item 划分
	\item 分配
	\item 授权
\end{enumerate}
\textbf{解说:}《中华人民共和国无线电频率划分规定》1.2.3对(无线电频率或无线电频道的)指配定义:将无线电频率或频道批准给无线电台在规定条件下使用。\\
\textbf{答案:}A

\textbf{问题:}在频率划分表中,一个频带被标明划分给多种业务时,这些业务被分为下述类别:
\begin{enumerate}[label=\Alph*), leftmargin=1cm]
	\item 主要业务和次要业务
	\item 业余业务和非业余业务
	\item 民用业务和军用业务
	\item 安全业务和一般业务
\end{enumerate}

\textbf{问题:}在频率划分表中,当一个频段划分给业余业务或卫星业余业务和多个其他业务,并且业余业务和卫星业余业务作为次要业务时,业余无线电台应该遵循的规则是:
\begin{enumerate}[label=\Alph*), leftmargin=1cm]
	\item 不得对主要业务电台产生有害干扰
	\item 可要求保护不受来自主要业务电台的有害干扰
	\item 不得对来自同一业务或其他次要业务电台的有害干扰提出保护要求
	\item 容许因设备技术问题对主要业务电台产生短时间有害干扰
\end{enumerate}

\textbf{问题:}在频率划分表中,当一个频段划分给业余业务或卫星业余业务和多个其他业务,并且业余业务和卫星业余业务作为次要业务时,业余无线电台遵循的规则是:
\begin{enumerate}[label=\Alph*), leftmargin=1cm]
	\item 不得对来自主要业务电台的有害干扰提出保护要求
	\item 可要求保护不受来自主要业务电台的有害干扰
	\item 不得对来自同一业务或其他次要业务电台的有害干扰提出保护要求
	\item 容许因设备技术问题对主要业务电台产生短时间有害干扰
\end{enumerate}

\textbf{问题:}在频率划分表中,当一个频段划分给业余业务或卫星业余业务和多个其他业务,并且业余业务和卫星业余业务作为次要业务时,业余无线电台遵循的规则是:
\begin{enumerate}[label=\Alph*), leftmargin=1cm]
	\item 可要求保护不受来自同一业务或其他次要业务电台的有害干扰
	\item 可要求保护不受来自主要业务电台的有害干扰
	\item 不得对来自同一业务或其他次要业务电台的有害干扰提出保护要求
	\item 容许因设备技术问题对主要业务电台产生短时间有害干扰
\end{enumerate}

\textbf{问题:}VHF段的频率范围是多少?
\begin{enumerate}[label=\Alph*), leftmargin=1cm]
	\item 30到300MHz
	\item 30到300kHz
	\item 300到3000kHz
	\item 300到3000MHz
\end{enumerate}

\textbf{问题:}UHF段的频率范围是多少?
\begin{enumerate}[label=\Alph*), leftmargin=1cm]
	\item 300到3000MHz
	\item 30到300MHz
	\item 300到3000kHz
	\item 30到300kHz
\end{enumerate}

\textbf{问题:}HF段的频率范围是多少?
\begin{enumerate}[label=\Alph*), leftmargin=1cm]
	\item 3到30MHz
	\item 30到300MHz
	\item 300到3,000MHz
	\item 300到3,000kHz
\end{enumerate}

% 以下貌似不考?按照现有划分表部分题目也无答案
%\textbf{问题:}我国分配给业余业务和卫星业余业务专用的频段有:
%\begin{enumerate}[label=\Alph*), leftmargin=1cm]
%  \item 7MHz、14MHz、21MHz、28MHz、47GHz频段
%  \item 7MHz、14MHz、21MHz、28MHz、144MHz频段
%  \item 3.5MHz、14MHz、21MHz、28MHz、10GHz频段
%  \item 7MHz、14MHz、28MHz、144MHz、430MHz频段
%\end{enumerate}

%\textbf{问题:}我国分配给业余业务和卫星业余业务与其他业务共用、并且业余业务和卫星业余业务作为主要业务的VHF和UHF频段有:
%\begin{enumerate}[label=\Alph*), leftmargin=1cm]
%  \item 50MHz、144MHz
%  \item 144MHz、430MHz
%  \item 50MHz、430MHz
%  \item 220MHz、430MHz
%\end{enumerate}

%\textbf{问题:}我国分配给业余业务和卫星业余业务与其他业务共用、并且业余业务和卫星业余业务作为唯一主要业务的频段的个数以及在3GHz以下的该类频段分别为:
%\begin{enumerate}[label=\Alph*), leftmargin=1cm]
%  \item 3个,144-146MHz
%  \item 4个,7.0-7.2MHz
%  \item 5个,50-54MHz
%  \item 5个,28-29.7MHz
%\end{enumerate}

%\textbf{问题:}我国分配给业余业务和卫星业余业务与其他业务共用、并且业余业务和卫星业余业务作为次要业务的1200MHz以下频段有:
%\begin{enumerate}[label=\Alph*), leftmargin=1cm]
%  \item 135.7kHz、10.1MHz、430MHz
%  \item 3.5MHz、7MHz、50MHz
%  \item 3.5MHz、18.068MHz、144MHz
%  \item 10.1MHz、24.89MHz、430MHz
%\end{enumerate}

%\textbf{问题:}俗称的6米业余波段的频率范围以及业余业务和卫星业余业务的使用状态分别为:
%\begin{enumerate}[label=\Alph*), leftmargin=1cm]
%  \item 50-54MHz,主要业务
%  \item 50-52MHz,次要业务
%  \item 51-54MHz,专用
%  \item 52-56MHz,次要业务
%\end{enumerate}

%\textbf{问题:}俗称的2米业余波段的频率范围以及我国业余业务和卫星业余业务的使用状态分别为:
%\begin{enumerate}[label=\Alph*), leftmargin=1cm]
%  \item 144-148MHz;其中144-146MHz为唯一主要业务,146-148MHz为与其他业务共同作为主要业务
%  \item 144-146MHz;专用
%  \item 144-148MHz;其中144-146MHz为专用,146-148MHz为次要业务
%  \item 144-148MHz;次要业务
%\end{enumerate}

%\textbf{问题:}俗称的0.7米业余波段的频率范围以及业余业务和卫星业余业务的使用状态分别为:
%\begin{enumerate}[label=\Alph*), leftmargin=1cm]
%  \item 430-440MHz,次要业务
%  \item 430-440MHz,主要业务
%  \item 430-440MHz,专用
%  \item 420-470MHz,次要业务
%\end{enumerate}

%\textbf{问题:}在我国和多数其他国家的频率分配中,业余业务在430-440MHz频段中作为次要业务与其他业务共用。这个频段中我国分配的主要业务是:
%\begin{enumerate}[label=\Alph*), leftmargin=1cm]
%  \item 无线电定位和航空无线电导航
%  \item 固定业务
%  \item 移动业务
%  \item 水上移动和航空移动
%\end{enumerate}

%\textbf{问题:}VHF业余无线电台在144MHz频段进行本地联络时应避免占用的频率为:
%\begin{enumerate}[label=\Alph*), leftmargin=1cm]
%  \item 144-144.035MHz和145.8-146MHz
%  \item 144.035-145.8MHz
%  \item 144.050-144.053MHz和145.100-145.750MHz
%  \item 144.035-144.053MHz和145.550-145.750MHz
%\end{enumerate}

%\textbf{问题:}UHF业余无线电台在430MHz频段进行本地联络时应避免占用的频率为:
%\begin{enumerate}[label=\Alph*), leftmargin=1cm]
%  \item 431.9-432.240MHz和435-438MHz
%  \item 430-431.9MHz和432.240-435MHz
%  \item 431-432MHz和438-440MHz
%  \item 430-431.2MHz和435-436MHz
%\end{enumerate}

%\textbf{问题:}430MHz业余频段中留给业余卫星通信使用,话音及其他通信方式不应占用的频率段为:
%\begin{enumerate}[label=\Alph*), leftmargin=1cm]
%  \item 435MHz至438MHz
%  \item 432MHz至434MHz
%  \item 438MHz至439MHz
%  \item 433MHz至435MHz
%\end{enumerate}

%\textbf{问题:}144MHz业余频段中留给业余卫星通信使用,话音及其他通信方式不应占用的频率段为:
%\begin{enumerate}[label=\Alph*), leftmargin=1cm]
%  \item 145.8MHz至146MHz
%  \item 144.8MHz至145MHz
%  \item 144.2MHz至144.5MHz
%  \item 145.4MHz至144.6MHz
%\end{enumerate}

\textbf{问题:}《业余无线电台管理办法》规定业余无线电台设置、正确使用业余无线电台呼号的办法规是:
\begin{enumerate}[label=\Alph*), leftmargin=1cm]
	\item 业余无线电台应当在每次通信建立及结束时,主动报出本台呼号,在发射过程中至少每十分钟报出本台呼号一次;对于通信对方,也应使用对方电台的呼号加以标识
	\item 业余电台在和熟悉的通信对象联络、已经从信号特征确认双方业余电台身份时,可以省略呼号的发送
	\item 业余电台在通信中可以用姓名、代号、适当的别名或者法规定呼号的部分数字和字母代替完整的业余电台呼号作为电台的标识
	\item 业余电台在通信中可以用自造的呼号作为无线电管理机构指配的业余电台呼号的补充,一起作为电台的标识
\end{enumerate}

\textbf{问题:}业余无线电台应当在每次通信建立及结束时,主动报出本台呼号,在发射过程中至少每十分钟报出本台呼号一次。这里的“呼号”是指:
\begin{enumerate}[label=\Alph*), leftmargin=1cm]
	\item 完整的电台呼号,如在设台地以外的地点进行异地发射操作,还应在前面加上字母B、操作地分区号和符号“/”
	\item 可以是完整的电台呼号,也可以是完整电台呼号的任何一部分
	\item 可以是完整的电台呼号,也可以是电台呼号的分区号加后缀
	\item 一般指无线电管理机构指配的电台呼号,但也可以是对方能够理解的民间自创呼号、代号、代码等
\end{enumerate}

\textbf{问题:}业余无线电台呼号的指配流程是:
\begin{enumerate}[label=\Alph*), leftmargin=1cm]
	\item 无线电管理机构核发业余无线电台执照时,同时指配业余无线电台呼号
	\item 在向无线电管理机构委托的受理服务机构提交设台申请窗口后,由服务机构指配呼号
	\item 无线电管理机构核发业余无线电台执照后,由申请人再向其申请指配呼号
	\item 业余无线电台设台人在提交设台申请的同时提出所要求指配的呼号,经服务机构同意后,报无线电管理机构正式指配
\end{enumerate}

\textbf{问题:}业余无线电爱好者对业已指配给自己的电台呼号不满意,是否可以申请另行指配业余无线电台呼号?
\begin{enumerate}[label=\Alph*), leftmargin=1cm]
	\item 不可以。核发业余无线电台执照的无线电管理机构已经为申请人指配业余无线电台呼号的,不另行指配其他业余无线电台呼号
	\item 更新所设置的业余无线电台类别时可以申请另行指配业余无线电台呼号
	\item 可以申请另行指配业余无线电台呼号,但须缴纳额外的费用
	\item 业余无线电台执照有效期届满、设台人向核发执照的无线电管理机构申请办理延续手续时可以申请另行指配业余无线电台呼号
\end{enumerate}

\textbf{问题:}各地业余无线电台呼号前缀字母和后缀字符的可用范围的确定方法是:
\begin{enumerate}[label=\Alph*), leftmargin=1cm]
	\item 由国家无线电管理机构编制和分配
	\item 地方无线电管理机构根据当地呼号资源的使用情况自行分配
	\item 地方无线电民间组织提出建议,当地无线电管理机构批准
	\item 由业余无线电爱好者根据需求提出建议,当地无线电管理机构批准
\end{enumerate}

\textbf{问题:}业余无线电爱好者是否可要求设台地所在地方无线电管理机构给予指配超出业已分配给该地方的前缀字母和后缀字符可用范围的业余无线电台呼号?
\begin{enumerate}[label=\Alph*), leftmargin=1cm]
	\item 不能,特殊业余无线电台呼号只能由国家无线电管理机构指配
	\item 可以,但只限于与在当地所举办的大型国际或国家级活动有关的特殊电台
	\item 可以,但只限于与当地政府组织的大型科技活动有关的特殊电台
	\item 可以,但只限于当地业余无线电台参加国际重要业余无线电活动的特殊情况
\end{enumerate}

\textbf{问题:}设台地迁入其他省、自治区或者直辖市时,业余电台呼号的指配方法为:
\begin{enumerate}[label=\Alph*), leftmargin=1cm]
	\item 由设台人选择:方法一,注销原电台呼号,指配迁入地的新电台呼号;方法二,申请在迁入地继续指配原来的电台呼号
	\item 必须继续指配原来的电台呼号
	\item 必须指配迁入地的新电台呼号
	\item 可以在保留原电台呼号的同时申请指配迁入地的新电台呼号
\end{enumerate}

\textbf{问题:}设台地迁入其他省、自治区或者直辖市时,申请在迁入地继续指配原来的电台呼号的手续为:
\begin{enumerate}[label=\Alph*), leftmargin=1cm]
	\item 先到原核发执照的无线电管理机构申请办理注销手续,缴回原电台执照,取得由迁入地指配原业余无线电台呼号的书面同意,再到迁入地的地方无线电管理机构办理相应的手续、重新指配原电台呼号,领取新的电台执照
	\item 不需办理任何手续即可把原电台呼号带到迁入地继续使用
	\item 只需到原核发执照的无线电管理机构申请申请办理为迁移后的电台继续使用原电台呼号的全部手续
	\item 只需到迁入地的地方无线电管理机构申请申请办理为迁移后的电台继续使用原电台呼号的全部手续
\end{enumerate}

\textbf{问题:}在实际通信中,是否可以把本台呼号的地区号码加后缀视作《业余电台管理办法》所说的“本台呼号”?
\begin{enumerate}[label=\Alph*), leftmargin=1cm]
	\item 不可以。不完整呼号不具有呼号的属性,不能视作呼号
	\item 在熟悉的友台之间呼叫和联络中可以把不完整呼号视作“呼号”
	\item 在VHF/UHF频段进行本地呼叫和联络时可以把不完整呼号视作“呼号”
	\item 在HF频段进行国内呼叫和联络时可以把不完整呼号视作“呼号”
\end{enumerate}

\textbf{问题:}由国家无线电管理机构批准设台的北京火腿的电台呼号为BH1AAA,把电台带到西安去使用,则本台呼号应该为:
\begin{enumerate}[label=\Alph*), leftmargin=1cm]
	\item B9/BH1AAA
	\item BH1AAA/9
	\item B1/BH9AAA
	\item BH1AAA/B9
\end{enumerate}

\textbf{问题:}某业余无线电爱好者,自己所设置的业余无线电台呼号为BH1ZZZ。现该爱好者到业余无线电台BH9YYY做客并在该台进行发射操作。应当使用的呼号为:
\begin{enumerate}[label=\Alph*), leftmargin=1cm]
	\item BH9YYY或者B9/BH1ZZZ
	\item BH1ZZZ或者B9/BH1ZZZ
	\item BH1ZZZ/9或者BH1ZZZ/BH9
	\item BH9/BH1ZZZ或者BH1ZZZ
\end{enumerate}

\textbf{问题:}某业余无线电爱好者,自己所设置的业余无线电台呼号为BH1ZZZ。现该爱好者将自己的业余无线电台带到湖南进行异地发射操作。应当使用的呼号为:
\begin{enumerate}[label=\Alph*), leftmargin=1cm]
	\item B7/BH1ZZZ
	\item BH1ZZZ
	\item BH1ZZZ/B7
	\item BH7/BH1ZZZ
\end{enumerate}

\textbf{问题:}某业余无线电爱好者,自己所设置的业余无线电台呼号为BH1ZZZ。现该爱好者到业余无线电台BH3YYY做客并并在该台进行发射操作。这种发射操作在业余无线电台管理中称为:
\begin{enumerate}[label=\Alph*), leftmargin=1cm]
	\item 客席发射操作
	\item 异地发射操作
	\item 违章发射操作
	\item 移动发射操作
\end{enumerate}

\textbf{问题:}某业余无线电爱好者,自己所设置的业余无线电台呼号为BH1ZZZ。现该爱好者将自己的业余无线电台带到广东进行发射操作。这种发射操作在业余无线电台管理中称为:
\begin{enumerate}[label=\Alph*), leftmargin=1cm]
	\item 异地发射操作
	\item 客席发射操作
	\item 违章发射操作
	\item 临时发射操作
\end{enumerate}

\textbf{问题:}BH1ZZZ由北京迁入河北省,并办妥了由河北无线电管理机构指配使用原电台呼号的全部手续,领取了新的业余无线电台执照。该台在日常通信时应使用呼号:
\begin{enumerate}[label=\Alph*), leftmargin=1cm]
	\item B3/BH1ZZZ
	\item 固定台址发射操作用BH1ZZZ,移动发射操用B3/BH1ZZZ
	\item 可任选使用呼号B3/BH1ZZZ或者BH1ZZZ
	\item BH1ZZZ
\end{enumerate}

\textbf{问题:}某业余电台操作者听到业余专用频率上出现某种显然出自非业余电台的人为干扰发射,于是按下话筒向该发射者宣传无线电管理法规知识。对这种做法的评论应该是:
\begin{enumerate}[label=\Alph*), leftmargin=1cm]
	\item 错误;违反“业余无线电台的通信对象应当限于业余无线电台”规定。
	\item 正确;但有点条乱,不予提倡
	\item 正确;抓机遇宣传法规,应该提倡
	\item 正确;但需注意态度耐心、用语文明
\end{enumerate}

\textbf{问题:}在业余无线电台中转发广播电台、互联网聊天、电话通话、其他电台的联络信号,这类行为的性质是:
\begin{enumerate}[label=\Alph*), leftmargin=1cm]
	\item 错误行为;违反“业余无线电台的通信对象应当限于业余无线电台”规定,因为通信中产生信息的一方不是通信业余无线电台本身
	\item 正确行为;既然可以联络,不必要限制向话筒送什么内容
	\item 如果转发的目的是进行技术调试、用转发信号作为测试信号的话,就是正常行为
	\item 不算错误但也不值得提倡
\end{enumerate}

\textbf{问题:}业余电台在通信中为其他人或者单位、组织转达信息。对这种做法的评论应该是:
\begin{enumerate}[label=\Alph*), leftmargin=1cm]
	\item 违法行为;违反“业余无线电台的通信对象应当限于业余无线电台”的规定
	\item 只要所转达的信息在内容上不违反《业余电台管理办法规》的禁止规定就是合法行为
	\item 只要转达信息是无偿的,就是合法行为
	\item 只要所转达的信息是有利于社会的公益信息,就是合法行为
\end{enumerate}

\textbf{问题:}某业余无线电协会在发射操作中向其会员播发公益性通知和技术训练讲座,但未得到相应无线电管理机构的批准。对这种做法的评论应该是:
\begin{enumerate}[label=\Alph*), leftmargin=1cm]
	\item 违法行为;违反“未经核发业余无线电台执照的无线电管理机构批准,业余无线电台不得以任何方式进行广播或者发射通播性质的信号”的规定
	\item 只要所播发的通知或讲座有利于当地业余无线电爱好者技术水平的提高,不能算违法行为
	\item 只要所播发的通知或讲座有利于当地业余无线电应急通信训练,不能算违法行为
	\item 只要所播发的通知或讲座是涉及宣传业余电台管理知识的,不能算违法行为
\end{enumerate}

\textbf{问题:}关于业余无线电台在通信过程中使用的语言,正确的做法为:
\begin{enumerate}[label=\Alph*), leftmargin=1cm]
	\item 任何时候都应当使用明语及业余无线电领域公认的缩略语和简语
	\item 可以使用虽然不是所有火腿通用、但在某些火腿圈子内部有一定可懂度的新编缩略语或暗语
	\item 语言要创新,可以使用自创的特殊缩略语,虽开始时象是暗语,用多了就会变明语
	\item 可提倡使用稀有语言或方言,尽量使特定通信对象以外的业余无线电台听不懂,以减少他台呼叫和插入的机会
\end{enumerate}

\textbf{问题:}业余无线电台实验新的编码、调制方式、数字通信协议或者交换尚未公开格式的数据文件,正确做法是:
\begin{enumerate}[label=\Alph*), leftmargin=1cm]
	\item 事先尽可能采取各种办法向信号可能覆盖范围内的业余无线电爱好者公开有关技术细节,并提交给核发其业余无线电台执照的地方无线电管理机构
	\item 事先尽可能采取各种办法向信号可能覆盖范围内的业余无线电爱好者公开有关技术细节,但不必提交给核发其业余无线电台执照的地方无线电管理机构
	\item 应事先提交给核发其业余无线电台执照的地方无线电管理机构,但不必向其他业余无线电爱好者公开有关技术细节
	\item 不必事先公开或者提交给核发其业余无线电台执照的地方无线电管理机构,以后再说
\end{enumerate}

\textbf{问题:}由国家无线电管理机构审批的业余无线电台在设台地以外的地点进行异地发射操作时,应该注意:
\begin{enumerate}[label=\Alph*), leftmargin=1cm]
	\item 既要符合业余电台执照所核定的各项参数约束,又要遵守操作所在地的地方无线电管理机构的相关规定
	\item 遵守的限制以业余电台执照所核定的各项参数和核发其业余电台执照的地方无线电管理机构的规定为准,与操作所在地的规定无关
	\item 遵守的限制以操作所在地的地方无线电管理机构的相关规定为准,与核发电台执照的地方无线电管理机构的规定无关
	\item 有了电台执照就是万事大吉,不必认真了解和遵守什么具体规定
\end{enumerate}

\textbf{问题:}具备国家无线电管理机构规定的操作技术能力并具有法律规定有效证明文件、但还没有获准设置自己的业余电台的人是否可以到业余电台进行发射操作?答案是:
\begin{enumerate}[label=\Alph*), leftmargin=1cm]
	\item 可以。使用所操作业余电台的呼号,由该业余电台的设台人对操作不妥而造成的有害干负责
	\item 可以。因为自己没有呼号,只能在通信中使用临时自编的呼号,或用姓名代替呼号
	\item 不可以
	\item 青少年可以,成人不可以
\end{enumerate}

\textbf{问题:}尚未考得《业余电台操作证书》的人在接受业余电台培训中实习发射操作应遵守的条件是什么?
\begin{enumerate}[label=\Alph*), leftmargin=1cm]
	\item 必须已接受法规等基础培训、必须由电台负责人现场辅导、必须在执照核定范围以及国家规定的操作权限内、进行短时间体验性发射操作实习
	\item 只要业余电台设置人或者其技术负责人能确认实际上已经具备操作技术能力,可以独立进行发射操作,并能为其操作不善造成的后果负责,可以独立发射操作
	\item 尚未取得关于具备操作技术能力有效证明文件的人任何情况下都不可以进行发射操作
	\item 尚未取得关于具备操作技术能力有效证明文件者如为青少年,可以在集体业余电台独立操作,如为成人则任何情况下都不可以进行发射操作
\end{enumerate}

\textbf{问题:}业余无线电台设置人应对其无线电发射设备担负的法定责任为:
\begin{enumerate}[label=\Alph*), leftmargin=1cm]
	\item 应当确保其无线电发射设备处于正常工作状态,避免对其他无线电业务造成有害干扰
	\item 应当确保其无线电发射设备随最先进型号更新,为其他业余电台树立求新的榜样
	\item 应当确保其无线电发射设备达到最大发射功率,以克服其他无线电业务的干扰
	\item 应当确保其无线电发射设备经常处于工作状态,以提高业余频率的实际占用度
\end{enumerate}

\textbf{问题:}业余无线电爱好者使用业余无线电收信设备应遵守的规定为:
\begin{enumerate}[label=\Alph*), leftmargin=1cm]
	\item 不得接收与业余业务和卫星业余业务无关的信号
	\item 只要不造成对其他业务的无线电干扰,接收无线电信号没有限制
	\item 只要不被查出来,可以接收任何无线电信号
	\item 只要出于个人对信息的兴趣而不涉及赢利,可以接收任何无线电信号
\end{enumerate}

\textbf{问题:}业余无线电爱好者无意接收到非业余业务和卫星业余业务的信息时,应遵守的规则为:
\begin{enumerate}[label=\Alph*), leftmargin=1cm]
	\item 不得传播、公布
	\item 只可以在业余无线电台间共享,不得在其他场合公开
	\item 只可以用非无线电方式在业余无线电爱好者之间交流,不得以无线电方式转发
	\item 既然自己可以收到,别人也一定可以收到,当然可以传播、公布或者利用
\end{enumerate}

\textbf{问题:}业余无线电台是否可以发射从广播电台收到的信号、音像节目的录音,或者故意转送电台周围的声音?
\begin{enumerate}[label=\Alph*), leftmargin=1cm]
	\item 不可以,不得发送与业余业务和卫星业余业务无关的信号
	\item 可以,因为该类信息没有保密性
	\item 可以,用于显示自己发射设备的信号质量
	\item 可以,用于提起其他有业余无线电台操作员精神,防止乏困
\end{enumerate}

\textbf{问题:}国家对于利用业余无线电台从事发布、传播违反法律或者公共道德的信息的行为的态度是:
\begin{enumerate}[label=\Alph*), leftmargin=1cm]
	\item 禁止
	\item 不提倡
	\item 容忍
	\item 不可以发布传、播违法信息。但违反公共道德的信息属于水平问题,不鼓励就是了
\end{enumerate}

\textbf{问题:}出租车安装业余电台并用来传递有关载客的信息,这种行为的性质是:
\begin{enumerate}[label=\Alph*), leftmargin=1cm]
	\item 违法行为,违反了严禁利用业余无线电台从事从事商业或者其他营利活动的规定
	\item 不太好,因为占用了其他业余电台通信的频率
	\item 只要不影响其他业余电台的正常通信就可以
	\item 只要管理部门不来查处就可以
\end{enumerate}

\textbf{问题:}利用业余无线电台通信来促销业余无线电产品或者推动与业余无线电活动有关的其他商业性活动,对这类行为的态度应该是:
\begin{enumerate}[label=\Alph*), leftmargin=1cm]
	\item 禁止
	\item 不提倡但也不禁止,毕竟有利于业余无线电活动发展
	\item 只要是业余无线电民间组织是获利方,即使从事商业或其他营利活动,应支持
	\item 如果设台人或者设台单位本身是以这类经营为生的,应适当理解和容忍
\end{enumerate}

\textbf{问题:}利用自己的业余电台强信号故意压制其他业余电台的正常通信,或者在业余无线电频率上转播音乐或广播节目,这些行为的性质属于:
\begin{enumerate}[label=\Alph*), leftmargin=1cm]
	\item 违法行为,违反了严禁阻碍其他无线电台通信的规定
	\item 不妥行为,没有考虑到他人的乐趣
	\item 正常现象,社会上一些人素质就是如此,应该谅解
	\item 不文明行为,对其他业余电台不够礼貌
\end{enumerate}

\textbf{问题:}业余无线电活动是否有序开展,会影响整个社会的无线电通信的安全和有效,使用不当甚至会导致生命财产损失。业余无线电爱好者在这方的法定责任是:
\begin{enumerate}[label=\Alph*), leftmargin=1cm]
	\item 业余无线电台设置、使用人应当加强自律
	\item 个人没有责任,只能依靠管理部门的监督检查和违法查处
	\item 个人没有责任,只能依靠业余无线电民间组织充当“协管”
	\item 有了电台执照,日常一切言行当然可以带到电台通信中,无责任可言
\end{enumerate}

\textbf{问题:}国际电联规定的确定发射电台辐射功率的原则为:
\begin{enumerate}[label=\Alph*), leftmargin=1cm]
	\item 发射电台只应辐射为保证满意服务所必要的功率
	\item 发射电台应辐射尽量大的功率以提供尽量好的信号质量
	\item HF频段发射电台应辐射尽量大的功率,VHF频段发射电台应辐射尽量小的功率
	\item VHF/UHF频段发射电台应辐射尽量大的功率,HF频段发射电台应辐射尽量小的功率
\end{enumerate}

\textbf{问题:}业余电台通信受到违法电台或者不明电台的有害干扰。正确的做法是:
\begin{enumerate}[label=\Alph*), leftmargin=1cm]
	\item 不予理睬,收集有关信息并向无线电管理机构举报
	\item 在频率上向其宣传无线电管理法,要求其停止干扰
	\item 立即报告无线电管理机构进行干涉
	\item 用大功率信号对其进行压制
\end{enumerate}

\textbf{问题:}按照我国规定,购置使用公众对讲机不需取得批准。业余无线电爱好者需要与公众对讲机用户通信时应该:
\begin{enumerate}[label=\Alph*), leftmargin=1cm]
	\item 业余无线电台不能用于与公众对讲机通信
	\item 将业余无线电台设置到公众对讲机的频率,以不大于业余无线电台执照核定的发射功率与之通信
	\item 将业余无线电台设置到公众对讲机的频率,以不大于0.5W的发射功率与之通信
	\item 将业余无线电台设置到公众对讲机的频率,但只能进行由业余无线电台到公众对讲机的单向发信
\end{enumerate}

\textbf{问题:}关于业余无线电台的应急通信,正确的叙述是:
\begin{enumerate}[label=\Alph*), leftmargin=1cm]
	\item 在突发重大自然灾害等紧急情况下,业余无线电台才可以和非业余无线电台进行规定内容的通信
	\item 在日常应急通信训练中,业余无线电台可以和各种非业余无线电台进行通信
	\item 在日常应急通信训练中,业余无线电台可以和地方公益性救援团体的非业余无线电台进行通信
	\item 在日常应急通信训练中,业余无线电台可以和地方公益性救援团体的非业余无线电台进行通信,但须经当地业余无线电协会同意
\end{enumerate}

\textbf{问题:}业余无线电台允许与非业余无线电台通信的条件是:
\begin{enumerate}[label=\Alph*), leftmargin=1cm]
	\item 在突发重大自然灾害等紧急情况下,内容限于与抢险救灾直接相关的紧急事务或者应急救援相关部门交办的任务
	\item 在当地政府或非盈利机构组织的公益活动中,内容限于与公益事务或者相关的活动组织机构交办的任务
	\item 在青少年科技教育活动中,仅可与青少年非业余无线电台通信,内容限于与青少年科技教育直接有关的事务
	\item 在无线电技术研究中,仅可与具备其他业务电台执照的对象通信,内容限于技术实验所需的信号
\end{enumerate}

\textbf{问题:}关于业余无线电台的应急通信,正确的叙述是:
\begin{enumerate}[label=\Alph*), leftmargin=1cm]
	\item 在突发重大自然灾害等紧急情况下,业余无线电台的通信内容可以涉及应急救援相关部门交办的任务
	\item 在平时的任何时侯,业余无线电台的通信内容可以涉及任何政府组织和非盈利机构交办的任务
	\item 平时在专门的应急通信训练活动中,业余无线电台的通信内容可以涉及应急救援相关部门和组织机构交办的任务
	\item 在日常公益性社会活动中,业余无线电台的通信内容可以涉及各种公益机构交办的任务
\end{enumerate}

\textbf{问题:}法规和国际业余无线电惯例要求业余电台日志记载的必要基本内容是:
\begin{enumerate}[label=\Alph*), leftmargin=1cm]
	\item 通信时间、通信频率、通信模式、对方呼号、双方信号报告
	\item 通信对方姓名、对方所在国家或城市、通信模式、双方信号报告
	\item 通信时间、通信频率、双方收发信设备和天线、对方台址
	\item 通信时间、通信模式、对方信号报告、对方台址、对方天气
\end{enumerate}

\textbf{问题:}法规和国际业余无线电惯例要求业余电台日志记载的必要基本内容是:
\begin{enumerate}[label=\Alph*), leftmargin=1cm]
	\item DATE、TIME、FREQ、MODE、CALL(对方)、RST(双方)
	\item DATE、FREQ、QTH(对方)、RIG(对方)、RST(双方)、WX(对方)
	\item DATE、TIME、MODE、CALL(对方)、QTH(对方)、RST(双方)
	\item CALL(通信对方)、T IME、FREQ、RIG(对方)、RST(双方)、PWR(双方)
\end{enumerate}

\textbf{问题:}填写和邮寄QSL卡片时的正确做法有:
\begin{enumerate}[label=\Alph*), leftmargin=1cm]
	\item 迫切需要方卡回寄卡片时,应直接向对方地址邮寄卡片并附加SASE
	\item 填写错误时应划去或使用涂改液覆盖错误内容并加以改正
	\item 自己的邮寄地址与电台的发射地点不同时,应在QTH栏目内填明详细邮寄地址
	\item 通过卡片管理局寄出卡片并希望对方回卡时,应在卡片上注明PSE QSL DIRECT
\end{enumerate}

\textbf{问题:}关于QSL卡片的正确用法是:
\begin{enumerate}[label=\Alph*), leftmargin=1cm]
	\item 不是作为联络或收听证明而交换QSL卡片时,应填上“Eye ball QSO”等有关说明,不应赠送空白卡片
	\item 空白QSL卡片可以当做照片或者名片,任意赠送、交换、散发
	\item 出于火腿互相帮助的目的,虽然对方没有联络到自己,也可以发去确认联络的QSL卡片
	\item 如果在联络中没有听清对方呼号,可以在寄发QSL卡片的对方台名栏中填写对方操作员姓名
\end{enumerate}

\textbf{问题:}业余中继台的设置和技术参数等应满足下列关键条件:
\begin{enumerate}[label=\Alph*), leftmargin=1cm]
	\item 符合国家以及设台地的地方无线电管理机构的规定
	\item 符合设台地的地方业余无线电民间组织的规划
	\item 仅需符合设台地的地方无线电管理机构的规划及相关规定
	\item 符合申请人关于设置中继台的客观需求和技术考虑
\end{enumerate}

\textbf{问题:}业余中继台必备的技术措施为:
\begin{enumerate}[label=\Alph*), leftmargin=1cm]
	\item 设专人负责监控和管理工作,配备有效的遥控手段,保证造成有害干扰时及时停止发射
	\item 技术加密措施,防止未经设台人允许的业余无线电台启用中继
	\item 尽量提高发射功率,以便压制覆盖区内的其他强信号干扰
	\item 设热备份系统,保证不间断工作
\end{enumerate}

\textbf{问题:}某团体依法设置了一部业余中继台。其正确做法是:
\begin{enumerate}[label=\Alph*), leftmargin=1cm]
	\item 向其覆盖区域内的所有业余无线电台提供平等的服务,并将使用业余中继台所需的各项技术参数公开
	\item 中继台是设台者出资建设和维护的,因此仅供经设置者允许的业余电台使用
	\item 中继台是设台者出资建设和维护的,因此仅供本团体成员优先使用,空闲时方供其他业余电台使用
	\item 为保证中继台正常运行,要求覆盖区内所有业余电台缴纳维护成本,否则不准使用
\end{enumerate}

\textbf{问题:}选择144MHz或430MHz业余模拟调频中继台同频段收发频差的原则是:
\begin{enumerate}[label=\Alph*), leftmargin=1cm]
	\item 采用业余无线电标准频差,即144MHz频段600kHz,430MHz频段5MHz
	\item 尽量采用非标准频差以阻止一般业余无线电台占用
	\item 采用经常变换频差的办法减少占用度
	\item 可以在国家《无线电频率划分规定》所规定业余频率范围内任意选择
\end{enumerate}

\textbf{问题:}业余中继台的使用原则是:
\begin{enumerate}[label=\Alph*), leftmargin=1cm]
	\item 除必要的短暂通信外,应保持业余中继台具有足够的空闲时间,以便随时响应突发灾害应急呼叫
	\item 应使中继台尽量处于接近饱和的忙碌状态,提高使用效率
	\item 鼓励业余无线电民间组织(协会)通过中继台向当地会员发布通知
	\item 鼓励青少年学生通过中继台交流解题方法和学习心得
\end{enumerate}

\textbf{问题:}如果你知道另一个电台的呼号,想要在中继上呼叫他,你应该怎么做?
\begin{enumerate}[label=\Alph*), leftmargin=1cm]
	\item 呼叫对方的呼号,并报出自己的呼号
	\item 呼叫“break break”,然后说出对方的呼号
	\item 呼叫“CQ”三次,然后说出对方的呼号
	\item 等待,直到你要呼叫的电台呼叫CQ后,立刻回答他
\end{enumerate}

\textbf{问题:}业余无线电台设置、使用人应当接受下列机构对业余无线电台及其使用情况的监督检查:
\begin{enumerate}[label=\Alph*), leftmargin=1cm]
	\item 无线电管理机构或者其委托单位的监督检查
	\item 业余无线电民间组织的独立监督检查
	\item 单位或所在居委会、村民委员会、物主委员会的监督检查
	\item 国家计量监督部门的监督检查
\end{enumerate}

\textbf{问题:}违反国家规定,擅自设置、使用无线电台(站),或者擅自占用频率,经责令停止使用后拒不停止使用,干扰无线电通信正常进行,造成严重后果的的,可被判犯扰乱无线电通信管理秩序罪,处三年以下有期徒刑、拘役或者管制,并处或者单处罚金。这个规定出自于下列法规律:
\begin{enumerate}[label=\Alph*), leftmargin=1cm]
	\item 中华人民共和国刑法
	\item 中华人民共和国民法通则
	\item 中华人民共和国无线电管理条例
	\item 中华人民共和国电信法
\end{enumerate}

\textbf{问题:}无线电管制是指在下列范围内依法采取的对无线电波的发射、辐射和传播实施的强制性管理:
\begin{enumerate}[label=\Alph*), leftmargin=1cm]
	\item 在特定时间和特定区域内
	\item 在全国范围、所有时间内
	\item 在特定范围、所有时间内
	\item 在例行范围和例行时间内
\end{enumerate}

\textbf{问题:}无线电管制是指在特定时间和特定区域内,依法采取的下列性质的管理:
\begin{enumerate}[label=\Alph*), leftmargin=1cm]
	\item 对无线电波的发射、辐射和传播实施的强制性管理
	\item 对无线电波的发射、辐射实施的指导和行业自律性管理
	\item 对无线电发射设备的生产、销售实施的强制性管理
	\item 对无线电发射设备的生产、销售实施的指导和行业自律性管理
\end{enumerate}

\textbf{问题:}在特定时间和特定区域内实施无线电管制时,与业余无线电有关的管理措施包括:
\begin{enumerate}[label=\Alph*), leftmargin=1cm]
	\item 限制或者禁止业余无线电台(站)的使用,以及对特定的无线电频率实施技术阻断等
	\item 限制或者禁止业余无线电台设备的生产和销售
	\item 限制、但不会禁止业余无线电台(站)的使用
	\item 依法设置的业余电台不在管制范围之内
\end{enumerate}

\textbf{问题:}决定实施无线电管制的机构为:
\begin{enumerate}[label=\Alph*), leftmargin=1cm]
	\item 在全国范围内或者跨省、自治区、直辖市实施,由国务院和中央军事委员会决定。在省、自治区、直辖市范围内实施,由省、自治区、直辖市人民政府和相关军区决定
	\item 在全国范围内或者跨省、自治区、直辖市实施,由国家无线电管理机构决定。在省、自治区、直辖市范围内实施,由相关地方无线电管理机构决定
	\item 在地、市、县实施,由地、市、县人民政府决定
	\item 在单位、居民区实施,由单位上级业务主管机构和区人民政府共同决定
\end{enumerate}

\textbf{问题:}违反无线电管制命令和无线电管制指令的,由下列机构依法进行处罚:
\begin{enumerate}[label=\Alph*), leftmargin=1cm]
	\item 国家无线电管理机构或者省、自治区、直辖市无线电管理机构;违反治安管理规定者由公安机关处罚
	\item 城管、工商、交通联合执法
	\item 当地业余无线电协会
	\item 所在军区派出的专门机构
\end{enumerate}

\textbf{问题:}业余电台违反无线电管制命令和无线电管制指令的,可以依法规受到下列处罚:
\begin{enumerate}[label=\Alph*), leftmargin=1cm]
	\item  责令改正;拒不改正的,关闭、查封、暂扣或者拆除相关设备;情节严重的,吊销电台执照;违反治安管理规定的,由公安机关处罚
	\item  处警告或者三万元以下的罚款
	\item  处警告或者一千元以上,五千元以下的罚款
	\item  责令改正;并开除业余无线电协会会籍、罚没无线电通信设备
\end{enumerate}

%%%%%%% 以下为新增的5题 %%%%%%%

\textbf{问题:}对擅自设置、使用业余无线电台的单位或个人,国家无线电管理机构或者地方无线电管理机构可以根据其具体情况给予下列处罚:
\begin{enumerate}[label=\Alph*), leftmargin=1cm]
	\item  警告、查封或者没收设备,没收非法所得;情节严重的,可以并处一千元以上,五千元以下的罚款
	\item  劝告拆除非法设置的电台;情节严重的,可以并处警告、查封或者没收设备
	\item  责令停止使用非法设置的电台;情节严重的,可以并处警告、查封或者没收设备
	\item  责令停止使用非法设置的电台并作出书面检查;情节严重的,可以并处一千元以下的罚款
\end{enumerate}

\textbf{问题:}业余电台干扰无线电业务的,国家无线电管理机构或者地方无线电管理机构可以根据其具体情况给予设置业余电台的单位或个人下列处罚:
\begin{enumerate}[label=\Alph*), leftmargin=1cm]
	\item  警告、查封或者没收设备、没收非法所得;情节严重的,可以并处一千元以上,五千元以下的罚款
	\item  告拆除非法设置的电台;情节严重的,可以并处警告、查封或者没收设备
	\item  责令停止使用非法设置的电台;情节严重的,可以并处警告、查封或者没收设备
	\item  责令停止使用非法设置的电台并作出书面检查;情节严重的,可以并处一千元以下的罚款
\end{enumerate}

\textbf{问题:}业余电台随意变更核定项目、发送和接收与业余无线电无关的信号的,国家无线电管理机构或者地方无线电管理机构可以根据其具体情况给予设置业余无线电台的单位或个人下列处罚:
\begin{enumerate}[label=\Alph*), leftmargin=1cm]
	\item  警告、查封或者没收设备、没收非法所得;情节严重的,可以并处一千元以上,五千元以下的罚款
	\item  劝告拆除非法设置的电台;情节严重的,可以并处警告、查封或者没收设备
	\item  责令停止使用非法设置的电台;情节严重的,可以并处警告、查封或者没收设备
	\item  责令停止使用非法设置的电台并作出书面检查;情节严重的,可以并处一千元以下的罚款
\end{enumerate}

\textbf{问题:}超出核定范围使用频率或者有其它违反频率管理有关规定的行为的,无线电管理机构可以根据其具体情况给予设置业余无线电台的单位或个人下列处罚:
\begin{enumerate}[label=\Alph*), leftmargin=1cm]
  \item 责令限期改正,可以处警告或者三万元以下的罚款
  \item 责令限期改正,可以处警告或者一千元以上,五千元以下的罚款
  \item 责令限期改正,可以处警告或者一千元以下的罚款
  \item 责令限期改正,情节严重的,可以并处警告、查封或者没收设备
\end{enumerate}

\textbf{问题:}对涂改、仿制、伪造、倒卖、出租、出借业余无线电台执照,或者以其他形式非法转让业余无线电台执照的,无线电管理机构可以给予下列处罚:
\begin{enumerate}[label=\Alph*), leftmargin=1cm]
  \item 应当责令限期改正,可以处警告或者三万元以下的罚款
  \item 应当责令限期改正,可以处警告或者一千元以上,五千元以下的罚款
  \item 应当责令限期改正,可以处警告或者一千元以下的罚款
  \item 应当责令限期改正,情节严重的,可以并处警告、查封或者没收设备
\end{enumerate}

\textbf{问题:}对盗用、出租、出借、转让、私自编制或者违法使用业余无线电台呼号的,无线电管理机构可以给予下列处罚:
\begin{enumerate}[label=\Alph*), leftmargin=1cm]
  \item 应当责令限期改正,可以处警告或者三万元以下的罚款
  \item 应当责令限期改正,可以处警告或者一千元以上,五千元以下的罚款
  \item 应当责令限期改正,情节严重的,可以并处警告、查封或者没收设备
  \item 应当责令限期改正,可以处警告或者一千元以下的罚款
\end{enumerate}

\textbf{问题:}对以不正当手段取得业余无线电台执照的,无线电管理机构可以给予下列处罚:
\begin{enumerate}[label=\Alph*), leftmargin=1cm]
  \item 责令限期改正,可以处警告或者三万元以下的罚款
  \item 责令限期改正,可以处警告或者一千元以上,五千元以下的罚款
  \item 责令限期改正,可以处警告或者一千元以下的罚款
  \item 责令限期改正,情节严重的,可以并处警告、查封或者没收设备
\end{enumerate}

\textbf{问题:}对向负责监督检查的无线电管理机构隐藏有关情况,提供虚假材料或者拒绝提供反映其活动情况的真实材料的,无线电管理机构可以给予下列处罚:
\begin{enumerate}[label=\Alph*), leftmargin=1cm]
  \item 责令限期改正,可以处警告或者三万元以下的罚款
  \item 责令限期改正,可以处警告或者一千元以上,五千元以下的罚款
  \item 责令限期改正,可以处警告或者一千元以下的罚款
  \item 责令限期改正,情节严重的,可以并处警告、查封或者没收设备
\end{enumerate}

\textbf{问题:}对违法使用业余无线电台造成严重后果的,无线电管理机构可以给予下列处罚:
\begin{enumerate}[label=\Alph*), leftmargin=1cm]
  \item 应当责令限期改正,可以处警告或者三万元以下的罚款
  \item 应当责令限期改正,可以处警告或者一千元以上,五千元以下的罚款
  \item 应当责令限期改正,可以处警告或者一千元以下的罚款
  \item 应当责令限期改正,情节严重的,可以并处警告、查封或者没收设备
\end{enumerate}
