\chapter{无线电通信的方法}



\noindent\textbf{问题:}某俱乐部约定了一个成员业余电台之间交流技术的网络频率,当遇有其他业余电台按通信惯例要求参加通信时,处理原则应为:
\begin{enumerate}[label=\Alph*), leftmargin=3em]
	\item 无条件欢迎加入,因为任何核准的业余电台对频率享有平等的频率使用权
	\item 要求其他业余电台在任何时间都不得使用俱乐部自己约定的专用通信频率
	\item 要求其他业余电台在俱乐部成员结束网络通信后再使用该频率
	\item 由俱乐部网络控制台决定是其他业余电台是否可以加入
\end{enumerate}

\bigskip


\noindent\textbf{问题:}业余电台在发起呼叫前不可缺少的操作步骤是:
\begin{enumerate}[label=\Alph*), leftmargin=3em]
	\item 先用吹话筒、吹口哨等方法发出连续信号检查天线驻波比
	\item 检查发射功率是否达到设备的额定输出功率
	\item 先用礼貌的语言请其他电台让出频率
	\item 先守听一段时间,确保没有其他电台正在使用频率
\end{enumerate}
\noindent\textbf{解说:}业余电台在发起呼叫前不可缺少的操作步骤是:\textbf{先守听一段时间,确保没有其他电台正在使用频率}。\\\noindent\textbf{答案:}D%%%%%%

\bigskip


\noindent\textbf{问题:}业余电台在发射调试信号进行发射功率和天线驻波比等检查时必须注意做到的是:
\begin{enumerate}[label=\Alph*), leftmargin=3em]	
	\item 先将天线的发射方向指向正北
	\item 先将频率设置到无人使用的空闲频率、偏离常用的热点频率
	\item 话筒离嘴距离在2公分以上,电键按键时间不短于5秒钟
	\item 先将收发信机的语音压缩功能打开
\end{enumerate}
\noindent\textbf{解说:}业余电台在发射调试信号进行发射功率和天线驻波比等检查时必须注意做到的是:\textbf{先将频率设置到无人使用的空闲频率、偏离常用的热点频率}。\\\noindent\textbf{答案:}B

\bigskip


\noindent\textbf{问题:}业余电台通过守听,没有听到信号还不足以确认频率空闲,因为有时听不到通信双方中的另一方,贸然呼叫会对已有的通信造成干扰。为避免这种情况,应该:
\begin{enumerate}[label=\Alph*), leftmargin=3em]	
	\item 因为守听没有听到别人的信号,可以放心发起呼叫
	\item 先询问“有人使用频率吗”?确认没有应答方能发起呼叫
	\item 可以先启动呼叫和进行通信,只要没有其他电台出来交涉就可放心使用
	\item 可以先启动呼叫和进行通信,等听到确实有电台先占用频率,再主动让出
\end{enumerate}
\noindent\textbf{解说:}为避免这种情况,应该\textbf{先询问“有人使用频率吗”?确认没有应答方能发起呼叫}。\\\noindent\textbf{答案:}B


\bigskip


\noindent\textbf{问题:}业余电台发起呼叫前应先守听一段时间,如没有听到信号,应再询问“有人使用频率吗”?确认没有应答方能发起呼叫。下列英语短句中不能正确表达这一询问的是:
\begin{enumerate}[label=\Alph*), leftmargin=3em]
	\item Calling you, Roger?
	\item Is any body in the frequency?
	\item Any body here?
	\item Is the frequency in use?
\end{enumerate}
\noindent\textbf{解说:}不能正确表达这一询问的是\textbf{“Calling you, Roger?”}。“Is any body in the frequency?”意为“有人在用这个频率吗”,“Any body here?”意为“有人在吗”,“Is the frequency in use?”意为“这个频率在被使用吗”。\\\noindent\textbf{答案:}A

\bigskip


\noindent\textbf{问题:}业余电台BH1ZZZ用话音发起CQ呼叫的正确格式为:
\begin{enumerate}[label=\Alph*), leftmargin=3em]
	\item CQ、CQ、CQ。我是1ZZZ。听到请回答
	\item CQ、CQ、CQ。BH1ZZZ呼叫。Bravo Hotel One Zulu Zulu Zulu呼叫,BH1ZZZ呼叫。听到请回答。
	\item CQ、CQ、CQ,CQ、CQ、CQ,CQ、CQ、CQ。BH1ZZZ呼叫。请过来。
	\item CQ、CQ、CQ。听到请回答。
\end{enumerate}
\noindent\textbf{解说:}业余电台BH1ZZZ用话音发起CQ呼叫的正确格式为\textbf{CQ、CQ、CQ。BH1ZZZ呼叫。Bravo Hotel One Zulu Zulu Zulu呼叫,BH1ZZZ呼叫。听到请回答。}\\\noindent\textbf{答案:}B

\bigskip


\noindent\textbf{问题:}业余电台BH1ZZZ用话音发起CQ呼叫的正确格式为:
\begin{enumerate}[label=\Alph*), leftmargin=3em]
	\item CQ CQ CQ, CQ CQ CQ, CQ CQ CQ. This is BH1ZZZ. Back to you.
	\item CQ CQ CQ.This is BH1ZZZ. Bravo Hotel One Zulu Zulu Zulu, BH1ZZZ is calling. I’m standing by.
	\item CQ CQ CQ. This One Zulu Zulu Zulu. Over.
	\item CQ CQ CQ. Go ahead please.
\end{enumerate}
\noindent\textbf{解说:}业余电台BH1ZZZ用话音发起CQ呼叫的正确格式为\textbf{CQ CQ CQ.This is BH1ZZZ. Bravo Hotel One Zulu Zulu Zulu, BH1ZZZ is calling. I’m standing by.}\\\noindent\textbf{答案:}B


\bigskip


\noindent\textbf{问题:}业余电台BH1ZZZ用话音呼叫BH8YYY的正确格式为:
\begin{enumerate}[label=\Alph*), leftmargin=3em]
	\item BH8YYY。我是BH1ZZZ,我是BH1ZZZ,我是BH1ZZZ。听到请回答。
	\item 8YYY、8YYY、8YYY。BH1ZZZ呼叫。请过来。
	\item BH8YYY、BH8YYY、BH8YYY。BH1ZZZ呼叫。Bravo Hotel One Zulu Zulu Zulu,BH1ZZZ呼叫。听到请回答。
	\item BH8YYY、BH8YYY、BH8YYY。我是1ZZZ。听到请回答
\end{enumerate}
\noindent\textbf{解说:}业余电台BH1ZZZ用话音呼叫BH8YYY的正确格式为\textbf{BH8YYY、BH8YYY、BH8YYY。BH1ZZZ呼叫。Bravo Hotel One Zulu Zulu Zulu,BH1ZZZ呼叫。听到请回答。}\\\noindent\textbf{答案:}C


\bigskip


\noindent\textbf{问题:}业余电台BH1ZZZ用话音呼叫BH8YYY的正确格式为:
\begin{enumerate}[label=\Alph*), leftmargin=3em]
	\item Bravo Hotel Eight Yankee Yankee Yankee, Bravo Hotel Eight Yankee Yankee Yankee, Bravo Hotel Eight Yankee Yankee Yankee.This is Bravo Hotel One Zulu Zulu Zulu. Bravo Hotel One Zulu Zulu Zulu, Bravo Hotel One Zulu Zulu Zulu is calling. I’m standing by.
	\item Bravo Hotel Eight Yankee Yankee Yankee, Bravo Hotel Eight Yankee Yankee Yankee, Bravo Hotel Eight Yankee Yankee Yankee. Go ahead please.
	\item 8YYY, 8YYY, YYY. This is BH1ZZZ. Over.
	\item BH8YYY, BH8YYY, BH8YYY. This One Zulu Zulu Zulu. Come in please.
\end{enumerate}
\noindent\textbf{解说:}业余电台BH1ZZZ用话音呼叫BH8YYY的正确格式为\textbf{Bravo Hotel Eight Yankee Yankee Yankee, Bravo Hotel Eight Yankee Yankee Yankee, Bravo Hotel Eight Yankee Yankee Yankee.This is Bravo Hotel One Zulu Zulu Zulu. Bravo Hotel One Zulu Zulu Zulu, Bravo Hotel One Zulu Zulu Zulu is calling. I’m standing by.}\\\noindent\textbf{答案:}A


\bigskip


\noindent\textbf{问题:}BH1ZZZ希望加入两个电台正在通信中的谈话,正确的方法为:
\begin{enumerate}[label=\Alph*), leftmargin=3em]
	\item 在双方对话的间隙,短暂发射一次“Break in!”或“插入!”,如得到响应,再说明本台呼号 “BH1ZZZ请求插入”,等对方正式表示邀请后,方能加入
	\item 只要双方都是自己熟悉的业余电台操作员,可直接插入谈话,不必拘泥礼节
	\item 在一方正在发射期间,短暂插入一次“Break in”,向正在收听的一方发出插入请求
	\item 短暂发射一次“Break in!”或“插入!”,如对方无反应,应加大功率反复作此发射
\end{enumerate}
\noindent\textbf{解说:}BH1ZZZ希望加入两个电台正在通信中的谈话,正确的方法为\textbf{在双方对话的间隙,短暂发射一次“Break in!”或“插入!”,如得到响应,再说明本台呼号 “BH1ZZZ请求插入”,等对方正式表示邀请后,方能加入}。\\\noindent\textbf{答案:}A


\bigskip


\noindent\textbf{问题:}以请求插入的方式加入两个电台正在通信中的谈话,应满足的起码条件是:
\begin{enumerate}[label=\Alph*), leftmargin=3em]
	\item 自己的信号质量不亚于原通信双方
	\item 自己的操作技巧不亚于原通信双方
	\item 确认自己的加入不会影响原通信双方的乐趣
	\item 自己拥有比原通信双方更有吸引力的谈话内容
\end{enumerate}
\noindent\textbf{解说:}以请求插入的方式加入两个电台正在通信中的谈话,应满足的起码条件是\textbf{确认自己的加入不会影响原通信双方的乐趣}。\\\noindent\textbf{答案:}C


\bigskip


\noindent\textbf{问题:}业余电台之间进行通信,必须相互正确发送和接收的信息为:
\begin{enumerate}[label=\Alph*), leftmargin=3em]
	\item 对方呼号、信号报告、设备情况
	\item 本台呼号、信号报告、QTH
	\item 本台呼号、对方呼号、QTH
	\item 本台呼号、对方呼号、信号报告
\end{enumerate}
\noindent\textbf{解说:}业余电台之间进行通信,必须相互正确发送和接收的信息为\textbf{本台呼号、对方呼号、信号报告}。\\\noindent\textbf{答案:}D


\bigskip


\noindent\textbf{问题:}如何回答一个CQ呼叫?
\begin{enumerate}[label=\Alph*), leftmargin=3em]
	\item 先给出信号报告,再报出自己的呼号
	\item 先报出对方的呼号,再报出自己的呼号
	\item 先报出自己的呼号,再报出对方的呼号
	\item 说:“CQ”,并报出对方的呼号
\end{enumerate}
\noindent\textbf{解说:}回答一个CQ呼叫的方法为\textbf{先报出对方的呼号,再报出自己的呼号}。\\\noindent\textbf{答案:}B


\bigskip


\noindent\textbf{问题:}当一部电台在呼叫CQ时,他的意思是?
\begin{enumerate}[label=\Alph*), leftmargin=3em]
	\item 非特指地呼叫任何一部电台
	\item 只有被呼叫的电台可以回答,其他人不能回答
	\item 呼叫重庆的电台
	\item 此电台正在测试天线,不需要任何电台回答这个呼叫
\end{enumerate}
\noindent\textbf{解说:}当一部电台在呼叫CQ时,他的意思是\textbf{非特指地呼叫任何一部电台}。\\\noindent\textbf{答案:}A


\bigskip


\noindent\textbf{问题:}如果其他电台报告你在2米波段的信号刚才非常强,但是突然变弱或不可辨,这时你应当怎么做?
\begin{enumerate}[label=\Alph*), leftmargin=3em]
	\item 将你电台中的镍氢电池换成锂电池
	\item 稍稍移动一下自己的位置,有时信号无规律反射造成的多径效应可能导致失真
	\item 打开哑音发射功能
	\item 请对方电台调整自己的静噪设置
\end{enumerate}
\noindent\textbf{解说:}如果其他电台报告你在2米波段的信号刚才非常强,但是突然变弱或不可辨,这时你应当\textbf{稍稍移动一下自己的位置,有时信号无规律反射造成的多径效应可能导致失真}。\\\noindent\textbf{答案:}B


\bigskip


\noindent\textbf{问题:}下列哪种方式可以让你快速切换到一个你经常使用的频率?
\begin{enumerate}[label=\Alph*), leftmargin=3em]
	\item 打开哑音输出
	\item 关闭哑音输出
	\item 使用快速扫描模式来切换到那个频率
	\item 将这个频率作为一个频道存储在电台中
\end{enumerate}
\noindent\textbf{解说:}\textbf{将这个频率作为一个频道存储在电台中}可以让你快速切换到一个你经常使用的频率。\\\noindent\textbf{答案:}D


\bigskip


\noindent\textbf{问题:}“谁在呼叫我”的业余无线电通信Q简语为:

\begin{enumerate}[label=\Alph*), leftmargin=3em]
	\item QRZ?
	\item QRA?
	\item QRZ
	\item QSL?
\end{enumerate}

\noindent\textbf{解说:}“谁在呼叫我”的业余无线电通信Q简语为“\textbf{QRZ?}”。其余各项分别为:“QRZ”为“……正在(用……kHz或……MHz)呼叫你”,“QRA?”为“你台的名称是什么?”,“QSL?”为“你能承认收妥吗?”。\\\noindent\textbf{答案:}A


\bigskip


\noindent\textbf{问题:}“我遇到他台干扰”的业余无线电通信Q简语为:

\begin{enumerate}[label=\Alph*), leftmargin=3em]
	\item QSB?
	\item QRM
	\item QRN
	\item QSM
\end{enumerate}
\noindent\textbf{解说:}“我遇到他台干扰”的业余无线电通信Q简语为“\textbf{QRM}”。其余各项分别为:“QSM”为“将你发给我的最末一份电报(或电报号数……)重发”,“QRN”为“我正受到……天电干扰”,“QSB?”为“我的信号有衰落吗?”。\\\noindent\textbf{答案:}B


\bigskip


\noindent\textbf{问题:}“我遇到天电干扰”的业余无线电通信Q简语为:

\begin{enumerate}[label=\Alph*), leftmargin=3em]
	\item QSN
	\item QRN
	\item QST
	\item QRM
\end{enumerate}

\noindent\textbf{解说:}“我遇到天电干扰”的业余无线电通信Q简语为\textbf{QRN}。其余各项分别为:“QST”为“以下是对所有业余无线电爱好者的广播消息”,“QSN”为“我已在……kHz(或MHz)上听到你(或……)”,“QRM”为“我现在正受到……干扰”。\\\noindent\textbf{答案:}B


\bigskip


\noindent\textbf{问题:}“我给你收据(QSL卡片)、我已收妥”的业余无线电通信Q简语为:

\begin{enumerate}[label=\Alph*), leftmargin=3em]
	\item QRG
	\item QRV
	\item QSL
	\item QSX
\end{enumerate}

\noindent\textbf{解说:}“我给你收据(QSL卡片)、我已收妥”的业余无线电通信Q简语为\textbf{QSL}。其余各项Q简语分别为:“QRG”为“你(或……)的准确频率是……kHz(或MHz)”,“QSX”为“我正用……kHz(或MHz)收听……”,“QRV”为“我已准备好”。\\\noindent\textbf{答案:}C


\bigskip


\noindent\textbf{问题:}“我的电台位置是××××”的业余无线电通信Q简语为:

\begin{enumerate}[label=\Alph*), leftmargin=3em]
	\item QSL ××××
	\item QTH ××××
	\item QRD ××××
	\item QSP ××××
\end{enumerate}

\noindent\textbf{解说:}“我的电台位置是××××”的业余无线电通信Q简语为\textbf{QTH ××××}。其余各项Q简语分别为:“QRD”为“我到……去,从……来”,“QSL”为“我现在承认收妥”,“QSP”为“我可以免费转发到……”。\\\noindent\textbf{答案:}B


\bigskip


\noindent\textbf{问题:}“天线”的业余无线电通信常用缩语是:

\begin{enumerate}[label=\Alph*), leftmargin=3em]
	\item ATT
	\item ATN
	\item ATR
	\item ANT
\end{enumerate}

\noindent\textbf{解说:}“天线”的业余无线电通信常用缩语是\textbf{ANT}。其余三项不符合题意。\\\noindent\textbf{答案:}D


\bigskip


\noindent\textbf{问题:}业余无线电常用缩语“ARDF”的意思是:

\begin{enumerate}[label=\Alph*), leftmargin=3em]
	\item 天线调谐器、天调
	\item 业余无线电测向
	\item 地址
	\item 天线测试仪
\end{enumerate}

\noindent\textbf{解说:}业余无线电常用缩语“ARDF”的意思是\textbf{业余无线电测向}。其余选项为:地址为“ADRS”。\\\noindent\textbf{答案:}B


\bigskip


\noindent\textbf{问题:}“频率”的业余无线电通信常用缩语是:
\begin{enumerate}[label=\Alph*), leftmargin=3em]
	\item FER
	\item FREQ
	\item FIND
	\item TUNE
\end{enumerate}

\noindent\textbf{解说:}“频率”的业余无线电通信常用缩语是\textbf{FREQ}。\\\noindent\textbf{答案:}B
%%%???

\bigskip


\noindent\textbf{问题:}业余无线电通信常用缩语“GND”的意思是:

\begin{enumerate}[label=\Alph*), leftmargin=3em]
	\item 格林威治时间
	\item 好运气
	\item 地线,地面
	\item 高兴
\end{enumerate}

\noindent\textbf{解说:}业余无线电通信常用缩语“GND”的意思是\textbf{地线,地面}”。其余选项为:格林威治时间为“GMT”,好运气的常用缩语为“GL”,高兴的常用缩语为“GLD”。\\\noindent\textbf{答案:}C


\bigskip


\noindent\textbf{问题:}业余无线电通信常用缩语“OM”的意思是:

\begin{enumerate}[label=\Alph*), leftmargin=3em]
	\item 欧姆
	\item 老朋友
	\item 老人
	\item 或者
\end{enumerate}

\noindent\textbf{解说:}业余无线电通信常用缩语“OM”的意思是\textbf{老朋友}。在莫尔斯电码通信中,“OM”为英语“old man”的缩略,指男性操作者(不分年龄)。\\\noindent\textbf{答案:}B


\bigskip


\noindent\textbf{问题:}“电台设备”的业余无线电通信常用缩语是:

\begin{enumerate}[label=\Alph*), leftmargin=3em]
	\item SB
	\item RIG
	\item EQP
	\item REG
\end{enumerate}

\noindent\textbf{解说:}“电台设备”的业余无线电通信常用缩语是\textbf{RIG}。\\\noindent\textbf{答案:}B


\bigskip


\noindent\textbf{问题:}“收信机”的业余无线电通信常用缩语是:

\begin{enumerate}[label=\Alph*), leftmargin=3em]
	\item RCVR,RX
	\item XMTR
	\item XCVR
	\item RMKS
\end{enumerate}

\noindent\textbf{解说:}“收信机”的业余无线电通信常用缩语是\textbf{RCVR,RX}。其余选项为:“XCVR”为“收发信机”,“XMTR”为“发信机”。\\\noindent\textbf{答案:}A


\bigskip


\noindent\textbf{问题:}“发信机”的业余无线电通信常用缩语是:

\begin{enumerate}[label=\Alph*), leftmargin=3em]
	\item TX、XMTR
	\item XTL
	\item VXCO
	\item VXO
\end{enumerate}

\noindent\textbf{解说:}“发信机”的业余无线电通信常用缩语是\textbf{TX、XMTR}。其余选项为:“VXO”为“可变频率晶体振荡器”。\\\noindent\textbf{答案:}A


\bigskip


\noindent\textbf{问题:}“收发信机”的业余无线电通信常用缩语是:

\begin{enumerate}[label=\Alph*), leftmargin=3em]
	\item XMTR
	\item XCVR
	\item XTL
	\item XVTR
\end{enumerate}

\noindent\textbf{解说:}“收发信机”的业余无线电通信常用缩语是\textbf{XCVR}。其余选项为:“XMTR”为“发信机”。\\\noindent\textbf{答案:}B


\bigskip


\noindent\textbf{问题:}业余无线电通信常用缩语“WX”的意思是:

\begin{enumerate}[label=\Alph*), leftmargin=3em]
	\item 联络、工作
	\item 瓦特
	\item 天气
	\item 星期
\end{enumerate}

\noindent\textbf{解说:}业余无线电通信常用缩语“WX”的意思是\textbf{天气}。其余选项为:瓦特为“W”,联络、工作为“WK”。\\\noindent\textbf{答案:}C


\bigskip


\noindent\textbf{问题:}业余无线电通信常用缩语“73”的意思是:

\begin{enumerate}[label=\Alph*), leftmargin=3em]
	\item 再见
	\item 向对方的致意、美好的祝愿
	\item 希望下次再见
	\item 谢谢你
\end{enumerate}

\noindent\textbf{解说:}业余无线电通信常用缩语“73”的意思是\textbf{向对方的致意、美好的祝愿}。\\\noindent\textbf{答案:}B


\bigskip


\noindent\textbf{问题:}业余无线电通话常用语“Roger”的用法是:

\begin{enumerate}[label=\Alph*), leftmargin=3em]
	\item 回答起始语,表示开始发话了,任何情况都可使用
	\item 回答起始语,相当于“听到”,用于能听到对方信号、但不一定能全部抄收的情况
	\item 回答起始语,相当于“明白”,仅在已完全抄收对方刚才发送的信息时使用
	\item 惯用口头语,相当于电话的“喂”,仅提起注意,不包含任何意义
\end{enumerate}

\noindent\textbf{解说:}。“Roger”为\textbf{回答起始语,相当于“明白”,仅在已完全抄收对方刚才发送的信息时使用}。\\\noindent\textbf{答案:}C


\bigskip


\noindent\textbf{问题:}业余无线电通信中常用的天线种类的缩写DP代表:

\begin{enumerate}[label=\Alph*), leftmargin=3em]
	\item 偶极天线
	\item 长线天线
	\item 垂直天线
	\item 定向天线
\end{enumerate}

\noindent\textbf{解说:}业余无线电通信中常用的天线种类的缩写DP代表\textbf{偶极天线}。其余选项分别为:长线天线的缩写为LW,定向天线的缩写为BEAM,垂直天线的缩写为VERT。\\\noindent\textbf{答案:}A


\bigskip


\noindent\textbf{问题:}业余无线电通信中常用的天线种类的缩写GP代表:

\begin{enumerate}[label=\Alph*), leftmargin=3em]
	\item 偶极天线
	\item 对数周期天线
	\item 定向天线
	\item 垂直接地天线
\end{enumerate}

\noindent\textbf{解说:}业余无线电通信中常用的天线种类的缩写GP代表\textbf{垂直接地天线}。其余选项分别为:对数周期天线的缩写为LP,偶极天线的缩写为DP,定向天线的缩写为BEAM。\\\noindent\textbf{答案:}D


\bigskip


\noindent\textbf{问题:}业余无线电通信中常用的天线种类的缩写BEAM代表:

\begin{enumerate}[label=\Alph*), leftmargin=3em]
	\item 专指八木天线
	\item 垂直天线
	\item 偶极天线
	\item 定向天线
\end{enumerate}

\noindent\textbf{解说:}业余无线电通信中常用的天线种类的缩写BEAM代表\textbf{定向天线}。其余选项分别为:八木天线的缩写为YAGI,偶级天线的缩写为DP,垂直天线的缩写为VERT。\\\noindent\textbf{答案:}D


\bigskip


\noindent\textbf{问题:}业余无线电通信中常用的天线种类的缩写YAGI代表:

\begin{enumerate}[label=\Alph*), leftmargin=3em]
	\item 偶极天线
	\item 垂直天线
	\item 八木天线
	\item 定向天线
\end{enumerate}

\noindent\textbf{解说:}业余无线电通信中常用的天线种类的缩写YAGI代表\textbf{八木天线}。其余三项分别为:定向天线的缩写为BEAM,偶级天线的缩写为DP,垂直天线的缩写为VERT。\\\noindent\textbf{答案:}C


\bigskip


\noindent\textbf{问题:}业余无线电通信中常用的天线种类的缩写VER代表:

\begin{enumerate}[label=\Alph*), leftmargin=3em]
	\item 垂直接地天线
	\item 垂直天线
	\item 偶极天线
	\item 定向天线
\end{enumerate}

\noindent\textbf{解说:}业余无线电通信中常用的天线种类的缩写VER代表\textbf{垂直天线}。其余三项分别为:垂直接地天线的缩写为GP,偶级天线的缩写为DP,定向天线的缩写为BEAM。\\\noindent\textbf{答案:}B


\bigskip


\noindent\textbf{问题:}已知北京时间,相应的UTC时间应为:

\begin{enumerate}[label=\Alph*), leftmargin=3em]
	\item 北京时间的小时数加8,如小时数大于24,则小时数减24,日期改为后一天。
	\item 北京时间的小时数减8,如小时数小于0,则小时数加24,日期改为后一天。
	\item 北京时间的小时数加8,如小时数大于24,则小时数减24,日期改为前一天。
	\item 北京时间的小时数减8,如小时数小于0,则小时数加24,日期改为前一天。
\end{enumerate}

\noindent\textbf{解说:}已知北京时间,相应的UTC时间计算方法为:\textbf{北京时间的小时数减8,如小时数小于0,则小时数加24,日期改为前一天}。\\\noindent\textbf{答案:}D


\bigskip


\noindent\textbf{问题:}已知UTC时间,相应的北京时间应为:

\begin{enumerate}[label=\Alph*), leftmargin=3em]
	\item UTC时间的小时数减8,如小时数小于0,则小时数加24,日期改为前一天。
	\item UTC时间的小时数加8,如小时数大于24,则小时数减24,日期改为前一天。
	\item UTC时间的小时数加8,如小时数大于24,则小时数减24,日期改为后一天。
	\item UTC时间的小时数减8,如小时数小于0,则小时数加24,日期改为后一天。
\end{enumerate}
%%%??? 原来的选项,有问题?
%\item 北京时间的小时数减8,如小时数小于0,则小时数加24,日期改为前一天。
%\item 北京时间的小时数加8,如小时数大于24,则小时数减24,日期改为前一天。
%\item UTC时间的小时数加8,如小时数大于24,则小时数减24,日期改为后一天。
%\item 北京时间的小时数减8,如小时数小于0,则小时数加24,日期改为后一天。

\noindent\textbf{解说:}已知UTC时间,相应的北京时间计算方法为:\textbf{UTC时间的小时数加8,如小时数大于24,则小时数减24,日期改为后一天}。\\\noindent\textbf{答案:}C


\bigskip


\noindent\textbf{问题:}我国所属的“CQ分区”有:

\begin{enumerate}[label=\Alph*), leftmargin=3em]
	\item 42、43、44、50
	\item 42、43、44
	\item 23、24
	\item 23、24、27
\end{enumerate}

\noindent\textbf{解说:}我国所属的“”有\textbf{23、24、27}。\\\noindent\textbf{答案:}D


\bigskip


\noindent\textbf{问题:}“ITU分区”是IARU的活动计算通信成绩的基础。我国所属的“ITU分区”有:

\begin{enumerate}[label=\Alph*), leftmargin=3em]
	\item 23、24、27
	\item 33、42、43、44、50
	\item 33、42、43、44
	\item 23、24
\end{enumerate}

\noindent\textbf{解说:}我国所属的“ITU分区”有\textbf{33、42、43、44、50}。\\\noindent\textbf{答案:}B


\bigskip


\noindent\textbf{问题:}业余无线电通信梅登海德网格定位系统(Maidenhead Grid Square Locator)是一种:

\begin{enumerate}[label=\Alph*), leftmargin=3em]
	\item 卫星定位系统
	\item 根据国际政治行政区划对地球表面进行网格划分和命名,用以标示地理位置的系统
	\item 根据国际呼号系列对地球表面进行网格划分和命名,用以标示地理位置的系统
	\item 根据经纬度坐标对地球表面进行网格划分和命名,用以标示地理位置的系统
\end{enumerate}

\noindent\textbf{解说:}业余无线电常用梅森海德网格定位系统(Maidenhead Grid Square Locator)来确定地理位置。\textbf{这种系统根据经纬度坐标对地球表面进行网格划分和命名,用以标示地理位置}。\\\noindent\textbf{答案:}D


\bigskip


\noindent\textbf{问题:}业余无线电通信常用的梅登海德网格定位系统网格名称的格式为:

\begin{enumerate}[label=\Alph*), leftmargin=3em]
	\item 4个字母或者6个字母
	\item 2个字母和2位数字、2个字母和2位数字再加2个字母
	\item 4位数字或者6位数字
	\item 呼号前缀字母加2位数字和2个字母
\end{enumerate}

\noindent\textbf{解说:}业余无线电通信常用的梅登海德网格定位系统网格名称的格式为:4字符(\textbf{2个字母和2位数字})或6字符(\textbf{2个字母和2位数字再加2个字母})。\\\noindent\textbf{答案:}B


\bigskip


\noindent\textbf{问题:}业余无线电通信常用的梅登海德网格定位系统网格名称的长度是4字符或6字符,两者定位精度不同,差别为:

\begin{enumerate}[label=\Alph*), leftmargin=3em]
	\item 4字符网格根据国际呼号系列区分,6字符网格在4字符基础上加以经纬度细分
	\item 4字符网格名称用于HF频段通信,6字符网格名称用于VHF/UHF通信
	\item 两者网格大小不同,4字符网格为经度2度和纬度1度,6字符网格为经度5分和纬度2.5分
	\item 4字符网格精确到国家分区,6字符网格精确到国家的城市或县乡
\end{enumerate}

\noindent\textbf{解说:}业余无线电通信常用的梅登海德网格定位系统网格名称的长度是4字符或6字符,两者定位精度不同,差别为:\textbf{两者网格大小不同,4字符网格为经度2度和纬度1度,6字符网格为经度5分和纬度2.5分}。\\\noindent\textbf{答案:}C

\bigskip


\noindent\textbf{问题:}业余无线电通信所说的“网格定位”是什么意思?

\begin{enumerate}[label=\Alph*), leftmargin=3em]
	\item 一个由一串字母和数字确定的方位角和仰角
	\item 一个由一串字母和数字确定的地理位置
	\item 用来调谐末级功放的设备
	\item 用于无线电测向运动的设备
\end{enumerate}

\noindent\textbf{解说:}业余无线电通信所说的“网格定位”是\textbf{一个由一串字母和数字确定地理位置}的系统。\\\noindent\textbf{答案:}B
