\chapter{附录}

\section{附录1 字母解释法}

\begin{tabular}{cc}%
\begin{tabular}[t]{|c|l|}
	\hline
	\textbf{字母} & \textbf{单词} \\
	\hline
	A & Alfa \\
	\hline
	B & Bravo \\
	\hline
	C & Charlie \\
	\hline
	D & Delta \\
	\hline
	E & Echo \\
	\hline
	F & Foxtrot \\
	\hline
	G & Golf \\
	\hline
	H & Hotel \\
	\hline
	I & India \\
	\hline
	J & Juliett \\
	\hline
	K & Kilo \\
	\hline
	L & Lima \\
	\hline
	M & Mike \\
	\hline
	N & November \\
	\hline
	O & Oscar \\
	\hline
	P & Papa \\
	\hline
	Q & Quebec \\
	\hline
	R & Romeo \\
	\hline
	S & Sierra \\
	\hline
	T & Tango \\
	\hline
	U & Uniform \\
	\hline
	V & Victor \\
	\hline
	W & Whiskey \\
	\hline
	X & X-ray \\
	\hline
	Y & Yankee \\
	\hline
	Z & Zulu \\
	\hline
\end{tabular} &
\begin{tabular}[t]{|c|l|}
	\hline
	\textbf{字母} & \textbf{单词} \\
	\hline
	0 & Zero \\
	\hline
	1 & One \\
	\hline
	2 & Two \\
	\hline
	3 & Three \\
	\hline
	4 & Four \\
	\hline
	5 & Five \\
	\hline
	6 & Six \\
	\hline
	7 & Seven \\
	\hline
	8 & Eight \\
	\hline
	9 & Nine \\
	\hline
\end{tabular} \tabularnewline
\end{tabular}

\newpage

\section{附录2 业余通信中常用的Q简语}

\begin{tabular}{|l|l|l|}
	\hline
	    & \textbf{提问} & \textbf{回答或建议} \\
	\hline
	QRM & 你遇到他台干扰吗 & 我遇到他台干扰 \\
	\hline
	QRN & 你遇到天电干扰吗 & 我遇到天电干扰 \\
	\hline
	QRZ & 谁在呼叫我 & \\
	\hline
	QSL & 你能给我收据(或 QSL 卡片)吗 & 我给你收据(QSL 卡片)、我已收妥 \\
	\hline
	QTH & 你的电台位置在哪里 & 我的电台位置是 ×××× \\
	\hline
\end{tabular}

\newpage



\section{附录3 业余通信常用缩语表}




\newpage

\section{附录4 无线电频段表}

\begin{tabular}{|c|c|c|c|c|}
	\hline
	\textbf{频段名称} & \textbf{缩写} & \textbf{频率范围} \\
	\hline
	甚高频 & VHF & 30–300 MHz \\
	\hline
	特高频 & UHF & 300–3,000 MHz \\
	\hline
	超高频 & SHF & 3–30 GHz \\
	\hline
	极高频 & EHF & 30–300 GHz \\
	\hline
\end{tabular}

\newpage






\section{附录5 业余无线电频率划分}


\begin{tabular}{|c|c|c|}
	\hline
	\textbf{波段名称} & \textbf{频率范围} \\
	%\hline
	%160米 & 1800–2000 kHz & 1.8-2 MHz \\
	\hline
	80米 & 3.5–4.0 MHz \\
	\hline
	40米 & 7.0–7.3 MHz \\
	\hline
	20米 & 14.0–14.35 MHz \\
	\hline
	15米 & 21–21.45 MHz \\
	\hline
	10米 & 28–29.7 MHz \\
	\hline
	6米 & 50–54 MHz \\
	\hline
	2米 & 144–148 MHz \\
	\hline
	0.7米 & 420–450 MHz \\
	\hline
\end{tabular}

\newpage

\section{附录6 计算公式}




\newpage

\section{附录7 国际单位制词头}

\begin{tabular}{|c|c|}
	\hline
	\textbf{符号} & \textbf{意义} \\
	\hline
	T & $10^{12}$ \\
	\hline
	G & $10^{9}$ \\
	\hline
	M & $10^{6}$ \\
	\hline
	k & $10^{3}$ \\
	\hline
	 & $10^{0}$ \\
	\hline
	m & $10^{-3}$ \\
	\hline
	μ & $10^{-6}$ \\
	\hline
	n & $10^{-9}$ \\
	\hline
	p & $10^{-12}$ \\
	\hline
\end{tabular}

