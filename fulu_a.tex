\chapter{附录}

\section{业余无线电台分区表}

\begin{center}
  \begin{tabular}[t]{|c|l|}
    \hline
    \multicolumn{1}{|c|}{\textbf{分区号}} & \multicolumn{1}{|c|}{\textbf{地区}} \\
    \hline
    第1区                                & 北京                                \\
    \hline
    第2区                                & 黑龙江、吉林、辽宁                         \\
    \hline
    第3区                                & 天津、内蒙古、河北、山西                      \\
    \hline
    第4区                                & 上海、山东、江苏                          \\
    \hline
    第5区                                & 浙江、江西、福建                          \\
    \hline
    第6区                                & 安徽、河南、湖北                          \\
    \hline
    第7区                                & 湖南、广东、广西、海南                       \\
    \hline
    第8区                                & 四川、重庆、贵州、云南                       \\
    \hline
    第9区                                & 陕西、甘肃、宁夏、青海                       \\
    \hline
    第0区                                & 新疆、西藏                             \\
    \hline
  \end{tabular}
\end{center}

\newpage

\section{字母解释法(节录)}

\begin{center}
  \begin{tabular}{cc}
    \begin{tabular}[t]{|c|l|}
      \hline
      \textbf{字母} & \textbf{单词} \\
      \hline
      A           & Alfa        \\
      \hline
      B           & Bravo       \\
      \hline
      C           & Charlie     \\
      \hline
      D           & Delta       \\
      \hline
      E           & Echo        \\
      \hline
      F           & Foxtrot     \\
      \hline
      G           & Golf        \\
      \hline
      H           & Hotel       \\
      \hline
      I           & India       \\
      \hline
      J           & Juliett     \\
      \hline
      K           & Kilo        \\
      \hline
      L           & Lima        \\
      \hline
      M           & Mike        \\
      \hline
      N           & November    \\
      \hline
      O           & Oscar       \\
      \hline
      P           & Papa        \\
      \hline
      Q           & Quebec      \\
      \hline
      R           & Romeo       \\
      \hline
      S           & Sierra      \\
      \hline
      T           & Tango       \\
      \hline
      U           & Uniform     \\
      \hline
      V           & Victor      \\
      \hline
      W           & Whiskey     \\
      \hline
      X           & X-ray       \\
      \hline
      Y           & Yankee      \\
      \hline
      Z           & Zulu        \\
      \hline
    \end{tabular} &
    \begin{tabular}[t]{|c|l|}
      \hline
      \textbf{字母} & \textbf{单词} \\
      \hline
      0           & Zero        \\
      \hline
      1           & One         \\
      \hline
      2           & Two         \\
      \hline
      3           & Three       \\
      \hline
      4           & Four        \\
      \hline
      5           & Five        \\
      \hline
      6           & Six         \\
      \hline
      7           & Seven       \\
      \hline
      8           & Eight       \\
      \hline
      9           & Nine        \\
      \hline
    \end{tabular}
    \begin{tabular}[t]{|c|l|}
      \hline
      \textbf{字母} & \textbf{单词}      \\
      \hline
      /           & Portable, Stroke \\
      \hline
      ?           & Question         \\
      \hline
      --          & Hyphen, Dash     \\
      \hline
      .           & Decimal          \\
      \hline
    \end{tabular}  \tabularnewline
  \end{tabular}
\end{center}

\newpage

\section{常用Q简语}

\begin{longtable}{|l|l|l|}
  \hline
               & \textbf{提问}                 & \textbf{回答或建议}                                     \\
  \hline
  \texttt{QRM} & 你遇到他台干扰吗?                   & 我遇到他台干扰                                            \\
  \hline
  \texttt{QRN} & 你遇到天电干扰吗?                   & 我遇到天电干扰                                            \\
  \hline
  \texttt{QRZ} & 谁在呼叫我?                      & ××××在××× \unit{\kHz}/\unit{\MHz} 上呼叫你\cite{Q_code} \\
  \hline
  \texttt{QSL} & 你能给我收据(或 \texttt{QSL} 卡片)吗? & 我给你收据(\texttt{QSL} 卡片)、我已收妥                        \\
  \hline
  \texttt{QTH} & 你的电台位置在哪里?                  & 我的电台位置是 ××××                                       \\
  \hline
\end{longtable}

\newpage

\section{常用俗语}

\begin{longtable}{|l|l|l|}
  \hline
  \textbf{词语} & \textbf{含义} & \textbf{例句}        \\
  \hline
  QRA         & 呼号          & 你的QRA再过来一遍         \\
  \hline
  QRZ         & 谁           & QRZ刚才是谁在呼叫我?       \\
  \hline
  QSY         & 移动          & 你什么时候从上海QSY到北京?    \\
  \hline
  QTH         & 位置          & 在我的QTH这里原装天线完全无法使用 \\
  \hline
  QRV         & 工作          & 我主要QRV在70厘米FM模式下   \\
  \hline
  QSO         & 通联          & 我和xxxxxx友台QSO过     \\
  \hline
  73          & 再见          & 我有点事,先73了          \\
  \hline
\end{longtable}

\newpage

\section{业余无线电通信常用缩语}

说明:黑体字的缩语为考到的重要缩语。

\begin{longtable}[l]{lll}
  \textbf{缩语}    & \textbf{原词}                          & \textbf{含义}         \\
  \endfirsthead
  \textbf{缩语}    & \textbf{原词}                          & \textbf{含义}         \\
  \endhead
  \textbf{AM}    & Amplitude modulation                 & 幅度调制                \\
  \textbf{ANT}   &                                      & 天线                  \\
  \textbf{ARDF}  & Amateur radio direction finding      & 业余无线电测向             \\
  \textbf{BEAM}  & Beam antenna                         & 定向天线                \\
  BREAK          &                                      & 我还没有说完              \\
  BREAK BREAK    &                                      & 我还没有说完              \\
  Break in       &                                      & 能在你的信号间隙中接收吗        \\
  CALL SIGN      &                                      & 呼号                  \\
  CQ             &                                      & 普遍呼叫                \\
  \textbf{CTCSS} & Continuous Tone-Coded Squelch System & 亚音调静噪               \\
  DATE           &                                      & 日期                  \\
  dB             & decibel                              & 增益(单位)              \\
  DTMF           & Dual-tone multi-frequency signaling  & 双音多频编码              \\
  \textbf{DP}    & Dipole antenna                       & 偶级天线                \\
  EHF            & Extremely high frequency             & 极高频                 \\
  FET            & Field-effect transistor              & 场效应管                \\
  FM             & Frequency modulation                 & 频率调制、调频             \\
  FREQ           & Frequency                            & 频率                  \\
  \unit{\GHz}    & gigahertz                            & 吉赫兹(单位)             \\
  \textbf{GND}   & Ground                               & 地线、地面               \\
  \textbf{GP}    & Ground-plane antenna                 & 垂直接地天线              \\
  HF             & High frequency                       & 高频、即SW(Shortwave)   \\
  Hz             & hertz                                & 赫兹(单位)              \\
  \unit{\kHz}    & kilohertz                            & 千赫兹(单位)             \\
  \textbf{LCD}   & Liquid-crystal display               & 液晶显示器               \\
  \textbf{LED}   & Light-emitting diode                 & 发光二极管               \\
  \unit{\MHz}    & megahertz                            & 兆赫兹(单位)             \\
  MODE           &                                      & 模式                  \\
  NFM            & Narrowband FM                        & 窄带调频                \\
  \textbf{OM}    & Old man                              & 老朋友                 \\%朋友
  OVER           &                                      & 我的传输已结束,期待回复        \\
  \textbf{PM}    & Phase modulation                     & 相位调制                \\
  \textbf{PTT}   & Push-to-talk                         & 按键发射                \\
  \textbf{RCVR}  & Receiver                             & 收信机                 \\
  \textbf{RF}    & Radio frequency                      & 无线电波                \\
  \textbf{RIG}   & Equipment                            & 电台设备                \\
  \textbf{ROGER} &                                      & 回答起始语,相当于“明白”,      \\
                 &                                      & 仅在已完全抄收对方刚才发送的信息时使用 \\
  RST            & Readability, Strength, Tone          & 可辨度、信号强度、音调系统       \\
  \textbf{RX}    & Receiver                             & 收信机                 \\
  SHF            & Super high frequency                 & 超高频                 \\
  \textbf{SSB}   & Single-sideband modulation           & 单边带调制               \\
  STANDBY        &                                      & 等待你的呼叫              \\
  \textbf{SQL}   & Squelch                              & 静噪控制                \\
  THIS IS        &                                      & 此消息发自以下xxxx电台       \\
  TIME           &                                      & 时间                  \\
  TNC            & Terminal node controller             & 终端节点控制器             \\
  \textbf{TX}    & Transmitter                          & 发信机                 \\
  UHF            & Ultra high frequency                 & 特高频                 \\
  UTC            & Coordinated Universal Time           & 协调世界时               \\
  \textbf{VER}   & Vertical antenna                     & 垂直天线                \\
  VHF            & Very high frequency                  & 甚高频                 \\
  \textbf{VOX}   & Voice-operated switch                & 发信机声控               \\
  WFM            & Wideband FM                          & 宽带调频                \\
  \textbf{WX}    & Weather                              & 天气                  \\
  \textbf{XMTR}  & Transmitter                          & 发信机                 \\
  \textbf{XCVR}  & Transceiver                          & 收发信机                \\
  \textbf{YAGI}  & Yagi–Uda antenna                     & 八木天线                \\
  \textbf{73}    &                                      & 向对方的致意、美好的祝愿        \\
\end{longtable}

\newpage

\section{术语表}

\begin{longtable}[l]{ll}
  \textbf{术语} & \textbf{含义} \\
  Emission    & 发射          \\
  Eyeball QSO & 眼球QSO       \\
  Radiation   & 辐射          \\
\end{longtable}

\newpage

\section{无线电频段表(节录)}

黑体字的缩写为考到的重要缩写。

\begin{longtable}[c]{|l|l|l|}
  \hline
  \textbf{频段名称} & \textbf{缩写}  & \textbf{频率范围}                               \\
  \hline
  高频            & \textbf{HF}  & \qtyrange[range-phrase=--]{3}{30}{\MHz}     \\
  \hline
  甚高频           & \textbf{VHF} & \qtyrange[range-phrase=--]{30}{300}{\MHz}   \\
  \hline
  特高频           & \textbf{UHF} & \qtyrange[range-phrase=--]{300}{3000}{\MHz} \\
  \hline
  %超高频           & SHF          & \qtyrange[range-phrase=--]{3}{30}{\GHz}      \\
  %\hline
  %极高频           & EHF          & \qtyrange[range-phrase=--]{30}{300}{\GHz}    \\
  %\hline
\end{longtable}

\newpage

\section{中国大陆各业余无线电频段}

\begin{longtable}{|l|l|}
  \hline
  \textbf{波段名称} & \textbf{频率范围}                                    \\
  \hline
  2200米         & \qtyrange[range-phrase=--]{135.7}{137.8}{\kHz}   \\
  \hline
  160米          & \qtyrange[range-phrase=--]{1800}{2000}{\kHz}     \\
  \hline
  80米           & \qtyrange[range-phrase=--]{3.5}{3.9}{\MHz}       \\
  \hline
  60米           & \qtyrange[range-phrase=--]{5.3515}{5.3665}{\MHz} \\
  \hline
  40米           & \qtyrange[range-phrase=--]{7.0}{7.2}{\MHz}       \\
  \hline
  30米           & \qtyrange[range-phrase=--]{10.1}{10.15}{\MHz}    \\
  \hline
  20米           & \qtyrange[range-phrase=--]{14.0}{14.35}{\MHz}    \\
  \hline
  17米           & \qtyrange[range-phrase=--]{18.068}{18.168}{\MHz} \\
  \hline
  15米           & \qtyrange[range-phrase=--]{21}{21.45}{\MHz}      \\
  \hline
  12米           & \qtyrange[range-phrase=--]{24.89}{24.99}{\MHz}   \\
  \hline
  10米           & \qtyrange[range-phrase=--]{28}{29.7}{\MHz}       \\
  \hline
  6米            & \qtyrange[range-phrase=--]{50}{54}{\MHz}         \\
  \hline
  2米            & \qtyrange[range-phrase=--]{144}{148}{\MHz}       \\
  \hline
  70厘米          & \qtyrange[range-phrase=--]{430}{440}{\MHz}       \\
  \hline
  23厘米          & \qtyrange[range-phrase=--]{1240}{1300}{\MHz}     \\
  \hline
\end{longtable}

\newpage

\section{国际单位制词头(节录)}

\begin{longtable}{|l|l|}
  \hline
  \textbf{符号} & \textbf{意义} \\
  \hline
  T           & \num{e12}   \\
  \hline
  G           & \num{e9}    \\
  \hline
  M           & \num{e6}    \\
  \hline
  k           & \num{e3}    \\
  \hline
              & \(10^{0}\)  \\
  \hline
  m           & \num{e-3}   \\
  \hline
  μ           & \num{e-6}   \\
  \hline
  n           & \num{e-9}   \\
  \hline
  p           & \num{e-12}  \\
  \hline
\end{longtable}

\newpage

\section{计算公式}

两数的乘积的对数等于两数的对数之和:
\[\log a\: b=\log a+\log b\]

某数的n次幂的对数等于该数的对数的n倍:
\[\log a^n =n \cdot \log a\]

直流电路欧姆定律:
\[\mbox{电流}I_{(\unit{\ampere})} = \frac{\mbox{电压}U_{(\unit{\volt})}}{\mbox{电阻值}R_{(\unit{\ohm})}}\]

峰值电压的计算公式:
\[\mbox{峰值电压}_{(\unit{\volt})} = \frac{\mbox{峰--峰值}}{2}\]

正弦交流电压的有效值计算公式:
\begin{equation*}
  \begin{aligned}
    \mbox{交流电压的有效值}_{(\unit{\volt})} & =\frac{\mbox{峰值}}{\sqrt{2}}   \\
                                     & \approx \mbox{峰值}\times 0.707
  \end{aligned}
\end{equation*}

功率增益的计算公式:
\[\mbox{功率增益}= 10 \cdot \log_{10} \left( {\frac{P_{ \mbox{输出} }}{P_{ \mbox{输入} }}}\right) \unit{\dB}\]

将功率(\unit{\watt})转换为功率(\unit[qualifier-mode=combine]{\deci\bel\of{W}})的计算公式:

\begin{equation*}
  \begin{aligned}
    P_{(\unit[qualifier-mode=combine]{\deci\bel\of{W}})} & = 10 \cdot \log_{10} P_{(\unit{\watt})} \\
  \end{aligned}
\end{equation*}

将功率(\unit{\watt})转换为功率(\unit[qualifier-mode=combine]{\deci\bel\of{m}})的计算公式:

\begin{equation*}
  \begin{aligned}
    P_{(\unit[qualifier-mode=combine]{\deci\bel\of{m}})} & = 10 \cdot \log_{10} \left( 1000 \cdot P_{(\unit{\watt})} \middle/ 1 \unit{\watt} \right) \\
  \end{aligned}
\end{equation*}

将功率(\unit{\watt})转换为功率(\unit[qualifier-mode=combine]{\deci\bel\of{μ}})的计算公式:

\begin{equation*}
  \begin{aligned}
    P_{(\unit[qualifier-mode=combine]{\deci\bel\of{μ}})} & = 10 \cdot \log_{10} \left( 1000000 \cdot P_{(\unit{\watt})} \middle/ 1 \unit{\watt} \right) \\
  \end{aligned}
\end{equation*}

\newpage

\section{常用对数表(节录)}

强烈建议背诵$\log_{10} 1.0$、$\log_{10} 2.0$、……$\log_{10} 10.0$等第0列的10个数值到小数点后三位,考试闭卷,无法使用计算器,如果不背诵,将无法计算。

\begin{longtable}[c]{|c|c|}
  \hline
  \textbf{N} & \textbf{0}  \\
  \hline
  %	\endfirsthead
  %	\hline
  %	\textbf{N} & \textbf{1.0} \\
  \endhead
  \num{1.0}  & \num{.000}  \\ \hline
  \num{2.0}  & \num{.301}  \\ \hline
  \num{3.0}  & \num{.477}  \\ \hline
  \num{4.0}  & \num{.602}  \\ \hline
  \num{5.0}  & \num{.699}  \\ \hline
  \num{6.0}  & \num{.778}  \\ \hline
  \num{7.0}  & \num{.845}  \\ \hline
  \num{8.0}  & \num{.903}  \\ \hline
  \num{9.0}  & \num{.954}  \\ \hline
  \num{10.0} & \num{1.000} \\ \hline
\end{longtable}

\newpage

\section{将呼号转换成ICAO字母解释法的单词组合的C程序}

%\lstinputlisting[language=C]{icao.c}

\begin{verbatim}
// 将呼号转换成ICAO字母解释法的单词组合的C程序
// 用法:
//   $ gcc -Wall -Wpedantic -Wextra -std=c99 -o icao icao.c
//   $ ./icao
// 特别鸣谢:ChatGPT
#include <stdio.h>
#include <ctype.h>
#include <string.h>

const char* natoTranslate(char c)
{
	switch (toupper(c))
	{
		case 'A':
		return "Alfa";
		case 'B':
		return "Bravo";
		case 'C':
		return "Charlie";
		case 'D':
		return "Delta";
		case 'E':
		return "Echo";
		case 'F':
		return "Foxtrot";
		case 'G':
		return "Golf";
		case 'H':
		return "Hotel";
		case 'I':
		return "India";
		case 'J':
		return "Juliett";
		case 'K':
		return "Kilo";
		case 'L':
		return "Lima";
		case 'M':
		return "Mike";
		case 'N':
		return "November";
		case 'O':
		return "Oscar";
		case 'P':
		return "Papa";
		case 'Q':
		return "Quebec";
		case 'R':
		return "Romeo";
		case 'S':
		return "Sierra";
		case 'T':
		return "Tango";
		case 'U':
		return "Uniform";
		case 'V':
		return "Victor";
		case 'W':
		return "Whiskey";
		case 'X':
		return "X-ray";
		case 'Y':
		return "Yankee";
		case '0':
		return "Zero";
		case '1':
		return "One";
		case '2':
		return "Two";
		case '3':
		return "Three";
		case '4':
		return "Four";
		case '5':
		return "Five";
		case '6':
		return "Six";
		case '7':
		return "Seven";
		case '8':
		return "Eight";
		case '9':
		return "Nine";
		case '/':
		return "Stroke";
		default:
		return 0;
	}
}

int main(void)
{
	printf("请输入要翻译的呼号:");
	char ch[64];
	int i;
	if (fgets(ch, 64, stdin))
	{
		ch[strcspn(ch, "\n")] = 0;
	}
	for (i = 0; ch[i] != '\0'; i++)
	{
		if (ch[i] == '/' || isalnum(ch[i]))
		printf("%c", ch[i]);
	}
	printf("的字母解释法是:");
	for (i = 0; ch[i] != '\0'; i++)
	{
		if (ch[i] == '/' || isalnum(ch[i]))
		printf("%s ", natoTranslate(ch[i]));
	}
	printf("\n");
	return 0;
}

\end{verbatim}

\newpage

\section{有用网址}

\begin{longtable}{|p{8cm}|p{8cm}|}
  \hline
  \textbf{名称}           & \textbf{网址}                                                     \\
  \hline
  业余无线电台操作技术能力验证考核报名系统  & \url{http://82.157.138.16:8091/CRAC/crac/login_student.html}    \\
  \hline
  复习试题和模拟考试系统下载         & \url{http://82.157.138.16:8091/CRAC/crac/pages/list_files.html} \\
  \hline
  中国无线电协会业余无线电分会资料下载    & \url{http://www.crac.org.cn/News/List?type=6}                   \\
  \hline
  国际电联《无线电规则》(2020年版)   & \url{https://www.itu.int/hub/publication/r-reg-rr-2020/}        \\
  \hline
  业余无线电爱好者的道德和操作守则(第3版) & \url{https://www.iaru-r1.org/wp-content/uploads/2019/10/        %E4%B8%9A%E4%BD%99%E6%97%A0%E7%BA%BF%E7%94%B5%E7%88%B1%E5%A5%BD%E8%80%85%E7%9A%84%E9%81%93%E5%BE%B7%E5%92%8C%E6%93%8D%E4%BD%9C%E5%AE%88%E5%88%99.pdf}                   \\
  \hline
\end{longtable}
