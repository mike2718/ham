\chapter{附录}

\section{字母解释法(节录)}

\begin{tabular}{cc}
\begin{tabular}[t]{|c|l|}
	\hline
	\textbf{字母} & \textbf{单词} \\
	\hline
	A & Alfa \\
	\hline
	B & Bravo \\
	\hline
	C & Charlie \\
	\hline
	D & Delta \\
	\hline
	E & Echo \\
	\hline
	F & Foxtrot \\
	\hline
	G & Golf \\
	\hline
	H & Hotel \\
	\hline
	I & India \\
	\hline
	J & Juliett \\
	\hline
	K & Kilo \\
	\hline
	L & Lima \\
	\hline
	M & Mike \\
	\hline
	N & November \\
	\hline
	O & Oscar \\
	\hline
	P & Papa \\
	\hline
	Q & Quebec \\
	\hline
	R & Romeo \\
	\hline
	S & Sierra \\
	\hline
	T & Tango \\
	\hline
	U & Uniform \\
	\hline
	V & Victor \\
	\hline
	W & Whiskey \\
	\hline
	X & X-ray \\
	\hline
	Y & Yankee \\
	\hline
	Z & Zulu \\
	\hline
\end{tabular} &
\begin{tabular}[t]{|c|l|}
	\hline
	\textbf{字母} & \textbf{单词} \\
	\hline
	0 & Zero \\
	\hline
	1 & One \\
	\hline
	2 & Two \\
	\hline
	3 & Three \\
	\hline
	4 & Four \\
	\hline
	5 & Five \\
	\hline
	6 & Six \\
	\hline
	7 & Seven \\
	\hline
	8 & Eight \\
	\hline
	9 & Nine \\
	\hline
\end{tabular} \tabularnewline
\end{tabular}

\newpage

\section{业余通信中常用的Q简语(节录)}

\begin{longtable}{|l|l|l|}
	\hline
	    & \textbf{提问} & \textbf{回答或建议} \\
	\hline
	QRM & 你遇到他台干扰吗 & 我遇到他台干扰 \\
	\hline
	QRN & 你遇到天电干扰吗 & 我遇到天电干扰 \\
	\hline
	QRZ & 谁在呼叫我 & \\
	\hline
	QSL & 你能给我收据(或 QSL 卡片)吗 & 我给你收据(QSL 卡片)、我已收妥 \\
	\hline
	QTH & 你的电台位置在哪里 & 我的电台位置是 ×××× \\
	\hline
\end{longtable}

\newpage



\section{业余通信常用缩语表(节录)}

说明:黑体字的缩语为考到的重要缩语。

\begin{longtable}[l]{lll}
	\textbf{缩语} & \textbf{原词} & \textbf{含义} \\
	\endfirsthead
	\textbf{缩语} & \textbf{原词} & \textbf{含义} \\
	\endhead
	\textbf{AM} & Amplitude modulation & 幅度调制 \\
	\textbf{ANT} & & 天线 \\
	\textbf{ARDF} & Amateur radio direction finding & 业余无线电测向 \\
	\textbf{BEAM} & Beam antenna & 定向天线 \\
	BREAK & & 我还没有说完 \\
	BREAK BREAK & & 我还没有说完 \\
	Break in & & 能在你的信号间隙中接收吗 \\
	CALL SIGN & & 呼号 \\
	CQ & & 普遍呼叫 \\
	\textbf{CTCSS} & Continuous Tone-Coded Squelch System & 亚音调静噪 \\
	DATE & & 日期 \\
	dB & decibel & 增益(单位) \\
	DTMF & Dual-tone multi-frequency signaling & 双音多频编码 \\
	\textbf{DP} & Dipole antenna & 偶级天线 \\
	EHF & Extremely high frequency & 极高频 \\
	FET & Field-effect transistor & 场效应管 \\
	FM & Frequency modulation & 频率调制、调频 \\
	FREQ & Frequency & 频率 \\
	GHz & gigahertz & 吉赫兹(单位) \\
	\textbf{GND} & Ground & 地线、地面 \\
	\textbf{GP} & Ground-plane antenna & 垂直接地天线 \\
	HF & High frequency & 高频、即SW(Shortwave) \\
	Hz & hertz & 赫兹(单位) \\
	kHz & kilohertz & 千赫兹(单位) \\
	\textbf{LCD} & Liquid-crystal display & 液晶显示器 \\
	\textbf{LED} & Light-emitting diode & 发光二极管 \\
	MHz & megahertz & 兆赫兹(单位) \\
	MODE & & 模式 \\
	NFM & Narrowband FM & 窄带调频 \\
	\textbf{OM} & Old man & 老朋友 \\%朋友
	OVER & & 我的传输已结束,期待回复 \\
	\textbf{PM} & Phase modulation & 相位调制 \\
	\textbf{PTT} & Push-to-talk & 按键发射 \\
	\textbf{RCVR} & Receiver & 收信机 \\
	\textbf{RF} & Radio frequency & 无线电波 \\
	\textbf{RIG} & Equipment & 电台设备 \\
	\textbf{ROGER} & & 回答起始语,相当于“明白”, \\
	&  & 仅在已完全抄收对方刚才发送的信息时使用 \\
	RST & Readability, Strength, Tone & 可辨度、信号强度、音调系统 \\
	\textbf{RX} & Receiver & 收信机 \\
	SHF & Super high frequency & 超高频 \\
	\textbf{SSB} & Single-sideband modulation & 单边带调制 \\
	STANDBY & & 等待你的呼叫 \\
	\textbf{SQL} & Squelch & 静噪控制 \\
	THIS IS & & 此消息发自以下xxxx电台 \\
	TIME & & 时间 \\
	TNC & Terminal node controller & 终端节点控制器 \\
	\textbf{TX} & Transmitter & 发信机 \\
	UHF & Ultra high frequency & 特高频 \\
	UTC & Coordinated Universal Time & 协调世界时 \\
	\textbf{VER} & Vertical antenna & 垂直天线 \\
	VHF & Very high frequency & 甚高频 \\
	\textbf{VOX} & Voice-operated switch & 发信机声控 \\
	WFM & Wideband FM & 宽带调频 \\
	\textbf{WX} & Weather & 天气 \\
	\textbf{XMTR} & Transmitter & 发信机 \\
	\textbf{XCVR} & Transceiver & 收发信机 \\
	\textbf{YAGI} & Yagi–Uda antenna & 八木天线 \\
	\textbf{73} & & 向对方的致意、美好的祝愿 \\
\end{longtable}


\newpage

\section{术语表}

\begin{longtable}[l]{ll}
	\textbf{术语} & \textbf{含义} \\
	Emission & 发射 \\
	Eyeball QSO & 眼球QSO \\
	GB & 中国强制性国家标准前缀 \\
	GB/T & 中国推荐性国家标准前缀 \\
	Radiation & 辐射 \\
\end{longtable}

\newpage

\section{无线电频段表(节录)}

说明:黑体字的缩写为考到的重要缩写。

\begin{longtable}{|c|c|c|}
	\hline
	\textbf{频段名称} & \textbf{缩写} & \textbf{频率范围} \\
	\hline
	高频 & \textbf{HF} & 3–30 MHz \\
	\hline
	甚高频 & \textbf{VHF} & 30–300 MHz \\
	\hline
	特高频 & \textbf{UHF} & 300–3,000 MHz \\
	\hline
	超高频 & SHF & 3–30 GHz \\
	\hline
	极高频 & EHF & 30–300 GHz \\
	\hline
\end{longtable}

\newpage






\section{业余无线电频率划分表(节录)}


\begin{longtable}{|c|c|}
	\hline
	\textbf{波段名称} & \textbf{频率范围} \\
	\hline
	80米 & 3.5–4.0 MHz \\
	\hline
	40米 & 7.0–7.3 MHz \\
	\hline
	20米 & 14.0–14.35 MHz \\
	\hline
	15米 & 21–21.45 MHz \\
	\hline
	10米 & 28–29.7 MHz \\
	\hline
	6米 & 50–54 MHz \\
	\hline
	2米 & 144–148 MHz \\
	\hline
	0.7米 & 420–450 MHz \\
	\hline
\end{longtable}

\newpage



\section{国际单位制词头(节录)}

\begin{longtable}{|c|c|}
	\hline
	\textbf{符号} & \textbf{意义} \\
	\hline
	T & $10^{12}$ \\
	\hline
	G & $10^{9}$ \\
	\hline
	M & $10^{6}$ \\
	\hline
	k & $10^{3}$ \\
	\hline
	 & $10^{0}$ \\
	\hline
	m & $10^{-3}$ \\
	\hline
	μ & $10^{-6}$ \\
	\hline
	n & $10^{-9}$ \\
	\hline
	p & $10^{-12}$ \\
	\hline
\end{longtable}

\newpage

\section{计算公式}

直流电路欧姆定律:
$$\mbox{电流}I(A)=\frac{\mbox{电压}U(V)}{\mbox{电阻值}R(\Omega)}$$

峰值电压的计算公式:
$$\mbox{峰值}(V)=\frac{\mbox{峰--峰值}}{2}$$

正弦交流电压的有效值计算公式:
$$\mbox{交流的有效值}(V)=\frac{\mbox{峰值}}{\sqrt{2}}=\mbox{峰值}\times0.707$$

功率增益的计算公式:
$$\mbox{功率增益}=10 \log_{10} \left( {\frac{P_{ \mbox{输出} }}{P_{ \mbox{输入} }}}\right)\ \mathrm{dB}$$

%电压增益的计算公式:
%$$\mbox{电压增益}=20 \log \left( {\frac{V_{ \mbox{输出} }}{V_{ \mbox{输入} }}} \right)\ \mathrm{dB}$$

%电流增益的计算公式:
%$$\mbox{电流增益}=20 \log \left( {\frac{I_{ \mbox{输出} }}{I_{ \mbox{输入} }}} \right)\ \mathrm{dB}$$

\newpage

\section{业余无线电台呼号所属分区信息查询C程序}

\lstinputlisting[language=C]{hhcx.c}

\newpage

\section{有用网址}

\begin{longtable}{|p{8cm}|p{8cm}|}
	\hline
	\textbf{名称} & \textbf{网址} \\
	\hline
	业余无线电台操作技术能力验证考核报名及信息管理系统 & \url{http://114.115.246.55:8091/CRAC/crac/index.html} \\
	\hline
	中国无线电协会业余无线电分会 & \url{http://www.crac.org.cn} \\
	\hline
\end{longtable}
