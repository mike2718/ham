\chapter{无线电通信的方法}




\noindent\textbf{问题:}自制业余无线电发射设备,在经无线电检测机构检测合格并取得电台执照之前,调试时天线输出端应连接(或串联必要的仪表后连接):
\begin{enumerate}[label=\Alph*), leftmargin=3em]
\item 假负载
\item VSWR严格等于1:1的驻波天线
\item VSWR严格等于1:1的行波天线
\item 测试专用的标准环形天线
\end{enumerate}

\bigskip


\noindent\textbf{问题:}某俱乐部约定了一个成员业余电台之间交流技术的网络频率,当遇有其他业余电台按通信惯例要求参加通信时,处理原则应为:
\begin{enumerate}[label=\Alph*), leftmargin=3em]
\item 无条件欢迎加入,因为任何核准的业余电台对频率享有平等的频率使用权
\item 要求其他业余电台在任何时间都不得使用俱乐部自己约定的专用通信频率
\item 要求其他业余电台在俱乐部成员结束网络通信后再使用该频率
\item 由俱乐部网络控制台决定是其他业余电台是否可以加入
\end{enumerate}

\bigskip


\noindent\textbf{问题:}业余电台在发起呼叫前不可缺少的操作步骤是:
\begin{enumerate}[label=\Alph*), leftmargin=3em]
\item 先守听一段时间,确保没有其他电台正在使用频率
\item 检查发射功率是否达到设备的额定输出功率
\item 先用礼貌的语言请其他电台让出频率
\item 先用吹话筒、吹口哨等方法发出连续信号检查天线驻波比
\end{enumerate}

\bigskip


\noindent\textbf{问题:}业余电台在发射调试信号进行发射功率和天线驻波比等检查时必须注意做到的是:
\begin{enumerate}[label=\Alph*), leftmargin=3em]
\item 先将频率设置到无人使用的空闲频率、偏离常用的热点频率
\item 先将天线的发射方向指向正北
\item 先将收发信机的语音压缩功能打开
\item 话筒离嘴距离在2公分以上,电键按键时间不短于5秒钟
\end{enumerate}

\bigskip


\noindent\textbf{问题:}单边带业余电台在测试检查天线驻波比需要发射平稳的连续信号。文明的作法是:
\begin{enumerate}[label=\Alph*), leftmargin=3em]
\item 先将电台设为CW方式按电键,或者设为AM或FM方式按PTT键(不对话筒说话),产生连续载波,测试结束后设回SSB方式
\item 将电台设为SSB方式,用平稳的气流对话筒吹口哨
\item 将电台设为SSB方式,深呼吸后用平稳的气流对话筒发长音“啊”
\item 将电台设为SSB方式,深呼吸后用平稳的气流对话筒发长音“嘻”
\end{enumerate}

\bigskip


\noindent\textbf{问题:}业余电台发起呼叫前应先守听一段时间,如没有听到信号,应再询问“有人使用频率吗”?确认没有应答方能发起呼叫。下列英语短句中不能正确表达这一询问的是:
\begin{enumerate}[label=\Alph*), leftmargin=3em]
\item Calling you, Roger?
\item Is the frequency in use?
\item Is any body in the frequency?
\item Any body here?
\end{enumerate}

\bigskip


\noindent\textbf{问题:}业余电台发起呼叫前应先守听一段时间,如没有听到信号,应再询问“有人使用频率吗”?确认没有应答方能发起呼叫。用CW表达这一询问的方法是:
\begin{enumerate}[label=\Alph*), leftmargin=3em]
\item QRL?
\item QRX?
\item QRZ?
\item QRV?
\end{enumerate}

\bigskip


\noindent\textbf{问题:}业余电台BH1ZZZ用话音发起CQ呼叫的正确格式为:
\begin{enumerate}[label=\Alph*), leftmargin=3em]
\item CQ、CQ、CQ。BH1ZZZ呼叫。Bravo Hotel One Zulu Zulu Zulu呼叫,BH1ZZZ呼叫。听到请回答。
\item CQ、CQ、CQ。听到请回答。
\item CQ、CQ、CQ。我是1ZZZ。听到请回答
\item CQ、CQ、CQ,CQ、CQ、CQ,CQ、CQ、CQ。BH1ZZZ呼叫。请过来。
\end{enumerate}

\bigskip


\noindent\textbf{问题:}业余电台BH1ZZZ用话音发起CQ呼叫的正确格式为:
\begin{enumerate}[label=\Alph*), leftmargin=3em]
\item CQ CQ CQ.This is BH1ZZZ. Bravo Hotel One Zulu Zulu Zulu, BH1ZZZ is calling. I’m standing by.
\item CQ CQ CQ. Go ahead please.
\item CQ CQ CQ. This One Zulu Zulu Zulu. Over.
\item CQ CQ CQ, CQ CQ CQ, CQ CQ CQ. This is BH1ZZZ. Back to you.
\end{enumerate}

\bigskip


\noindent\textbf{问题:}业余电台BH1ZZZ用话音呼叫BH8YYY的正确格式为:
\begin{enumerate}[label=\Alph*), leftmargin=3em]
\item BH8YYY、BH8YYY、BH8YYY。BH1ZZZ呼叫。Bravo Hotel One Zulu Zulu Zulu,BH1ZZZ呼叫。听到请回答。
\item BH8YYY。我是BH1ZZZ,我是BH1ZZZ,我是BH1ZZZ。听到请回答。
\item BH8YYY、BH8YYY、BH8YYY。我是1ZZZ。听到请回答
\item 8YYY、8YYY、8YYY。BH1ZZZ呼叫。请过来。
\end{enumerate}

\bigskip


\noindent\textbf{问题:}业余电台BH1ZZZ用话音呼叫BH8YYY的正确格式为:
\begin{enumerate}[label=\Alph*), leftmargin=3em]
\item Bravo Hotel Eight Yankee Yankee Yankee, Bravo Hotel Eight Yankee Yankee Yankee, Bravo Hotel Eight Yankee Yankee Yankee.This is Bravo Hotel One Zulu Zulu Zulu. Bravo Hotel One Zulu Zulu Zulu, Bravo Hotel One Zulu Zulu Zulu is calling. I’m standing by.
\item Bravo Hotel Eight Yankee Yankee Yankee, Bravo Hotel Eight Yankee Yankee Yankee, Bravo Hotel Eight Yankee Yankee Yankee. Go ahead please.
\item BH8YYY, BH8YYY, BH8YYY. This One Zulu Zulu Zulu. Come in please.
\item 8YYY, 8YYY, YYY. This is BH1ZZZ. Over.
\end{enumerate}

\bigskip


\noindent\textbf{问题:}BH1ZZZ希望加入两个电台正在通信中的谈话,正确的方法为:
\begin{enumerate}[label=\Alph*), leftmargin=3em]
\item 在双方对话的间隙,短暂发射一次“Break in!”或“插入!”,如得到响应,再说明本台呼号 “BH1ZZZ请求插入”,等对方正式表示邀请后,方能加入
\item 在一方正在发射期间,短暂插入一次“Break in”,向正在收听的一方发出插入请求
\item 短暂发射一次“Break in!”或“插入!”,如对方无反应,应加大功率反复作此发射
\item 只要双方都是自己熟悉的业余电台操作员,可直接插入谈话,不必拘泥礼节
\end{enumerate}

\bigskip


\noindent\textbf{问题:}以请求插入的方式加入两个电台正在通信中的谈话,应满足的起码条件是:
\begin{enumerate}[label=\Alph*), leftmargin=3em]
\item 确认自己的加入不会影响原通信双方的乐趣
\item 自己的信号质量不亚于原通信双方
\item 自己的操作技巧不亚于原通信双方
\item 自己拥有比原通信双方更有吸引力的谈话内容
\end{enumerate}

\bigskip


\noindent\textbf{问题:}参加DX网络通信有助于与一些稀有电台建立通信。正确做法是:
\begin{enumerate}[label=\Alph*), leftmargin=3em]
\item 事前了解网络规则,未经主控台允许不能随意发起呼叫,根据主控台要求进行登录,然后需随时注意主控台的安排,在主控台安排DX电台呼叫自己时及时回答联络
\item 听到DX网络通信后,应抓住机会立即对听到的电台发起呼叫
\item 当两个电台在网络主控台安排下互联联络时,自己可以通过Break in插入通信
\item DX网络时间内肯定有很多DX电台在守听,利用该频点呼叫CQ定有收获
\end{enumerate}

\bigskip


\noindent\textbf{问题:}14022KHz有很多电台争相报出自己的呼号,原来是想呼叫发射频率为14020KHz的某稀有台。如要加入对该稀有台的呼叫,应该:
\begin{enumerate}[label=\Alph*), leftmargin=3em]
\item 守听14020KHz,在稀有台结束和其他电台联络或者呼叫CQ和QRZ时,在14022KHz快速准确地发送自己的呼号
\item 守听14020KHz,在稀有台结束和其他电台联络或者呼叫CQ和QRZ时,在14020KHz快速准确地发送自己的呼号
\item 在14022KHz不断发送自己的呼号
\item 在14020KHz呼叫该稀有台
\end{enumerate}

\bigskip


\noindent\textbf{问题:}业余电台之间进行通信,必须相互正确发送和接收的信息为:
\begin{enumerate}[label=\Alph*), leftmargin=3em]
\item 本台呼号、对方呼号、信号报告
\item 本台呼号、对方呼号、QTH
\item 本台呼号、信号报告、QTH
\item 对方呼号、信号报告、设备情况
\end{enumerate}

\bigskip


\noindent\textbf{问题:}法规要求业余电台在通信建立及结束时主动发送本台呼号。允许用发送呼号的一部分来代替发送完整呼号的情况是:
\begin{enumerate}[label=\Alph*), leftmargin=3em]
\item 在任何情况下都必须用完整呼号作为电台标识
\item 在熟悉的友台之间呼叫可以仅使用呼号后缀作为电台标识
\item 在VHF/UHF频段进行本地呼叫时可以仅使用呼号后缀作为电台标识
\item 在HF频段进行国内呼叫时可以仅使用呼号后缀作为电台标识
\end{enumerate}

\bigskip


\noindent\textbf{问题:}如何回答一个CQ呼叫?
\begin{enumerate}[label=\Alph*), leftmargin=3em]
\item 先报出对方的呼号,再报出自己的呼号
\item 先报出自己的呼号,再报出对方的呼号
\item 说:“CQ”,并报出对方的呼号
\item 先给出信号报告,再报出自己的呼号
\end{enumerate}

\bigskip


\noindent\textbf{问题:}当一部电台在呼叫CQ时,他的意思是?
\begin{enumerate}[label=\Alph*), leftmargin=3em]
\item 非特指地呼叫任何一部电台
\item 此电台正在测试天线,不需要任何电台回答这个呼叫
\item 只有被呼叫的电台可以回答,其他人不能回答
\item 呼叫重庆的电台
\end{enumerate}

\bigskip


\noindent\textbf{问题:}如果其他电台报告你在2米波段的信号刚才非常强,但是突然变弱或不可辨,这时你应当怎么做?
\begin{enumerate}[label=\Alph*), leftmargin=3em]
\item 稍稍移动一下自己的位置,有时信号无规律反射造成的多径效应可能导致失真
\item 打开哑音发射功能
\item 请对方电台调整自己的静噪设置
\item 将你电台中的镍氢电池换成锂电池
\end{enumerate}

\bigskip


\noindent\textbf{问题:}下列哪种方式可以让你快速切换到一个你经常使用的频率?
\begin{enumerate}[label=\Alph*), leftmargin=3em]
\item 将这个频率作为一个频道存储在电台中
\item 打开哑音输出
\item 关闭哑音输出
\item 使用快速扫描模式来切换到那个频率
\end{enumerate}

\bigskip


\noindent\textbf{问题:}“你和我还有事吗”的业余无线电通信Q简语为:
\begin{enumerate}[label=\Alph*), leftmargin=3em]
\item QRL?
\item QRU
\item QRU?
\item QRB?
\end{enumerate}
\noindent\textbf{解说:}“你和我还有事吗”的业余无线电通信Q简语为\textbf{QRU?}。其余选项为:QRU为“我和你无事了。”,QRL?为“你忙吗?”。\\\noindent\textbf{答案:}C


\bigskip


\noindent\textbf{问题:}“我和你无事了”的业余无线电通信Q简语为:
\begin{enumerate}[label=\Alph*), leftmargin=3em]
\item QRU?
\item QRS
\item QRL
\item QRU
\end{enumerate}
\noindent\textbf{解说:}“我和你无事了”的业余无线电通信Q简语为\textbf{QRU}。其余选项为:QRU?为“你和我还有事吗”,QRS为“请发得慢一些。”,QRL为“我很忙,请不要干扰。”。\\\noindent\textbf{答案:}D



\bigskip


\noindent\textbf{问题:}“谁在呼叫我”的业余无线电通信Q简语为:
\begin{enumerate}[label=\Alph*), leftmargin=3em]
\item QRZ?
\item QRZ
\item QSL?
\item QRA?
\end{enumerate}
\noindent\textbf{解说:}“谁在呼叫我”的业余无线电通信Q简语为\textbf{QRZ?}。其余各项为:“QRZ”为“……正在(用……kHz或……MHz)呼叫你。”,“QRA?”为“你台的名称是什么?”,“QSL?”为“你能承认收妥吗?”。\\\noindent\textbf{答案:}A

\bigskip


\noindent\textbf{问题:}“要我增加功率吗”的业余无线电通信Q简语为:
\begin{enumerate}[label=\Alph*), leftmargin=3em]
\item QRO?
\item QRS?
\item QRO
\item QSO?
\end{enumerate}
\noindent\textbf{解说:}“要我增加功率吗”的业余无线电通信Q简语为\textbf{QRO?}。其余选项为:QRO为“请增加发信机功率。”,QRS?为“要我发得慢一些吗?”,QSO?为“你是否能和……直接通信?”。\\\noindent\textbf{答案:}A


\bigskip


\noindent\textbf{问题:}“要我减小功率吗”的业余无线电通信Q简语为:
\begin{enumerate}[label=\Alph*), leftmargin=3em]
\item QRP?
\item QSP?
\item QRS?
\item QRP
\end{enumerate}
\noindent\textbf{解说:}“要我减小功率吗”的业余无线电通信Q简语为\textbf{QRP?}。其余选项为:QRP为“请减低发信机功率。”,QRS?为“要我发得慢一些吗?”,QSP?为“你可否免费转发到……?”。\\\noindent\textbf{答案:}A


\bigskip


\noindent\textbf{问题:}“我能直接和×××电台通信”的业余无线电通信Q简语为:
\begin{enumerate}[label=\Alph*), leftmargin=3em]
\item QSO ×××
\item QRV ×××
\item QSP ×××
\item QRU ×××
\end{enumerate}

\bigskip


\noindent\textbf{问题:}“你能直接和×××电台通信吗”的业余无线电通信Q简语为:
\begin{enumerate}[label=\Alph*), leftmargin=3em]
\item QSO ××× ?
\item QRV ××× ?
\item QRL ××× ?
\item QRT ××× ?
\end{enumerate}

\bigskip


\noindent\textbf{问题:}“我遇到他台干扰”的业余无线电通信Q简语为:
\begin{enumerate}[label=\Alph*), leftmargin=3em]
\item QRM
\item QSM
\item QRN
\item QSB?
\end{enumerate}

\bigskip


\noindent\textbf{问题:}“你遇到他台干扰吗”的业余无线电通信Q简语为:
\begin{enumerate}[label=\Alph*), leftmargin=3em]
\item QRM?
\item QSM?
\item QSN?
\item QSD?
\end{enumerate}

\bigskip


\noindent\textbf{问题:}“你遇到天电干扰吗”的业余无线电通信Q简语为:
\begin{enumerate}[label=\Alph*), leftmargin=3em]
\item QRN?
\item QSM?
\item QSN?
\item QRV?
\end{enumerate}

\bigskip


\noindent\textbf{问题:}“要我加快发送速度吗”的业余无线电通信Q简语为:
\begin{enumerate}[label=\Alph*), leftmargin=3em]
\item QRQ?
\item QSQ?
\item QRS?
\item QRT?
\end{enumerate}

\bigskip


\noindent\textbf{问题:}“请加快发送速度”的业余无线电通信Q简语为:
\begin{enumerate}[label=\Alph*), leftmargin=3em]
\item QRQ
\item QSM
\item QSV
\item QSQ
\end{enumerate}

\bigskip


\noindent\textbf{问题:}“要我减慢发送速度吗”的业余无线电通信Q简语为:
\begin{enumerate}[label=\Alph*), leftmargin=3em]
\item QRS?
\item QSQ?
\item QRQ?
\item QRT?
\end{enumerate}

\bigskip


\noindent\textbf{问题:}“请减慢发送速度”的业余无线电通信Q简语为:
\begin{enumerate}[label=\Alph*), leftmargin=3em]
\item QRS
\item QSM
\item QSV
\item QRO
\end{enumerate}

\bigskip


\noindent\textbf{问题:}“你是否已准备好”的业余无线电通信Q简语为:
\begin{enumerate}[label=\Alph*), leftmargin=3em]
\item QRV?
\item QSV?
\item QRL?
\item QRU?
\end{enumerate}

\bigskip


\noindent\textbf{问题:}“我已准备好”的业余无线电通信Q简语为:
\begin{enumerate}[label=\Alph*), leftmargin=3em]
\item QRV
\item QSV
\item QRL
\item QRU
\end{enumerate}

\bigskip


\noindent\textbf{问题:}“要我停止发送吗”的业余无线电通信Q简语为:
\begin{enumerate}[label=\Alph*), leftmargin=3em]
\item QRT?
\item QST?
\item QRT
\item QSX
\end{enumerate}

\bigskip


\noindent\textbf{问题:}“请停止发送”的业余无线电通信Q简语为:
\begin{enumerate}[label=\Alph*), leftmargin=3em]
\item QRT
\item QST
\item QRV
\item QSB
\end{enumerate}

\bigskip


\noindent\textbf{问题:}“我的信号有衰落吗”的业余无线电通信Q简语为:
\begin{enumerate}[label=\Alph*), leftmargin=3em]
\item QSB?
\item QSD?
\item QRB?
\item QSP?
\end{enumerate}

\bigskip


\noindent\textbf{问题:}“你的信号有衰落”的业余无线电通信Q简语为:
\begin{enumerate}[label=\Alph*), leftmargin=3em]
\item QSB
\item QSX
\item QRE
\item QSP
\end{enumerate}

\bigskip


\noindent\textbf{问题:}“我给你收据(QSL卡片)、我已收妥”的业余无线电通信Q简语为:
\begin{enumerate}[label=\Alph*), leftmargin=3em]
\item QSL
\item QRG
\item QSX
\item QRV
\end{enumerate}

\bigskip


\noindent\textbf{问题:}“我的电台位置是××××”的业余无线电通信Q简语为:
\begin{enumerate}[label=\Alph*), leftmargin=3em]
\item QTH ××××
\item QRD ××××
\item QSL ××××
\item QSP ××××
\end{enumerate}

\bigskip


\noindent\textbf{问题:}“我遇到天电干扰”的业余无线电通信Q简语为:
\begin{enumerate}[label=\Alph*), leftmargin=3em]
\item QRN
\item QST
\item QSN
\item QRM
\end{enumerate}

\bigskip


\noindent\textbf{问题:}“我发报的手法有毛病吗”的业余无线电通信Q简语为:
\begin{enumerate}[label=\Alph*), leftmargin=3em]
\item QSD?
\item QSB?
\item QRT?
\item QSV?
\end{enumerate}

\bigskip


\noindent\textbf{问题:}“你发报的手法有毛病”的业余无线电通信Q简语为:
\begin{enumerate}[label=\Alph*), leftmargin=3em]
\item QSD
\item QRC
\item QSU
\item QSK
\end{enumerate}

\bigskip


\noindent\textbf{问题:}“你正忙着吗”的业余无线电通信Q简语为:
\begin{enumerate}[label=\Alph*), leftmargin=3em]
\item QRL?
\item QRX?
\item QSU?
\item QRU?
\end{enumerate}

\bigskip


\noindent\textbf{问题:}“我正忙着”的业余无线电通信Q简语为:
\begin{enumerate}[label=\Alph*), leftmargin=3em]
\item QRL
\item QRX
\item QSV
\item QRB
\end{enumerate}

\bigskip


\noindent\textbf{问题:}“能在你的信号间隙中接收吗(即QSK插入方式)”的业余无线电通信Q简语为:
\begin{enumerate}[label=\Alph*), leftmargin=3em]
\item QSK?
\item QRF?
\item QSG?
\item QRJ?
\end{enumerate}

\bigskip


\noindent\textbf{问题:}“我在发射的信号间隙中接收(即QSK插入方式)”的业余无线电通信Q简语为:
\begin{enumerate}[label=\Alph*), leftmargin=3em]
\item QSK
\item QRK
\item QSM
\item QRE
\end{enumerate}

\bigskip


\noindent\textbf{问题:}“你能给我收据(或QSL卡片)吗”的业余无线电通信Q简语为:
\begin{enumerate}[label=\Alph*), leftmargin=3em]
\item QSL?
\item QRL?
\item QSA?
\item QSD?
\end{enumerate}

\bigskip


\noindent\textbf{问题:}“你能传信到×××电台吗”的业余无线电通信Q简语为:
\begin{enumerate}[label=\Alph*), leftmargin=3em]
\item QSP ×××?
\item QRD ×××?
\item QSX ×××?
\item QRV ×××?
\end{enumerate}

\bigskip


\noindent\textbf{问题:}“我能传信到×××电台”的业余无线电通信Q简语为:
\begin{enumerate}[label=\Alph*), leftmargin=3em]
\item QSP ×××
\item QSH ×××
\item QRL ×××
\item QRP ×××
\end{enumerate}

\bigskip


\noindent\textbf{问题:}“你将在nnnn KHz(或MHz)频率守听×××电台吗”的业余无线电通信Q简语为:
\begin{enumerate}[label=\Alph*), leftmargin=3em]
\item QSX ××× ON nnnn KHz(或MHz)?
\item QRU ××× ON nnnn KHz(或MHz)?
\item QSL ××× ON nnnn KHz(或MHz)?
\item QRV ××× ON nnnn KHz(或MHz)?
\end{enumerate}

\bigskip


\noindent\textbf{问题:}“我将在nnnn KHz(或MHz)频率守听×××电台”的业余无线电通信Q简语为:
\begin{enumerate}[label=\Alph*), leftmargin=3em]
\item QSX ××× ON nnnn KHz(或MHz)
\item QRX ××× ON nnnn KHz(或MHz)
\item QRZ ××× ON nnnn KHz(或MHz)
\item QSP ××× ON nnnn KHz(或MHz)
\end{enumerate}

\bigskip


\noindent\textbf{问题:}“要我将频率改到nnnn频率吗”的业余无线电通信Q简语为:
\begin{enumerate}[label=\Alph*), leftmargin=3em]
\item QSY nnnn KHz(或MHz)?
\item QRY nnnn KHz(或MHz)?
\item QSV nnnn KHz(或MHz)?
\item QRV nnnn KHz(或MHz)?
\end{enumerate}

\bigskip


\noindent\textbf{问题:}“请将频率改到nnnn频率”的业余无线电通信Q简语为:
\begin{enumerate}[label=\Alph*), leftmargin=3em]
\item QSY nnnn KHz(或MHz)
\item QRY nnnn KHz(或MHz)
\item QSU nnnn KHz(或MHz)
\item QRO nnnn KHz(或MHz)
\end{enumerate}

\bigskip


\noindent\textbf{问题:}“你的电台位置在哪里”的业余无线电通信Q简语为:
\begin{enumerate}[label=\Alph*), leftmargin=3em]
\item QTH?
\item QRA?
\item QSA?
\item QSZ?
\end{enumerate}

\bigskip


\noindent\textbf{问题:}“我的信号强度如何”的业余无线电通信Q简语为:
\begin{enumerate}[label=\Alph*), leftmargin=3em]
\item QSA?
\item QSB?
\item QSD?
\item QTU?
\end{enumerate}

\bigskip


\noindent\textbf{问题:}“你的信号强度为×级(1-5级)”的业余无线电通信Q简语为:
\begin{enumerate}[label=\Alph*), leftmargin=3em]
\item QSA ×
\item QSL ×
\item QRM ×
\item QRY
\end{enumerate}

\bigskip


\noindent\textbf{问题:}业余无线电通信常用缩语“ABT”的意思是:
\begin{enumerate}[label=\Alph*), leftmargin=3em]
\item 在…之上
\item 关于、大约
\item 电池
\item 衰减
\end{enumerate}
\noindent\textbf{解说:}业余无线电通信常用缩语“ABT”的意思是\textbf{关于、大约}。\\\noindent\textbf{答案:}B


\bigskip


\noindent\textbf{问题:}“地址”的业余无线电通信常用缩语是:
\begin{enumerate}[label=\Alph*), leftmargin=3em]
\item ABT
\item ABV
\item ATT
\item ADR或ADDR
\end{enumerate}
\noindent\textbf{解说:}“地址”的业余无线电通信常用缩语是\textbf{ADR或ADDR}。\\\noindent\textbf{答案:}D




\bigskip


\noindent\textbf{问题:}业余无线电常用缩语“ATT”的意思是:
\begin{enumerate}[label=\Alph*), leftmargin=3em]
\item 衰落
\item 地址
\item 衰减
\item 关于、大约
\end{enumerate}
\noindent\textbf{解说:}业余无线电常用缩语“ATT”的意思是\textbf{衰减}。\\\noindent\textbf{答案:}C



\bigskip


\noindent\textbf{问题:}业余无线电常用缩语“PWR”的意思是:
\begin{enumerate}[label=\Alph*), leftmargin=3em]
\item 功率
\item 报告
\item 读写
\item 中继台
\end{enumerate}
\noindent\textbf{解说:}业余无线电常用缩语“PWR”的意思是\textbf{功率}。\\\noindent\textbf{答案:}A



\bigskip


\noindent\textbf{问题:}“再”、“再来一次”的业余无线电通信常用缩语是:
\begin{enumerate}[label=\Alph*), leftmargin=3em]
\item AGN
\item GA
\item ABT
\item ABV
\end{enumerate}
\noindent\textbf{解说:}“再”、“再来一次”的业余无线电通信常用缩语是\textbf{AGN}。\\\noindent\textbf{答案:}A



\bigskip


\noindent\textbf{问题:}业余无线电通信常用缩语“GA”的意思是:
\begin{enumerate}[label=\Alph*), leftmargin=3em]
\item 公安
\item 继续、请过来
\item 垂直地网天线
\item 姑娘
\end{enumerate}
\noindent\textbf{解说:}业余无线电通信常用缩语“GA”的意思是\textbf{垂直地网天线}。\\\noindent\textbf{答案:}C



\bigskip


\noindent\textbf{问题:}业余无线电通信常用缩语“AHR”的意思是:
\begin{enumerate}[label=\Alph*), leftmargin=3em]
\item 另一个
\item 地址
\item 天线
\item 这里
\end{enumerate}
\noindent\textbf{解说:}业余无线电通信常用缩语“AHR”的意思是\textbf{另一个}。\\\noindent\textbf{答案:}A

\bigskip


\noindent\textbf{问题:}“天线”的业余无线电通信常用缩语是:
\begin{enumerate}[label=\Alph*), leftmargin=3em]
\item ATR
\item ANT
\item ATT
\item ATN
\end{enumerate}
\noindent\textbf{解说:}“天线”的业余无线电通信常用缩语是\textbf{ANT}。\\\noindent\textbf{答案:}B



\bigskip


\noindent\textbf{问题:}业余无线电常用缩语“ARDF”的意思是:
\begin{enumerate}[label=\Alph*), leftmargin=3em]
\item 天线测试仪
\item 业余无线电测向
\item 地址
\item 天线调谐器、天调
\end{enumerate}
\noindent\textbf{解说:}业余无线电常用缩语“ARDF”的意思是\textbf{业余无线电测向}。\\\noindent\textbf{答案:}B



\bigskip


\noindent\textbf{问题:}“收听”的业余无线电常用缩语是:
\begin{enumerate}[label=\Alph*), leftmargin=3em]
\item GA
\item HR
\item KP
\item RCV
\end{enumerate}
\noindent\textbf{解说:}“收听”的业余无线电常用缩语是\textbf{KP}。其余选项为:GA为“发过来”,HR为“这里、小时”。\\\noindent\textbf{答案:}C
%%%???KP保持

\bigskip


\noindent\textbf{问题:}业余无线电常用缩语“HST”的意思是:
\begin{enumerate}[label=\Alph*), leftmargin=3em]
\item 快速收发报
\item 信号报告
\item 通播
\item 这里、听到
\end{enumerate}
\noindent\textbf{解说:}业余无线电常用缩语“HST”的意思是\textbf{快速收发报}。\\\noindent\textbf{答案:}A



\bigskip


\noindent\textbf{问题:}业余无线电CW通信常用缩语“AS”(经常连发在一起)的意思是:
\begin{enumerate}[label=\Alph*), leftmargin=3em]
\item 请稍等
\item 天线
\item 全部
\item 关于
\end{enumerate}
\noindent\textbf{解说:}业余无线电CW通信常用缩语“AS”(经常连发在一起)的意思是\textbf{请稍等}。\\\noindent\textbf{答案:}A

\bigskip


\noindent\textbf{问题:}业余无线电通信常用缩语“AS”的意思有:
\begin{enumerate}[label=\Alph*), leftmargin=3em]
\item 关于
\item 回答
\item 请稍等、亚洲、如同
\item 天线开关
\end{enumerate}
\noindent\textbf{解说:}业余无线电通信常用缩语“AS”的意思有\textbf{请稍等、亚洲、如同}。其余选项为:关于的缩略语为ABT,回答的缩略语为ANS,天线的缩略语为ANT。\\\noindent\textbf{答案:}C

\bigskip


\noindent\textbf{问题:}业余无线电通信常用词语“BEST”的意思是:
\begin{enumerate}[label=\Alph*), leftmargin=3em]
\item 最好的
\item 通播电报
\item 信号报告
\item 电池组
\end{enumerate}
\noindent\textbf{解说:}业余无线电通信常用词语“BEST”的意思是\textbf{最好的}。\\\noindent\textbf{答案:}A

\bigskip


\noindent\textbf{问题:}业余无线电通信常用缩语“BJT”的意思是:
\begin{enumerate}[label=\Alph*), leftmargin=3em]
\item 北京时间
\item 结型场效应半导体管
\item 双基极二极管
\item 双极型半导体管
\end{enumerate}
\noindent\textbf{解说:}业余无线电通信常用缩语“BJT”的意思是\textbf{北京时间}。\\\noindent\textbf{答案:}A


\bigskip


\noindent\textbf{问题:}业余无线电通信常用缩语“BK”的意思是:
\begin{enumerate}[label=\Alph*), leftmargin=3em]
\item 插入、打断
\item 结束工作
\item 千字节(单位)
\item 请马上回答
\end{enumerate}
\noindent\textbf{解说:}业余无线电通信常用缩语“BK”的意思是\textbf{插入、打断}。\\\noindent\textbf{答案:}A

\bigskip


\noindent\textbf{问题:}“QSL卡片管理局”的业余无线电通信常用缩语是:
\begin{enumerate}[label=\Alph*), leftmargin=3em]
\item BROU
\item BOUR
\item BURO
\item BRUO
\end{enumerate}
\noindent\textbf{解说:}“QSL卡片管理局”的业余无线电通信常用缩语是\textbf{BURO}。\\\noindent\textbf{答案:}C%布若

\bigskip


\noindent\textbf{问题:}“遇到”、“见面”的业余无线电通信常用缩语是:
\begin{enumerate}[label=\Alph*), leftmargin=3em]
\item MTRS
\item C
\item MET
\item SE
\end{enumerate}
\noindent\textbf{解说:}“遇到”、“见面”的业余无线电通信常用缩语是\textbf{C}。\\\noindent\textbf{答案:}B
%%%???


\bigskip


\noindent\textbf{问题:}业余无线电通信常用缩语“CFM”的意思是:
\begin{enumerate}[label=\Alph*), leftmargin=3em]
\item 法拉
\item 确认
\item 调频
\item 呼叫
\end{enumerate}
\noindent\textbf{解说:}业余无线电通信常用缩语“CFM”的意思是\textbf{确认}。\\\noindent\textbf{答案:}B

\bigskip


\noindent\textbf{问题:}业余无线电通信常用词语“CHEERIO”的意思是:
\begin{enumerate}[label=\Alph*), leftmargin=3em]
\item 圣诞节
\item 英文字符
\item 再次见面
\item 再会、祝贺
\end{enumerate}
\noindent\textbf{解说:}业余无线电通信常用词语“CHEERIO”的意思是\textbf{再会、祝贺}。\\\noindent\textbf{答案:}D



\bigskip


\noindent\textbf{问题:}业余无线电通信常用缩语“CL”、“CLS”、“CLG”的意思分别是:
\begin{enumerate}[label=\Alph*), leftmargin=3em]
\item 关闭(或呼叫)、呼号、呼叫
\item 计数、云层、呼号
\item 确认、呼叫、清除
\item 呼号、清除、关闭
\end{enumerate}
\noindent\textbf{解说:}业余无线电通信常用缩语“CL”的意思是\textbf{关闭(或呼叫)},“CLS”的意思是\textbf{呼号},“CLG”的意思是\textbf{呼叫}。\\\noindent\textbf{答案:}A

\bigskip


\noindent\textbf{问题:}业余无线电通信常用词语“DATE”的意思是:
\begin{enumerate}[label=\Alph*), leftmargin=3em]
\item 时间
\item 日期
\item 地址
\item 频率
\end{enumerate}
\noindent\textbf{解说:}业余无线电通信常用词语“DATE”的意思是\textbf{日期}。其余选项为:时间为TIME,地址为ADR、ADS,频率为FQ、FREQ\\\noindent\textbf{答案:}B

\bigskip


\noindent\textbf{问题:}业余无线电通信常用缩语“DR”的意思是:
\begin{enumerate}[label=\Alph*), leftmargin=3em]
\item 亲爱的
\item 从
\item 远距离
\item 二极管
\end{enumerate}
\noindent\textbf{解说:}业余无线电通信常用缩语“DR”的意思是\textbf{亲爱的}。\\\noindent\textbf{答案:}A

\bigskip


\noindent\textbf{问题:}单元(常用于天线振子)的业余无线电通信常用缩语是:
\begin{enumerate}[label=\Alph*), leftmargin=3em]
\item UNIT
\item YAGI
\item ANT
\item EL、ELE、ELS
\end{enumerate}
\noindent\textbf{解说:}单元(常用于天线振子)的业余无线电通信常用缩语是\textbf{EL、ELE、ELS}。其余选项为:ANT为天线,YAGI为八木天线。\\\noindent\textbf{答案:}D

\bigskip


\noindent\textbf{问题:}业余无线电CW通信常用缩语“ES”的意思是:
\begin{enumerate}[label=\Alph*), leftmargin=3em]
\item 从
\item 请等待
\item 是
\item 和
\end{enumerate}
\noindent\textbf{解说:}业余无线电CW通信常用缩语“ES”的意思是\textbf{和}。\\\noindent\textbf{答案:}D
%%%???

\bigskip


\noindent\textbf{问题:}业余无线电通信常用缩语“FB”的意思是:
\begin{enumerate}[label=\Alph*), leftmargin=3em]
\item 美好的祝愿
\item 再见
\item 很好的
\item 腐败
\end{enumerate}
\noindent\textbf{解说:}业余无线电通信常用缩语“FB”的意思是\textbf{很好的}。其余选项为:再见为GB,美好的祝愿为73。\\\noindent\textbf{答案:}C

\bigskip


\noindent\textbf{问题:}“频率”的业余无线电通信常用缩语是:
\begin{enumerate}[label=\Alph*), leftmargin=3em]
\item TUNE
\item FREQ
\item FIND
\item FER
\end{enumerate}
\noindent\textbf{解说:}“频率”的业余无线电通信常用缩语是\textbf{FREQ}。FER是“为了,对于”。\\\noindent\textbf{答案:}B
%%%???

\bigskip


\noindent\textbf{问题:}业余无线电通信常用缩语“GND”的意思是:
\begin{enumerate}[label=\Alph*), leftmargin=3em]
\item 格林威治时间
\item 地线,地面
\item 高兴
\item 好运气
\end{enumerate}
\noindent\textbf{解说:}业余无线电通信常用缩语“GND”的意思是\textbf{地线,地面}”。其余选项为:格林威治时间为GMT,好运气为GL,高兴为GLD。\\\noindent\textbf{答案:}B

\bigskip


\noindent\textbf{问题:}业余无线电通信常用缩语“OM”的意思是:
\begin{enumerate}[label=\Alph*), leftmargin=3em]
\item 老人
\item 欧姆
\item 老朋友
\item 或者
\end{enumerate}
\noindent\textbf{解说:}业余无线电通信常用缩语“OM”的意思是\textbf{老朋友}。\\\noindent\textbf{答案:}C

\bigskip


\noindent\textbf{问题:}“电台设备”的业余无线电通信常用缩语是:
\begin{enumerate}[label=\Alph*), leftmargin=3em]
\item REG
\item RIG
\item SB
\item EQP
\end{enumerate}
\noindent\textbf{解说:}“电台设备”的业余无线电通信常用缩语是\textbf{RIG}。\\\noindent\textbf{答案:}B

\bigskip


\noindent\textbf{问题:}业余无线电通信常用词语“FINE”的意思是:
\begin{enumerate}[label=\Alph*), leftmargin=3em]
\item 好的,精细的
\item 调谐
\item 发现
\item 确认
\end{enumerate}

\bigskip


\noindent\textbf{问题:}业余无线电通信常用缩语“FR”、“FER”的意思是:
\begin{enumerate}[label=\Alph*), leftmargin=3em]
\item 频率
\item 希望
\item 好的,精细的
\item 为了,对于
\end{enumerate}
\noindent\textbf{解说:}业余无线电通信常用缩语“FR”、“FER”的意思是\textbf{为了,对于}。\\\noindent\textbf{答案:}D


\bigskip


\noindent\textbf{问题:}“下午好”的业余无线电通信常用缩语是:
\begin{enumerate}[label=\Alph*), leftmargin=3em]
\item GN
\item GA
\item GE
\item GM
\end{enumerate}
\noindent\textbf{解说:}“下午好”的业余无线电通信常用缩语是\textbf{GA}。其余选项为:GM为早晨好,GE为晚上好,GN为晚安。\\\noindent\textbf{答案:}B

\bigskip


\noindent\textbf{问题:}“早晨好”的业余无线电通信常用缩语是:
\begin{enumerate}[label=\Alph*), leftmargin=3em]
\item GM
\item GL
\item GA
\item GB
\end{enumerate}
\noindent\textbf{解说:}“早晨好”的业余无线电通信常用缩语是\textbf{GM}。其余选项为:GA为下午好,GB为再见,GL为好运气。\\\noindent\textbf{答案:}A


\bigskip


\noindent\textbf{问题:}“晚上好”的业余无线电通信常用缩语是:
\begin{enumerate}[label=\Alph*), leftmargin=3em]
\item GN
\item GA
\item GE
\item GM
\end{enumerate}
\noindent\textbf{解说:}“晚上好”的业余无线电通信常用缩语是\textbf{GE}。其余选项为:GM为早晨好,GA为下午好,GN为晚安。\\\noindent\textbf{答案:}C



\bigskip


\noindent\textbf{问题:}业余无线电通信常用缩语“GN”的意思是:
\begin{enumerate}[label=\Alph*), leftmargin=3em]
\item 早晨好
\item 晚安
\item 好运气
\item 高兴
\end{enumerate}
\noindent\textbf{解说:}业余无线电通信常用缩语“GN”的意思是\textbf{晚安}。其余选项为:早晨好是GM,好运气是GL,高兴是GLD。\\\noindent\textbf{答案:}B

\bigskip


\noindent\textbf{问题:}“再见”的业余无线电通信常用缩语是:
\begin{enumerate}[label=\Alph*), leftmargin=3em]
\item GE
\item GB
\item GL
\item GA
\end{enumerate}
\noindent\textbf{解说:}“再见”的业余无线电通信常用缩语是\textbf{GB}。其余选项为:GE为晚上好,GL为好运气,GA为下午好。\\\noindent\textbf{答案:}B



\bigskip


\noindent\textbf{问题:}业余无线电通信常用缩语“GL”的意思是:
\begin{enumerate}[label=\Alph*), leftmargin=3em]
\item 晚安
\item 早安
\item 好运气
\item 再见
\end{enumerate}
\noindent\textbf{解说:}业余无线电通信常用缩语“GL”的意思是\textbf{好运气}。其余选项为:早安为GM,再见为GB,晚安为GN。\\\noindent\textbf{答案:}C



\bigskip


\noindent\textbf{问题:}业余无线电通信常用缩语“GLD”的意思是:
\begin{enumerate}[label=\Alph*), leftmargin=3em]
\item 高兴
\item 地线,地面
\item 再见
\item 好运气
\end{enumerate}
\noindent\textbf{解说:}业余无线电通信常用缩语“GLD”的意思是\textbf{高兴}。其余选项为:好运气为GL,再见为GB,地线、地面为GND。\\\noindent\textbf{答案:}A



\bigskip


\noindent\textbf{问题:}业余无线电通信常用缩语“GMT”的意思是:
\begin{enumerate}[label=\Alph*), leftmargin=3em]
\item 格林威治时间
\item 地线,地面
\item 好运气
\item 高兴
\end{enumerate}
\noindent\textbf{解说:}业余无线电通信常用缩语“GMT”的意思是\textbf{格林威治时间}。地线、地面为GND,好运气为GL,高兴为GLD。\\\noindent\textbf{答案:}A



\bigskip


\noindent\textbf{问题:}“抄收”的业余无线电通信常用缩语是:
\begin{enumerate}[label=\Alph*), leftmargin=3em]
\item CFM
\item CPI
\item HPI
\item HPY
\end{enumerate}
\noindent\textbf{解说:}“抄收”的业余无线电通信常用缩语是\textbf{CPI}。其余选项为:HPI为愉快,CFM为确认、认为。\\\noindent\textbf{答案:}B



\bigskip


\noindent\textbf{问题:}“希望”的业余无线电通信常用缩语是:
\begin{enumerate}[label=\Alph*), leftmargin=3em]
\item CPI
\item HPI
\item HPE
\item HPY
\end{enumerate}
\noindent\textbf{解说:}“希望”的业余无线电通信常用缩语是\textbf{HPE}。HPI为愉快。\\\noindent\textbf{答案:}C



\bigskip


\noindent\textbf{问题:}业余无线电通信常用缩语“HPY”、“HPI”的意思是:
\begin{enumerate}[label=\Alph*), leftmargin=3em]
\item 这里
\item 抄收
\item 幸福
\item 希望
\end{enumerate}
\noindent\textbf{解说:}业余无线电通信常用缩语“HPY”、“HPI”的意思是\textbf{幸福}。其余选项为:这里、听到为HR,抄收为CPI,希望为HPE。\\\noindent\textbf{答案:}C



\bigskip


\noindent\textbf{问题:}业余无线电通信常用缩语“HR”的意思是:
\begin{enumerate}[label=\Alph*), leftmargin=3em]
\item 这里、听到
\item 幸福
\item 号码
\item 希望
\end{enumerate}
\noindent\textbf{解说:}业余无线电通信常用缩语“HR”的意思是\textbf{这里、听到}。其余选项为:希望为HPE,号码为NR,幸福为HPY、HPI。\\\noindent\textbf{答案:}A

%%%???
%号码为NR

\bigskip


\noindent\textbf{问题:}“怎样”、“如何”的业余无线电通信常用缩语是:
\begin{enumerate}[label=\Alph*), leftmargin=3em]
\item CW
\item HW
\item HPI
\item HR
\end{enumerate}
\noindent\textbf{解说:}“怎样”、“如何”的业余无线电通信常用缩语是\textbf{HW}。其余选项为:CW为等幅电报,HPI为幸福,HR为这里、听到。\\\noindent\textbf{答案:}B



\bigskip


\noindent\textbf{问题:}“很多”的业余无线电通信常用缩语是:
\begin{enumerate}[label=\Alph*), leftmargin=3em]
\item NAME
\item MNY、MNI
\item VY
\item ALL
\end{enumerate}
\noindent\textbf{解说:}“很多”的业余无线电通信常用缩语是\textbf{MNY、MNI}。其余选项的常用缩语为:AL为全部,VY为很。NAME为名字的英语。\\\noindent\textbf{答案:}B



\bigskip


\noindent\textbf{问题:}业余无线电通信常用缩语“MTRS”的意思是:
\begin{enumerate}[label=\Alph*), leftmargin=3em]
\item 太太
\item 小姐
\item 米
\item 先生
\end{enumerate}
\noindent\textbf{解说:}业余无线电通信常用缩语“MTRS”的意思是\textbf{米}。\\\noindent\textbf{答案:}C



\bigskip


\noindent\textbf{问题:}“方式”的业余无线电通信常用英语是:
\begin{enumerate}[label=\Alph*), leftmargin=3em]
\item NAME
\item NICE
\item MTRS
\item MODE
\end{enumerate}
\noindent\textbf{解说:}“方式”的业余无线电通信常用英语是\textbf{MODE}。其余选项的常用英语为:NAME为名称,NICE为良好的。MTRS为米的常用缩语。\\\noindent\textbf{答案:}D



\bigskip


\noindent\textbf{问题:}“名字”的业余无线电通信常用英语是:
\begin{enumerate}[label=\Alph*), leftmargin=3em]
\item NICE
\item MNI
\item NAME
\item MODE
\end{enumerate}
\noindent\textbf{解说:}“名字”的业余无线电通信常用英语是\textbf{NAME}。其余选项的常用英语为:MODE为方式,NICE为良好的。MNI为许多的常用缩语。\\\noindent\textbf{答案:}C



\bigskip


\noindent\textbf{问题:}业余无线电通信常用词语“NICE”的意思是:
\begin{enumerate}[label=\Alph*), leftmargin=3em]
\item 良好的
\item 名字
\item 鼠标
\item 方式
\end{enumerate}
\noindent\textbf{解说:}业余无线电通信常用词语“NICE”的意思是\textbf{良好的}。其余选项为:方式为MODE,名字为NAME,鼠标为MOUSE。\\\noindent\textbf{答案:}A



\bigskip


\noindent\textbf{问题:}业余无线电通信常用缩语“NW”的意思是:
\begin{enumerate}[label=\Alph*), leftmargin=3em]
\item 新的
\item 怎样
\item 现在
\item 不
\end{enumerate}
\noindent\textbf{解说:}业余无线电通信常用缩语“NW”的意思是\textbf{现在}。其余选项为:怎样的常用缩语是HW,不的英语为NO,新的英语为NEW。\\\noindent\textbf{答案:}C



\bigskip


\noindent\textbf{问题:}“操作员”的业余无线电通信常用缩语是:
\begin{enumerate}[label=\Alph*), leftmargin=3em]
\item OP、OPR
\item CPI
\item OM
\item RPT
\end{enumerate}
\noindent\textbf{解说:}“操作员”的业余无线电通信常用缩语是\textbf{OP、OPR}。其余选项的常用缩语为:RPT为重复,OM为老朋友,CPI为抄收。\\\noindent\textbf{答案:}A



\bigskip


\noindent\textbf{问题:}“邮政信箱”的业余无线电通信常用缩语是:
\begin{enumerate}[label=\Alph*), leftmargin=3em]
\item QTH
\item MAIL
\item BURO
\item P O BOX
\end{enumerate}
\noindent\textbf{解说:}“邮政信箱”的业余无线电通信常用缩语是\textbf{P O BOX}。其余选项为:BURO为管理局的常用缩语,MAIL为邮件的英语,QTH为我的位置是纬度……经度……的Q简语。\\\noindent\textbf{答案:}D



\bigskip


\noindent\textbf{问题:}业余无线电通信常用缩语“RMKS”的意思是:
\begin{enumerate}[label=\Alph*), leftmargin=3em]
\item 报告
\item 备注、注释
\item 业余无线电测向
\item 中继台
\end{enumerate}
\noindent\textbf{解说:}业余无线电通信常用缩语“RMKS”的意思是\textbf{备注、注释}。其余选项为:报告的常用缩语为RPRT,业余无线电测向的简称为ARDF,中继台的英语为REPEATER。\\\noindent\textbf{答案:}B



\bigskip


\noindent\textbf{问题:}“收信机”的业余无线电通信常用缩语是:
\begin{enumerate}[label=\Alph*), leftmargin=3em]
\item XMTR
\item XCVR
\item RCVR,RX
\item RMKS
\end{enumerate}
\noindent\textbf{解说:}“收信机”的业余无线电通信常用缩语是\textbf{RCVR,RX}。其余选项的常用缩语为:XCVR为收发信机,XMTR为发信机,RMKS为备注、注释。\\\noindent\textbf{答案:}C



\bigskip


\noindent\textbf{问题:}“发信机”的业余无线电通信常用缩语是:
\begin{enumerate}[label=\Alph*), leftmargin=3em]
\item VXO
\item VXCO
\item XTL
\item TX、XMTR
\end{enumerate}
\noindent\textbf{解说:}“发信机”的业余无线电通信常用缩语是\textbf{TX、XMTR}。其余选项的常用缩语为:VXO为可变频率晶体振荡器。\\\noindent\textbf{答案:}D



\bigskip


\noindent\textbf{问题:}“收发信机”的业余无线电通信常用缩语是:
\begin{enumerate}[label=\Alph*), leftmargin=3em]
\item XTL
\item XMTR
\item XCVR
\item XVTR
\end{enumerate}
\noindent\textbf{解说:}“收发信机”的业余无线电通信常用缩语是\textbf{XCVR}。其余选项的常用缩语为:XMTR为发信机,XVTR为变频器。\\\noindent\textbf{答案:}C





\bigskip


\noindent\textbf{问题:}业余无线电通信常用缩语“WX”的意思是:
\begin{enumerate}[label=\Alph*), leftmargin=3em]
\item 星期
\item 天气
\item 瓦特
\item 联络、工作
\end{enumerate}
\noindent\textbf{解说:}业余无线电通信常用缩语“WX”的意思是\textbf{天气}。其余常用缩语选项为:瓦特为“W”,联络、工作为“WK”。星期的英语为WEEK。\\\noindent\textbf{答案:}B


\bigskip


\noindent\textbf{问题:}业余无线电通信常用缩语“73”的意思是:
\begin{enumerate}[label=\Alph*), leftmargin=3em]
\item 再见
\item 希望下次再见
\item 谢谢你
\item 向对方的致意、美好的祝愿
\end{enumerate}
\noindent\textbf{解说:}业余无线电通信常用缩语“73”的意思是\textbf{向对方的致意、美好的祝愿}。其余选项的常用缩语为:再见为GB,谢谢你为TU。\\\noindent\textbf{答案:}D



\bigskip


\noindent\textbf{问题:}业余无线电通话常用语“Roger”的用法是:
\begin{enumerate}[label=\Alph*), leftmargin=3em]
\item 回答起始语,相当于“听到”,用于能听到对方信号、但不一定能全部抄收的情况
\item 回答起始语,表示开始发话了,任何情况都可使用
\item 惯用口头语,相当于电话的“喂”,仅提起注意,不包含任何意义
\item 回答起始语,相当于“明白”,仅在已完全抄收对方刚才发送的信息时使用
\end{enumerate}
\noindent\textbf{解说:}。业余无线电通话常用语“Roger”为\textbf{回答起始语,相当于“明白”,仅在已完全抄收对方刚才发送的信息时使用}。\\\noindent\textbf{答案:}D


\bigskip


\noindent\textbf{问题:}业余电台通信中常用到缩写“SASE”,其意义是:
\begin{enumerate}[label=\Alph*), leftmargin=3em]
\item 写好收信人地址的信封
\item 请勿通过卡片管理局交换QSL卡
\item 国际邮资券
\item 请尽快寄出QSL卡片
\end{enumerate}
\noindent\textbf{解说:}业余电台通信中常用缩写“SASE”的意义是\textbf{写好收信人地址的信封}。\\\noindent\textbf{答案:}A



\bigskip


\noindent\textbf{问题:}“报告”的业余无线电通信常用缩语是:
\begin{enumerate}[label=\Alph*), leftmargin=3em]
\item RPRT
\item RMKS
\item MSG
\item PRT
\end{enumerate}
\noindent\textbf{解说:}“报告”的业余无线电通信常用缩语是\textbf{RPRT}。其余选项的常用缩语为:RMKS为备注、注释,MSG为电报、消息。\\\noindent\textbf{答案:}A





\bigskip


\noindent\textbf{问题:}业余无线电通信常用缩语“SK”(通常在CW中连在一起拍发)的意思是:
\begin{enumerate}[label=\Alph*), leftmargin=3em]
\item 下次再见
\item 谢谢
\item 开关
\item 结束通信
\end{enumerate}
\noindent\textbf{解说:}业余无线电通信常用缩语“SK”的意思是\textbf{结束通信}。其余选项的常用缩语为:谢谢为TNX。\\\noindent\textbf{答案:}D




\bigskip


\noindent\textbf{问题:}“对不起”的业余无线电通信常用缩语是:
\begin{enumerate}[label=\Alph*), leftmargin=3em]
\item SK
\item AS
\item TNX
\item SRI,SRY
\end{enumerate}
\noindent\textbf{解说:}“对不起”的业余无线电通信常用缩语是\textbf{SRI,SRY}。其余选项的常用缩语为:TNX为谢谢,SK为结束通信,AS为等待。\\\noindent\textbf{答案:}D



\bigskip


\noindent\textbf{问题:}“电台”的业余无线电通信常用缩语是:
\begin{enumerate}[label=\Alph*), leftmargin=3em]
\item SRY
\item QTH
\item ANT
\item STN
\end{enumerate}
\noindent\textbf{解说:}“电台”的业余无线电通信常用缩语是\textbf{STN}。其他选项为:SRY为对不起的常用缩语,ANT为天线的常用缩语,QTH为我的位置是纬度……经度……的Q简语。\\\noindent\textbf{答案:}D



\bigskip


\noindent\textbf{问题:}业余无线电通信常用缩语“SURE”的意思是:
\begin{enumerate}[label=\Alph*), leftmargin=3em]
\item 确实
\item 电台
\item 短波收听者
\item 对不起
\end{enumerate}
\noindent\textbf{解说:}业余无线电通信常用缩语“SURE”的意思是\textbf{确实}。其余选项的常用缩语为:电台为STN,短波收听者为SWL,对不起为SRY。\\\noindent\textbf{答案:}A


\bigskip


\noindent\textbf{问题:}业余无线电常用缩语“SWL”的意思是:
\begin{enumerate}[label=\Alph*), leftmargin=3em]
\item 确认
\item 对不起
\item 短波收听者
\item 高兴
\end{enumerate}
\noindent\textbf{解说:}业余无线电常用缩语“SWL”的意思是\textbf{短波收听者}。其余选项的常用缩语为:确认为CFM,对不起为SRY,高兴为GLD。\\\noindent\textbf{答案:}C



\bigskip


\noindent\textbf{问题:}“温度”的业余无线电通信常用缩语是:
\begin{enumerate}[label=\Alph*), leftmargin=3em]
\item TEMP
\item WX
\item TMPO
\item TUNE
\end{enumerate}
\noindent\textbf{解说:}“温度”的业余无线电通信常用缩语是\textbf{TEMP}。其余选项的常用缩语为:WX为天气。\\\noindent\textbf{答案:}A


\bigskip


\noindent\textbf{问题:}“谢谢”的业余无线电通信常用缩语是:
\begin{enumerate}[label=\Alph*), leftmargin=3em]
\item SRI,SRY
\item 73
\item TRY
\item TNX,TKS
\end{enumerate}
\noindent\textbf{解说:}“谢谢”的业余无线电通信常用缩语是\textbf{TNX,TKS}。其余选项的常用缩语为:SRI,SRY为对不起,73为美好的祝愿。TRY为尝试的英语。\\\noindent\textbf{答案:}D



\bigskip


\noindent\textbf{问题:}业余无线电通信常用缩语“TU”的意思是:
\begin{enumerate}[label=\Alph*), leftmargin=3em]
\item 电子管
\item 天调、天线调谐器
\item 谢谢你
\item 发信机
\end{enumerate}
\noindent\textbf{解说:}业余无线电通信常用缩语“TU”的意思是\textbf{谢谢你}。其余选项的常用缩语为:天调、天线调谐器为ATU,发信机为TX。电子管的英语为TUBE。\\\noindent\textbf{答案:}C

%%%???



\bigskip


\noindent\textbf{问题:}“世界协调时”的业余无线电通信常用缩语是:
\begin{enumerate}[label=\Alph*), leftmargin=3em]
\item UCT
\item TUC
\item CUT
\item UTC
\end{enumerate}
\noindent\textbf{解说:}“世界协调时”的业余无线电通信常用缩语是\textbf{UTC}。\\\noindent\textbf{答案:}D



\bigskip


\noindent\textbf{问题:}业余无线电通信常用缩语“VIA”的意思是:
\begin{enumerate}[label=\Alph*), leftmargin=3em]
\item 美国之音%谁出的作死题?
\item 经、由
\item 邮寄
\item 声控
\end{enumerate}
\noindent\textbf{解说:}业余无线电通信常用缩语“VIA”的意思是\textbf{经、由}。其余选项的常用缩语为:邮寄为SD,声控为VOX。美国之音的英语缩写为VOA。\\\noindent\textbf{答案:}B


\bigskip


\noindent\textbf{问题:}“很”、“非常”的业余无线电通信常用缩语是:
\begin{enumerate}[label=\Alph*), leftmargin=3em]
\item MNI
\item VIA
\item VY
\item ALL
\end{enumerate}
\noindent\textbf{解说:}“很”、“非常”的业余无线电通信常用缩语是\textbf{VY}。其他选项为:VIA为经、由,MNI为许多。ALL为全部的英语。\\\noindent\textbf{答案:}C



\bigskip


\noindent\textbf{问题:}业余无线电通信常用缩语“WK”的意思是:
\begin{enumerate}[label=\Alph*), leftmargin=3em]
\item 瓦特
\item 星期、工作
\item 插入
\item 圣诞节
\end{enumerate}
\noindent\textbf{解说:}业余无线电通信常用缩语“WK”的意思是\textbf{星期、工作}。其他选项的常用缩语为:瓦特为W,圣诞节为XMAS,插入为BK。\\\noindent\textbf{答案:}B


\bigskip


\noindent\textbf{问题:}业余无线电通信常用缩语“WKD”的意思是:
\begin{enumerate}[label=\Alph*), leftmargin=3em]
\item 圣诞节
\item 星期
\item 联络过、工作过
\item 天气
\end{enumerate}
\noindent\textbf{解说:}业余无线电通信常用缩语“WKD”的意思是\textbf{联络过、工作过}。其他选项的常用缩语为:星期为WK,天气为WX,圣诞节为XMAS。\\\noindent\textbf{答案:}C



\bigskip


\noindent\textbf{问题:}业余无线电通信常用缩语“WTS”的意思是:
\begin{enumerate}[label=\Alph*), leftmargin=3em]
\item 瓦特
\item 工作、联络
\item 天气
\item 星期
\end{enumerate}
\noindent\textbf{解说:}业余无线电通信常用缩语“WTS”的意思是\textbf{瓦特}。其他选项的常用缩语为:工作、联络为WKD,天气为WX,星期为WK。\\\noindent\textbf{答案:}A


\bigskip


\noindent\textbf{问题:}业余无线电通信常用缩语“XMAS”的意思是:
\begin{enumerate}[label=\Alph*), leftmargin=3em]
\item 收发信机
\item 晶体
\item 圣诞节
\item 发信机
\end{enumerate}
\noindent\textbf{解说:}业余无线电通信常用缩语“XMAS”的意思是\textbf{圣诞节}。其他选项的常用缩语为:发信机为XMTR,收发信机为XCVR。晶体的英语为CRYSTAL。\\\noindent\textbf{答案:}C




\bigskip


\noindent\textbf{问题:}业余无线电通信常用缩语“XYL”的意思是:
\begin{enumerate}[label=\Alph*), leftmargin=3em]
\item 妻子、已婚女子
\item 姑娘
\item 晶体
\item 发信机
\end{enumerate}
\noindent\textbf{解说:}业余无线电通信常用缩语“XYL”的意思是\textbf{妻子、已婚女子}。其他选项的常用缩语为:发信机为XMTR。\\\noindent\textbf{答案:}A



\bigskip


\noindent\textbf{问题:}业余无线电通信常用缩语“YL”的意思是:
\begin{enumerate}[label=\Alph*), leftmargin=3em]
\item 呼叫
\item 好运气
\item 小姐、女士
\item 你的
\end{enumerate}
\noindent\textbf{解说:}业余无线电通信常用缩语“YL”的意思是\textbf{小姐、女士}。其他选项的常用缩语为:呼叫为CL,好运气为GL,你的为YR。\\\noindent\textbf{答案:}C




\bigskip


\noindent\textbf{问题:}业余无线电通信常用缩语“TU”的意思是:
\begin{enumerate}[label=\Alph*), leftmargin=3em]
\item 结束联络
\item 美好的祝愿
\item 再见
\item 谢谢你
\end{enumerate}
\noindent\textbf{解说:}业余无线电通信常用缩语“TU”的意思是\textbf{谢谢你}。其他选项的常用缩语为:美好的祝愿为73,结束联络、再见为SK。\\\noindent\textbf{答案:}D



\bigskip


\noindent\textbf{问题:}“你的”或者“你是”的业余无线电通信常用缩语是:
\begin{enumerate}[label=\Alph*), leftmargin=3em]
\item FB
\item UR
\item US
\item TU
\end{enumerate}
\noindent\textbf{解说:}“你的”或者“你是”的业余无线电通信常用缩语是\textbf{UR}。其他选项的常用缩语为:FB为很好,TU为谢谢你。\\\noindent\textbf{答案:}B



\bigskip


\noindent\textbf{问题:}业余无线电通信常用缩语“88”的意思是:
\begin{enumerate}[label=\Alph*), leftmargin=3em]
\item 祝对方发达、发财
\item 再见
\item 谢谢你
\item 向对方异性操作员的致意、美好的祝愿
\end{enumerate}
\noindent\textbf{解说:}业余无线电通信常用缩语“88”的意思是\textbf{向对方异性操作员的致意、美好的祝愿}。其他选项的常用缩语为:再见为SK,谢谢你为TU。\\\noindent\textbf{答案:}D



\bigskip


\noindent\textbf{问题:}业余无线电通信中常用的天线种类的缩写DP代表:
\begin{enumerate}[label=\Alph*), leftmargin=3em]
\item 定向天线
\item 垂直天线
\item 偶极天线
\item 长线天线
\end{enumerate}
\noindent\textbf{解说:}业余无线电通信中常用的天线种类的缩写DP代表\textbf{偶极天线}。其余选项的缩写分别为:长线天线为LW,定向天线为BEAM,垂直天线为VERT。\\\noindent\textbf{答案:}C



\bigskip


\noindent\textbf{问题:}业余无线电通信中常用的天线种类的缩写LW代表:
\begin{enumerate}[label=\Alph*), leftmargin=3em]
\item 长线天线
\item 偶极天线
\item 定向天线
\item 垂直天线
\end{enumerate}
\noindent\textbf{解说:}业余无线电通信中常用的天线种类的缩写LW代表\textbf{长线天线}。其余选项的缩写分别为:偶极天线为DP,定向天线为BEAM,垂直天线为VERT。\\\noindent\textbf{答案:}A



\bigskip


\noindent\textbf{问题:}业余无线电通信中常用的天线种类的缩写GP代表:
\begin{enumerate}[label=\Alph*), leftmargin=3em]
\item 垂直接地天线
\item 偶极天线
\item 对数周期天线
\item 定向天线
\end{enumerate}
\noindent\textbf{解说:}业余无线电通信中常用的天线种类的缩写GP代表\textbf{垂直接地天线}。其余选项的缩写分别为:偶极天线为DP,对数周期天线为LP,定向天线为BEAM。\\\noindent\textbf{答案:}A



\bigskip


\noindent\textbf{问题:}业余无线电通信中常用的天线种类的缩写BEAM代表:
\begin{enumerate}[label=\Alph*), leftmargin=3em]
\item 定向天线
\item 专指八木天线
\item 偶极天线
\item 垂直天线
\end{enumerate}

\bigskip


\noindent\textbf{问题:}业余无线电通信中常用的天线种类的缩写YAGI代表:
\begin{enumerate}[label=\Alph*), leftmargin=3em]
\item 八木天线
\item 定向天线
\item 偶极天线
\item 垂直天线
\end{enumerate}

\bigskip


\noindent\textbf{问题:}业余无线电通信中常用的天线种类的缩写VER代表:
\begin{enumerate}[label=\Alph*), leftmargin=3em]
\item 垂直天线
\item 垂直接地天线
\item 定向天线
\item 偶极天线
\end{enumerate}

\bigskip


\noindent\textbf{问题:}业余无线电技术常提到的天线种类的缩写LP代表:
\begin{enumerate}[label=\Alph*), leftmargin=3em]
\item 对数周期天线
\item 垂直天线
\item 垂直接地天线
\item 定向天线
\end{enumerate}

\bigskip


\noindent\textbf{问题:}业余无线电通信方式缩写CW的英文原词意义是:
\begin{enumerate}[label=\Alph*), leftmargin=3em]
\item 等幅电报
\item 莫尔斯编码
\item 幅度键控
\item 移频键控
\end{enumerate}

\bigskip


\noindent\textbf{问题:}为了便于计算时间,将地球划分为若干个时区,各理论时区的划分方法是:
\begin{enumerate}[label=\Alph*), leftmargin=3em]
\item 全球划分为24个时区,每个理论时区宽度为经度15度,本初子午线通过0区的中心
\item 全球划分为12个时区,每个理论时区宽度为经度30度,本初子午线通过0区的中心
\item 全球划分为24个时区,每个理论时区宽度为经度15度,其边界为东西经度为15的整倍数的子午线
\item 全球划分为12个时区,每个理论时区宽度为经度30度,其边界为东西经度为30的整倍数的子午线
\end{enumerate}

\bigskip


\noindent\textbf{问题:}为了便于计算时间,将地球划分为若干个时区,各理论时区的命名规则是:
\begin{enumerate}[label=\Alph*), leftmargin=3em]
\item 本初子午线通过其中心的为0区,向东依次为东1区、东2区…东12区,向西依次为西1区、西2区…西12区
\item 本初子午线通过中心的为0区,向东依次为1区、2区…24区
\item 本初子午线通过中心的为0区,向西依次为1区、2区…24区
\item 本初子午线通过中心的为0区,如向东数则依次称为东1区、东2区…东24区,如向西数则依次称为西1区、西2区…西24区
\end{enumerate}

\bigskip


\noindent\textbf{问题:}为了便于计算时间,将地球划分为若干个时区,北京的情况是:
\begin{enumerate}[label=\Alph*), leftmargin=3em]
\item 北京处于东8区,地方时间比0时区的时间早8小时
\item 北京处于东8区,地方时间比0时区的时间晚8小时
\item 北京处于西8区,地方时间比0时区的时间早8小时
\item 北京处于西8区,地方时间比0时区的时间晚8小时
\end{enumerate}

\bigskip


\noindent\textbf{问题:}为了便于计算时间,将地球划分为若干个时区,其理论分区为每区宽经度15度。北京、西安和乌鲁木齐实际所属的时区应为:
\begin{enumerate}[label=\Alph*), leftmargin=3em]
\item 世界上实际使用法定分区,北京、西安、乌鲁木齐都属于东8区
\item 根据所在经度推算,北京、西安、乌鲁木齐分别处于东8区、东7区和东6区
\item 根据所在经度推算,北京、西安、乌鲁木齐分别处于西8区、西7区和西6区
\item 根据所在经度推算,北京、西安、乌鲁木齐分别处于东6区、东7区和东8区
\end{enumerate}

\bigskip


\noindent\textbf{问题:}已知北京时间,相应的UTC时间应为:
\begin{enumerate}[label=\Alph*), leftmargin=3em]
\item 北京时间的小时数减8,如小时数小于0,则小时数加24,日期改为前一天。
\item 北京时间的小时数减8,如小时数小于0,则小时数加24,日期改为后一天。
\item 北京时间的小时数加8,如小时数大于24,则小时数减24,日期改为前一天。
\item 北京时间的小时数加8,如小时数大于24,则小时数减24,日期改为后一天。
\end{enumerate}

\bigskip


\noindent\textbf{问题:}已知UTC时间,相应的北京时间应为:
\begin{enumerate}[label=\Alph*), leftmargin=3em]
\item UTC时间的小时数加8,如小时数大于24,则小时数减24,日期改为后一天。
\item 北京时间的小时数减8,如小时数小于0,则小时数加24,日期改为后一天。
\item 北京时间的小时数加8,如小时数大于24,则小时数减24,日期改为前一天。
\item 北京时间的小时数减8,如小时数小于0,则小时数加24,日期改为前一天。
\end{enumerate}

\bigskip


\noindent\textbf{问题:}已知某业余电台处于西N时区(N为0-12间的整数),该台的当地时间应比北京时间:
\begin{enumerate}[label=\Alph*), leftmargin=3em]
\item 晚8+N小时
\item 晚8-N小时
\item 早8+N小时
\item 早8-N小时
\end{enumerate}

\bigskip


\noindent\textbf{问题:}已知某业余电台处于东N时区(N为0-12间的整数),该台的当地时间应比北京时间:
\begin{enumerate}[label=\Alph*), leftmargin=3em]
\item 晚8-N小时
\item 晚8+N小时
\item 早8-N小时
\item 早8+N小时
\end{enumerate}

\bigskip


\noindent\textbf{问题:}为划分无线电频率,国际电信联盟《无线电规则》进行了如下的区域划分:
\begin{enumerate}[label=\Alph*), leftmargin=3em]
\item 将世界划分为3个区域,中国位于第3区
\item 将世界划分为40个区域,中国位于第24、25区
\item 将世界划分为89个区域,中国位于第33、42、43、44、45、50区
\item 将世界划分为17个区域,中国位于第8区
\end{enumerate}

\bigskip


\noindent\textbf{问题:}在业余无线电通信中,经常用到把全球分为三个区域的分区办法。制定该分区的国际机构及其公布的文件分别为:
\begin{enumerate}[label=\Alph*), leftmargin=3em]
\item 国际电信联盟ITU,《无线电规则》
\item 美国业余无线电协会ARRL,《业余无线电手册》
\item 国际业余无线电协会IARU,《IARU新闻》
\item 美国《CQ》杂志,《WAZ奖状规则》
\end{enumerate}

\bigskip


\noindent\textbf{问题:}ITU的区域划分有一套详细的规则,粗略地描述大体是:
\begin{enumerate}[label=\Alph*), leftmargin=3em]
\item 欧洲、俄罗斯亚洲部分、蒙古及部分西北亚国家为一区,南北美洲为二区,亚洲(除俄罗斯、蒙古和部分西北亚洲国家)和大洋洲为三区
\item 欧洲、俄罗斯亚洲部分、蒙古及部分西北亚国家为一区,亚洲(除俄罗斯、蒙古和部分西北亚洲国家)和大洋洲为二区,南北美洲为三区,
\item 南北美洲为一区,欧洲、俄罗斯亚洲部分、蒙古及部分西北亚国家为二区,亚洲(除俄罗斯、蒙古和部分西北亚洲国家)和大洋洲为三区
\item 南北美洲为一区,亚洲(除俄罗斯、蒙古和部分西北亚洲国家)和大洋洲为二区,欧洲、俄罗斯亚洲部分、蒙古及部分西北亚国家为三区
\end{enumerate}

\bigskip


\noindent\textbf{问题:}业余无线电通信计算成绩时,经常用到“CQ分区”。制定该分区的民间机构及其公布的文件分别为:
\begin{enumerate}[label=\Alph*), leftmargin=3em]
\item 美国《CQ》杂志,《WAZ奖状规则》
\item 美国业余无线电协会ARRL,《业余无线电手册》
\item 国际业余无线电协会ITU,《IARU新闻》
\item 英国业余无线电协会RSGB,《无线电通信》杂志
\end{enumerate}

\bigskip


\noindent\textbf{问题:}我国所属的“CQ分区”有:
\begin{enumerate}[label=\Alph*), leftmargin=3em]
\item 23、24、27
\item 42、43、44
\item 23、24
\item 42、43、44、50
\end{enumerate}

\bigskip


\noindent\textbf{问题:}我国黄岩岛、东沙岛、钓鱼岛分别属于“CQ分区”的:
\begin{enumerate}[label=\Alph*), leftmargin=3em]
\item 27、24、24
\item 24、24、25
\item 27、27、24
\item 44、44、50
\end{enumerate}

\bigskip


\noindent\textbf{问题:}“ITU分区”是IARU的活动计算通信成绩的基础。我国所属的“ITU分区”有:
\begin{enumerate}[label=\Alph*), leftmargin=3em]
\item 33、42、43、44、50
\item 33、42、43、44
\item 23、24
\item 23、24、27
\end{enumerate}

\bigskip


\noindent\textbf{问题:}“ITU分区”是IARU的活动计算通信成绩的基础。我国黄岩岛、东沙岛、钓鱼岛分别属于“ITU分区”的:
\begin{enumerate}[label=\Alph*), leftmargin=3em]
\item 50、44、44
\item 24、24、25
\item 44、44、44
\item 27、24、24
\end{enumerate}

\bigskip


\noindent\textbf{问题:}业余无线电通信梅登海德网格定位系统(Maidenhead Grid Square Locator)是一种:
\begin{enumerate}[label=\Alph*), leftmargin=3em]
\item 根据经纬度坐标对地球表面进行网格划分和命名,用以标示地理位置的系统
\item 卫星定位系统
\item 根据国际呼号系列对地球表面进行网格划分和命名,用以标示地理位置的系统
\item 根据国际政治行政区划对地球表面进行网格划分和命名,用以标示地理位置的系统
\end{enumerate}

\bigskip


\noindent\textbf{问题:}业余无线电通信常用的梅登海德网格定位系统网格名称的格式为:
\begin{enumerate}[label=\Alph*), leftmargin=3em]
\item 2个字母和2位数字、2个字母和2位数字再加2个字母
\item 4位数字或者6位数字
\item 4个字母或者6个字母
\item 呼号前缀字母加2位数字和2个字母
\end{enumerate}

\bigskip


\noindent\textbf{问题:}业余无线电通信常用的梅登海德网格定位系统网格名称的长度是4字符或6字符,两者定位精度不同,差别为:
\begin{enumerate}[label=\Alph*), leftmargin=3em]
\item 两者网格大小不同,4字符网格为经度2度和纬度1度,6字符网格为经度5分和纬度2.5分
\item 4字符网格精确到国家分区,6字符网格精确到国家的城市或县乡
\item 4字符网格根据国际呼号系列区分,6字符网格在4字符基础上加以经纬度细分
\item 4字符网格名称用于HF频段通信,6字符网格名称用于VHF/UHF通信
\end{enumerate}

\bigskip

\noindent\textbf{问题:}以下呼号前缀中,所属CQ分区与埃及相同的是:
\begin{enumerate}[label=\Alph*), leftmargin=3em]
	\item 5X
	\item 5A
	\item 5T
	\item 5W
\end{enumerate}
\noindent\textbf{解说:}\textbf{5A}(利比亚)与埃及同属34分区。其余选项为:5W(萨摩亚)CQ分区为32,5T(毛里塔尼亚)CQ分区为35,5X(乌干达)CQ分区为37。\\\noindent\textbf{答案:}B

\bigskip


\noindent\textbf{问题:}对于中国HAM来说,属于既稀有又困难的是:
\begin{enumerate}[label=\Alph*), leftmargin=3em]
	\item VK0HR
	\item KP5A
	\item P5/4L4FN
	\item KP1A
\end{enumerate}
\noindent\textbf{解说:}对于中国HAM来说,属于既稀有又困难的是\textbf{KP5A},其余选项为:KP1A为美国呼号,VK0HR为澳大利亚呼号,P5/4L4FN为格鲁吉亚电台在朝鲜民主主义人民共和国使用的呼号。\\\noindent\textbf{答案:}B
%%%???题目有错???

\bigskip


\noindent\textbf{问题:}3V、4X、5A、6Y字头所代表的国家是:
\begin{enumerate}[label=\Alph*), leftmargin=3em]
	\item Libya、Israel、Jamaica、Guinea
	\item Jamaica、Israel、Libya、Senegal
	\item Guinea、Israel、Fiji Islands、Senegal
	\item Tunisa、Israel、Libya、Jamaica%原本Tunis、Israel、Libya、Jamaica
\end{enumerate}
\noindent\textbf{解说:}3V字头代表\textbf{Tunisia}(突尼斯),4X字头代表\textbf{Israel}(以色列),5A字头代表\textbf{Libya}(利比亚),6Y字头代表\textbf{Jamaica}(牙买加)。\\\noindent\textbf{答案:}D


\bigskip


\noindent\textbf{问题:}业余无线电通信所说的“网格定位”是什么意思?
\begin{enumerate}[label=\Alph*), leftmargin=3em]
	\item 一个由一串字母和数字确定的地理位置
	\item 用来调谐末级功放的设备
	\item 用于无线电测向运动的设备
	\item 一个由一串字母和数字确定的方位角和仰角
\end{enumerate}
\noindent\textbf{解说:}业余无线电通信所说的“网格定位”是\textbf{一个由一串字母和数字确定地理位置}的系统。\\\noindent\textbf{答案:}A

\bigskip

