\documentclass{ctexbook}
\usepackage[a4paper,twoside,left=2.5cm,top=2.5cm,bottom=2cm,
right=2cm]{geometry}

%%%%%%%%%%%%%%%%%%%%%%%%%%%%%%%%%%%%%%%%%% 页眉和页脚样式(仿某书风格)
\usepackage{fancyhdr}
\pagestyle{fancy}
\fancyhf{}
\fancyhead[LE]{	\textsf{ \small 业余无线电操作技能考核攻略本(类别:B)} }
\fancyhead[RO]{ \textsf{ \small \rightmark } }
\fancyfoot[LE,RO]{ \textrm{ \small --\ \thepage\ -- } }
%%%%%%%%%%%%%%%%%%%%%%%%%%%%%%%%%%%%%%%%%% 页眉和页脚样式(仿某书风格)

%\sectionmark 的重定义需要在 \pagestyle 之后生效
\renewcommand\chaptermark[1]{
	\markright{\CTEXifname{\CTEXthechapter\ }{}#1}}
\usepackage{graphicx}
\usepackage{epstopdf}
\usepackage{xcolor}
\graphicspath{ {./images/} }
\usepackage{enumitem}
\usepackage{amsmath}
\usepackage{mathtools}
\usepackage{lipsum}
\usepackage{wasysym}
\usepackage{hyperref}
\hypersetup{
	colorlinks=true,
	linkcolor=blue,
	urlcolor=red,
	pdftitle={业余无线电操作技能考核攻略本(类别:B)},
}
\usepackage{circuitikz}
\usepackage{multirow}
\usepackage{textcomp}
\usepackage{xstring}
\usepackage{catchfile}
\usepackage{siunitx}
\usepackage[yyyymmdd]{datetime}
\renewcommand{\dateseparator}{-}
\usepackage{longtable}
\usepackage{listings}
% 解决无法嵌入eps文件的问题
%\usepackage[outdir=./]{epstopdf}
\usepackage{hologo}
\usepackage{gbt7714}  % 参考文献用
\setmainfont{Noto Serif}
\setsansfont{Noto Sans}
\setmonofont{Fira Code}
\setCJKmainfont{Noto Serif SC}
\setCJKsansfont{Noto Sans SC}
\setCJKmonofont{Noto Sans SC}
\setenumerate[0]{label=(\Alph*)}
\ctexset{
	chapter/name = {第,章},
	chapter/number = \arabic{chapter},
	chapter/numberformat = \color{blue}\zihao{0}\emph,
	section/name = {},
	section/number = \Roman{section},
%	autoindent=false, %禁用自动调整功能
}
\setlist{nosep}

\CatchFileDef{\headfull}{.git/HEAD}{}
\StrGobbleRight{\headfull}{1}[\head]
\StrBehind[2]{\head}{/}[\branch]
\IfFileExists{.git/refs/heads/\branch}{%
	\CatchFileDef{\commit}{.git/refs/heads/\branch}{}}{%
	\newcommand{\commit}{\dots~(in \emph{packed-refs})}}

\definecolor{codegreen}{HTML}{408080}
\definecolor{codegray}{HTML}{72777D}
\definecolor{codepurple}{HTML}{BA2121}
\definecolor{backcolour}{HTML}{F8F9FA}
\definecolor{keyword}{HTML}{008000}

\lstdefinestyle{mystyle}{
	backgroundcolor=\color{backcolour},
	commentstyle=\color{codegreen},
	keywordstyle=\color{keyword},
	numberstyle=\tiny\color{codegray},
	stringstyle=\color{codepurple},
	basicstyle=\ttfamily\footnotesize,
	breakatwhitespace=false,
	breaklines=true,
	captionpos=b,
	keepspaces=true,
	numbers=left,
	numbersep=5pt,
	showspaces=false,
	showstringspaces=false,
	showtabs=false,
	tabsize=4
}

\lstset{style=mystyle}

%\setlength{\headheight}{13.06885pt}

\begin{document}%内容开始


\title{\sffamily {\Huge 业余无线电操作技能考核攻略本\\(类别:B)}}
\author{\large \texttt{BG7XTQ}(原\texttt{JJ1DSB})编著}
\date{\large{日期:\texttt{\today}}
	\\版本:\texttt{\commit}在分支\texttt{\branch}上
	\\使用 \textrm{\hologo{XeLaTeX}} 编译
}

\maketitle %标题页


%献辞
\thispagestyle{empty}
\vfil
\ \\
\vspace{15em}
\begin{center}
	{\Large 献给我的母亲。}
\end{center}
%献辞

\newpage

\tableofcontents%目录

% 前言
\include{qianyan}

% 编著者的话
\chapter*{编著者的话}

无线电资源是全人类共同的财产。提到无线电,我们再熟悉不过的是日常生活中的手机和Wi-Fi,在军事上,人们利用无线电控制导弹、飞机,%用来杀人
在救险活动上,人们利用无线电辅助实施灾害时的救援,%用来救人
在业余无线电领域,爱好者们互相通信,以提高技能,同时学习新知识。

笔者原本对业余无线电一无所知,因为精通业余无线电的朋友的介绍,才逐渐开始对其有所了解。在日本留学期间,笔者考取了日本的操作证书和电台执照,建立了第一个自己的业余无线电台,开始了业余无线电爱好者的旅途。

在归国后,笔者通过业余无线电操作证考试拿到了A类的操作证。在操作证考试应试学习过程中,笔者深深感到,国内现有的操作证考试应试书籍对于很多小学生读者来说,缺乏细致的解释,题目里的术语艰涩难懂,计算题不知道如何计算,用这些书籍学习的读者,想必难以通过操作证考试。在这样的背景下,笔者萌生了撰写一本老少皆能读懂的操作证考试的应试书籍的想法。

本书在写作过程中,为了让业余无线电知识几乎完全不了解的初学者也能读懂,笔者经过了反复的推敲,尽可能的把复杂的业余无线电知识简单易懂地展现给读者们。本书在解题的过程中,适当地介绍相关的术语,并把重点难点用加粗的字体标出,方便应试者快速记忆概念、理解计算方法。

希望本书能帮助您顺利通过考试。

%咋的,编不出来了?


% 第一章 无线电的相关法律 开始

\chapter{无线电的相关法律}

\newpage



\noindent\textbf{试题:}我国现行法律体系中专门针对无线电管理的最高法律文件及其立法机关是:
\begin{enumerate}[leftmargin=3em]
\item 中华人民共和国电信条例,国务院
\item 中华人民共和国业余无线电台管理办法,工业和信息化部
\item 中华人民共和国无线电管理条例,国务院和中央军委
\item 中华人民共和国无线电管理办法,工业和信息化部
\end{enumerate}
\noindent\textbf{解说:}我国专门针对无线电管理的最高法律文件是\textbf{《中华人民共和国无线电管理条例》},其立法机关是\textbf{中华人民共和国国务院和中华人民共和国中央军事委员会}。\\\noindent\textbf{答案:}C

\bigskip


\noindent\textbf{试题:}我国现行法律体系中专门针对业余无线电台管理的最高法律文件及其立法机关是:
\begin{enumerate}[leftmargin=3em]
\item 业余无线电台管理办法,工业和信息化部
\item 个人业余无线电台管理暂行办法,国家体委和国家无委
\item 中华人民共和国电信条例,国务院
\item 业余无线电台管理暂行规定,国家体委和国家无委
\end{enumerate}
\noindent\textbf{解说:}我国专门针对业余无线电台管理的最高法律文件是\textbf{《业余无线电台管理办法》},其立法机关是\textbf{中华人民共和国工业和信息化部}。\\\noindent\textbf{答案:}A

\bigskip


\noindent\textbf{试题:}我国的无线电主管部门是:
\begin{enumerate}[leftmargin=3em]
\item 各级无线电管理机构
\item 各地业余无线电协会
\item 各级体育管理机构
\item 各地电信管理局
\end{enumerate}
\noindent\textbf{解说:}《业余无线电台管理办法》第三条规定:\\\textbf{国家无线电管理机构和省、自治区、直辖市无线电管理机构(以下简称地方无线电管理机构)依法对业余无线电台实施监督管理。\\
国家无线电管理机构和地方无线电管理机构统称无线电管理机构}。\\\noindent\textbf{答案:}A

\bigskip


\noindent\textbf{试题:}我国依法负责对业余无线电台实施监督管理的机构是:
\begin{enumerate}[leftmargin=3em]
\item 国家体育管理机构和地方体育管理机构
\item 在国家或地方民政部门注册的业余无线电协会
\item 国家无线电管理机构和地方无线电管理机构
\item 国家和地方公安部门
\end{enumerate}
\noindent\textbf{解说:}《业余无线电台管理办法》第三条规定:\\\textbf{国家无线电管理机构和省、自治区、直辖市无线电管理机构(以下简称地方无线电管理机构)依法对业余无线电台实施监督管理}。\\\noindent\textbf{答案:}C

\bigskip


\noindent\textbf{试题:}《业余无线电台管理办法》所说的“地方无线电管理机构”指的是:
\begin{enumerate}[leftmargin=3em]
\item 各地方与无线电设备生产销售和无线电应用有关的行政管理机构
\item 地方业余无线电协会或者类似组织机构
\item 省、自治区、直辖市无线电管理机构
\item 地市县(区)及以下各级无线电管理机构
\end{enumerate}
\noindent\textbf{解说:}《业余无线电台管理办法》第三条规定:\textbf{省、自治区、直辖市无线电管理机构}简称地方无线电管理机构。\\\noindent\textbf{答案:}C

\bigskip


\noindent\textbf{试题:}国家鼓励和支持业余无线电台开展下列活动:
\begin{enumerate}[leftmargin=3em]
\item 无线电通信技术研究、普及活动以及突发重大自然灾害等紧急情况下的应急通信活动
\item 机动车辆行车服务性通信活动
\item 休闲娱乐性交谈
\item 作为日常公益活动的通信工具
\end{enumerate}
\noindent\textbf{解说:}《业余无线电台管理办法》第四条规定:国家鼓励和支持\textbf{业余无线电通信技术的研究、普及和突发重大自然灾害等紧急情况下的应急无线电通信活动。}\\\noindent\textbf{答案:}A

\bigskip


\noindent\textbf{试题:}关于业余电台管理的正确说法是:
\begin{enumerate}[leftmargin=3em]
\item 通过法律手段限制业余无线电台的设置
\item 在业余电台与其他业务电台遇到干扰纠纷时无条件优先保护其他业务电台
\item 依法设置的业余无线电台受国家法律保护
\item 业余无线电爱好者的一切行为都受国家法律保护
\end{enumerate}
\noindent\textbf{解说:}《业余无线电台管理办法》第四条规定:\textbf{依法设置的业余无线电台受国家法律保护。}\\\noindent\textbf{答案:}C

\bigskip


%题库中修改的题目
\noindent\textbf{试题:}无线电频率的使用必须得到各级无线电管理机构的批准,基本依据是“无线电频谱资源属于国家所有”,出自于下列法律的第二百五十二条:
\begin{enumerate}[leftmargin=3em]
\item 中华人民共和国民法典
\item 中华人民共和国电信法
\item 中华人民共和国宪法
\item 中华人民共和国刑法
\end{enumerate}
\noindent\textbf{解说:}《\textbf{中华人民共和国民法典}》第二百五十二条规定:无线电频谱资源属于国家所有。\\\noindent\textbf{答案:}A

\bigskip


\noindent\textbf{试题:}我国对无线电管理术语“业余业务”、“卫星业余业务”和“业余无线电台”做出具体定义的法规文件是
\begin{enumerate}[leftmargin=3em]
\item 无线电台执照管理规定
\item 中华人民共和国无线电管理条例
\item 中华人民共和国无线电频率划分规定
\item 中华人民共和国电信条例
\end{enumerate}
\textbf{解说:}《\textbf{中华人民共和国无线电频率划分规定}》第1.3.39款、第1.3.40款和第1.4.38款具体定义了“业余业务”、“卫星业余业务”和“业余无线电台”。\\\noindent\textbf{答案:}A

\bigskip


\noindent\textbf{试题:}业余电台的法定用途为:
\begin{enumerate}[leftmargin=3em]
\item 供公民在业余时间进行休闲娱乐
\item 供公民在业余时间进行与个人生活事务有关的通信
\item 供业余无线电爱好者进行自我训练、相互通信和技术研究
\item 供私家车主或者相应组织作为行车安全保障和途中消遣工具
\end{enumerate}
\noindent\textbf{解说:}《业余无线电台管理办法》第二十八条规定:业余无线电台供其设置人、使用人用于\textbf{相互通信、技术研究和自我训练}。\\\noindent\textbf{答案:}C

\bigskip


\noindent\textbf{试题:}无线电业余业务是供业余无线电爱好者作下列用途的无线电通信业务:
\begin{enumerate}[leftmargin=3em]
\item 救灾抢险、车队联络和技术学习
\item 技术教学、民兵训练和公益通信
\item 娱乐休闲、报告路况和公益服务
\item 自我训练、相互通信和技术研究
\end{enumerate}
\noindent\textbf{解说:}《中华人民共和国无线电频率划分规定》第1.3.39款对业余无线电业务的定义:供业余无线电爱好者进行\textbf{自我训练、相互通信和技术研究}的无线电通信业务。\\\noindent\textbf{答案:}D

\bigskip


\noindent\textbf{试题:}关于无线电通信的正确说法:
\begin{enumerate}[leftmargin=3em]
\item 产生无线电波并用其加热属于无线电通信的一种应用
\item 无线电通信包括利用光在内的所有电磁波所进行的各种通信
\item 无线电通信是指利用无线电波进行的符号、信号、文字、图像、声音或其他信息的传输、发射或接收。
\item 利用无线电波进行的符号、信号、文字、图像、声音以外的信息传输不属于无线电通信
\end{enumerate}
\noindent\textbf{解说:}ITU《无线电规则》第1.3条对电信的定义:\textbf{利用导线、无线电、光学或其他电磁系统进行的对符号、信号、文字、图像、声音或任何性质信息的传输、发射或接收}。\\\noindent\textbf{答案:}C

\bigskip


\noindent\textbf{试题:}无线电波是指:
\begin{enumerate}[leftmargin=3em]
\item 频率为3,000\unit{\GHz}以下的在空间传播的电磁波
\item 频率为3,000 Hz至3,000 \unit{\MHz}的电磁波
\item 频率为30 Hz至30\unit{\GHz}的在空间传播的电磁波
\item 频率为3,000\unit{\GHz}以下的所有电磁波
\end{enumerate}
\noindent\textbf{解说:}ITU《无线电规则》第1.5条对无线电波或赫兹波的定义:\textbf{不用人工波导而在空间传播的、频率规定在3000\unit{\GHz}以下的电磁波}。\\\noindent\textbf{答案:}A



\bigskip


\noindent\textbf{试题:}个人申请设置具有发信功能的业余无线电台的年龄条件是:
\begin{enumerate}[leftmargin=3em]
\item 年满十八周岁
\item 年满十六周岁
\item 具备《业余无线电台操作证书》者申请设置业余无线电台不受年龄限制
\item 年满十四周岁
\end{enumerate}
\noindent\textbf{解说:}《业余无线电台管理办法》第六条规定:个人申请设置具有发信功能的业余无线电台的,应当\textbf{年满十八周岁}。\\\noindent\textbf{答案:}A



\bigskip


\noindent\textbf{试题:}申请设置业余无线电台应当具备的条件有:
\begin{enumerate}[leftmargin=3em]
\item 熟悉无线电管理规定、具备国家规定的操作技术能力、发射设备符合国家技术标准、法律和行政法规规定的其他条件
\item 熟悉无线电管理规定、具备当地无线电管理机构委托的考试机构设置的操作技术能力标准、发射设备符合国家技术标准、法律和行政法规规定的其他条件
\item 加入指定协会、具备当地无线电管理机构规定的操作技术能力、发射设备符合国家技术标准、法律和行政法规规定的其他条件
\item 熟悉无线电管理规定、具备国家规定的操作技术能力、发射设备符合国家技术标准、当地无线电管理机构委托的受理机构设置的其他条件
\end{enumerate}
\noindent\textbf{解说:}《业余无线电台管理办法》第六条规定:申请设置业余无线电台,应当具备下列条件:
\begin{enumerate}[label=, leftmargin=0.8em]
\item (一)\textbf{熟悉无线电管理规定};
\item (二)\textbf{具备国家无线电管理机构规定的操作技术能力};
\item (三)\textbf{无线电发射设备符合国家相关技术标准};
\item (四)\textbf{法律、行政法规规定的其他条件}。
\end{enumerate}
要点:记忆关键字“\textbf{规定}”、“\textbf{国家}”、“\textbf{国家}”、“\textbf{法律}”。\\\noindent\textbf{答案:}A


\bigskip


\noindent\textbf{试题:}使用业余无线电台应当具备的条件有:
\begin{enumerate}[leftmargin=3em]
\item 具备国家或地方无线电管理机构核发的业余无线电台执照
\item 熟悉无线电管理规定、实际上具备国家规定的操作技术能力但不必需取得相应的证明
\item 使用具有发信功能的业余无线电台的,应当年满十八周岁
\item 熟悉无线电管理规定、具备国家规定的操作技术能力并取得相应操作技术能力证明
\end{enumerate}
\noindent\textbf{解说:}《业余无线电台管理办法》第十八条规定:使用业余无线电台,应当具备下列条件:
\begin{enumerate}[label=, leftmargin=0.8em]
\item (一)\textbf{熟悉无线电管理规定};
\item (二)\textbf{具备国家无线电管理机构规定的操作技术能力,取得相应操作技术能力证明};
\item (三)无线电发射设备符合国家相关技术标准;
\item (四)法律、行政法规规定的其他条件。
\end{enumerate}
\noindent\textbf{答案:}D


\bigskip


\noindent\textbf{试题:}按照《业余电台管理办法》规定,申请设置使用配备有多台业余无线电发射设备的业余无线电台,应该:
\begin{enumerate}[leftmargin=3em]
\item 视为一个业余电台,指配一个电台呼号,但所有设备均应经过核定并将参数载入电台执照
\item 每台设备视为一个业余电台,各指配一个电台呼号,并都应经过核定并将参数载入电台执照
\item 视为一个业余电台,指配一个电台呼号,每个频段选择一台设备加以核定并将参数载入电台执照
\item 视为一个业余电台,指配一个电台呼号,其中只需有一台设备加以核定并将参数载入电台执照
\end{enumerate}
\noindent\textbf{解说:}《业余无线电台管理办法》第三十二条规定:\textbf{核发业余无线电台执照的无线电管理机构已经为设置人指配业余无线电台呼号的,不另行为其指配其他业余无线电台呼号}。\\
《业余无线电台管理办法》第十四条规定:\textbf{业余无线电台执照应当载明所核定的技术参数和发射设备等信息}。\\
《工业和信息化部关千实施〈业余无线电台管理办法〉若干事项的通知(工信部无〔2013〕43号)》第十一款规定:\textbf{业余无线电台执照应当载明所核定使用的全部发射设备}。\\\noindent\textbf{答案:}A


\bigskip


\noindent\textbf{试题:}申请设置下列业余无线电台时应在《业余无线电台设置(变更)申请表》 的“台站种类”选择“特殊”类:
\begin{enumerate}[leftmargin=3em]
\item 中继台、信标台、空间台
\item 用于业余卫星通信的地面业余无线电台
\item 需要到外地移动操作的手持台
\item 移动操作的车载台
\end{enumerate}
\noindent\textbf{解说:}《工业和信息化部关千实施〈业余无线电台管理办法〉若干事项的通知(工信部无〔2013〕43号)》第九款规定:《业余无线电台设置(变更)申请表》和《业余无线电台技术资料申报表》中特殊台站是指\textbf{业余信标台、空间业余无线电台、特殊实验电台等}特殊业余无线电台。\\\noindent\textbf{答案:}A




\bigskip


\noindent\textbf{试题:}申请设置信标台、空间台和技术参数需要超出管理办法规定的特殊业余电台的办法为:
\begin{enumerate}[leftmargin=3em]
\item 必须由地方业余无线电协会作为申请单位,经本地无线电管理机构办理批准设台
\item 先按设置一般业余电台的办法申请,然后再到本地无线电管理机构办理变更执照核定内容
\item 按照设置一般业余电台的办法申请即可,然后根据需要操作就可以
\item 在《业余无线电台设置(变更)申请表》 的“台站种类”选择“特殊”类,由地方无线电管理机构受理和初审后交国家无线电管理机构审批
\end{enumerate}
\noindent\textbf{解说:}申请设置信标台、空间台和技术参数需要超出管理办法规定的特殊业余电台的办法为:\textbf{在《业余无线电台设置(变更)申请表》 的“台站种类”选择“特殊”类,由地方无线电管理机构受理和初审后交国家无线电管理机构审批}。\\\noindent\textbf{答案:}D

%%%???


\bigskip


\noindent\textbf{试题:}设置通信范围涉及两个以上的省、自治区、直辖市或者涉及境外的一般业余无线电台,审批机构是下列中:
\begin{enumerate}[leftmargin=3em]
\item 设台地的地方无线电民间组织
\item 设台地地方无线电管理机构
\item 国家无线电管理机构委托的设台地地方无线电民间组织
\item 国家无线电管理机构或其委托的设台地的地方无线电管理机构
\end{enumerate}
\noindent\textbf{解说:}《业余无线电台管理办法》第八条规定:\textbf{设置通信范围涉及两个以上的省、自治区、直辖市或者涉及境外的业余无线电台,由国家无线电管理机构审批。国家无线电管理机构可以委托设台地地方无线电管理机构负责除业余信标台、用于卫星业余业务的空间业余无线电台等特殊业余无线电台以外的业余无线电台的设置审批,核发业余无线电台执照}。\\\noindent\textbf{答案:}D

\bigskip


\noindent\textbf{试题:}按照在省、自治区、直辖市范围内通信所申请设置的业余无线电台,如想要将通信范围扩大至涉及两个以上的省、自治区、直辖市或者涉及境外,或者要到设台地以外进行异地发射操作,须办理下列手续:
\begin{enumerate}[leftmargin=3em]
\item 事先向核发执照的无线电管理机构申请办理变更手续,按相关流程经国家无线电管理机构或其委托的设台地的地方无线电管理机构批准后,换发业余无线电台执照
\item 反正已经有了电台执照,可先扩大操作起来,等执照有效期届满时再申请办理变更手续,换发业余无线电台执照
\item 只要不会被发现,可以不申请办理变更手续,悄悄越限操作
\item 反正已经有了电台执照,只需向核发执照的无线电管理机构通报变更情况即可,不必申请办理变更和换发执照
\end{enumerate}
\noindent\textbf{解说:}《业余无线电台管理办法》第十四条规定:需要变更业余无线电台执照核定内容的,应当\textbf{向核发执照的无线电管理机构申请办理变更手续,换发业余无线电台执照}。\\\noindent\textbf{答案:}A

\bigskip


\noindent\textbf{试题:}业余无线电台执照有效期届满后需要继续使用的,应当在下列期限内向核发执照的无线电管理机构申请办理延续手续:
\begin{enumerate}[leftmargin=3em]
\item 有效期届满一个月前
\item 有效期届满一个月之内
\item 有效期届满三个月之内
\item 有效期届满二十天前
\end{enumerate}
\noindent\textbf{解说:}《业余无线电台管理办法》第十三条规定:业余无线电台执照的有效期不超过五年。业余无线电台执照有效期届满后需要继续使用的,应当在\textbf{有效期届满三十日前}向核发执照的无线电管理机构申请办理延续手续。\\\noindent\textbf{答案:}A
%%%最新版法律:有效期届满前三十日以前

\bigskip


\noindent\textbf{试题:}因改进或调整业余发射设备使业余无线电台的技术参数超出其业余无线电台执照所核定的范围时,应当办理下列手续:
\begin{enumerate}[leftmargin=3em]
\item 等执照有效期届满时向核发执照的无线电管理机构申请办理变更手续,换发业余无线电台执照
\item 只需及时向核发执照的无线电管理机构通报变更情况,进行备案即可
\item 及时向核发执照的无线电管理机构申请办理变更手续,换发业余无线电台执照
\item 只要设备型号和产品序列号没有改变,不必申请办理变更手续
\end{enumerate}
\noindent\textbf{解说:}《业余无线电台管理办法》第十四条规定:业余无线电台的技术参数不得超出其业余无线电台执照所核定的范围。需要变更业余无线电台执照核定内容的,\textbf{应当向核发执照的无线电管理机构申请办理变更手续,换发业余无线电台执照}。\\\noindent\textbf{答案:}C




\bigskip


\noindent\textbf{试题:}终止使用业余无线电台的,应当向下列机构申请注销执照:
\begin{enumerate}[leftmargin=3em]
\item 国家无线电管理机构
\item 受国家无线电管理机构委托的全国性业余无线电民间组织
\item 受国家无线电管理机构委托的地方业余无线电民间组织
\item 核发业余无线电台执照的无线电管理机构
\end{enumerate}
\noindent\textbf{解说:}《业余无线电台管理办法》第十五条规定:终止使用业余无线电台的,应当向\textbf{核发业余无线电台执照的无线电管理机构}申请注销执照。\\\noindent\textbf{答案:}D




\bigskip


\noindent\textbf{试题:}业余无线电台专用无线电发射设备的重要特征是:
\begin{enumerate}[leftmargin=3em]
\item 发射频率不得超出业余频段
\item 发射频率必须覆盖所有业余频段
\item 必须具有数字对讲方式
\item 发射方式必须包含调频
\end{enumerate}
\noindent\textbf{解说:}《业余无线电台管理办法》第十四条规定:业余无线电台的技术参数\textbf{不得超出}其业余无线电台执照所核定的范围。\\\noindent\textbf{答案:}A

%%%?



\bigskip


\noindent\textbf{试题:}业余无线电发射设备的下列指标必须符合国家的相关规定:
\begin{enumerate}[leftmargin=3em]
\item 频率调制频偏和调制度
\item 频率容限和带外发射
\item 指配频带和必要带宽
\item 频率容限和杂散域发射功率
\end{enumerate}
\noindent\textbf{解说:}《中华人民共和国无线电频率划分规定》第2.3款规定:发射电台应符合附录 1 中规定的\textbf{频率容限}。第2.4款规定:发射电台应符合附录 2 中规定的\textbf{杂散发射或杂散域中无用发射的最大允许功率电平}。\\\noindent\textbf{答案:}D





\bigskip


\noindent\textbf{试题:}业余无线电台使用的发射设备必须符合下列条件:
\begin{enumerate}[leftmargin=3em]
\item 商品设备应当具备《无线电发射设备型号核准证》,自制、改装、拼装设备应通过国家相关技术标准的检测
\item 必须具备《无线电发射设备型号核准证》
\item 国产商品设备应当具备《无线电发射设备型号核准证》,国外商品设备符合国际流行技术标准即可
\item 商品设备应当具备《无线电发射设备型号核准证》,自制、改装、拼装设备不受限制
\end{enumerate}
\noindent\textbf{解说:}《业余无线电台管理办法》第十一条规定:\textbf{业余无线电台无线电发射设备应当依法取得《中华人民共和国无线电发射设备型号核准证》。申请人可以使用符合国家相关技术标准的自制、改装、拼装的无线电发射设备办理审批手续。}\\\noindent\textbf{答案:}A




\bigskip


\noindent\textbf{试题:}对业余无线电台专用无线电发射设备进行型号核准的依据为:
\begin{enumerate}[leftmargin=3em]
\item 国家《无线电频率划分规定》中有关无线电发射设备技术指标的规定
\item 国家关于专业无线电通信发射设备的技术标准
\item 地方无线电管理机构制订的技术标准
\item 经国家认证的检测单位所制订的技术标准
\end{enumerate}
\noindent\textbf{解说:}《业余无线电台管理办法》第十一条规定:对业余无线电台专用无线电发射设备进行型号核准,应当以\textbf{《中华人民共和国无线电频率划分规定》中有关无线电发射设备技术指标的规定}为依据。\\\noindent\textbf{答案:}A




\bigskip


\noindent\textbf{试题:}业余无线电台专用无线电发射设备的发射频率必须满足的条件是:
\begin{enumerate}[leftmargin=3em]
\item 发射频率包含所有业余业务或者卫星业余业务频段
\item 发射频率不能超越业余业务或者卫星业余业务频段
\item 发射频率可以在业余频段和非业余频段之间选择
\item 发射频率包含至少一个业余业务或者卫星业余业务频段
\end{enumerate}
\noindent\textbf{解说:}《业余无线电台管理办法》第十一条规定:业余无线电台专用无线电发射设备不得用于其他无线电业务,其\textbf{发射频率应当在业余业务或者卫星业余业务频段内}。\\\noindent\textbf{答案:}B




\bigskip


\noindent\textbf{试题:}业余电台的无线电发射设备应符国家规定的下列主要技术指标:
\begin{enumerate}[leftmargin=3em]
\item 杂散发射不低于最大允许功率电平、电源电压及频率符合国家电网标准、采用标准天线阻抗
\item 符合频率容限、符合杂散发射最大允许功率电平
\item 杂散发射不低于最大允许功率电平、频率漂移不低于频率容限、电源利用效率满足节能要求
\item 工作频率范围足够宽、杂散发射不低于最大允许功率电平、带宽大于允许最低值
\end{enumerate}
\noindent\textbf{解说:}《中华人民共和国无线电频率划分规定》第2.3款规定:发射电台应符合附录 1 中规定的\textbf{频率容限}。第2.4款规定:业余电台的无线电发射设备应符国家规定的下列主要技术指标:发射电台应符合附录 2 中规定的\textbf{杂散发射或杂散域中无用发射的最大允许功率电平}。\\要点:记忆关键字“\textbf{符合}”、“\textbf{符合}”。\\\noindent\textbf{答案:}B




\bigskip


\noindent\textbf{试题:}频率容限是发射设备的重要指标,通常用下述单位来表示:
\begin{enumerate}[leftmargin=3em]
\item dB
\item 百万分之几(或者赫兹)
\item 瓦
\item 百分之几(或者兆赫)
\end{enumerate}
\noindent\textbf{解说:}ITU《无线电规则》(2020年版)1.151款对\textbf{频率容限}的定义:发射所占频段中心频率偏离护配频率,或发射的特征频率偏离基准频率的最大容许偏差。\textbf{频率容限以百万分之几或以赫兹表示}。\\\noindent\textbf{答案:}B




\bigskip


\noindent\textbf{试题:}杂散域发射功率是发射设备的重要指标,通常用下述单位来表示:
\begin{enumerate}[leftmargin=3em]
\item 绝对功率(瓦)
\item 千赫(或者赫兹)%赫芝
\item 绝对功率\unit[qualifier-mode=combine]{\deci\bel\of{m}}、低于载波发射功率的分贝值dBc、低于PEP发射功率的相对值dB
\item 百分之几
\end{enumerate}
\noindent\textbf{解说:}杂散域发射功率通常用\textbf{绝对功率\unit[qualifier-mode=combine]{\deci\bel\of{m}}、低于载波发射功率的分贝值dBc、低于PEP发射功率的相对值dB}来表示。\\\noindent\textbf{答案:}C




\bigskip


\noindent\textbf{试题:}杂散发射是指必要带宽之外的一个或多个频率的发射,其发射电平可降低而不致影响相应信息的传输。一台发射机,工作频率为145.000\unit{\MHz},但在435.000\unit{\MHz}的频率上也有发射。这种发射属于:
\begin{enumerate}[leftmargin=3em]
\item 杂散发射
\item 带内发射
\item 谐波发射
\item 带外发射
\end{enumerate}
\noindent\textbf{解说:}这种发射属于\textbf{杂散发射}。\\\noindent\textbf{答案:}A




\bigskip


\noindent\textbf{试题:}业余无线电专用发射设备必须满足的主要技术指标要求包括:
\begin{enumerate}[leftmargin=3em]
\item 频率容限和杂散辐射不超过限值,发射频率不超出国家规定的业余频率
\item 频率容限不低于限值,杂散辐射不超过限值,发射频率不超出国家规定的业余频率
\item 发射功率不低于功率限额,输出阻抗符合工业标准
\item 频率容限和杂散辐射不超过限值,发射频率包括国家规定的业余频率
\end{enumerate}
\noindent\textbf{解说:}《中华人民共和国无线电频率划分规定》第2.3款规定:发射电台应符合附录 1 中规定的\textbf{频率容限}。第2.4款规定:业余电台的无线电发射设备应符国家规定的下列主要技术指标:发射电台应符合附录 2 中规定的\textbf{杂散发射或杂散域中无用发射的最大允许功率电平}。\\《业余无线电台管理办法》第十一条规定:业余无线电台专用无线电发射设备不得用于其他无线电业务,其\textbf{发射频率应当在业余业务或者卫星业余业务频段内}。\\要点:记忆关键字“\textbf{不超过}”、“\textbf{不超出}”。\\\noindent\textbf{答案:}A
%%%

\bigskip


\noindent\textbf{试题:}业余无线电台使用的频率应当符合下述规定:
\begin{enumerate}[leftmargin=3em]
\item ITU《无线电规则》第IV节“频率划分表”
\item 一般业余无线电书籍所叙述的频率
\item IARU三区“频率规划”
\item 《中华人民共和国无线电频率划分规定》
\end{enumerate}
\noindent\textbf{解说:}《中华人民共和国无线电管理条例》第四十二条规定:研制无线电发射设备使用的无线电频率,应当符合\textbf{国家无线电频率划分规定}。\\\noindent\textbf{答案:}D


\bigskip


\noindent\textbf{试题:}业余无线电台在业余业务、卫星业余业务作为次要业务使用的频率或者与其他主要业务共同使用的频率上发射操作时,应当注意:
\begin{enumerate}[leftmargin=3em]
\item 首先守听频率是否已由其他业务电台占用,如听不到,即可按照先来先用的原则放心使用
\item 只要遵守了《中华人民共和国无线电频率划分规定》的有关规定即可放心使用
\item 遵守无线电管理机构对该频率的使用规定
\item 可以任意使用,但在遇到其他业务电台使用时要主动避让
\end{enumerate}
\noindent\textbf{解说:}《业余无线电台管理办法》第十九条规定:业余业务、卫星业余业务作为次要业务使用频率或者与其他主要业务共同使用频率的,应当\textbf{遵守无线电管理机构对该频率的使用规定}。\\\noindent\textbf{答案:}C



\bigskip


\noindent\textbf{试题:}关于业余频率的使用,正确的叙述是:
\begin{enumerate}[leftmargin=3em]
\item 任何业余无线电台在任何频段都享有平等的频率使用权
\item 业余无线电台在无线电管理机构核准其使用的频段内,享有平等的频率使用权
\item 依法成立的地方业余无线电民间组织的业余电台,在其常用的台网频率上享有比其他个人设置的业余电台优先的使用权
\item 业余无线电台在无线电管理机构核准其使用的频段内,不同类别的业余电台享有不同优先程度的频率使用权
\end{enumerate}
\noindent\textbf{解说:}《业余无线电台管理办法》第十九条规定:\textbf{业余无线电台在无线电管理机构核准其使用的频段内,享有平等的频率使用权}。\\\noindent\textbf{答案:}B



\bigskip


\noindent\textbf{试题:}在无线电管理中,由国家将某个特定的频带列入频率划分表,规定该频带可在指定的条件下供业余业务或者卫星业余业务使用,这个过程称为:
\begin{enumerate}[leftmargin=3em]
\item 分配
\item 划分
\item 指配
\item 授权
\end{enumerate}
\noindent\textbf{解说:}根据《中华人民共和国无线电频率划分规定》的相关定义:\\将某个特定的频段列入频率划分表,规定该频段可在指定的条件下供一种或多种地面或空间无线电通信业务或射电天文业务使用的过程称为\textbf{划分}。\\将无线电频率或频道规定由一个或多个部门,在指定的区域内供地面或空间无线电通信业务在指定条件下使用的过程称为\textbf{分配}。\\将无线电频率或频道批准给无线电台在规定条件下使用的过程称为\textbf{指配}。\\
\textbf{答案:}B


\bigskip


\noindent\textbf{试题:}在无线电管理中,将无线电频率或频道规定由一个或多个部门,在指定的区域内供地面或空间无线电通信业务在指定条件下使用,这个过程称为:
\begin{enumerate}[leftmargin=3em]
\item 划分
\item 指配
\item 分配
\item 授权
\end{enumerate}
\noindent\textbf{解说:}根据《中华人民共和国无线电频率划分规定》的相关定义:\\将无线电频率或频道规定由一个或多个部门,在指定的区域内供地面或空间无线电通信业务在指定条件下使用的过程称为\textbf{分配}。\\将某个特定的频段列入频率划分表,规定该频段可在指定的条件下供一种或多种地面或空间无线电通信业务或射电天文业务使用的过程称为\textbf{划分}。\\将无线电频率或频道批准给无线电台在规定条件下使用的过程称为\textbf{指配}。\\
\textbf{答案:}C


\bigskip


\noindent\textbf{试题:}在无线电管理中,将无线电频率或频道批准给具体的业余无线电台在规定条件下使用,这个过程称为:
\begin{enumerate}[leftmargin=3em]
\item 分配
\item 授权
\item 划分
\item 指配
\end{enumerate}
\noindent\textbf{解说:}根据《中华人民共和国无线电频率划分规定》的相关定义:\\将无线电频率或频道批准给无线电台在规定条件下使用的过程称为\textbf{指配}。\\将某个特定的频段列入频率划分表,规定该频段可在指定的条件下供一种或多种地面或空间无线电通信业务或射电天文业务使用的过程称为\textbf{划分}。\\将无线电频率或频道规定由一个或多个部门,在指定的区域内供地面或空间无线电通信业务在指定条件下使用的过程称为\textbf{分配}。\\
\textbf{答案:}D



\bigskip


\noindent\textbf{试题:}必要带宽(necessary bandwidth)是指:对给定的发射类别而言,其恰好足以保证在相应速率及在指定条件下具有所要求质量的信息传输的所需带宽。业余电台单边带话音通信SSB、低速莫尔斯电码通信CW、调频话音通信FM和残余边带业余电视VSB ATV的必要带宽分别是:
\begin{enumerate}[leftmargin=3em]
\item 3000Hz、400Hz、5\unit{\MHz}以上、12.5\unit{\kHz}
\item 5\unit{\MHz}、3000Hz、400Hz、12.5\unit{\kHz}
\item 3000Hz、400Hz、12.5\unit{\kHz}、5\unit{\MHz}以上 %原文:5MH以上
\item 12.5\unit{\kHz}、5\unit{\MHz}以上、400Hz、2700Hz
\end{enumerate}
\noindent\textbf{解说:}业余电台单边带话音通信SSB、低速莫尔斯电码通信CW、调频话音通信FM和残余边带业余电视VSB ATV的必要带宽分别是\textbf{3000Hz、400Hz、12.5\unit{\kHz}、5\unit{\MHz}以上}。\\\noindent\textbf{答案:}C
%%%???


\bigskip


\noindent\textbf{试题:}在频率划分表中,一个频带被标明划分给多种业务时,这些业务被分为下述类别:
\begin{enumerate}[leftmargin=3em]
\item 民用业务和军用业务
\item 安全业务和一般业务
\item 主要业务和次要业务
\item 业余业务和非业余业务
\end{enumerate}
\noindent\textbf{解说:}《中华人民共和国无线电频率划分规定》第3.2.1条:在本频率划分表中,一个频段在世界范围或区域范围内被标明划分给多种业务时,这些业务按下列顺序排列:
\begin{enumerate}[leftmargin=3em, label=\alph*)]
\item \textbf{主要业务};
\item \textbf{次要业务}。
\end{enumerate}
\textbf{答案:}C



\bigskip


\noindent\textbf{试题:}在频率划分表中,当一个频段划分给业余业务或卫星业余业务和多个其他业务,并且业余业务和卫星业余业务作为次要业务时,业余无线电台应该遵循的规则是:
\begin{enumerate}[leftmargin=3em]
\item 不得对主要业务电台产生有害干扰
\item 不得对来自同一业务或其他次要业务电台的有害干扰提出保护要求
\item 可要求保护不受来自主要业务电台的有害干扰
\item 容许因设备技术问题对主要业务电台产生短时间有害干扰
\end{enumerate}
\noindent\textbf{解说:}《中华人民共和国无线电频率划分规定》第3.2.1条3(a)款规定:\textbf{不得对业经指配或将来可能指配频率的主要业务电台产生有害干扰}。\\
\textbf{答案:}A


\bigskip


\noindent\textbf{试题:}在频率划分表中,当一个频段划分给业余业务或卫星业余业务和多个其他业务,并且业余业务和卫星业余业务作为次要业务时,业余无线电台遵循的规则是:
\begin{enumerate}[leftmargin=3em]
\item 不得对来自同一业务或其他次要业务电台的有害干扰提出保护要求
\item 不得对来自主要业务电台的有害干扰提出保护要求
\item 可要求保护不受来自主要业务电台的有害干扰
\item 容许因设备技术问题对主要业务电台产生短时间有害干扰
\end{enumerate}
\noindent\textbf{解说:}《中华人民共和国无线电频率划分规定》第3.2.1条3(b)款规定:\textbf{不得对来自业经指配或将来可能指配频率的主要业务电台的有害干扰提出保护要求}。\\\textbf{答案:}B



\bigskip


\noindent\textbf{试题:}在频率划分表中,当一个频段划分给业余业务或卫星业余业务和多个其他业务,并且业余业务和卫星业余业务作为次要业务时,业余无线电台遵循的规则是:
\begin{enumerate}[leftmargin=3em]
\item 可要求保护不受来自主要业务电台的有害干扰
\item 不得对来自同一业务或其他次要业务电台的有害干扰提出保护要求
\item 容许因设备技术问题对主要业务电台产生短时间有害干扰
\item 可要求保护不受来自同一业务或其他次要业务电台的有害干扰
\end{enumerate}
\noindent\textbf{解说:}《中华人民共和国无线电频率划分规定》第3.2.1条3(c)款规定:\textbf{可要求保护不受来自将来可能指配频率的同一业务或其他次要业务电台的有害干扰}。\\\textbf{答案:}D



\bigskip


\noindent\textbf{试题:}分配给业余业务的某频段的频率下限为\(\textrm{F1}\),业余电台实际可以工作的发信频率应为:
\begin{enumerate}[leftmargin=3em]
\item \(\textrm{F1} + \textrm{信号下边带的频率宽度}\)
\item \(\textrm{F1} - \textrm{信号下边带的频率宽度}\)
\item \(\textrm{F1} - 2 \times \textrm{信号下边带的频率宽度}\)
\item \(\textrm{F1}\)
\end{enumerate}
\noindent\textbf{解说:}业余电台实际可以工作的发信频率应为\(\textbf{F1} + \textbf{信号下边带的频率宽度}\)。
\\\noindent\textbf{答案:}A





\bigskip


\noindent\textbf{试题:}分配给业余业务的某频段的频率上限为\(\textrm{F2}\),业余电台实际可以工作的发信频率应为:
\begin{enumerate}[leftmargin=3em]
\item \(\textrm{F2} + \textrm{信号上边带的频率宽度}\)
\item \(\textrm{F2}\)
\item \(\textrm{F2} - \textrm{信号上边带的频率宽度}\)
\item \(\textrm{F2} - 2 \times \textrm{信号上边带的频率宽度}\)
\end{enumerate}
\noindent\textbf{解说:}业余电台实际可以工作的发信频率应为\(\textbf{F2} - \textbf{信号上边带的频率宽度}\)。
\\\noindent\textbf{答案:}C





\bigskip


\noindent\textbf{试题:}为了满足我国《无线电频率划分规定》“电台的技术特性”关于无线电通信“把带宽保持在技术状态和该项业务的性质所允许的最低值上”的要求,业余电台操作者应了解各种通信方式的必要带宽。决定必要带宽的因素是:
\begin{enumerate}[leftmargin=3em]
\item 所要传输的信息速率越高、整个通信系统的噪声干扰越大,必要带宽越宽
\item 发射设备的功率越大,必要带宽越宽
\item 通信距离越近,必要带宽越宽
\item 接收设备的灵敏度越高,必要带宽越宽
\end{enumerate}
\noindent\textbf{解说:}决定必要带宽的因素是:\textbf{所要传输的信息速率越高、整个通信系统的噪声干扰越大,必要带宽越宽}。\\\noindent\textbf{答案:}A



\bigskip


\noindent\textbf{试题:}我国分配给业余业务和卫星业余业务专用的频段有:
\begin{enumerate}[leftmargin=3em]
\item 3.5\unit{\MHz}、14\unit{\MHz}、21\unit{\MHz}、28\unit{\MHz}、10\unit{\GHz}频段
\item 7\unit{\MHz}、14\unit{\MHz}、21\unit{\MHz}、28\unit{\MHz}、47\unit{\GHz}频段
\item 7\unit{\MHz}、14\unit{\MHz}、21\unit{\MHz}、28\unit{\MHz}、144\unit{\MHz}频段
\item 7\unit{\MHz}、14\unit{\MHz}、28\unit{\MHz}、144\unit{\MHz}、430\unit{\MHz}频段
\end{enumerate}
\noindent\textbf{解说:}《中华人民共和国无线电频率划分规定》规定:分配给业余业务和卫星业余业务专用的频段有\textbf{7\unit{\MHz}、14\unit{\MHz}、21\unit{\MHz}、28\unit{\MHz}、47\unit{\GHz}频段}。\\\noindent\textbf{答案:}B




\bigskip


\noindent\textbf{试题:}我国分配给业余业务和卫星业余业务与其他业务共用、并且业余业务和卫星业余业务作为主要业务之一的30\unit{\MHz}以下频段有:
\begin{enumerate}[leftmargin=3em]
\item 1.8\unit{\MHz}、3.5\unit{\MHz}、14.25\unit{\MHz}、18.068\unit{\MHz}、24.89\unit{\MHz}频段
\item 3.5\unit{\MHz}、7\unit{\MHz}、14.25\unit{\MHz}、21\unit{\MHz}、24.89\unit{\MHz}频段
\item 1.8\unit{\MHz}、10.1\unit{\MHz}、14.25\unit{\MHz}、18.068\unit{\MHz}、21.45\unit{\MHz}频段
\item 3.5\unit{\MHz}、10.1\unit{\MHz}、14.25\unit{\MHz}、18.068\unit{\MHz}、29.7\unit{\MHz}频段
\end{enumerate}
\noindent\textbf{解说:}我国分配给业余业务和卫星业余业务与其他业务共用、并且业余业务和卫星业余业务作为主要业务之一的30\unit{\MHz}以下频段有\textbf{1.8\unit{\MHz}、3.5\unit{\MHz}、14.25\unit{\MHz}、18.068\unit{\MHz}、24.89\unit{\MHz}频段}。\\\noindent\textbf{答案:}A
%%%???

\bigskip


\noindent\textbf{试题:}我国在VHF和UHF范围内分配给业余业务和卫星业余业务与其他业务共用并设业务类别为主要业务与次要业务,以下那些频率分配给业余业务和卫星业余业务与其他业务共用并且业余业务和卫星业余业务作为主要业务:
\begin{enumerate}[leftmargin=3em]
\item 50\unit{\MHz}、430\unit{\MHz}
\item 50\unit{\MHz}、144\unit{\MHz}
\item 220\unit{\MHz}、430\unit{\MHz}
\item 144\unit{\MHz}、430\unit{\MHz}
\end{enumerate}
\noindent\textbf{解说:}《中华人民共和国无线电频率划分规定》规定,频率\textbf{50\unit{\MHz}、144\unit{\MHz}}分配给业余业务和卫星业余业务与其他业务共用并且业余业务和卫星业余业务作为主要业务。\\\noindent\textbf{答案:}B

%%%???

\bigskip


\noindent\textbf{试题:}我国分配给业余业务和卫星业余业务与其他业务共用、并且业余业务和卫星业余业务作为唯一主要业务的频段的个数以及在3\unit{\GHz}以下的该类频段分别为:
\begin{enumerate}[leftmargin=3em]
\item 4个,7.0-7.2\unit{\MHz}
\item 5个,50-54\unit{\MHz}
\item 5个,28-29.7\unit{\MHz}
\item 3个,144-146\unit{\MHz}
\end{enumerate}
\noindent\textbf{解说:}《中华人民共和国无线电频率划分规定》规定:分配给业余业务和卫星业余业务与其他业务共用、并且业余业务和卫星业余业务作为唯一主要业务的频段的个数以及在3\unit{\GHz}以下的该类频段分别为:\textbf{3个,144-146\unit{\MHz}}。\\\noindent\textbf{答案:}D



\bigskip


\noindent\textbf{试题:}我国分配给业余业务和卫星业余业务与其他业务共用、并且业余业务和卫星业余业务作为次要业务的1200\unit{\MHz}以下频段有:
\begin{enumerate}[leftmargin=3em]
\item 10.1\unit{\MHz}、24.89\unit{\MHz}、430\unit{\MHz}
\item 135.7\unit{\kHz}、10.1\unit{\MHz}、430\unit{\MHz}
\item 3.5\unit{\MHz}、18.068\unit{\MHz}、144\unit{\MHz}
\item 3.5\unit{\MHz}、7\unit{\MHz}、50\unit{\MHz}
\end{enumerate}
\noindent\textbf{解说:}《中华人民共和国无线电频率划分规定》规定:分配给业余业务和卫星业余业务与其他业务共用、并且业余业务和卫星业余业务作为次要业务的1200\unit{\MHz}以下频段有:\textbf{135.7\unit{\kHz}、10.1\unit{\MHz}、430\unit{\MHz}}。\\\noindent\textbf{答案:}B




\bigskip


\noindent\textbf{试题:}俗称的6米业余波段的频率范围以及业余业务和卫星业余业务的使用状态分别为:
\begin{enumerate}[leftmargin=3em]
\item 51-54\unit{\MHz},专用
\item 50-54\unit{\MHz},主要业务
\item 50-52\unit{\MHz},次要业务
\item 52-56\unit{\MHz},次要业务
\end{enumerate}
\noindent\textbf{解说:}俗称的6米业余波段的频率范围是\textbf{50-54\unit{\MHz}}。根据《中华人民共和国无线电频率划分规定》,在该波段中业余业务和卫星业余业务的使用状态为\textbf{主要业务}。\\\noindent\textbf{答案:}B



\bigskip


\noindent\textbf{试题:}俗称的2米业余波段的频率范围以及我国业余业务和卫星业余业务的使用状态分别为:
\begin{enumerate}[leftmargin=3em]
\item 144-148\unit{\MHz};其中144-146\unit{\MHz}为唯一主要业务,146-148\unit{\MHz}为与其他业务共同作为主要业务
\item 144-148\unit{\MHz};其中144-146\unit{\MHz}为专用,146-148\unit{\MHz}为次要业务
\item 144-148\unit{\MHz};次要业务
\item 144-146\unit{\MHz};专用
\end{enumerate}
\noindent\textbf{解说:}俗称的2米业余波段的频率范围是\textbf{144-148\unit{\MHz}}。根据《中华人民共和国无线电频率划分规定》,在该波段中业余业务和卫星业余业务的使用状态为:\textbf{其中144-146\unit{\MHz}为唯一主要业务,146-148\unit{\MHz}为与其他业务共同作为主要业务}。\\\noindent\textbf{答案:}A



\bigskip


\noindent\textbf{试题:}俗称的0.7米业余波段的频率范围以及业余业务和卫星业余业务的使用状态分别为:
\begin{enumerate}[leftmargin=3em]
\item 430-440\unit{\MHz},主要业务
\item 420-470\unit{\MHz},次要业务
\item 430-440\unit{\MHz},次要业务
\item 430-440\unit{\MHz},专用
\end{enumerate}
\noindent\textbf{解说:}俗称的0.7米业余波段的频率范围是\textbf{430-440\unit{\MHz}}。《中华人民共和国无线电频率划分规定》规定,在该波段中业余业务和卫星业余业务的使用状态为\textbf{次要业务}。\\\noindent\textbf{答案:}C



\bigskip


\noindent\textbf{试题:}在我国和多数其他国家的频率分配中,业余业务在430-440\unit{\MHz}频段中作为次要业务与其他业务共用。这个频段中我国分配的主要业务是:
\begin{enumerate}[leftmargin=3em]
\item 移动业务
\item 水上移动和航空移动
\item 固定业务
\item 无线电定位和航空无线电导航
\end{enumerate}
\noindent\textbf{解说:}《中华人民共和国无线电频率划分规定》规定,在430-440\unit{\MHz}频段中我国分配的主要业务是\textbf{无线电定位和航空无线电导航}。\\\noindent\textbf{答案:}D



\bigskip


\noindent\textbf{试题:}VHF业余无线电台在144\unit{\MHz}频段进行本地联络时应避免占用的频率为:
\begin{enumerate}[leftmargin=3em]
\item 144.035-145.8\unit{\MHz}
\item 144.050-144.053\unit{\MHz}和145.100-145.750\unit{\MHz}
\item 144.035-144.053\unit{\MHz}和145.550-145.750\unit{\MHz}
\item 144-144.035\unit{\MHz}和145.8-146\unit{\MHz}
\end{enumerate}
\noindent\textbf{解说:}《中华人民共和国无线电频率划分规定》规定,VHF业余无线电台在144\unit{\MHz}频段进行本地联络时应避免占用的频率为:\textbf{144-144.035\unit{\MHz}和145.8-146\unit{\MHz}}。\\\noindent\textbf{答案:}D



\bigskip


\noindent\textbf{试题:}UHF业余无线电台在430\unit{\MHz}频段进行本地联络时应避免占用的频率为:
\begin{enumerate}[leftmargin=3em]
\item 431.9-432.240\unit{\MHz}和435-438\unit{\MHz}
\item 431-432\unit{\MHz}和438-440\unit{\MHz}
\item 430-431.9\unit{\MHz}和432.240-435\unit{\MHz}
\item 430-431.2\unit{\MHz}和435-436\unit{\MHz}
\end{enumerate}
\noindent\textbf{解说:}《中华人民共和国无线电频率划分规定》规定,UHF业余无线电台在430\unit{\MHz}频段进行本地联络时应避免占用的频率为:\textbf{431.9-432.240\unit{\MHz}和435-438\unit{\MHz}}。\\\noindent\textbf{答案:}A



\bigskip


\noindent\textbf{试题:}430\unit{\MHz}业余频段中留给业余卫星通信使用,话音及其他通信方式不应占用的频率段为:
\begin{enumerate}[leftmargin=3em]
\item 438\unit{\MHz}至439\unit{\MHz}
\item 435\unit{\MHz}至438\unit{\MHz}
\item 432\unit{\MHz}至434\unit{\MHz}
\item 433\unit{\MHz}至435\unit{\MHz}
\end{enumerate}
\noindent\textbf{解说:}《中华人民共和国无线电频率划分规定》,430\unit{\MHz}业余频段中留给业余卫星通信使用,话音及其他通信方式不应占用的频率段为:\textbf{435\unit{\MHz}至438\unit{\MHz}}。\\\noindent\textbf{答案:}B



\bigskip


\noindent\textbf{试题:}144\unit{\MHz}业余频段中留给业余卫星通信使用,话音及其他通信方式不应占用的频率段为:
\begin{enumerate}[leftmargin=3em]
\item 145.4\unit{\MHz}至144.6\unit{\MHz}
\item 145.8\unit{\MHz}至146\unit{\MHz}
\item 144.8\unit{\MHz}至145\unit{\MHz}
\item 144.2\unit{\MHz}至144.5\unit{\MHz}
\end{enumerate}
\noindent\textbf{解说:}《中华人民共和国无线电频率划分规定》,144\unit{\MHz}业余频段中留给业余卫星通信使用,话音及其他通信方式不应占用的频率段为:\textbf{145.8\unit{\MHz}至146\unit{\MHz}}。\\\noindent\textbf{答案:}B


\bigskip


\noindent\textbf{试题:}我国无线电频率划分表划分给业余业务使用的最接近无线宽带WiFi频率的频带为2,300-2,450\unit{\MHz},属于无线电频谱的下列频带(波段):
\begin{enumerate}[leftmargin=3em]
\item 超高频(厘米波)
\item 甚高频(米波)
\item 高频(短波)
\item 特高频(分米波)
\end{enumerate}
\noindent\textbf{解说:}2,300-2,450\unit{\MHz}属于无线电频谱的\textbf{特高频(分米波)}频带(波段)。\\\noindent\textbf{答案:}D

%%%???

\bigskip


\noindent\textbf{试题:}我国无线电频率划分表划分给业余业务使用的最接近无线宽带WiFi频率的频带为2,300-2,450\unit{\MHz},属于无线电频谱的下列频带(波段):
\begin{enumerate}[leftmargin=3em]
\item UHF
\item HF
\item VHF
\item SHF
\end{enumerate}
\noindent\textbf{解说:}2,300-2,450\unit{\MHz}属于无线电频谱的\textbf{UHF}频带(波段)。\\\noindent\textbf{答案:}A

%%%???

\bigskip


\noindent\textbf{试题:}我国无线电频率划分表划分给业余业务使用的最接近C波段卫星电视广播频率的频带是5.650-5.850\unit{\GHz},属于无线电频谱的下列频带(波段):
\begin{enumerate}[leftmargin=3em]
\item 极高频(毫米波)
\item 超高频(厘米波)
\item 甚高频(米波)
\item 特高频(分米波)
\end{enumerate}
\noindent\textbf{解说:}5.650-5.850\unit{\GHz}属于无线电频谱的\textbf{超高频(厘米波)}频带(波段)。\\\noindent\textbf{答案:}B

%%%???

\bigskip


\noindent\textbf{试题:}我国无线电频率划分表划分给业余业务使用的最接近C波段卫星电视广播频率的频带是5.650-5.850\unit{\GHz},属于无线电频谱的下列频带(波段):
\begin{enumerate}[leftmargin=3em]
\item EHF
\item SHF
\item UHF
\item VHF
\end{enumerate}
\noindent\textbf{解说:}5.650-5.850\unit{\GHz}属于无线电频谱的\textbf{SHF}频带(波段)。\\\noindent\textbf{答案:}B

%%%???

\bigskip


\noindent\textbf{试题:}我国无线电频率划分表划分给业余业务使用的最接近Ku波段卫星电视广播频率的频带10-10.5\unit{\GHz},属于无线电频谱的下列频带(波段):
\begin{enumerate}[leftmargin=3em]
\item 极高频(毫米波)
\item 超高频(厘米波)
\item 特高频(分米波)
\item 甚高频(米波)
\end{enumerate}
\noindent\textbf{解说:}10-10.5\unit{\GHz}属于无线电频谱的\textbf{超高频(厘米波)}频带(波段)。\\\noindent\textbf{答案:}B

%%%???

\bigskip


\noindent\textbf{试题:}我国无线电频率划分表划分给业余业务使用的最接近Ku波段卫星电视广播频率的频带10-10.5\unit{\GHz},属于无线电频谱的下列频带(波段):
\begin{enumerate}[leftmargin=3em]
\item UHF
\item EHF
\item VHF
\item SHF
\end{enumerate}
\noindent\textbf{解说:}10-10.5\unit{\GHz}属于无线电频谱的\textbf{SHF}频带(波段)。\\\noindent\textbf{答案:}D

%%%???

\bigskip


\noindent\textbf{试题:}我国无线电频率划分表划分给业余业务使用的最高频带为241\unit{\GHz}-250\unit{\GHz},属于无线电频谱的下列频带(波段):
\begin{enumerate}[leftmargin=3em]
\item 极高频(毫米波)
\item 特高频(分米波)
\item 至高频(丝米波或亚毫米波)
\item 超高频(厘米波)
\end{enumerate}
\noindent\textbf{解说:}241\unit{\GHz}-250\unit{\GHz}属于无线电频谱的\textbf{极高频(毫米波)}频带(波段)。\\\noindent\textbf{答案:}A




\bigskip


\noindent\textbf{试题:}我国无线电频率划分表划分给业余业务使用的最高频带为241\unit{\GHz}-250\unit{\GHz},属于无线电频谱的下列频带(波段):
\begin{enumerate}[leftmargin=3em]
\item UHF
\item SHF
\item EHF
\item THF
\end{enumerate}
\noindent\textbf{解说:}241\unit{\GHz}-250\unit{\GHz}属于无线电频谱的\textbf{EHF}频带(波段)。\\\noindent\textbf{答案:}C




\bigskip


\noindent\textbf{试题:}我国无线电频率划分表划分给业余业务使用的最低频带为135.7-137.8 \unit{\kHz},属于无线电频谱的下列频带(波段):
\begin{enumerate}[leftmargin=3em]
\item 特低频(特长波)
\item 超低频(超长波)
\item 甚低频(甚长波)
\item 低频(长波)
\end{enumerate}
\noindent\textbf{解说:}135.7-137.8 \unit{\kHz}属于无线电频谱的\textbf{低频(长波)}频带(波段)。\\\noindent\textbf{答案:}D



\bigskip


\noindent\textbf{试题:}我国无线电频率划分表划分给业余业务使用的最低频带为135.7-137.8 \unit{\kHz},属于无线电频谱的下列频带(波段):
\begin{enumerate}[leftmargin=3em]
\item ULF
\item LF
\item VLF
\item SLF
\end{enumerate}
\noindent\textbf{解说:}135.7-137.8 \unit{\kHz}属于无线电频谱的\textbf{LF}频带(波段)。\\\noindent\textbf{答案:}B




\bigskip


\noindent\textbf{试题:}我国无线电频率划分表划分给业余业务使用的1,800\unit{\kHz}-2,000\unit{\kHz}属于无线电频谱的下列频带(波段):
\begin{enumerate}[leftmargin=3em]
\item 高频(短波)
\item 低频(长波)
\item 中频(中波)
\item 甚高频(米波)
\end{enumerate}
\noindent\textbf{解说:}1,800\unit{\kHz}-2,000\unit{\kHz}属于无线电频谱的\textbf{中频(中波)}频带(波段)。\\\noindent\textbf{答案:}C




\bigskip


\noindent\textbf{试题:}我国无线电频率划分表划分给业余业务使用的1,800\unit{\kHz}-2,000\unit{\kHz}属于无线电频谱的下列频带(波段):
\begin{enumerate}[leftmargin=3em]
\item VHF
\item MF
\item LF
\item HF
\end{enumerate}
\noindent\textbf{解说:}1,800\unit{\kHz}-2,000\unit{\kHz}属于无线电频谱的\textbf{MF}频带(波段)。\\\noindent\textbf{答案:}B



\bigskip


\noindent\textbf{试题:}我国无线电频率划分表划分给业余业务使用的28\unit{\MHz}-29.7\unit{\MHz}属于无线电频谱的下列频带(波段):
\begin{enumerate}[leftmargin=3em]
\item 甚高频(米波)
\item 低频(长波)
\item 高频(短波)
\item 中频(中波)
\end{enumerate}
\noindent\textbf{解说:}28\unit{\MHz}-29.7\unit{\MHz}属于无线电频谱的\textbf{高频(短波)}频带(波段)。\\\noindent\textbf{答案:}C




\bigskip


\noindent\textbf{试题:}我国无线电频率划分表划分给业余业务使用的28\unit{\MHz}-29.7\unit{\MHz}属于无线电频谱的下列频带(波段):
\begin{enumerate}[leftmargin=3em]
\item MF
\item HF
\item VHF
\item LF
\end{enumerate}
\noindent\textbf{解说:}28\unit{\MHz}-29.7\unit{\MHz}属于无线电频谱的\textbf{HF}频带(波段)。\\\noindent\textbf{答案:}B



\bigskip


\noindent\textbf{试题:}我国无线电频率划分表划分给业余业务使用的50\unit{\MHz}-54\unit{\MHz}属于无线电频谱的下列频带(波段):
\begin{enumerate}[leftmargin=3em]
\item 超高频(厘米波)
\item 甚高频(米波)
\item 高频(短波)
\item 特高频(分米波)
\end{enumerate}
\noindent\textbf{解说:}50\unit{\MHz}-54\unit{\MHz}属于无线电频谱的\textbf{甚高频(米波)}频带(波段)。\\\noindent\textbf{答案:}B



\bigskip


\noindent\textbf{试题:}我国无线电频率划分表划分给业余业务使用的50\unit{\MHz}-54\unit{\MHz}属于无线电频谱的下列频带(波段):
\begin{enumerate}[leftmargin=3em]
\item VHF
\item UHF
\item SHF
\item HF
\end{enumerate}
\noindent\textbf{解说:}50\unit{\MHz}-54\unit{\MHz}属于无线电频谱的\textbf{VHF}频带(波段)。\\\noindent\textbf{答案:}A



\bigskip


\noindent\textbf{试题:}国际业余无线电界把WARC-76增加分配给业余业务和卫星业余业务的三个HF频段俗称为WARC频段,它们的频率范围是:
\begin{enumerate}[leftmargin=3em]
\item 10.1-10.15\unit{\MHz}、18.068-18.168\unit{\MHz}、24.89-24.99\unit{\MHz}
\item 10.89-10.88\unit{\MHz}、18.1-18.15\unit{\MHz}、24.068-24.168\unit{\MHz}
\item 10.1-10.15\unit{\MHz}、18.89-18.99\unit{\MHz}、24.068-24.168\unit{\MHz}
\item 10.068-10.168\unit{\MHz}、18.1-18.15\unit{\MHz}、24.89-24.99\unit{\MHz}
\end{enumerate}
\noindent\textbf{解说:}三个WARC频段的频率范围分别是\textbf{10.100–10.150\unit{\MHz}、18.068–18.168\unit{\MHz}、24.890–24.990\unit{\MHz}}。\\\noindent\textbf{答案:}A



\bigskip


\noindent\textbf{试题:}俗称的40米业余波段,其在ITU1、2、3区的频率范围以及业余业务和卫星业余业务的使用状态分别为:
\begin{enumerate}[leftmargin=3em]
\item 7.0-7.1\unit{\MHz}、7.0-7.2\unit{\MHz}、7.0-7.3\unit{\MHz},专用
\item 7.0-7.2\unit{\MHz}、7.0-7.3\unit{\MHz}、7.0-7.2\unit{\MHz},专用
\item 7.0-7.3\unit{\MHz}、7.0-7.3\unit{\MHz}、7.0-7.3\unit{\MHz},专用
\item 7.0-7.3\unit{\MHz}、7.0-7.3\unit{\MHz}、7.0-7.2\unit{\MHz},专用
\end{enumerate}
\noindent\textbf{解说:}俗称的40米业余波段,其在ITU1、2、3区的频率范围以及业余业务和卫星业余业务的使用状态分别为\textbf{7.0-7.2\unit{\MHz}、7.0-7.3\unit{\MHz}、7.0-7.2\unit{\MHz},专用}。\\\noindent\textbf{答案:}B




\bigskip


\noindent\textbf{试题:}俗称的160米业余波段的频率范围以及业余业务和卫星业余业务的使用状态分别为:
\begin{enumerate}[leftmargin=3em]
\item 1900-2000\unit{\kHz},主要业务
\item 1800-2000\unit{\kHz},主要业务
\item 1700-1900\unit{\kHz},专用业务
\item 1800-1900\unit{\kHz},次要业务
\end{enumerate}
\noindent\textbf{解说:}俗称的160米业余波段的频率范围以及业余业务和卫星业余业务的使用状态分别为\textbf{1800-2000\unit{\kHz},主要业务}。\\\noindent\textbf{答案:}B



\bigskip


\noindent\textbf{试题:}俗称的80米业余波段的频率范围以及业余业务和卫星业余业务的使用状态分别为:
\begin{enumerate}[leftmargin=3em]
\item 3.5-4.0\unit{\MHz},主要业务
\item 3.5-3.9\unit{\MHz},次要业务
\item 3.5-3.9\unit{\MHz},主要业务
\item 3.5-3.6\unit{\MHz},专用业务
\end{enumerate}
\noindent\textbf{解说:}俗称的80米业余波段的频率范围以及业余业务和卫星业余业务的使用状态分别为\textbf{3.5-3.9\unit{\MHz},主要业务}。\\\noindent\textbf{答案:}C



\bigskip


\noindent\textbf{试题:}俗称的20米业余波段的频率范围以及业余业务和卫星业余业务的使用状态分别为:
\begin{enumerate}[leftmargin=3em]
\item 14-14.25\unit{\MHz}为专用,14.25-14.35为主要业务
\item 14-14.35\unit{\MHz}为专用,14.35-14.45为主要业务
\item 14-14.15\unit{\MHz}为专用,14.15-14.25为主要业务
\item 14-14.35\unit{\MHz},专用
\end{enumerate}
\noindent\textbf{解说:}俗称的20米业余波段的频率范围以及业余业务和卫星业余业务的使用状态分别为\textbf{14-14.25\unit{\MHz}为专用,14.25-14.35为主要业务}。\\\noindent\textbf{答案:}A



\bigskip


\noindent\textbf{试题:}俗称的15米业余波段的频率范围以及业余业务和卫星业余业务的使用状态分别为:
\begin{enumerate}[leftmargin=3em]
\item 21-21.45\unit{\MHz},主要业务
\item 21-21.35\unit{\MHz},专用
\item 21-21.45\unit{\MHz},次要业务
\item 21-21.45\unit{\MHz},专用
\end{enumerate}
\noindent\textbf{解说:}俗称的15米业余波段的频率范围以及业余业务和卫星业余业务的使用状态分别为\textbf{21-21.45\unit{\MHz},专用}。\\\noindent\textbf{答案:}D



\bigskip


\noindent\textbf{试题:}俗称的10米业余波段的频率范围以及业余业务和卫星业余业务的使用状态分别为:
\begin{enumerate}[leftmargin=3em]
\item 28-29.7\unit{\MHz},专用
\item 28-29.6\unit{\MHz},专用
\item 28-29.7\unit{\MHz},主要业务
\item 28-30\unit{\MHz},次要业务
\end{enumerate}
\noindent\textbf{解说:}俗称的10米业余波段的频率范围以及业余业务和卫星业余业务的使用状态分别为\textbf{28-29.7\unit{\MHz},专用}。\\\noindent\textbf{答案:}A



\bigskip


\noindent\textbf{试题:}短波业余电台应避免在IARU信标工作频率\( \pm 500Hz\)的范围内发射电波。这些频率是:%±
\begin{enumerate}[leftmargin=3em]
\item 7.150\unit{\MHz}、14.110\unit{\MHz}、18.150\unit{\MHz}、21.150\unit{\MHz}、28.150\unit{\MHz}
\item 14.100\unit{\MHz}、18.110\unit{\MHz}、21.150\unit{\MHz}、24.930\unit{\MHz}、28.200\unit{\MHz}
\item 14.150\unit{\MHz}、18.100\unit{\MHz}、21.200\unit{\MHz}、24.930\unit{\MHz}、28.200\unit{\MHz}
\item 7.100\unit{\MHz}、10.070\unit{\MHz}、14.100\unit{\MHz}、21.100\unit{\MHz}、28.200\unit{\MHz}
\end{enumerate}
\noindent\textbf{解说:}短波业余电台应避免在IARU信标工作频率\( \pm 500Hz\)的范围内发射电波。这些频率是\textbf{14.100\unit{\MHz}、18.110\unit{\MHz}、21.150\unit{\MHz}、24.930\unit{\MHz}、28.200\unit{\MHz}}。\\\noindent\textbf{答案:}B%±



\bigskip


\noindent\textbf{试题:}不能用于通话的HF业余频段为:
\begin{enumerate}[leftmargin=3em]
\item 10\unit{\MHz}业余频段
\item 14\unit{\MHz}业余频段
\item 1.8\unit{\MHz}业余频段
\item 18\unit{\MHz}业余频段
\end{enumerate}
\noindent\textbf{解说:}不能用于通话的HF业余频段为\textbf{10\unit{\MHz}业余频段}。\\\noindent\textbf{答案:}A



\bigskip


\noindent\textbf{试题:}我国短波业余电台在7\unit{\MHz}频段进行LSB通话时可以实际占用的频率为:
\begin{enumerate}[leftmargin=3em]
\item 7.000-7.200\unit{\MHz}
\item 7.030-7.200\unit{\MHz}
\item 7.023-7.200\unit{\MHz}
\item 7.000-7.100\unit{\MHz}
\end{enumerate}
\noindent\textbf{解说:}我国短波业余电台在7\unit{\MHz}频段进行LSB通话时可以实际占用的频率为\textbf{7.030-7.200\unit{\MHz}}。\\\noindent\textbf{答案:}B



\bigskip


\noindent\textbf{试题:}短波业余电台在14\unit{\MHz}频段进行USB通话时可以实际占用的频率为:
\begin{enumerate}[leftmargin=3em]
\item 14.100-14.350\unit{\MHz}
\item 14.070-14.250\unit{\MHz}
\item 14.030-14.350\unit{\MHz}
\item 14.000-14.250\unit{\MHz}
\end{enumerate}
\noindent\textbf{解说:}短波业余电台在14\unit{\MHz}频段进行USB通话时可以实际占用的频率为\textbf{14.100-14.350\unit{\MHz}}。\\\noindent\textbf{答案:}A



\bigskip


\noindent\textbf{试题:}短波业余电台在18\unit{\MHz}频段进行USB通话时可以实际占用的频率为:
\begin{enumerate}[leftmargin=3em]
\item 18.1105-18.168\unit{\MHz}
\item 18.110-18.170\unit{\MHz}
\item 18.1005-18.180\unit{\MHz}
\item 18.068-18.186\unit{\MHz}
\end{enumerate}
\noindent\textbf{解说:}短波业余电台在18\unit{\MHz}频段进行USB通话时可以实际占用的频率为\textbf{18.1105-18.168\unit{\MHz}}。\\\noindent\textbf{答案:}A



\bigskip


\noindent\textbf{试题:}短波业余电台在21\unit{\MHz}频段进行USB通话时可以实际占用的频率为:
\begin{enumerate}[leftmargin=3em]
\item 21.125-21.45\unit{\MHz},除去21.1495-21.1505
\item 21.125-21.45\unit{\MHz}
\item 21-21.45\unit{\MHz}
\item 21-21.35\unit{\MHz}
\end{enumerate}
\noindent\textbf{解说:}短波业余电台在21\unit{\MHz}频段进行USB通话时可以实际占用的频率为\textbf{21.125-21.45\unit{\MHz},除去21.1495-21.1505}。\\\noindent\textbf{答案:}A




\bigskip


\noindent\textbf{试题:}短波业余电台在24\unit{\MHz}频段进行USB通话时可以实际占用的频率为:
\begin{enumerate}[leftmargin=3em]
\item 24.9205-24.99\unit{\MHz}
\item 24.890-24.98\unit{\MHz}
\item 24.9305-24.99\unit{\MHz}
\item 24.928-24.988\unit{\MHz}
\end{enumerate}
\noindent\textbf{解说:}短波业余电台在24\unit{\MHz}频段进行USB通话时可以实际占用的频率为\textbf{24.9305-24.99\unit{\MHz}}。\\\noindent\textbf{答案:}C




\bigskip


\noindent\textbf{试题:}短波业余电台在29\unit{\MHz}频段进行USB通话时可以实际占用的频率为:
\begin{enumerate}[leftmargin=3em]
\item 28.250-29.7\unit{\MHz}
\item 28.3-29.3\unit{\MHz}
\item 28.2-29.6\unit{\MHz}
\item 28-29.7\unit{\MHz}
\end{enumerate}
\noindent\textbf{解说:}短波业余电台在29\unit{\MHz}频段进行USB通话时可以实际占用的频率为\textbf{28.3-29.3\unit{\MHz}}。\\\noindent\textbf{答案:}B



\bigskip


\noindent\textbf{试题:}短波业余电台在29\unit{\MHz}频段进行FM通话时可以实际占用的频率为:
\begin{enumerate}[leftmargin=3em]
\item 29.51-29.7\unit{\MHz}
\item 28.3-29.510\unit{\MHz}%28.3-29.510\unit{\MHz} z
\item 29.3-29.7\unit{\MHz}
\item 28-29.7\unit{\MHz}
\end{enumerate}
\noindent\textbf{解说:}短波业余电台在29\unit{\MHz}频段进行FM通话时可以实际占用的频率为\textbf{29.51-29.7\unit{\MHz}}。\\\noindent\textbf{答案:}A



\bigskip


\noindent\textbf{试题:}为什么不能在低于1.2\unit{\GHz}的业余频段进行常规的ATV通信?
\begin{enumerate}[leftmargin=3em]
\item 多数话音通信集中在较低频段,易对ATV通信产生严重干扰
\item 多数话音通信集中在较低频段,而ATV画面发射时间通常较长,所以要主动避让
\item ATV通信需占5\unit{\MHz}以上带宽,较低业余频段不足以容纳
\item 频率越高,ATV传输画面的质量越稳定
\end{enumerate}
\noindent\textbf{解说:}因为\textbf{ATV通信需占5\unit{\MHz}左右带宽,较低业余频段不足以容纳}。\\\noindent\textbf{答案:}C




\bigskip


\noindent\textbf{试题:}按照有关规定,144\unit{\MHz}和430\unit{\MHz}频段业余中继台的上下行频差应分别为:
\begin{enumerate}[leftmargin=3em]
\item 0.6\unit{\MHz}、5\unit{\MHz}%0.6\unit{\MHz}、5\unit{\MHz}。
\item 5\unit{\MHz}、5\unit{\MHz}
\item 7.5\unit{\MHz}、10\unit{\MHz}
\item 2\unit{\MHz}、10\unit{\MHz}
\end{enumerate}
\noindent\textbf{解说:}按照有关规定,144\unit{\MHz}和430\unit{\MHz}频段业余中继台的上下行频差应分别为\textbf{0.6\unit{\MHz}、5\unit{\MHz}}。\\\noindent\textbf{答案:}A



\bigskip


\noindent\textbf{试题:}28\unit{\MHz}业余频段中留给业余卫星通信、话音及其他通信方式不应占用的频率段为:
\begin{enumerate}[leftmargin=3em]
\item 28.3\unit{\MHz}至28.61\unit{\MHz}
\item 29.3\unit{\MHz}至29.51\unit{\MHz} % 原文 29.3MMz至29.51MHz
\item 29.15\unit{\MHz}至29.35\unit{\MHz}
\item 28.7\unit{\MHz}至28.95\unit{\MHz}
\end{enumerate}
\noindent\textbf{解说:}28\unit{\MHz}业余频段中留给业余卫星通信、话音及其他通信方式不应占用的频率段为\textbf{29.3MMz至29.51\unit{\MHz}}。\\\noindent\textbf{答案:}B



\bigskip


\noindent\textbf{试题:}《业余无线电台管理办法》规定业余无线电台设置、正确使用业余无线电台呼号的办法规是:
\begin{enumerate}[leftmargin=3em]
\item 业余无线电台应当在每次通信建立及结束时,主动报出本台呼号,在发射过程中至少每十分钟报出本台呼号一次;对于通信对方,也应使用对方电台的呼号加以标识
\item 业余电台在通信中可以用自造的呼号作为无线电管理机构指配的业余电台呼号的补充,一起作为电台的标识
\item 业余电台在和熟悉的通信对象联络、已经从信号特征确认双方业余电台身份时,可以省略呼号的发送
\item 业余电台在通信中可以用姓名、代号、适当的别名或者法规定呼号的部分数字和字母代替完整的业余电台呼号作为电台的标识
\end{enumerate}
\noindent\textbf{解说:}《业余无线电台管理办法》附录3第8条:\textbf{业余无线电台在每次通信建立及结束时,应当主动发送本台呼号;在发信过程中应当至少每十分钟发送本台呼号一次。对于通信对方,也应使用对方电台的呼号加以标识}。\\\noindent\textbf{答案:}A



\bigskip


\noindent\textbf{试题:}业余无线电台应当在每次通信建立及结束时,主动报出本台呼号,在发射过程中至少每十分钟报出本台呼号一次。这里的“呼号”是指:
\begin{enumerate}[leftmargin=3em]
\item 一般指无线电管理机构指配的电台呼号,但也可以是对方能够理解的民间自创呼号、代号、代码等
\item 可以是完整的电台呼号,也可以是完整电台呼号的任何一部分
\item 可以是完整的电台呼号,也可以是电台呼号的分区号加后缀
\item 完整的电台呼号,如在设台地以外的地点进行异地发射操作,还应在前面加上字母B、操作地分区号和符号“/”
\end{enumerate}
\noindent\textbf{解说:}这里的“呼号”是指\textbf{完整的电台呼号}。《业余无线电台管理办法》第三十三条规定:由国家无线电管理机构审批的业余无线电台\textbf{在设台地以外的地点进行异地发射操作的,应当使用“字母B、操作地业余无线电台分区号、符号/、本人业余无线电台的呼号”格式的呼号}。\\\noindent\textbf{答案:}D



\bigskip


\noindent\textbf{试题:}业余无线电台呼号的指配流程是:
\begin{enumerate}[leftmargin=3em]
\item 无线电管理机构核发业余无线电台执照后,由申请人再向其申请指配呼号
\item 业余无线电台设台人在提交设台申请的同时提出所要求指配的呼号,经服务机构同意后,报无线电管理机构正式指配
\item 无线电管理机构核发业余无线电台执照时,同时指配业余无线电台呼号
\item 在向无线电管理机构委托的受理服务机构提交设台申请窗口后,由服务机构指配呼号
\end{enumerate}
\noindent\textbf{解说:}《业余无线电台管理办法》第三十二条规定:\textbf{无线电管理机构核发业余无线电台执照,应当同时指配业余无线电台呼号。}\\\noindent\textbf{答案:}C



\bigskip


\noindent\textbf{试题:}业余无线电爱好者对业已指配给自己的电台呼号不满意,是否可以申请另行指配业余无线电台呼号?
\begin{enumerate}[leftmargin=3em]
\item 业余无线电台执照有效期届满、设台人向核发执照的无线电管理机构申请办理延续手续时可以申请另行指配业余无线电台呼号
\item 不可以。核发业余无线电台执照的无线电管理机构已经为申请人指配业余无线电台呼号的,不另行指配其他业余无线电台呼号
\item 可以申请另行指配业余无线电台呼号,但须缴纳额外的费用
\item 更新所设置的业余无线电台类别时可以申请另行指配业余无线电台呼号
\end{enumerate}
\textbf{解说:不可以}。《业余无线电台管理办法》第三十二条规定:\textbf{核发业余无线电台执照的无线电管理机构已经为设置人指配业余无线电台呼号的,不另行为其指配其他业余无线电台呼号。}\\\noindent\textbf{答案:}B




\bigskip


\noindent\textbf{试题:}各地业余无线电台呼号前缀字母和后缀字符的可用范围的确定方法是:
\begin{enumerate}[leftmargin=3em]
\item 地方无线电管理机构根据当地呼号资源的使用情况自行分配
\item 地方无线电民间组织提出建议,当地无线电管理机构批准
\item 由国家无线电管理机构编制和分配
\item 由业余无线电爱好者根据需求提出建议,当地无线电管理机构批准
\end{enumerate}
\noindent\textbf{解说:}《业余无线电台管理办法》第三十二条规定:业余无线电台呼号\textbf{由国家无线电管理机构编制和分配}。\\\noindent\textbf{答案:}C



\bigskip


\noindent\textbf{试题:}业余无线电爱好者是否可要求设台地所在地方无线电管理机构给予指配超出业已分配给该地方的前缀字母和后缀字符可用范围的业余无线电台呼号?
\begin{enumerate}[leftmargin=3em]
\item 可以,但只限于当地业余无线电台参加国际重要业余无线电活动的特殊情况
\item 不能,特殊业余无线电台呼号只能由国家无线电管理机构指配
\item 可以,但只限于与在当地所举办的大型国际或国家级活动有关的特殊电台
\item 可以,但只限于与当地政府组织的大型科技活动有关的特殊电台
\end{enumerate}
\textbf{解说:不能}。《业余无线电台管理办法》第三十二条规定:业余信标台和空间业余无线电台等\textbf{特殊业余无线电台呼号由国家无线电管理机构指配},其他业余无线电台呼号由地方无线电管理机构指配。\\\noindent\textbf{答案:}B



\bigskip


\noindent\textbf{试题:}经地方无线电管理机构批准设置的业余无线电台,设台地迁入其他省、自治区或者直辖市时,应办理的手续为:
\begin{enumerate}[leftmargin=3em]
\item 持原电台执照直接到原核发执照的无线电管理机构申请办理变更手续
\item 不需要办理任何手续
\item 持原电台执照直接到迁入地的地方无线电管理机构申请办理变更手续
\item 先到原核发执照的无线电管理机构办理申请注销原业余无线电台,再到迁入地的地方无线电管理机构办理申请设置业余无线电台的手续
\end{enumerate}
\noindent\textbf{解说:}《业余无线电台管理办法》第十五条规定:\textbf{终止使用业余无线电台的,应当向核发业余无线电台执照的无线电管理机构申请注销执照}。\\
《业余无线电台管理办法》第八条规定:\textbf{设置在省、自治区、直辖市范围内通信的业余无线电台,由设台地地方无线电管理机构审批}。\\
《业余无线电台管理办法》第七条规定:\textbf{申请设置业余无线电台,应当向设台地地方无线电管理机构提交材料}。\\\noindent\textbf{答案:}D
%%%


\bigskip


\noindent\textbf{试题:}经国家无线电管理机构批准设置的业余无线电台,设台地迁入其他省、自治区或者直辖市时,应办理的手续为:
\begin{enumerate}[leftmargin=3em]
\item 先到原核发执照的无线电管理机构申请办理注销手续,缴回原电台执照,领取国家无线电管理机构已批准设台的证明,凭证明到迁入地的地方无线电管理机构完成申请变更手续,领取新电台执照
\item 持原电台执照直接到原核发执照的无线电管理机构申请办理变更手续
\item 持原电台执照直接到迁入地的地方无线电管理机构申请办理变更手续
\item 不需要办理任何手续
\end{enumerate}
\noindent\textbf{解说:}《业余无线电台管理办法》第十五条规定:\textbf{终止使用业余无线电台的,应当向核发业余无线电台执照的无线电管理机构申请注销执照}。\\
《业余无线电台管理办法》第八条规定:\textbf{设置通信范围涉及两个以上的省、自治区、直辖市或者涉及境外的业余无线电台,由国家无线电管理机构审批}。\\
《业余无线电台管理办法》第七条规定:\textbf{申请设置业余无线电台,应当向设台地地方无线电管理机构提交材料}。\\\noindent\textbf{答案:}A
%%%



\bigskip


\noindent\textbf{试题:}设台地迁入其他省、自治区或者直辖市时,业余电台呼号的指配方法为:
\begin{enumerate}[leftmargin=3em]
\item 可以在保留原电台呼号的同时申请指配迁入地的新电台呼号
\item 必须继续指配原来的电台呼号
\item 必须指配迁入地的新电台呼号
\item 由设台人选择:方法一,注销原电台呼号,指配迁入地的新电台呼号;方法二,申请在迁入地继续指配原来的电台呼号
\end{enumerate}
\noindent\textbf{解说:}设台地迁入其他省、自治区或者直辖市时,业余电台呼号的指配方法为:\textbf{由设台人选择:方法一,注销原电台呼号,指配迁入地的新电台呼号;方法二,申请在迁入地继续指配原来的电台呼号}。\\\noindent\textbf{答案:}D



\bigskip


\noindent\textbf{试题:}设台地迁入其他省、自治区或者直辖市时,申请在迁入地继续指配原来的电台呼号的手续为:
\begin{enumerate}[leftmargin=3em]
\item 先到原核发执照的无线电管理机构申请办理注销手续,缴回原电台执照,取得由迁入地指配原业余无线电台呼号的书面同意,再到迁入地的地方无线电管理机构办理相应的手续、重新指配原电台呼号,领取新的电台执照
\item 只需到迁入地的地方无线电管理机构申请申请办理为迁移后的电台继续使用原电台呼号的全部手续
\item 只需到原核发执照的无线电管理机构申请申请办理为迁移后的电台继续使用原电台呼号的全部手续
\item 不需办理任何手续即可把原电台呼号带到迁入地继续使用
\end{enumerate}
\noindent\textbf{解说:}设台地迁入其他省、自治区或者直辖市时,申请在迁入地继续指配原来的电台呼号的手续为:\textbf{先到原核发执照的无线电管理机构申请办理注销手续,缴回原电台执照,取得由迁入地指配原业余无线电台呼号的书面同意,再到迁入地的地方无线电管理机构办理相应的手续、重新指配原电台呼号,领取新的电台执照}。\\\noindent\textbf{答案:}A
%%%



\bigskip


\noindent\textbf{试题:}在实际通信中,是否可以把本台呼号的地区号码加后缀视作《业余电台管理办法》所说的“本台呼号”?
\begin{enumerate}[leftmargin=3em]
\item 在HF频段进行国内呼叫和联络时可以把不完整呼号视作“呼号”
\item 在熟悉的友台之间呼叫和联络中可以把不完整呼号视作“呼号”
\item 在VHF/UHF频段进行本地呼叫和联络时可以把不完整呼号视作“呼号”
\item 不可以。不完整呼号不具有呼号的属性,不能视作呼号
\end{enumerate}
\textbf{解说:不可以。不完整呼号不具有呼号的属性,不能视作呼号}。\\\noindent\textbf{答案:}D



\bigskip


\noindent\textbf{试题:}由国家无线电管理机构批准设台的北京火腿的电台呼号为BH1AAA,把电台带到西安去使用,则本台呼号应该为:
\begin{enumerate}[leftmargin=3em]
\item B9/BH1AAA
\item BH1AAA/B9
\item B1/BH9AAA
\item BH1AAA/9
\end{enumerate}
\noindent\textbf{解说:}业余无线电爱好者把电台带到西安去使用,《业余无线电台管理办法》附录3第8条规定:业余无线电台在设台地以外的地点进行异地发射操作的,应当使用“字母B、操作地业余无线电台分区号、符号/、本人业余无线电台的呼号”格式的呼号。西安属于陕西省,按照业余无线电台分区表,属于第9区,因此本台呼号应该为\textbf{B9/BH1AAA}。\\\noindent\textbf{答案:}A



\bigskip


\noindent\textbf{试题:}某业余无线电爱好者,自己所设置的业余无线电台呼号为BH1ZZZ。现该爱好者到业余无线电台BH9YYY做客并在该台进行发射操作。应当使用的呼号为:
\begin{enumerate}[leftmargin=3em]
\item BH1ZZZ/9或者BH1ZZZ/BH9
\item BH9YYY或者B9/BH1ZZZ
\item BH1ZZZ或者B9/BH1ZZZ
\item BH9/BH1ZZZ或者BH1ZZZ
\end{enumerate}
\noindent\textbf{解说:}到业余无线电台BH9YYY做客并在该台进行发射操作,《业余无线电台管理办法》附录3第8条规定:由国家无线电管理机构审批的业余无线电台在设台地以外的地点进行异地发射操作时,应当使用所操作业余无线电台的呼号,因此可以使用呼号\textbf{BH9YYY};《业余无线电台管理办法》附录3第7条规定:已经取得业余无线电台执照的个人,可以使用“字母B、操作地业余无线电台分区号、符号/、本人业余无线电台呼号”格式的呼号。进行客席发射操作的业余无线电台BH9YYY属于第9区,因此也可以使用呼号\textbf{B9/BH1ZZZ}。\\\noindent\textbf{答案:}B



\bigskip


\noindent\textbf{试题:}某业余无线电爱好者,自己所设置的业余无线电台呼号为BH1ZZZ。现该爱好者将自己的业余无线电台带到湖南进行异地发射操作。应当使用的呼号为:
\begin{enumerate}[leftmargin=3em]
\item BH1ZZZ
\item B7/BH1ZZZ
\item BH7/BH1ZZZ
\item BH1ZZZ/B7
\end{enumerate}
\noindent\textbf{解说:}将自己的业余无线电台带到湖南进行异地发射操作,《业余无线电台管理办法》附录3第8条规定:由国家无线电管理机构审批的业余无线电台在设台地以外的地点进行异地发射操作的,应当使用“字母B、操作地业余无线电台分区号、符号/、本人业余无线电台的呼号”格式的呼号。按照业余无线电台分区表,湖南属于第7区,因此本台呼号应该为\textbf{B7/BH1ZZZ}。\\\noindent\textbf{答案:}B



\bigskip


\noindent\textbf{试题:}某业余无线电爱好者,自己所设置的业余无线电台呼号为BH1ZZZ。现该爱好者到业余无线电台BH3YYY做客并并在该台进行发射操作。这种发射操作在业余无线电台管理中称为:
\begin{enumerate}[leftmargin=3em]
\item 异地发射操作
\item 移动发射操作
\item 客席发射操作
\item 违章发射操作
\end{enumerate}
\noindent\textbf{解说:}在他人设置的业余无线电台上进行发射操作,这种发射操作在业余无线电台管理中称为\textbf{客席发射操作}。\\\noindent\textbf{答案:}C
%%%



\bigskip


\noindent\textbf{试题:}某业余无线电爱好者,自己所设置的业余无线电台呼号为BH1ZZZ。现该爱好者将自己的业余无线电台带到广东进行发射操作。这种发射操作在业余无线电台管理中称为:
\begin{enumerate}[leftmargin=3em]
\item 违章发射操作
\item 客席发射操作
\item 临时发射操作
\item 异地发射操作
\end{enumerate}
\noindent\textbf{解说:}《业余无线电台管理办法》附录3第8条规定:由国家无线电管理机构审批的业余无线电台在设台地以外的地点进行\textbf{异地发射操作}的,应当使用“字母B、操作地业余无线电台分区号、符号/、本人业余无线电台的呼号”格式的呼号。\\\noindent\textbf{答案:}D






\bigskip


\noindent\textbf{试题:}BH1ZZZ由北京迁入河北省,并办妥了由河北无线电管理机构指配使用原电台呼号的全部手续,领取了新的业余无线电台执照。该台在日常通信时应使用呼号:
\begin{enumerate}[leftmargin=3em]
\item BH1ZZZ
\item B3/BH1ZZZ
\item 固定台址发射操作用BH1ZZZ,移动发射操用B3/BH1ZZZ
\item 可任选使用呼号B3/BH1ZZZ或者BH1ZZZ
\end{enumerate}
\noindent\textbf{解说:}《业余无线电台管理办法》附录3第8条规定:由国家无线电管理机构审批的业余无线电台在设台地以外的地点进行异地发射操作的,应当使用“字母B、操作地业余无线电台分区号、符号/、本人业余无线电台的呼号”格式的呼号。河北为第3区,故该台在日常通信时应使用呼号\textbf{B3/BH1ZZZ}。\\\noindent\textbf{答案:}B




\bigskip


\noindent\textbf{试题:}发射类别(class of emission)是指用标准符号标示的某发射的一组特性,例如主载波调制方式,调制信号,被发送信息的类型以及其他适用的信号特性。表示CW报的发射类别是:
\begin{enumerate}[leftmargin=3em]
\item A1A
\item J3E
\item F2B
\item G2B
\end{enumerate}
\noindent\textbf{解说:}表示CW报的发射类别是\textbf{A1A}。\\\noindent\textbf{答案:}A



\noindent\textbf{试题:}发射类别(class of emission)是指用标准符号标示的某发射的一组特性,例如主载波调制方式,调制信号,被发送信息的类型以及其他适用的信号特性。表示单边带话的发射类别是:
\begin{enumerate}[leftmargin=3em]
\item J3E
\item A1A
\item F2B
\item G2B
\end{enumerate}
\noindent\textbf{解说:}表示单边带话的发射类别是\textbf{J3E}。\\\noindent\textbf{答案:}A



\bigskip


\noindent\textbf{试题:}发射类别(class of emission)是指用标准符号标示的某发射的一组特性,例如主载波调制方式,调制信号,被发送信息的类型以及其他适用的信号特性。表示用单边带话传输的RTTY信号的发射类别是:
\begin{enumerate}[leftmargin=3em]
\item G2B
\item A1A
\item F2B
\item J3E
\end{enumerate}
\noindent\textbf{解说:}表示用单边带话传输的RTTY信号的发射类别是\textbf{F2B}。\\\noindent\textbf{答案:}C


\bigskip


\noindent\textbf{试题:}发射类别(class of emission)是指用标准符号标示的某发射的一组特性,例如主载波调制方式,调制信号,被发送信息的类型以及其他适用的信号特性。表示用单边带话传输的PSK31信号的发射类别是:
\begin{enumerate}[leftmargin=3em]
\item G2B
\item A1A
\item F2B
\item J3E
\end{enumerate}
\noindent\textbf{解说:}表示用单边带话传输的PSK31信号的发射类别是\textbf{G2B}。\\\noindent\textbf{答案:}A



\bigskip


\noindent\textbf{试题:}发射类别(class of emission)是指用标准符号标示的某发射的一组特性,例如主载波调制方式,调制信号,被发送信息的类型以及其他适用的信号特性。表示调频话的发射类别是:
\begin{enumerate}[leftmargin=3em]
\item J3E
\item F3E
\item F3F
\item F2B
\end{enumerate}
\noindent\textbf{解说:}表示调频话的发射类别是\textbf{F3E}。\\\noindent\textbf{答案:}B


\bigskip


\noindent\textbf{试题:}某业余电台操作者听到业余专用频率上出现某种显然出自非业余电台的人为干扰发射,于是按下话筒向该发射者宣传无线电管理法规知识。对这种做法的评论应该是:
\begin{enumerate}[leftmargin=3em]
\item 正确;但需注意态度耐心、用语文明
\item 正确;抓机遇宣传法规,应该提倡
\item 正确;但有点条乱,不予提倡
\item 错误;违反“业余无线电台的通信对象应当限于业余无线电台”规定%错误;违反“业余无线电台的通信对象应当限于业余无线电台”规定。
\end{enumerate}
\noindent\textbf{解说:}这种做法\textbf{错误},\textbf{违反了《业余无线电台管理办法》第二十条:“业余无线电台的通信对象应当限于业余无线电台”规定}。\\\noindent\textbf{答案:}D



\bigskip


\noindent\textbf{试题:}在业余无线电台中转发广播电台、互联网聊天、电话通话、其他电台的联络信号,这类行为的性质是:
\begin{enumerate}[leftmargin=3em]
\item 错误行为;违反“业余无线电台的通信对象应当限于业余无线电台”规定,因为通信中产生信息的一方不是通信业余无线电台本身
\item 如果转发的目的是进行技术调试、用转发信号作为测试信号的话,就是正常行为
\item 正确行为;既然可以联络,不必要限制向话筒送什么内容
\item 不算错误但也不值得提倡
\end{enumerate}
\noindent\textbf{解说:}是\textbf{错误行为,违反了《业余无线电台管理办法》第二十条:“业余无线电台的通信对象应当限于业余无线电台”规定,因为通信中产生信息的一方不是通信业余无线电台本身}。\\\noindent\textbf{答案:}A



\bigskip


\noindent\textbf{试题:}业余电台在通信中为其他人或者单位、组织转达信息。对这种做法的评论应该是:
\begin{enumerate}[leftmargin=3em]
\item 违法行为;违反“业余无线电台的通信对象应当限于业余无线电台”的规定
\item 只要所转达的信息在内容上不违反《业余电台管理办法规》的禁止规定就是合法行为
\item 只要所转达的信息是有利于社会的公益信息,就是合法行为
\item 只要转达信息是无偿的,就是合法行为
\end{enumerate}
\noindent\textbf{解说:}是\textbf{违法行为,违反《业余无线电台管理办法》第二十条:“业余无线电台的通信对象应当限于业余无线电台”的规定}。\\\noindent\textbf{答案:}A



\bigskip


\noindent\textbf{试题:}某业余无线电协会在发射操作中向其会员播发公益性通知和技术训练讲座,但未得到相应无线电管理机构的批准。对这种做法的评论应该是:
\begin{enumerate}[leftmargin=3em]
\item 只要所播发的通知或讲座是涉及宣传业余电台管理知识的,不能算违法行为
\item 违法行为;违反“未经核发业余无线电台执照的无线电管理机构批准,业余无线电台不得以任何方式进行广播或者发射通播性质的信号”的规定
\item 只要所播发的通知或讲座有利于当地业余无线电爱好者技术水平的提高,不能算违法行为
\item 只要所播发的通知或讲座有利于当地业余无线电应急通信训练,不能算违法行为
\end{enumerate}
\noindent\textbf{解说:}是\textbf{违法行为,违反《业余无线电台管理办法》第二十一条:“未经所在地地方无线电管理机构批准,业余无线电台不得以任何方式进行广播或者发射通播性质的信号”的规定}。\\\noindent\textbf{答案:}B





\bigskip


\noindent\textbf{试题:}关于业余无线电台在通信过程中使用的语言,正确的做法为:
\begin{enumerate}[leftmargin=3em]
\item 任何时候都应当使用明语及业余无线电领域公认的缩略语和简语
\item 语言要创新,可以使用自创的特殊缩略语,虽开始时象是暗语,用多了就会变明语
\item 可提倡使用稀有语言或方言,尽量使特定通信对象以外的业余无线电台听不懂,以减少他台呼叫和插入的机会
\item 可以使用虽然不是所有火腿通用、但在某些火腿圈子内部有一定可懂度的新编缩略语或暗语
\end{enumerate}
\noindent\textbf{解说:}《业余无线电台管理办法》第二十二条规定:\textbf{业余无线电台在通信过程中应当使用明语及业余无线电领域公认的缩略语和简语}。\\\noindent\textbf{答案:}A




\bigskip


\noindent\textbf{试题:}业余无线电台实验新的编码、调制方式、数字通信协议或者交换尚未公开格式的数据文件,正确做法是:
\begin{enumerate}[leftmargin=3em]
\item 事先尽可能采取各种办法向信号可能覆盖范围内的业余无线电爱好者公开有关技术细节,但不必提交给核发其业余无线电台执照的地方无线电管理机构
\item 应事先提交给核发其业余无线电台执照的地方无线电管理机构,但不必向其他业余无线电爱好者公开有关技术细节
\item 不必事先公开或者提交给核发其业余无线电台执照的地方无线电管理机构,以后再说
\item 事先尽可能采取各种办法向信号可能覆盖范围内的业余无线电爱好者公开有关技术细节,并提交给核发其业余无线电台执照的地方无线电管理机构
\end{enumerate}
\noindent\textbf{解说:}《业余无线电台管理办法》第二十二条规定:业余无线电台在通信过程中应当使用明语及业余无线电领域公认的缩略语和简语,\textbf{数据文件交换应当使用公开的方式}。\textbf{业余无线电台试验新的编码、调制方式和数字通信协议等,应当事先公开并向所在地地方无线电管理机构提交相关技术信息。}\\\noindent\textbf{答案:}D




\bigskip


\noindent\textbf{试题:}由国家无线电管理机构审批的业余无线电台在设台地以外的地点进行异地发射操作时,应该注意:
\begin{enumerate}[leftmargin=3em]
\item 有了电台执照就是万事大吉,不必认真了解和遵守什么具体规定
\item 既要符合业余电台执照所核定的各项参数约束,又要遵守操作所在地的地方无线电管理机构的相关规定
\item 遵守的限制以业余电台执照所核定的各项参数和核发其业余电台执照的地方无线电管理机构的规定为准,与操作所在地的规定无关
\item 遵守的限制以操作所在地的地方无线电管理机构的相关规定为准,与核发电台执照的地方无线电管理机构的规定无关
\end{enumerate}
\noindent\textbf{解说:}《业余无线电台管理办法》第十四条规定:\textbf{业余无线电台的技术参数不得超出其业余无线电台执照所核定的范围}。\\
《业余无线电台管理办法》第三十四条规定:\textbf{依法设置的通信范围涉及两个以上的省、自治区、直辖市或者涉及境外的业余无线电台,可以在设台地以外的地点进行异地发射操作,但应当遵守所在地地方无线电管理机构的相关规定}。\\\noindent\textbf{答案:}B
%%%


\bigskip


\noindent\textbf{试题:}具备国家无线电管理机构规定的操作技术能力并具有法律规定有效证明文件、但还没有获准设置自己的业余电台的人是否可以到业余电台进行发射操作?答案是:
\begin{enumerate}[leftmargin=3em]
\item 不可以
\item 可以。使用所操作业余电台的呼号,由该业余电台的设台人对操作不妥而造成的有害干负责
\item 青少年可以,成人不可以
\item 可以。因为自己没有呼号,只能在通信中使用临时自编的呼号,或用姓名代替呼号
\end{enumerate}
\noindent\textbf{解说:可以。}《业余无线电台管理办法》第六条规定:使用业余无线电台,应当具备下列条件:
\begin{enumerate}[label=, leftmargin=0.8em]
\item (一)熟悉无线电管理规定;
\item (二)具备国家无线电管理机构规定的操作技术能力,取得相应操作技术能力证明;
\item (三)无线电发射设备符合国家相关技术标准;
\item (四)法律、行政法规规定的其他条件。
\end{enumerate}
因此可以\textbf{使用所操作业余电台的呼号进行发射操作,由该业余电台的设台人应该对操作不妥而造成的有害干扰负责}。\\\noindent\textbf{答案:}B



\bigskip


\noindent\textbf{试题:}尚未考得《业余电台操作证书》的人在接受业余电台培训中实习发射操作应遵守的条件是什么?
\begin{enumerate}[leftmargin=3em]
\item 尚未取得关于具备操作技术能力有效证明文件的人任何情况下都不可以进行发射操作
\item 只要业余电台设置人或者其技术负责人能确认实际上已经具备操作技术能力,可以独立进行发射操作,并能为其操作不善造成的后果负责,可以独立发射操作
\item 尚未取得关于具备操作技术能力有效证明文件者如为青少年,可以在集体业余电台独立操作,如为成人则任何情况下都不可以进行发射操作
\item 必须已接受法规等基础培训、必须由电台负责人现场辅导、必须在执照核定范围以及国家规定的操作权限内、进行短时间体验性发射操作实习
\end{enumerate}
\noindent\textbf{解说:}尚未考得《业余电台操作证书》的人在接受业余电台培训中实习发射操作应遵守:\textbf{必须已接受法规等基础培训、必须由电台负责人现场辅导、必须在执照核定范围以及国家规定的操作权限内、进行短时间体验性发射操作实习}。\\\noindent\textbf{答案:}D
%%%



\bigskip


\noindent\textbf{试题:}业余无线电台设置人应对其无线电发射设备担负的法定责任为:
\begin{enumerate}[leftmargin=3em]
\item 应当确保其无线电发射设备随最先进型号更新,为其他业余电台树立求新的榜样
\item 应当确保其无线电发射设备处于正常工作状态,避免对其他无线电业务造成有害干扰
\item 应当确保其无线电发射设备达到最大发射功率,以克服其他无线电业务的干扰
\item 应当确保其无线电发射设备经常处于工作状态,以提高业余频率的实际占用度
\end{enumerate}
\noindent\textbf{解说:}业余无线电台设置人应对其无线电发射设备担负的法定责任为:\textbf{应当确保其无线电发射设备处于正常工作状态,避免对其他无线电业务造成有害干扰}。\\\noindent\textbf{答案:}B


\bigskip


\noindent\textbf{试题:}业余无线电爱好者使用业余无线电收信设备应遵守的规定为:
\begin{enumerate}[leftmargin=3em]
\item 只要不被查出来,可以接收任何无线电信号
\item 只要不造成对其他业务的无线电干扰,接收无线电信号没有限制
\item 不得接收与业余业务和卫星业余业务无关的信号
\item 只要出于个人对信息的兴趣而不涉及赢利,可以接收任何无线电信号
\end{enumerate}
\noindent\textbf{解说:}《业余无线电台管理办法》第二十七条规定:业余无线电台通信\textbf{不得发送、接收与业余业务和卫星业余业务无关的信号}。\\\noindent\textbf{答案:}C



\bigskip


\noindent\textbf{试题:}业余无线电爱好者无意接收到非业余业务和卫星业余业务的信息时,应遵守的规则为:
\begin{enumerate}[leftmargin=3em]
\item 只可以在业余无线电台间共享,不得在其他场合公开
\item 不得传播、公布
\item 既然自己可以收到,别人也一定可以收到,当然可以传播、公布或者利用
\item 只可以用非无线电方式在业余无线电爱好者之间交流,不得以无线电方式转发
\end{enumerate}
\noindent\textbf{解说:}《业余无线电台管理办法》第二十七条规定:\textbf{不得传播、公布}无意接收的非业余业务和卫星业余业务的信息。\\\noindent\textbf{答案:}B



\bigskip


\noindent\textbf{试题:}业余无线电台是否可以发射从广播电台收到的信号、音像节目的录音,或者故意转送电台周围的声音?
\begin{enumerate}[leftmargin=3em]
\item 可以,用于提起其他有业余无线电台操作员精神,防止乏困
\item 可以,因为该类信息没有保密性
\item 不可以,不得发送与业余业务和卫星业余业务无关的信号
\item 可以,用于显示自己发射设备的信号质量
\end{enumerate}
\textbf{解说:不可以}。《业余无线电台管理办法》第二十七条规定:业余无线电台通信\textbf{不得发送、接收与业余业务和卫星业余业务无关的信号}。\\\noindent\textbf{答案:}C





\bigskip


\noindent\textbf{试题:}国家对于利用业余无线电台从事发布、传播违反法律或者公共道德的信息的行为的态度是:
\begin{enumerate}[leftmargin=3em]
\item 不提倡
\item 不可以发布传、播违法信息。但违反公共道德的信息属于水平问题,不鼓励就是了
\item 禁止
\item 容忍
\end{enumerate}
\noindent\textbf{解说:}《业余无线电台管理办法》第二十八条规定:\textbf{禁止}利用业余无线电台从事发布、传播违反法律或者公共道德的信息。\\\noindent\textbf{答案:}C



\bigskip


\noindent\textbf{试题:}出租车安装业余电台并用来传递有关载客的信息,这种行为的性质是:
\begin{enumerate}[leftmargin=3em]
\item 只要管理部门不来查处就可以
\item 不太好,因为占用了其他业余电台通信的频率
\item 只要不影响其他业余电台的正常通信就可以
\item 违法行为,违反了严禁利用业余无线电台从事商业或者其他营利活动的规定%违法行为,违反了严禁利用业余无线电台从事从事商业或者其他营利活动的规定
\end{enumerate}
\noindent\textbf{解说:}是\textbf{违法行为,违反了《业余无线电台管理办法》第二十八条“禁止利用业余无线电台从事商业或者其他与营利有关的活动”的规定}。\\\noindent\textbf{答案:}D



\bigskip


\noindent\textbf{试题:}利用业余无线电台通信来促销业余无线电产品或者推动与业余无线电活动有关的其他商业性活动,对这类行为的态度应该是:
\begin{enumerate}[leftmargin=3em]
\item 不提倡但也不禁止,毕竟有利于业余无线电活动发展
\item 禁止
\item 只要是业余无线电民间组织是获利方,即使从事商业或其他营利活动,应支持
\item 如果设台人或者设台单位本身是以这类经营为生的,应适当理解和容忍
\end{enumerate}
\noindent\textbf{解说:}《业余无线电台管理办法》第二十八条规定:\textbf{禁止}利用业余无线电台从事商业或者其他与营利有关的活动。\\\noindent\textbf{答案:}B



\bigskip


\noindent\textbf{试题:}利用自己的业余电台强信号故意压制其他业余电台的正常通信,或者在业余无线电频率上转播音乐或广播节目,这些行为的性质属于:
\begin{enumerate}[leftmargin=3em]
\item 不文明行为,对其他业余电台不够礼貌
\item 违法行为,违反了严禁阻碍其他无线电台通信的规定
\item 正常现象,社会上一些人素质就是如此,应该谅解
\item 不妥行为,没有考虑到他人的乐趣
\end{enumerate}
\noindent\textbf{解说:}属于\textbf{违法行为,违反了《业余无线电台管理办法》第二十八条“禁止利用业余无线电台阻碍其他无线电台通信”的规定}。\\\noindent\textbf{答案:}B



\bigskip


\noindent\textbf{试题:}业余无线电活动是否有序开展,会影响整个社会的无线电通信的安全和有效,使用不当甚至会导致生命财产损失。业余无线电爱好者在这方的法定责任是:
\begin{enumerate}[leftmargin=3em]
\item 个人没有责任,只能依靠管理部门的监督检查和违法查处
\item 有了电台执照,日常一切言行当然可以带到电台通信中,无责任可言
\item 业余无线电台设置、使用人应当加强自律
\item 个人没有责任,只能依靠业余无线电民间组织充当“协管”
\end{enumerate}
\noindent\textbf{解说:}《业余无线电台管理办法》第二十九条规定:\textbf{业余无线电台设置人、使用人应当加强自律},接受无线电管理机构或者其委托单位的指导、监督和检查。\\\noindent\textbf{答案:}C



\bigskip


\noindent\textbf{试题:}国际电联规定的确定发射电台辐射功率的原则为:
\begin{enumerate}[leftmargin=3em]
\item VHF/UHF频段发射电台应辐射尽量大的功率,HF频段发射电台应辐射尽量小的功率
\item 发射电台只应辐射为保证满意服务所必要的功率
\item HF频段发射电台应辐射尽量大的功率,VHF频段发射电台应辐射尽量小的功率
\item 发射电台应辐射尽量大的功率以提供尽量好的信号质量
\end{enumerate}
\noindent\textbf{解说:}ITU《无线电规则》(2020年版)第15.2款:\textbf{发射电台只应辐射为保证满意服务所必要的功率}。\\\noindent\textbf{答案:}B



\bigskip


\noindent\textbf{试题:}业余电台通信受到违法电台或者不明电台的有害干扰。正确的做法是:
\begin{enumerate}[leftmargin=3em]
\item 不予理睬,收集有关信息并向无线电管理机构举报
\item 在频率上向其宣传无线电管理法,要求其停止干扰
\item 用大功率信号对其进行压制
\item 立即报告无线电管理机构进行干涉
\end{enumerate}
\noindent\textbf{解说:}正确的做法是:\textbf{不予理睬,收集有关信息并向无线电管理机构举报}。\\\noindent\textbf{答案:}A



\bigskip


\noindent\textbf{试题:}按照我国规定,购置使用公众对讲机不需取得批准。业余无线电爱好者需要与公众对讲机用户通信时应该:
\begin{enumerate}[leftmargin=3em]
\item 将业余无线电台设置到公众对讲机的频率,以不大于业余无线电台执照核定的发射功率与之通信
\item 将业余无线电台设置到公众对讲机的频率,但只能进行由业余无线电台到公众对讲机的单向发信
\item 业余无线电台不能用于与公众对讲机通信
\item 将业余无线电台设置到公众对讲机的频率,以不大于0.5W的发射功率与之通信
\end{enumerate}
\noindent\textbf{解说:}\textbf{业余无线电台不能用于与公众对讲机通信}。《业余无线电台管理办法》第十一条规定:\textbf{业余无线电台专用无线电发射设备不得用于其他无线电业务,其发射频率应当在业余业务或者卫星业余业务频段内}。\\\noindent\textbf{答案:}C




\bigskip


\noindent\textbf{试题:}业余中继台的设置和技术参数等应满足下列关键条件:
\begin{enumerate}[leftmargin=3em]
\item 符合申请人关于设置中继台的客观需求和技术考虑
\item 符合国家以及设台地的地方无线电管理机构的规定
\item 符合设台地的地方业余无线电民间组织的规划
\item 仅需符合设台地的地方无线电管理机构的规划及相关规定
\end{enumerate}
\noindent\textbf{解说:}《业余无线电台管理办法》第十条规定:业余中继台的设置和技术参数等应当\textbf{符合国家以及设台地地方无线电管理机构的规定}。\\\noindent\textbf{答案:}B



\bigskip


\noindent\textbf{试题:}业余中继台必备的技术措施为:
\begin{enumerate}[leftmargin=3em]
\item 设热备份系统,保证不间断工作
\item 技术加密措施,防止未经设台人允许的业余无线电台启用中继
\item 设专人负责监控和管理工作,配备有效的遥控手段,保证造成有害干扰时及时停止发射
\item 尽量提高发射功率,以便压制覆盖区内的其他强信号干扰
\end{enumerate}
\noindent\textbf{解说:}《业余无线电台管理办法》第十条规定:业余中继台应当\textbf{设专人负责监控和管理工作,配备有效的遥控手段 }。\\
《业余无线电台管理办法》第二十三条规定:业余无线电台设置人应当对无线电发射设备进行有效监控,确保正常工作,\textbf{保证能够及时停止其造成的有害干扰}。 \\\noindent\textbf{答案:}C


\bigskip


\noindent\textbf{试题:}某团体依法设置了一部业余中继台。其正确做法是:
\begin{enumerate}[leftmargin=3em]
\item 为保证中继台正常运行,要求覆盖区内所有业余电台缴纳维护成本,否则不准使用
\item 中继台是设台者出资建设和维护的,因此仅供经设置者允许的业余电台使用
\item 向其覆盖区域内的所有业余无线电台提供平等的服务,并将使用业余中继台所需的各项技术参数公开
\item 中继台是设台者出资建设和维护的,因此仅供本团体成员优先使用,空闲时方供其他业余电台使用
\end{enumerate}
\noindent\textbf{解说:}《业余无线电台管理办法》第二十四条规定:业余中继台应当\textbf{向其覆盖区域内的所有业余无线电台提供平等的服务,并将使用业余中继台所需的各项技术参数公开}。\\\noindent\textbf{答案:}C



\bigskip


\noindent\textbf{试题:}选择144\unit{\MHz}或430\unit{\MHz}业余模拟调频中继台同频段收发频差的原则是:
\begin{enumerate}[leftmargin=3em]
\item 采用业余无线电标准频差,即144\unit{\MHz}频段600\unit{\kHz},430\unit{\MHz}频段5\unit{\MHz}
\item 可以在国家《无线电频率划分规定》所规定业余频率范围内任意选择
\item 采用经常变换频差的办法减少占用度
\item 尽量采用非标准频差以阻止一般业余无线电台占用
\end{enumerate}
\noindent\textbf{解说:}选择144\unit{\MHz}或430\unit{\MHz}业余模拟调频中继台同频段收发频差的原则是:\textbf{采用业余无线电标准频差,即144\unit{\MHz}频段600\unit{\kHz},430\unit{\MHz}频段5\unit{\MHz}}。\\\noindent\textbf{答案:}A
%%%

\bigskip


\noindent\textbf{试题:}关于业余无线电台的应急通信,正确的叙述是:
\begin{enumerate}[leftmargin=3em]
\item 在突发重大自然灾害等紧急情况下,业余无线电台才可以和非业余无线电台进行规定内容的通信
\item 在日常应急通信训练中,业余无线电台可以和地方公益性救援团体的非业余无线电台进行通信
\item 在日常应急通信训练中,业余无线电台可以和地方公益性救援团体的非业余无线电台进行通信,但须经当地业余无线电协会同意
\item 在日常应急通信训练中,业余无线电台可以和各种非业余无线电台进行通信
\end{enumerate}
\noindent\textbf{解说:}《业余无线电台管理办法》第二十条规定:\textbf{在突发重大自然灾害等紧急情况下,业余无线电台可以和非业余无线电台通信},但应当及时向所在地地方无线电管理机构报告,\textbf{其通信内容应当限于与抢险救灾直接相关的紧急事务或者应急救援相关部门交办的任务}。\\\noindent\textbf{答案:}A



\bigskip


\noindent\textbf{试题:}业余无线电台允许与非业余无线电台通信的条件是:
\begin{enumerate}[leftmargin=3em]
\item 在当地政府或非盈利机构组织的公益活动中,内容限于与公益事务或者相关的活动组织机构交办的任务
\item 在突发重大自然灾害等紧急情况下,内容限于与抢险救灾直接相关的紧急事务或者应急救援相关部门交办的任务
\item 在无线电技术研究中,仅可与具备其他业务电台执照的对象通信,内容限于技术实验所需的信号
\item 在青少年科技教育活动中,仅可与青少年非业余无线电台通信,内容限于与青少年科技教育直接有关的事务
\end{enumerate}
\noindent\textbf{解说:}《业余无线电台管理办法》第二十条规定:\textbf{在突发重大自然灾害等紧急情况下,业余无线电台可以和非业余无线电台通信},但应当及时向所在地地方无线电管理机构报告,\textbf{其通信内容应当限于与抢险救灾直接相关的紧急事务或者应急救援相关部门交办的任务}。\\\noindent\textbf{答案:}B



\bigskip


\noindent\textbf{试题:}关于业余无线电台的应急通信,正确的叙述是:
\begin{enumerate}[leftmargin=3em]
\item 在平时的任何时侯,业余无线电台的通信内容可以涉及任何政府组织和非盈利机构交办的任务
\item 在突发重大自然灾害等紧急情况下,业余无线电台的通信内容可以涉及应急救援相关部门交办的任务
\item 在日常公益性社会活动中,业余无线电台的通信内容可以涉及各种公益机构交办的任务
\item 平时在专门的应急通信训练活动中,业余无线电台的通信内容可以涉及应急救援相关部门和组织机构交办的任务
\end{enumerate}
\noindent\textbf{解说:}《业余无线电台管理办法》第二十条规定:\textbf{在突发重大自然灾害等紧急情况下,业余无线电台可以和非业余无线电台通信},但应当及时向所在地地方无线电管理机构报告,\textbf{其通信内容应当限于与抢险救灾直接相关的紧急事务或者应急救援相关部门交办的任务}。\\\noindent\textbf{答案:}B




\bigskip


\noindent\textbf{试题:}法规和国际业余无线电惯例要求业余电台日志记载的必要基本内容是:
\begin{enumerate}[leftmargin=3em]
\item 通信时间、通信频率、双方收发信设备和天线、对方台址
\item 通信时间、通信频率、通信模式、对方呼号、双方信号报告
\item 通信对方姓名、对方所在国家或城市、通信模式、双方信号报告
\item 通信时间、通信模式、对方信号报告、对方台址、对方天气
\end{enumerate}
\noindent\textbf{解说:}《业余无线电台管理办法》第三十条规定:业余无线电台的\textbf{通信时间、通信频率、通信模式和通信对象等}内容应当记入电台日志。\\\noindent\textbf{答案:}B
%%%



\bigskip


\noindent\textbf{试题:}法规和国际业余无线电惯例要求业余电台日志记载的必要基本内容是:
\begin{enumerate}[leftmargin=3em]
\item DATE、FREQ、QTH(对方)、RIG(对方)、RST(双方)、WX(对方)
\item DATE、TIME、FREQ、MODE、CALL(对方)、RST(双方)
\item CALL(通信对方)、T IME、FREQ、RIG(对方)、RST(双方)、PWR(双方)
\item DATE、TIME、MODE、CALL(对方)、QTH(对方)、RST(双方)
\end{enumerate}
\noindent\textbf{解说:}法规和国际业余无线电惯例要求业余电台日志记载的必要基本内容是\textbf{DATE、TIME、FREQ、MODE、CALL(对方)、RST(双方)}。\\\noindent\textbf{答案:}B
%%%


\bigskip


\noindent\textbf{试题:}负责协调国际无线电管理的政府间组织是:
\begin{enumerate}[leftmargin=3em]
\item 联合国大会
\item 国际业余无线电联盟
\item 国际电信联盟
\item 联合国科教文组织
\end{enumerate}
\noindent\textbf{解说:}负责协调国际无线电管理的政府间组织是\textbf{国际电信联盟}。\\\noindent\textbf{答案:}C



\bigskip


\noindent\textbf{试题:}业余中继台的使用原则是:
\begin{enumerate}[leftmargin=3em]
\item 除必要的短暂通信外,应保持业余中继台具有足够的空闲时间,以便随时响应突发灾害应急呼叫
\item 应使中继台尽量处于接近饱和的忙碌状态,提高使用效率
\item 鼓励青少年学生通过中继台交流解题方法和学习心得
\item 鼓励业余无线电民间组织(协会)通过中继台向当地会员发布通知
\end{enumerate}
\noindent\textbf{解说:}业余中继台的使用原则是:\textbf{除必要的短暂通信外,应保持业余中继台具有足够的空闲时间,以便随时响应突发灾害应急呼叫}。\\\noindent\textbf{答案:}A



\bigskip


\noindent\textbf{试题:}填写和邮寄QSL卡片时的正确做法有:
\begin{enumerate}[leftmargin=3em]
\item 填写错误时应划去或使用涂改液覆盖错误内容并加以改正
\item 通过卡片管理局寄出卡片并希望对方回卡时,应在卡片上注明PSE QSL DIRECT
\item 自己的邮寄地址与电台的发射地点不同时,应在QTH栏目内填明详细邮寄地址
\item 迫切需要方卡回寄卡片时,应直接向对方地址邮寄卡片并附加SASE
\end{enumerate}
\noindent\textbf{解说:}正确做法是:\textbf{迫切需要方卡回寄卡片时,应直接向对方地址邮寄卡片并附加SASE}。\\\noindent\textbf{答案:}D



\bigskip


\noindent\textbf{试题:}如果收到国外寄来的QSL卡片中夹带有一张或多张IRC,应该:
\begin{enumerate}[leftmargin=3em]
\item 尽快检查电台日志确认联络的真实性,并通过国内的QSL卡片管理局寄出自己的QSL卡片
\item 尽快检查电台日志确认联络的真实性,并通过邮局直接向对方地址寄出自己的QSL卡片
\item 不必理会
\item 不必确认联络的真实性,尽快通过邮局直接向对方地址寄出自己的QSL卡片
\end{enumerate}
\noindent\textbf{解说:}应该\textbf{尽快检查电台日志确认联络的真实性,并通过邮局直接向对方地址寄出自己的QSL卡片}。\\\noindent\textbf{答案:}B




\bigskip


\noindent\textbf{试题:}关于QSL卡片的正确用法是:
\begin{enumerate}[leftmargin=3em]
\item 不是作为联络或收听证明而交换QSL卡片时,应填上“Eye ball QSO”等有关说明,不应赠送空白卡片
\item 如果在联络中没有听清对方呼号,可以在寄发QSL卡片的对方台名栏中填写对方操作员姓名
\item 出于火腿互相帮助的目的,虽然对方没有联络到自己,也可以发去确认联络的QSL卡片
\item 空白QSL卡片可以当做照片或者名片,任意赠送、交换、散发
\end{enumerate}
\noindent\textbf{解说:}正确用法是:\textbf{不是作为联络或收听证明而交换QSL卡片时,应填上“Eye ball QSO”等有关说明,不应赠送空白卡片}。\\\noindent\textbf{答案:}A

\bigskip


\noindent\textbf{试题:}如果你知道另一个电台的呼号,想要在中继上呼叫他,你应该怎么做?
\begin{enumerate}[leftmargin=3em]
\item 呼叫“break break”,然后说出对方的呼号
\item 呼叫对方的呼号,并报出自己的呼号
\item 呼叫“CQ”三次,然后说出对方的呼号
\item 等待,直到你要呼叫的电台呼叫CQ后,立刻回答他
\end{enumerate}
\noindent\textbf{解说:}应该\textbf{呼叫对方的呼号,并报出自己的呼号}。\\\noindent\textbf{答案:}B


\bigskip


\noindent\textbf{试题:}业余无线电台设置、使用人应当接受下列机构对业余无线电台及其使用情况的监督检查:
\begin{enumerate}[leftmargin=3em]
\item 国家计量监督部门的监督检查
\item 单位或所在居委会、村民委员会、物主委员会的监督检查
\item 业余无线电民间组织的独立监督检查
\item 无线电管理机构或者其委托单位的监督检查
\end{enumerate}
\noindent\textbf{解说:}《业余无线电台管理办法》第二十九条规定:业余无线电台设置人、使用人应当加强自律,接受\textbf{无线电管理机构或者其委托单位的指导、监督和检查}。\\\noindent\textbf{答案:}D



\bigskip


\noindent\textbf{试题:}对擅自设置、使用业余无线电台的单位或个人,国家无线电管理机构或者地方无线电管理机构可以根据其具体情况给予下列处罚:
\begin{enumerate}[leftmargin=3em]
\item 责令停止使用非法设置的电台并作出书面检查;情节严重的,可以并处一千元以下的罚款
\item 劝告拆除非法设置的电台;情节严重的,可以并处警告、查封或者没收设备
\item 警告、查封或者没收设备、没收非法所得;情节严重的,可以并处一千元以上,五千元以下的罚款
\item 责令停止使用非法设置的电台;情节严重的,可以并处警告、查封或者没收设备
\end{enumerate}
\noindent\textbf{解说:}《中华人民共和国无线电管理条例》(1993年9月11日发布)第四十三条规定:对有下列行为之一的单位和个人,国家无线电管理机构或者地方无线电管理机构可以根据具体情况给予\textbf{警告、查封或者没收设备、没收非法所得的处罚;情节严重的,可以并处一千元以上、五千元以下的罚款}或者吊销其电台执照:
\begin{enumerate}[leftmargin=3em, label=]
	\item(一)\textbf{擅自设置、使用无线电台(站)的};
	\item(二)违反本条例规定研制、生产、进口无线电发射设备的;
	\item(三)干扰无线电业务的;
	\item(四)随意变更核定项目,发送和接收与工作无关的信号的;
	\item(五)不遵守频率管理的有关规定,擅自出租、转让频率的。
\end{enumerate}
\textbf{答案:}D

%最新版法律:第七十条 违反本条例规定,未经许可擅自使用无线电频率,或者擅自设置、使用无线电台(站)的,由无线电管理机构责令改正,没收从事违法活动的设备和违法所得,可以并处5万元以下的罚款;拒不改正的,并处5万元以上20万元以下的罚款;擅自设置、使用无线电台(站)从事诈骗等违法活动,尚不构成犯罪的,并处20万元以上50万元以下的罚款。

\bigskip


\noindent\textbf{试题:}业余电台干扰无线电业务的,国家无线电管理机构或者地方无线电管理机构可以根据其具体情况给予设置业余无线电台的单位或个人下列处罚:
\begin{enumerate}[leftmargin=3em]
\item 责令停止使用非法设置的电台并作出书面检查;情节严重的,可以并处一千元以下的罚款
\item 警告、查封或者没收设备、没收非法所得;情节严重的,可以并处一千元以上,五千元以下的罚款
\item 劝告拆除非法设置的电台;情节严重的,可以并处警告、查封或者没收设备
\item 责令停止使用非法设置的电台;情节严重的,可以并处警告、查封或者没收设备
\end{enumerate}
\noindent\textbf{解说:}《中华人民共和国无线电管理条例》(1993年9月11日发布)第四十三条规定:对有下列行为之一的单位和个人,国家无线电管理机构或者地方无线电管理机构可以根据具体情况给予\textbf{警告、查封或者没收设备、没收非法所得的处罚;情节严重的,可以并处一千元以上、五千元以下的罚款}或者吊销其电台执照:
\begin{enumerate}[leftmargin=3em, label=]
	\item(一)擅自设置、使用无线电台(站)的;
	\item(二)违反本条例规定研制、生产、进口无线电发射设备的;
	\item(三)\textbf{干扰无线电业务的};
	\item(四)随意变更核定项目,发送和接收与工作无关的信号的;
	\item(五)不遵守频率管理的有关规定,擅自出租、转让频率的。
\end{enumerate}
\textbf{答案:}B



\bigskip


\noindent\textbf{试题:}业余电台随意变更核定项目、发送和接收与业余无线电无关的信号的,国家无线电管理机构或者地方无线电管理机构可以根据其具体情况给予设置业余无线电台的单位或个人下列处罚:
\begin{enumerate}[leftmargin=3em]
\item 警告、查封或者没收设备、没收非法所得;情节严重的,可以并处一千元以上,五千元以下的罚款
\item 责令停止使用非法设置的电台并作出书面检查;情节严重的,可以并处一千元以下的罚款
\item 责令停止使用非法设置的电台;情节严重的,可以并处警告、查封或者没收设备
\item 劝告拆除非法设置的电台;情节严重的,可以并处警告、查封或者没收设备
\end{enumerate}
\noindent\textbf{解说:}《中华人民共和国无线电管理条例》(1993年9月11日发布)第四十三条规定:对有下列行为之一的单位和个人,国家无线电管理机构或者地方无线电管理机构可以根据具体情况给予\textbf{警告、查封或者没收设备、没收非法所得的处罚;情节严重的,可以并处一千元以上、五千元以下的罚款}或者吊销其电台执照:
\begin{enumerate}[leftmargin=3em, label=]
	\item(一)擅自设置、使用无线电台(站)的;
	\item(二)违反本条例规定研制、生产、进口无线电发射设备的;
	\item(三)干扰无线电业务的;
	\item(四)\textbf{随意变更核定项目,发送和接收与工作无关的信号的};
	\item(五)不遵守频率管理的有关规定,擅自出租、转让频率的。
\end{enumerate}
\textbf{答案:}A



\bigskip


\noindent\textbf{试题:}超出核定范围使用频率或者有其他违反频率管理有关规定的行为的,无线电管理机构可以根据其具体情况给予设置业余无线电台的单位或个人下列处罚:
\begin{enumerate}[leftmargin=3em]
\item 责令限期改正,可以处警告或者三万元以下的罚款
\item 责令限期改正,可以处警告或者一千元以上,五千元以下的罚款
\item 责令限期改正,情节严重的,可以并处警告、查封或者没收设备
\item 责令限期改正,可以处警告或者一千元以下的罚款
\end{enumerate}
\noindent\textbf{解说:}《业余无线电台管理办法》第四十一条规定:对超出核定范围使用频率或者有其他违反频率管理有关规定的行为的,无线电管理机构应当依据职权\textbf{责令限期改正,可以处警告或者三万元以下的罚款}。\\\textbf{答案:}A



\bigskip


\noindent\textbf{试题:}对涂改、仿制、伪造、倒卖、出租、出借业余无线电台执照,或者以其他形式非法转让业余无线电台执照的,无线电管理机构可以给予下列处罚:
\begin{enumerate}[leftmargin=3em]
\item 应当责令限期改正,可以处警告或者一千元以下的罚款
\item 应当责令限期改正,可以处警告或者一千元以上,五千元以下的罚款
\item 应当责令限期改正,情节严重的,可以并处警告、查封或者没收设备
\item 应当责令限期改正,可以处警告或者三万元以下的罚款
\end{enumerate}
\noindent\textbf{解说:}《业余无线电台管理办法》第四十一条规定:对涂改、仿制、伪造业余无线电台执照,或者倒卖、出租、出借及以其他形式非法转让业余无线电台执照的,无线电管理机构\textbf{应当依据职权责令限期改正,可以处警告或者三万元以下的罚款}。\\\textbf{答案:}D



\bigskip


\noindent\textbf{试题:}对盗用、出租、出借、转让、私自编制或者违法使用业余无线电台呼号的,无线电管理机构可以给予下列处罚:
\begin{enumerate}[leftmargin=3em]
\item 应当责令限期改正,情节严重的,可以并处警告、查封或者没收设备
\item 应当责令限期改正,可以处警告或者三万元以下的罚款
\item 应当责令限期改正,可以处警告或者一千元以上,五千元以下的罚款
\item 应当责令限期改正,可以处警告或者一千元以下的罚款
\end{enumerate}
\noindent\textbf{解说:}《业余无线电台管理办法》第四十一条规定:对盗用、出租、出借、转让、私自编制或者违法使用业余无线电台呼号的,无线电管理机构\textbf{应当依据职权责令限期改正,可以处警告或者三万元以下的罚款}。\\\textbf{答案:}B



\bigskip


\noindent\textbf{试题:}对以不正当手段取得业余无线电台执照的,无线电管理机构可以给予下列处罚:
\begin{enumerate}[leftmargin=3em]
\item 责令限期改正,可以处警告或者一千元以下的罚款
\item 责令限期改正,可以处警告或者三万元以下的罚款
\item 责令限期改正,情节严重的,可以并处警告、查封或者没收设备
\item 责令限期改正,可以处警告或者一千元以上,五千元以下的罚款
\end{enumerate}
\noindent\textbf{解说:}《业余无线电台管理办法》第四十一条规定:对以不正当手段取得业余无线电台执照的,无线电管理机构应当依据职权\textbf{责令限期改正,可以处警告或者三万元以下的罚款}。\\\textbf{答案:}B



\bigskip


\noindent\textbf{试题:}对向负责监督检查的无线电管理机构隐瞒有关情况、提供虚假材料或者拒绝提供反映其活动情况的真实材料的,无线电管理机构可以给予下列处罚:
\begin{enumerate}[leftmargin=3em]
\item 责令限期改正,情节严重的,可以并处警告、查封或者没收设备
\item 责令限期改正,可以处警告或者一千元以上,五千元以下的罚款
\item 责令限期改正,可以处警告或者三万元以下的罚款
\item 责令限期改正,可以处警告或者一千元以下的罚款
\end{enumerate}
\noindent\textbf{解说:}《业余无线电台管理办法》第四十一条规定:对向负责监督检查的无线电管理机构隐瞒有关情况、提供虚假材料或者拒绝提供反映其活动情况的真实材料的,无线电管理机构应当依据职权\textbf{责令限期改正,可以处警告或者三万元以下的罚款}。\\\textbf{答案:}C



\bigskip


\noindent\textbf{试题:}对违法使用业余无线电台造成严重后果的,无线电管理机构可以给予下列处罚:
\begin{enumerate}[leftmargin=3em]
\item 应当责令限期改正,可以处警告或者一千元以上,五千元以下的罚款
\item 应当责令限期改正,情节严重的,可以并处警告、查封或者没收设备
\item 应当责令限期改正,可以处警告或者一千元以下的罚款
\item 应当责令限期改正,可以处警告或者三万元以下的罚款
\end{enumerate}
\noindent\textbf{解说:}《业余无线电台管理办法》第四十一条规定:对违法使用业余无线电台造成严重后果的,无线电管理机构应当依据职权\textbf{责令限期改正,可以处警告或者三万元以下的罚款}。\\\textbf{答案:}D



\bigskip

% 更新过的题目
\noindent\textbf{试题:}违反国家规定,擅自设置、使用无线电台(站),或者擅自占用频率,经责令停止使用后拒不停止使用,干扰无线电通信正常进行,造成严重后果的的,可被判犯扰乱无线电通信管理秩序罪,处三年以下有期徒刑、拘役或者管制,并处或者单处罚金。这个规定出自于下列法规律:
\begin{enumerate}[leftmargin=3em]
\item 中华人民共和国无线电管理条例
\item 中华人民共和国电信法
\item 中华人民共和国民法通则
\item 中华人民共和国刑法
\end{enumerate}
\textbf{解说:}《\textbf{中华人民共和国刑法}》第二百八十八条规定:违反国家规定,擅自设置、使用无线电台(站),或者擅自占用频率,经责令停止使用后拒不停止使用,干扰无线电通讯正常进行,造成严重后果的,处三年以下有期徒刑、拘役或者管制,并处或者单处罚金。\\\noindent\textbf{答案:}D



\bigskip


\noindent\textbf{试题:}无线电管制是指在下列范围内依法采取的对无线电波的发射、辐射和传播实施的强制性管理:
\begin{enumerate}[leftmargin=3em]
\item 在全国范围、所有时间内
\item 在特定时间和特定区域内
\item 在例行范围和例行时间内
\item 在特定范围、所有时间内
\end{enumerate}
\noindent\textbf{解说:}《中华人民共和国无线电管制规定》第二条规定:本规定所称无线电管制,是指在\textbf{特定时间和特定区域}内,依法采取限制或者禁止无线电台(站)、无线电发射设备和辐射无线电波的非无线电设备的使用,以及对特定的无线电频率实施技术阻断等措施,对无线电波的发射、辐射和传播实施的强制性管理。\\\noindent\textbf{答案:}B



\bigskip


\noindent\textbf{试题:}无线电管制是指在特定时间和特定区域内,依法采取的下列性质的管理:
\begin{enumerate}[leftmargin=3em]
\item 对无线电波的发射、辐射和传播实施的强制性管理
\item 对无线电波的发射、辐射实施的指导和行业自律性管理
\item 对无线电发射设备的生产、销售实施的指导和行业自律性管理
\item 对无线电发射设备的生产、销售实施的强制性管理
\end{enumerate}
\noindent\textbf{解说:}《中华人民共和国无线电管制规定》第二条规定:本规定所称无线电管制,是指在特定时间和特定区域内,依法采取限制或者禁止无线电台(站)、无线电发射设备和辐射无线电波的非无线电设备的使用,以及对特定的无线电频率实施技术阻断等措施,\textbf{对无线电波的发射、辐射和传播实施的强制性管理}。\\\noindent\textbf{答案:}A



\bigskip


\noindent\textbf{试题:}在特定时间和特定区域内实施无线电管制时,与业余无线电有关的管理措施包括:
\begin{enumerate}[leftmargin=3em]
\item 限制或者禁止业余无线电台设备的生产和销售
\item 限制或者禁止业余无线电台(站)的使用,以及对特定的无线电频率实施技术阻断等
\item 依法设置的业余电台不在管制范围之内
\item 限制、但不会禁止业余无线电台(站)的使用
\end{enumerate}
\noindent\textbf{解说:}《中华人民共和国无线电管制规定》第二条规定:本规定所称无线电管制,是指在特定时间和特定区域内,依法采取\textbf{限制或者禁止无线电台(站)、无线电发射设备和辐射无线电波的非无线电设备的使用,以及对特定的无线电频率实施技术阻断等}措施,对无线电波的发射、辐射和传播实施的强制性管理。\\\noindent\textbf{答案:}B


\bigskip


\noindent\textbf{试题:}决定实施无线电管制的机构为:
\begin{enumerate}[leftmargin=3em]
\item 在地、市、县实施,由地、市、县人民政府决定
\item 在全国范围内或者跨省、自治区、直辖市实施,由国务院和中央军事委员会决定。在省、自治区、直辖市范围内实施,由省、自治区、直辖市人民政府和相关军区决定
\item 在全国范围内或者跨省、自治区、直辖市实施,由国家无线电管理机构决定。在省、自治区、直辖市范围内实施,由相关地方无线电管理机构决定
\item 在单位、居民区实施,由单位上级业务主管机构和区人民政府共同决定
\end{enumerate}
\noindent\textbf{解说:}《中华人民共和国无线电管制规定》第三条规定:\textbf{在全国范围内或者跨省、自治区、直辖市实施无线电管制,由国务院和中央军事委员会决定。在省、自治区、直辖市范围内实施无线电管制,由省、自治区、直辖市人民政府和相关军区决定},并报国务院和中央军事委员会备案。\\\noindent\textbf{答案:}B



\bigskip


\noindent\textbf{试题:}违反无线电管制命令和无线电管制指令的,由下列机构依法进行处罚:
\begin{enumerate}[leftmargin=3em]
\item 所在军区派出的专门机构
\item 城管、工商、交通联合执法
\item 当地业余无线电协会
\item 国家无线电管理机构或者省、自治区、直辖市无线电管理机构;违反治安管理规定者由公安机关处罚
\end{enumerate}
\noindent\textbf{解说:}《中华人民共和国无线电管制规定》第十二条规定:违反无线电管制命令和无线电管制指令的,由\textbf{国家无线电管理机构或者省、自治区、直辖市无线电管理机构责令改正;拒不改正的,可以关闭、查封、暂扣或者拆除相关设备;情节严重的,吊销无线电台(站)执照和无线电频率使用许可证;违反治安管理规定的,由公安机关依法给予处罚}。\\\noindent\textbf{答案:}D



\bigskip


\noindent\textbf{试题:}业余电台违反无线电管制命令和无线电管制指令的,可以依法规受到下列处罚:
\begin{enumerate}[leftmargin=3em]
\item  处警告或者三万元以下的罚款
\item  责令改正;并开除业余无线电协会会籍、罚没无线电通信设备
\item  处警告或者一千元以上,五千元以下的罚款
\item  责令改正;拒不改正的,关闭、查封、暂扣或者拆除相关设备;情节严重的,吊销电台执照;违反治安管理规定的,由公安机关处罚
\end{enumerate}
\noindent\textbf{解说:}《中华人民共和国无线电管制规定》第十二条规定:违反无线电管制命令和无线电管制指令的,由国家无线电管理机构或者省、自治区、直辖市无线电管理机构\textbf{责令改正;拒不改正的,可以关闭、查封、暂扣或者拆除相关设备;情节严重的,吊销无线电台(站)执照和无线电频率使用许可证;违反治安管理规定的,由公安机关依法给予处罚}。\\\noindent\textbf{答案:}D



\bigskip


\noindent\textbf{试题:}B类业余无线电台允许发射的发射频率为:
\begin{enumerate}[leftmargin=3em]
\item 限于HF范围内的各业余业务和卫星业余业务频段
\item 各VHF和UHF频段
\item 30-3000\unit{\MHz}范围外的各业余业务和卫星业余业务频段
\item 各业余业务和卫星业余业务频段
\end{enumerate}
\noindent\textbf{解说:}《工业和信息化部关千实施〈业余无线电台管理办法〉若干事项的通知(工信部无〔2013〕43号)》第一条第一款规定:B类业余无线电台可以在\textbf{各业余业务和卫星业余业务频段}内发射工作。\\\noindent\textbf{答案:}D


\bigskip


\noindent\textbf{试题:}B类业余无线电台允许发射的最大发射功率为不大于:
\begin{enumerate}[leftmargin=3em]
\item 25瓦
\item 100瓦
\item 30\unit{\MHz}以上业余频段不大于100瓦,30\unit{\MHz}以下业余频段不大于25瓦
\item 30\unit{\MHz}以下业余频段不大于100瓦,30\unit{\MHz}以上业余频段不大于25瓦
\end{enumerate}
\noindent\textbf{解说:}《工业和信息化部关千实施〈业余无线电台管理办法〉若干事项的通知(工信部无〔2013〕43号)》第一条第一款规定:B类业余无线电台可以在业余业务和卫星业余业务频段内发射工作,\textbf{30\unit{\MHz}以下频段最大发射功率不大于100瓦,30\unit{\MHz}以上频段最大发射功率不大于25瓦}。\\\noindent\textbf{答案:}D


\bigskip


\noindent\textbf{试题:}负责组织A类和B类业余无线电台所需操作技术能力的验证的机构是:
\begin{enumerate}[leftmargin=3em]
\item 地方教育、体育机构及其相关民间组织
\item 地方无线电管理机构(或其委托单位)
\item 地方业余无线电协会
\item 国家无线电管理机构和地方无线电管理机构(或其委托单位)
\end{enumerate}
\noindent\textbf{解说:}《工业和信息化部关于实施〈业余无线电台管理办法〉若干事项的通知(工信部无〔2013〕43号)》第四条规定:各\textbf{省、自治区、直辖市无线电管理机构或其委托的机构}负责组织对A类和B类业余无线电台所需操作技术能力进行验证,以及核发验证合格证明等工作。\\《业余无线电台管理办法》第三条规定:省、自治区、直辖市无线电管理机构简称\textbf{地方无线电管理机构}。\\\noindent\textbf{答案:}B

% 第一章 无线电的相关法律 结束








% 第二章 无线电通信的方法 开始

\chapter{无线电通信的方法}

\newpage

\noindent\textbf{试题:}自制业余无线电发射设备,在经无线电检测机构检测合格并取得电台执照之前,调试时天线输出端应连接(或串联必要的仪表后连接):
\begin{enumerate}[leftmargin=3em]
\item VSWR严格等于1:1的驻波天线
\item 测试专用的标准环形天线
\item VSWR严格等于1:1的行波天线
\item 假负载
\end{enumerate}
\noindent\textbf{解说:}自制业余无线电发射设备,在经无线电检测机构检测合格并取得电台执照之前,调试时天线输出端应连接(或串联必要的仪表后连接)\textbf{假负载}。\\\noindent\textbf{答案:}D



\bigskip


\noindent\textbf{试题:}某俱乐部约定了一个成员业余电台之间交流技术的网络频率,当遇有其他业余电台按通信惯例要求参加通信时,处理原则应为:
\begin{enumerate}[leftmargin=3em]
\item 由俱乐部网络控制台决定是其他业余电台是否可以加入
\item 无条件欢迎加入,因为任何核准的业余电台对频率享有平等的频率使用权
\item 要求其他业余电台在任何时间都不得使用俱乐部自己约定的专用通信频率
\item 要求其他业余电台在俱乐部成员结束网络通信后再使用该频率
\end{enumerate}
\noindent\textbf{解说:}《业余无线电台管理办法》第十九条规定:\textbf{业余无线电台在无线电管理机构核准其使用的频段内,享有平等的频率使用权}。因此处理原则应为\textbf{无条件欢迎加入}。\\\noindent\textbf{答案:}B



\bigskip


\noindent\textbf{试题:}业余电台在发起呼叫前不可缺少的操作步骤是:
\begin{enumerate}[leftmargin=3em]
\item 先用吹话筒、吹口哨等方法发出连续信号检查天线驻波比
\item 先守听一段时间,确保没有其他电台正在使用频率
\item 检查发射功率是否达到设备的额定输出功率
\item 先用礼貌的语言请其他电台让出频率
\end{enumerate}
\noindent\textbf{解说:}业余电台在发起呼叫前不可缺少的操作步骤是\textbf{先守听一段时间,确保没有其他电台正在使用频率}。\\\noindent\textbf{答案:}B




\bigskip


\noindent\textbf{试题:}业余电台在发射调试信号进行发射功率和天线驻波比等检查时必须注意做到的是:
\begin{enumerate}[leftmargin=3em]
\item 先将收发信机的语音压缩功能打开
\item 先将频率设置到无人使用的空闲频率、偏离常用的热点频率
\item 话筒离嘴距离在2公分以上,电键按键时间不短于5秒钟
\item 先将天线的发射方向指向正北
\end{enumerate}
\noindent\textbf{解说:}业余电台在发射调试信号进行发射功率和天线驻波比等检查时必须注意做到的是\textbf{先将频率设置到无人使用的空闲频率、偏离常用的热点频率}。\\\noindent\textbf{答案:}B



\bigskip


\noindent\textbf{试题:}单边带业余电台在测试检查天线驻波比需要发射平稳的连续信号。文明的作法是:
\begin{enumerate}[leftmargin=3em]
\item 先将电台设为CW方式按电键,或者设为AM或FM方式按PTT键(不对话筒说话),产生连续载波,测试结束后设回SSB方式
\item 将电台设为SSB方式,用平稳的气流对话筒吹口哨
\item 将电台设为SSB方式,深呼吸后用平稳的气流对话筒发长音“啊”
\item 将电台设为SSB方式,深呼吸后用平稳的气流对话筒发长音“嘻”
\end{enumerate}
\noindent\textbf{解说:}文明的作法是:\textbf{先将电台设为CW方式按电键,或者设为AM或FM方式按PTT键(不对话筒说话),产生连续载波,测试结束后设回SSB方式}。\\\noindent\textbf{答案:}A



\bigskip


\noindent\textbf{试题:}业余电台发起呼叫前应先守听一段时间,如没有听到信号,应再询问“有人使用频率吗”?确认没有应答方能发起呼叫。下列英语短句中不能正确表达这一询问的是:
\begin{enumerate}[leftmargin=3em]
\item Is the frequency in use?
\item Calling you, Roger?%出题人就不能找母语使用者出个不要这样Chinenlish的错误选项么
\item Is any body in the frequency?
\item Any body here?
\end{enumerate}
\noindent\textbf{解说:}不能正确表达这一询问的是\textbf{Calling you, Roger?}。\\\noindent\textbf{答案:}B



\bigskip


\noindent\textbf{试题:}业余电台发起呼叫前应先守听一段时间,如没有听到信号,应再询问“有人使用频率吗”?确认没有应答方能发起呼叫。用CW表达这一询问的方法是:
\begin{enumerate}[leftmargin=3em]
\item QRL?
\item QRX?
\item QRZ?
\item QRV?
\end{enumerate}
\noindent\textbf{解说:}用CW表达这一询问的方法是\textbf{QRL?}。\\\noindent\textbf{答案:}A


\bigskip


\noindent\textbf{试题:}业余电台BH1ZZZ用话音发起CQ呼叫的正确格式为:
\begin{enumerate}[leftmargin=3em]
\item CQ、CQ、CQ。BH1ZZZ呼叫。Bravo Hotel One Zulu Zulu Zulu呼叫,BH1ZZZ呼叫。听到请回答。
\item CQ、CQ、CQ,CQ、CQ、CQ,CQ、CQ、CQ。BH1ZZZ呼叫。请过来。
\item CQ、CQ、CQ。听到请回答。
\item CQ、CQ、CQ。我是1ZZZ。听到请回答
\end{enumerate}
\noindent\textbf{解说:}业余电台BH1ZZZ用话音发起CQ呼叫的正确格式为\textbf{CQ、CQ、CQ。BH1ZZZ呼叫。Bravo Hotel One Zulu Zulu Zulu呼叫,BH1ZZZ呼叫。听到请回答。}\\\noindent\textbf{答案:}A



\bigskip


\noindent\textbf{试题:}业余电台BH1ZZZ用话音发起CQ呼叫的正确格式为:
\begin{enumerate}[leftmargin=3em]
\item CQ CQ CQ, CQ CQ CQ, CQ CQ CQ. This is BH1ZZZ. Back to you.
\item CQ CQ CQ. This One Zulu Zulu Zulu. Over.
\item CQ CQ CQ. Go ahead please.
\item CQ CQ CQ.This is BH1ZZZ. Bravo Hotel One Zulu Zulu Zulu, BH1ZZZ is calling. I’m standing by.
\end{enumerate}
\noindent\textbf{解说:}业余电台BH1ZZZ用话音发起CQ呼叫的正确格式为\textbf{CQ CQ CQ.This is BH1ZZZ. Bravo Hotel One Zulu Zulu Zulu, BH1ZZZ is calling. I’m standing by.}。\\\noindent\textbf{答案:}D



\bigskip


\noindent\textbf{试题:}业余电台BH1ZZZ用话音呼叫BH8YYY的正确格式为:
\begin{enumerate}[leftmargin=3em]
\item BH8YYY。我是BH1ZZZ,我是BH1ZZZ,我是BH1ZZZ。听到请回答。
\item 8YYY、8YYY、8YYY。BH1ZZZ呼叫。请过来。
\item BH8YYY、BH8YYY、BH8YYY。我是1ZZZ。听到请回答
\item BH8YYY、BH8YYY、BH8YYY。BH1ZZZ呼叫。Bravo Hotel One Zulu Zulu Zulu,BH1ZZZ呼叫。听到请回答。
\end{enumerate}
\noindent\textbf{解说:}业余电台BH1ZZZ用话音呼叫BH8YYY的正确格式为\textbf{BH8YYY、BH8YYY、BH8YYY。BH1ZZZ呼叫。Bravo Hotel One Zulu Zulu Zulu,BH1ZZZ呼叫。听到请回答。}。\\\noindent\textbf{答案:}D


\bigskip


\noindent\textbf{试题:}业余电台BH1ZZZ用话音呼叫BH8YYY的正确格式为:
\begin{enumerate}[leftmargin=3em]
\item 8YYY, 8YYY, YYY. This is BH1ZZZ. Over.
\item Bravo Hotel Eight Yankee Yankee Yankee, Bravo Hotel Eight Yankee Yankee Yankee, Bravo Hotel Eight Yankee Yankee Yankee. Go ahead please.
\item Bravo Hotel Eight Yankee Yankee Yankee, Bravo Hotel Eight Yankee Yankee Yankee, Bravo Hotel Eight Yankee Yankee Yankee.This is Bravo Hotel One Zulu Zulu Zulu. Bravo Hotel One Zulu Zulu Zulu, Bravo Hotel One Zulu Zulu Zulu is calling. I’m standing by.
\item BH8YYY, BH8YYY, BH8YYY. This One Zulu Zulu Zulu. Come in please.
\end{enumerate}
\noindent\textbf{解说:}业余电台BH1ZZZ用话音呼叫BH8YYY的正确格式为\textbf{Bravo Hotel Eight Yankee Yankee Yankee, Bravo Hotel Eight Yankee Yankee Yankee, Bravo Hotel Eight Yankee Yankee Yankee.This is Bravo Hotel One Zulu Zulu Zulu. Bravo Hotel One Zulu Zulu Zulu, Bravo Hotel One Zulu Zulu Zulu is calling. I’m standing by.}。\\\noindent\textbf{答案:}C




\bigskip


\noindent\textbf{试题:}BH1ZZZ希望加入两个电台正在通信中的谈话,正确的方法为:
\begin{enumerate}[leftmargin=3em]
\item 只要双方都是自己熟悉的业余电台操作员,可直接插入谈话,不必拘泥礼节
\item 在一方正在发射期间,短暂插入一次“Break in”,向正在收听的一方发出插入请求
\item 短暂发射一次“Break in!”或“插入!”,如对方无反应,应加大功率反复作此发射
\item 在双方对话的间隙,短暂发射一次“Break in!”或“插入!”,如得到响应,再说明本台呼号 “BH1ZZZ请求插入”,等对方正式表示邀请后,方能加入
\end{enumerate}
\noindent\textbf{解说:}正确的方法为\textbf{在双方对话的间隙,短暂发射一次“Break in!”或“插入!”,如得到响应,再说明本台呼号 “BH1ZZZ请求插入”,等对方正式表示邀请后,方能加入}。\\\noindent\textbf{答案:}D



\bigskip


\noindent\textbf{试题:}以请求插入的方式加入两个电台正在通信中的谈话,应满足的起码条件是:
\begin{enumerate}[leftmargin=3em]
\item 自己的信号质量不亚于原通信双方
\item 自己拥有比原通信双方更有吸引力的谈话内容
\item 自己的操作技巧不亚于原通信双方
\item 确认自己的加入不会影响原通信双方的乐趣
\end{enumerate}
\noindent\textbf{解说:}以请求插入的方式加入两个电台正在通信中的谈话,应满足的起码条件是\textbf{确认自己的加入不会影响原通信双方的乐趣}。\\\noindent\textbf{答案:}D



\bigskip


\noindent\textbf{试题:}参加DX网络通信有助于与一些稀有电台建立通信。正确做法是:
\begin{enumerate}[leftmargin=3em]
\item 事前了解网络规则,未经主控台允许不能随意发起呼叫,根据主控台要求进行登录,然后需随时注意主控台的安排,在主控台安排DX电台呼叫自己时及时回答联络
\item 当两个电台在网络主控台安排下互联联络时,自己可以通过Break in插入通信
\item 听到DX网络通信后,应抓住机会立即对听到的电台发起呼叫
\item DX网络时间内肯定有很多DX电台在守听,利用该频点呼叫CQ定有收获
\end{enumerate}
\noindent\textbf{解说:}正确做法是\textbf{事前了解网络规则,未经主控台允许不能随意发起呼叫,根据主控台要求进行登录,然后需随时注意主控台的安排,在主控台安排DX电台呼叫自己时及时回答联络}。\\\noindent\textbf{答案:}A



\bigskip


\noindent\textbf{试题:}14022\si{\kHz}有很多电台争相报出自己的呼号,原来是想呼叫发射频率为14020\si{\kHz}的某稀有台。如要加入对该稀有台的呼叫,应该:
\begin{enumerate}[leftmargin=3em]
\item 守听14020\si{\kHz},在稀有台结束和其他电台联络或者呼叫CQ和QRZ时,在14022\si{\kHz}快速准确地发送自己的呼号
\item 在14020\si{\kHz}呼叫该稀有台
\item 守听14020\si{\kHz},在稀有台结束和其他电台联络或者呼叫CQ和QRZ时,在14020\si{\kHz}快速准确地发送自己的呼号
\item 在14022\si{\kHz}不断发送自己的呼号
\end{enumerate}
\noindent\textbf{解说:}如要加入对该稀有台的呼叫,应该\textbf{守听14020\si{\kHz},在稀有台结束和其他电台联络或者呼叫CQ和QRZ时,在14022\si{\kHz}快速准确地发送自己的呼号}。\\\noindent\textbf{答案:}A




\bigskip


\noindent\textbf{试题:}业余电台之间进行通信,必须相互正确发送和接收的信息为:
\begin{enumerate}[leftmargin=3em]
\item 对方呼号、信号报告、设备情况
\item 本台呼号、对方呼号、信号报告
\item 本台呼号、对方呼号、QTH
\item 本台呼号、信号报告、QTH
\end{enumerate}
\noindent\textbf{解说:}业余电台之间进行通信,必须相互正确发送和接收的信息为\textbf{本台呼号、对方呼号、信号报告}。\\\noindent\textbf{答案:}B




\bigskip


\noindent\textbf{试题:}法规要求业余电台在通信建立及结束时主动发送本台呼号。允许用发送呼号的一部分来代替发送完整呼号的情况是:
\begin{enumerate}[leftmargin=3em]
\item 在VHF/UHF频段进行本地呼叫时可以仅使用呼号后缀作为电台标识
\item 在熟悉的友台之间呼叫可以仅使用呼号后缀作为电台标识
\item 在HF频段进行国内呼叫时可以仅使用呼号后缀作为电台标识
\item 在任何情况下都必须用完整呼号作为电台标识
\end{enumerate}
\noindent\textbf{解说:}\textbf{在任何情况下都必须用完整呼号作为电台标识}。\\\noindent\textbf{答案:}D



\bigskip


\noindent\textbf{试题:}如何回答一个CQ呼叫?
\begin{enumerate}[leftmargin=3em]
\item 先报出对方的呼号,再报出自己的呼号
\item 说:“CQ”,并报出对方的呼号
\item 先给出信号报告,再报出自己的呼号
\item 先报出自己的呼号,再报出对方的呼号
\end{enumerate}
\noindent\textbf{解说:}回答一个CQ呼叫的程序是:\textbf{先报出对方的呼号,再报出自己的呼号}。\\\noindent\textbf{答案:}A



\bigskip


\noindent\textbf{试题:}当一部电台在呼叫CQ时,他的意思是?
\begin{enumerate}[leftmargin=3em]
\item 只有被呼叫的电台可以回答,其他人不能回答
\item 此电台正在测试天线,不需要任何电台回答这个呼叫
\item 非特指地呼叫任何一部电台
\item 呼叫重庆的电台
\end{enumerate}
\noindent\textbf{解说:}当一部电台在呼叫CQ时,他的意思是\textbf{非特指地呼叫任何一部电台}。\\\noindent\textbf{答案:}C




\bigskip


\noindent\textbf{试题:}如果其他电台报告你在2米波段的信号刚才非常强,但是突然变弱或不可辨,这时你应当怎么做?
\begin{enumerate}[leftmargin=3em]
\item 将你电台中的镍氢电池换成锂电池
\item 打开哑音发射功能
\item 稍稍移动一下自己的位置,有时信号无规律反射造成的多径效应可能导致失真
\item 请对方电台调整自己的静噪设置
\end{enumerate}
\noindent\textbf{解说:}这时应当\textbf{稍稍移动一下自己的位置,有时信号无规律反射造成的多径效应可能导致失真}。\\\noindent\textbf{答案:}C



\bigskip


\noindent\textbf{试题:}下列哪种方式可以让你快速切换到一个你经常使用的频率?
\begin{enumerate}[leftmargin=3em]
\item 使用快速扫描模式来切换到那个频率
\item 关闭哑音输出
\item 将这个频率作为一个频道存储在电台中
\item 打开哑音输出
\end{enumerate}
\noindent\textbf{解说:}\textbf{将这个频率作为一个频道存储在电台中}后,可以让你快速切换到一个你经常使用的频率。\\\noindent\textbf{答案:}C



\bigskip


\noindent\textbf{试题:}“你和我还有事吗”的业余无线电通信Q简语为:
\begin{enumerate}[leftmargin=3em]
\item QRL?
\item QRU
\item QRU?
\item QRB?
\end{enumerate}
\noindent\textbf{解说:}“你和我还有事吗”的业余无线电通信Q简语为\textbf{QRU?}。\\\noindent\textbf{答案:}C


\bigskip


\noindent\textbf{试题:}“我和你无事了”的业余无线电通信Q简语为:
\begin{enumerate}[leftmargin=3em]
\item QRU?
\item QRS
\item QRL
\item QRU
\end{enumerate}
\noindent\textbf{解说:}“我和你无事了”的业余无线电通信Q简语为\textbf{QRU}。\\\noindent\textbf{答案:}D



\bigskip


\noindent\textbf{试题:}“谁在呼叫我”的业余无线电通信Q简语为:
\begin{enumerate}[leftmargin=3em]
\item QRZ?
\item QRZ
\item QSL?
\item QRA?
\end{enumerate}
\noindent\textbf{解说:}“谁在呼叫我”的业余无线电通信Q简语为\textbf{QRZ?}。\\\noindent\textbf{答案:}A

\bigskip


\noindent\textbf{试题:}“要我增加功率吗”的业余无线电通信Q简语为:
\begin{enumerate}[leftmargin=3em]
\item QRO?
\item QRS?
\item QRO
\item QSO?
\end{enumerate}
\noindent\textbf{解说:}“要我增加功率吗”的业余无线电通信Q简语为\textbf{QRO?}。\\\noindent\textbf{答案:}A


\bigskip


\noindent\textbf{试题:}“要我减小功率吗”的业余无线电通信Q简语为:
\begin{enumerate}[leftmargin=3em]
\item QRP?
\item QSP?
\item QRS?
\item QRP
\end{enumerate}
\noindent\textbf{解说:}“要我减小功率吗”的业余无线电通信Q简语为\textbf{QRP?}。\\\noindent\textbf{答案:}A


\bigskip


\noindent\textbf{试题:}“我能直接和×××电台通信”的业余无线电通信Q简语为:
\begin{enumerate}[leftmargin=3em]
\item QSO ×××
\item QRV ×××
\item QSP ×××
\item QRU ×××
\end{enumerate}
\noindent\textbf{解说:}“我能直接和×××电台通信”的业余无线电通信Q简语为\textbf{QSO ×××}。\\\noindent\textbf{答案:}A


\bigskip


\noindent\textbf{试题:}“你能直接和×××电台通信吗”的业余无线电通信Q简语为:
\begin{enumerate}[leftmargin=3em]
\item QSO ××× ?
\item QRV ××× ?
\item QRT ××× ?
\item QRL ××× ?
\end{enumerate}
\noindent\textbf{解说:}“你能直接和×××电台通信吗”的业余无线电通信Q简语为\textbf{QSO ××× ?}。\\\noindent\textbf{答案:}A



\bigskip


\noindent\textbf{试题:}“我遇到他台干扰”的业余无线电通信Q简语为:
\begin{enumerate}[leftmargin=3em]
\item QSM
\item QRM
\item QRN
\item QSB?
\end{enumerate}
\noindent\textbf{解说:}“我遇到他台干扰”的业余无线电通信Q简语为\textbf{QRM}。\\\noindent\textbf{答案:}B


\bigskip


\noindent\textbf{试题:}“你遇到他台干扰吗”的业余无线电通信Q简语为:
\begin{enumerate}[leftmargin=3em]
\item QSD?
\item QSN?
\item QSM?
\item QRM?
\end{enumerate}
\noindent\textbf{解说:}“你遇到他台干扰吗”的业余无线电通信Q简语为\textbf{QRM?}。\\\noindent\textbf{答案:}D



\bigskip


\noindent\textbf{试题:}“你遇到天电干扰吗”的业余无线电通信Q简语为:
\begin{enumerate}[leftmargin=3em]
\item QSN?
\item QRN?
\item QRV?
\item QSM?
\end{enumerate}
\noindent\textbf{解说:}“你遇到天电干扰吗”的业余无线电通信Q简语为\textbf{QRN?}。\\\noindent\textbf{答案:}B



\bigskip


\noindent\textbf{试题:}“要我加快发送速度吗”的业余无线电通信Q简语为:
\begin{enumerate}[leftmargin=3em]
\item QSQ?
\item QRS?
\item QRT?
\item QRQ?
\end{enumerate}
\noindent\textbf{解说:}“要我加快发送速度吗”的业余无线电通信Q简语为\textbf{QRQ?}。\\\noindent\textbf{答案:}D



\bigskip


\noindent\textbf{试题:}“请加快发送速度”的业余无线电通信Q简语为:
\begin{enumerate}[leftmargin=3em]
\item QSQ
\item QSV
\item QSM
\item QRQ
\end{enumerate}
\noindent\textbf{解说:}“请加快发送速度”的业余无线电通信Q简语为\textbf{QRQ}。\\\noindent\textbf{答案:}D



\bigskip


\noindent\textbf{试题:}“要我减慢发送速度吗”的业余无线电通信Q简语为:
\begin{enumerate}[leftmargin=3em]
\item QRS?
\item QRT?
\item QRQ?
\item QSQ?
\end{enumerate}
\noindent\textbf{解说:}“要我减慢发送速度吗”的业余无线电通信Q简语为\textbf{QRS?}。\\\noindent\textbf{答案:}A




\bigskip


\noindent\textbf{试题:}“请减慢发送速度”的业余无线电通信Q简语为:
\begin{enumerate}[leftmargin=3em]
\item QSM
\item QSV
\item QRS
\item QRO
\end{enumerate}
\noindent\textbf{解说:}“请减慢发送速度”的业余无线电通信Q简语为\textbf{QRS}。\\\noindent\textbf{答案:}C




\bigskip


\noindent\textbf{试题:}“你是否已准备好”的业余无线电通信Q简语为:
\begin{enumerate}[leftmargin=3em]
\item QRU?
\item QRV?
\item QSV?
\item QRL?
\end{enumerate}
\noindent\textbf{解说:}“你是否已准备好”的业余无线电通信Q简语为\textbf{QRV?}。\\\noindent\textbf{答案:}B




\bigskip


\noindent\textbf{试题:}“我已准备好”的业余无线电通信Q简语为:
\begin{enumerate}[leftmargin=3em]
\item QSV
\item QRV
\item QRL
\item QRU
\end{enumerate}
\noindent\textbf{解说:}“我已准备好”的业余无线电通信Q简语为\textbf{QRV}。\\\noindent\textbf{答案:}B



\bigskip


\noindent\textbf{试题:}“要我停止发送吗”的业余无线电通信Q简语为:
\begin{enumerate}[leftmargin=3em]
\item QSX
\item QRT?
\item QRT
\item QST?
\end{enumerate}
\noindent\textbf{解说:}“要我停止发送吗”的业余无线电通信Q简语为\textbf{QRT?}。\\\noindent\textbf{答案:}B



\bigskip


\noindent\textbf{试题:}“请停止发送”的业余无线电通信Q简语为:
\begin{enumerate}[leftmargin=3em]
\item QRV
\item QSB
\item QRT
\item QST
\end{enumerate}
\noindent\textbf{解说:}“请停止发送”的业余无线电通信Q简语为\textbf{QRT}。\\\noindent\textbf{答案:}C



\bigskip


\noindent\textbf{试题:}“我的信号有衰落吗”的业余无线电通信Q简语为:
\begin{enumerate}[leftmargin=3em]
\item QRB?
\item QSP?
\item QSD?
\item QSB?
\end{enumerate}
\noindent\textbf{解说:}“我的信号有衰落吗”的业余无线电通信Q简语为\textbf{QSB?}。\\\noindent\textbf{答案:}D



\bigskip


\noindent\textbf{试题:}“你的信号有衰落”的业余无线电通信Q简语为:
\begin{enumerate}[leftmargin=3em]
\item QSP
\item QSX
\item QSB
\item QRE
\end{enumerate}
\noindent\textbf{解说:}“你的信号有衰落”的业余无线电通信Q简语为\textbf{QSB}。\\\noindent\textbf{答案:}C



\bigskip


\noindent\textbf{试题:}“我给你收据(QSL卡片)、我已收妥”的业余无线电通信Q简语为:
\begin{enumerate}[leftmargin=3em]
\item QRV
\item QSX
\item QSL
\item QRG
\end{enumerate}
\noindent\textbf{解说:}“我给你收据(QSL卡片)、我已收妥”的业余无线电通信Q简语为\textbf{QSL}。\\\noindent\textbf{答案:}C

\bigskip


\noindent\textbf{试题:}“我的电台位置是××××”的业余无线电通信Q简语为:
\begin{enumerate}[leftmargin=3em]
\item QRD ××××
\item QSP ××××
\item QTH ××××
\item QSL ××××
\end{enumerate}
\noindent\textbf{解说:}“我的电台位置是××××”的业余无线电通信Q简语为\textbf{QTH ××××}。\\\noindent\textbf{答案:}C


\bigskip


\noindent\textbf{试题:}“我遇到天电干扰”的业余无线电通信Q简语为:
\begin{enumerate}[leftmargin=3em]
\item QRM
\item QST
\item QRN
\item QSN
\end{enumerate}
\noindent\textbf{解说:}“我遇到天电干扰”的业余无线电通信Q简语为\textbf{QRN}。\\\noindent\textbf{答案:}C



\bigskip


\noindent\textbf{试题:}“我发报的手法有毛病吗”的业余无线电通信Q简语为:
\begin{enumerate}[leftmargin=3em]
\item QSV?
\item QSB?
\item QSD?
\item QRT?
\end{enumerate}
\noindent\textbf{解说:}“我发报的手法有毛病吗”的业余无线电通信Q简语为\textbf{QSD?}。\\\noindent\textbf{答案:}C




\bigskip


\noindent\textbf{试题:}“你发报的手法有毛病”的业余无线电通信Q简语为:
\begin{enumerate}[leftmargin=3em]
\item QSD
\item QRC
\item QSU
\item QSK
\end{enumerate}
\noindent\textbf{解说:}“你发报的手法有毛病”的业余无线电通信Q简语为\textbf{QSD}。\\\noindent\textbf{答案:}A



\bigskip


\noindent\textbf{试题:}“你正忙着吗”的业余无线电通信Q简语为:
\begin{enumerate}[leftmargin=3em]
\item QRU?
\item QSU?
\item QRL?
\item QRX?
\end{enumerate}
\noindent\textbf{解说:}“你正忙着吗”的业余无线电通信Q简语为\textbf{QRL?}。\\\noindent\textbf{答案:}C



\bigskip


\noindent\textbf{试题:}“我正忙着”的业余无线电通信Q简语为:
\begin{enumerate}[leftmargin=3em]
\item QRL
\item QRB
\item QSV
\item QRX
\end{enumerate}
\noindent\textbf{解说:}“我正忙着”的业余无线电通信Q简语为\textbf{QRL}。\\\noindent\textbf{答案:}A



\bigskip


\noindent\textbf{试题:}“能在你的信号间隙中接收吗(即QSK插入方式)”的业余无线电通信Q简语为:
\begin{enumerate}[leftmargin=3em]
\item QSK?
\item QSG?
\item QRJ?
\item QRF?
\end{enumerate}
\noindent\textbf{解说:}“能在你的信号间隙中接收吗(即QSK插入方式)”的业余无线电通信Q简语为\textbf{QSK?}。\\\noindent\textbf{答案:}A



\bigskip


\noindent\textbf{试题:}“我在发射的信号间隙中接收(即QSK插入方式)”的业余无线电通信Q简语为:
\begin{enumerate}[leftmargin=3em]
\item QRE
\item QSK
\item QSM
\item QRK
\end{enumerate}
\noindent\textbf{解说:}“我在发射的信号间隙中接收(即QSK插入方式)”的业余无线电通信Q简语为\textbf{QSK}。\\\noindent\textbf{答案:}B



\bigskip


\noindent\textbf{试题:}“你能给我收据(或QSL卡片)吗”的业余无线电通信Q简语为:
\begin{enumerate}[leftmargin=3em]
\item QSA?
\item QRL?
\item QSL?
\item QSD?
\end{enumerate}
\noindent\textbf{解说:}“你能给我收据(或QSL卡片)吗”的业余无线电通信Q简语为\textbf{QSL?}。\\\noindent\textbf{答案:}C




\bigskip


\noindent\textbf{试题:}“你能传信到×××电台吗”的业余无线电通信Q简语为:
\begin{enumerate}[leftmargin=3em]
\item QSX ×××?
\item QRD ×××?
\item QSP ×××?
\item QRV ×××?
\end{enumerate}
\noindent\textbf{解说:}“你能传信到×××电台吗”的业余无线电通信Q简语为\textbf{QSP ×××?}。\\\noindent\textbf{答案:}C



\bigskip


\noindent\textbf{试题:}“我能传信到×××电台”的业余无线电通信Q简语为:
\begin{enumerate}[leftmargin=3em]
\item QSP ×××
\item QSH ×××
\item QRL ×××
\item QRP ×××
\end{enumerate}
\noindent\textbf{解说:}“我能传信到×××电台”的业余无线电通信Q简语为\textbf{QSP ×××}。\\\noindent\textbf{答案:}A




\bigskip


\noindent\textbf{试题:}“你将在nnnn \si{\kHz}(或\unit{\MHz})频率守听×××电台吗”的业余无线电通信Q简语为:
\begin{enumerate}[leftmargin=3em]
\item QRV ××× ON nnnn \si{\kHz}(或\unit{\MHz})?
\item QRU ××× ON nnnn \si{\kHz}(或\unit{\MHz})?
\item QSX ××× ON nnnn \si{\kHz}(或\unit{\MHz})?
\item QSL ××× ON nnnn \si{\kHz}(或\unit{\MHz})?
\end{enumerate}
\noindent\textbf{解说:}“你将在nnnn \si{\kHz}(或\unit{\MHz})频率守听×××电台吗”的业余无线电通信Q简语为\textbf{QSX ××× ON nnnn \si{\kHz}(或\unit{\MHz})?}。\\\noindent\textbf{答案:}C



\bigskip


\noindent\textbf{试题:}“我将在nnnn \si{\kHz}(或\unit{\MHz})频率守听×××电台”的业余无线电通信Q简语为:
\begin{enumerate}[leftmargin=3em]
\item QRZ ××× ON nnnn \si{\kHz}(或\unit{\MHz})
\item QRX ××× ON nnnn \si{\kHz}(或\unit{\MHz})
\item QSX ××× ON nnnn \si{\kHz}(或\unit{\MHz})
\item QSP ××× ON nnnn \si{\kHz}(或\unit{\MHz})
\end{enumerate}
\noindent\textbf{解说:}“我将在nnnn \si{\kHz}(或\unit{\MHz})频率守听×××电台”的业余无线电通信Q简语为\textbf{QSX ××× ON nnnn \si{\kHz}(或\unit{\MHz})}。\\\noindent\textbf{答案:}C




\bigskip


\noindent\textbf{试题:}“要我将频率改到nnnn频率吗”的业余无线电通信Q简语为:
\begin{enumerate}[leftmargin=3em]
\item QRV nnnn \si{\kHz}(或\unit{\MHz})?
\item QSY nnnn \si{\kHz}(或\unit{\MHz})?
\item QRY nnnn \si{\kHz}(或\unit{\MHz})?
\item QSV nnnn \si{\kHz}(或\unit{\MHz})?
\end{enumerate}
\noindent\textbf{解说:}“要我将频率改到nnnn频率吗”的业余无线电通信Q简语为\textbf{QSY nnnn \si{\kHz}(或\unit{\MHz})?}。\\\noindent\textbf{答案:}B



\bigskip


\noindent\textbf{试题:}“请将频率改到nnnn频率”的业余无线电通信Q简语为:
\begin{enumerate}[leftmargin=3em]
\item QRO nnnn \si{\kHz}(或\unit{\MHz})
\item QSU nnnn \si{\kHz}(或\unit{\MHz})
\item QSY nnnn \si{\kHz}(或\unit{\MHz})
\item QRY nnnn \si{\kHz}(或\unit{\MHz})
\end{enumerate}
\noindent\textbf{解说:}“请将频率改到nnnn频率”的业余无线电通信Q简语为\textbf{QSY nnnn \si{\kHz}(或\unit{\MHz})}。\\\noindent\textbf{答案:}C



\bigskip


\noindent\textbf{试题:}“你的电台位置在哪里”的业余无线电通信Q简语为:
\begin{enumerate}[leftmargin=3em]
\item QSZ?
\item QRA?
\item QSA?
\item QTH?
\end{enumerate}
\noindent\textbf{解说:}“你的电台位置在哪里”的业余无线电通信Q简语为\textbf{QTH?}。\\\noindent\textbf{答案:}D



\bigskip


\noindent\textbf{试题:}“我的信号强度如何”的业余无线电通信Q简语为:
\begin{enumerate}[leftmargin=3em]
\item QSB?
\item QSA?
\item QSD?
\item QTU?
\end{enumerate}
\noindent\textbf{解说:}“我的信号强度如何”的业余无线电通信Q简语为\textbf{QSA?}。\\\noindent\textbf{答案:}B



\bigskip


\noindent\textbf{试题:}“你的信号强度为×级(1-5级)”的业余无线电通信Q简语为:
\begin{enumerate}[leftmargin=3em]
\item QSL ×
\item QSA ×
\item QRY
\item QRM ×
\end{enumerate}
\noindent\textbf{解说:}“你的信号强度为×级(1-5级)”的业余无线电通信Q简语为\textbf{QSA ×}。\\\noindent\textbf{答案:}B



\bigskip


\noindent\textbf{试题:}业余无线电通信常用缩语“ABT”的意思是:
\begin{enumerate}[leftmargin=3em]
\item 在…之上
\item 关于、大约
\item 电池
\item 衰减
\end{enumerate}
\noindent\textbf{解说:}业余无线电通信常用缩语“ABT”的意思是\textbf{关于、大约}。\\\noindent\textbf{答案:}B


\bigskip


\noindent\textbf{试题:}“地址”的业余无线电通信常用缩语是:
\begin{enumerate}[leftmargin=3em]
\item ABT
\item ABV
\item ATT
\item ADR或ADDR
\end{enumerate}
\noindent\textbf{解说:}“地址”的业余无线电通信常用缩语是\textbf{ADR或ADDR}。\\\noindent\textbf{答案:}D




\bigskip


\noindent\textbf{试题:}业余无线电常用缩语“ATT”的意思是:
\begin{enumerate}[leftmargin=3em]
\item 衰落
\item 地址
\item 衰减
\item 关于、大约
\end{enumerate}
\noindent\textbf{解说:}业余无线电常用缩语“ATT”的意思是\textbf{衰减}。\\\noindent\textbf{答案:}C



\bigskip


\noindent\textbf{试题:}业余无线电常用缩语“PWR”的意思是:
\begin{enumerate}[leftmargin=3em]
\item 功率
\item 报告
\item 读写
\item 中继台
\end{enumerate}
\noindent\textbf{解说:}业余无线电常用缩语“PWR”的意思是\textbf{功率}。\\\noindent\textbf{答案:}A



\bigskip


\noindent\textbf{试题:}“再”、“再来一次”的业余无线电通信常用缩语是:
\begin{enumerate}[leftmargin=3em]
\item AGN
\item GA
\item ABT
\item ABV
\end{enumerate}
\noindent\textbf{解说:}“再”、“再来一次”的业余无线电通信常用缩语是\textbf{AGN}。\\\noindent\textbf{答案:}A



\bigskip


\noindent\textbf{试题:}业余无线电通信常用缩语“GA”的意思是:
\begin{enumerate}[leftmargin=3em]
\item 公安
\item 继续、请过来
\item 垂直地网天线
\item 姑娘
\end{enumerate}
\noindent\textbf{解说:}业余无线电通信常用缩语“GA”的意思是\textbf{继续、请过来}。\\\noindent\textbf{答案:}B



\bigskip


\noindent\textbf{试题:}业余无线电通信常用缩语“AHR”的意思是:
\begin{enumerate}[leftmargin=3em]
\item 另一个
\item 地址
\item 天线
\item 这里
\end{enumerate}
\noindent\textbf{解说:}业余无线电通信常用缩语“AHR”的意思是\textbf{另一个}。\\\noindent\textbf{答案:}A

\bigskip


\noindent\textbf{试题:}“天线”的业余无线电通信常用缩语是:
\begin{enumerate}[leftmargin=3em]
\item ATR
\item ANT
\item ATT
\item ATN
\end{enumerate}
\noindent\textbf{解说:}“天线”的业余无线电通信常用缩语是\textbf{ANT}。\\\noindent\textbf{答案:}B



\bigskip


\noindent\textbf{试题:}业余无线电常用缩语“ARDF”的意思是:
\begin{enumerate}[leftmargin=3em]
\item 天线测试仪
\item 业余无线电测向
\item 地址
\item 天线调谐器、天调
\end{enumerate}
\noindent\textbf{解说:}业余无线电常用缩语“ARDF”的意思是\textbf{业余无线电测向}。\\\noindent\textbf{答案:}B



\bigskip


\noindent\textbf{试题:}“收听”的业余无线电常用缩语是:
\begin{enumerate}[leftmargin=3em]
\item GA
\item HR
\item KP
\item RCV
\end{enumerate}
\noindent\textbf{解说:}“收听”的业余无线电常用缩语是\textbf{KP}。\\\noindent\textbf{答案:}C
%%% KP保持keep

\bigskip


\noindent\textbf{试题:}业余无线电常用缩语“HST”的意思是:
\begin{enumerate}[leftmargin=3em]
\item 快速收发报
\item 信号报告
\item 通播
\item 这里、听到
\end{enumerate}
\noindent\textbf{解说:}业余无线电常用缩语“HST”的意思是\textbf{快速收发报}。\\\noindent\textbf{答案:}A



\bigskip


\noindent\textbf{试题:}业余无线电CW通信常用缩语“AS”(经常连发在一起)的意思是:
\begin{enumerate}[leftmargin=3em]
\item 请稍等
\item 天线
\item 全部
\item 关于
\end{enumerate}
\noindent\textbf{解说:}业余无线电CW通信常用缩语“$\overline{\rm AS}$”的意思是\textbf{请稍等}。\\\noindent\textbf{答案:}A

\bigskip


\noindent\textbf{试题:}业余无线电通信常用缩语“AS”的意思有:
\begin{enumerate}[leftmargin=3em]
\item 关于
\item 回答
\item 请稍等、亚洲、如同
\item 天线开关
\end{enumerate}
\noindent\textbf{解说:}业余无线电通信常用缩语“AS”的意思有\textbf{请稍等、亚洲、如同}。\\\noindent\textbf{答案:}C

\bigskip


\noindent\textbf{试题:}业余无线电通信常用词语“BEST”的意思是:
\begin{enumerate}[leftmargin=3em]
\item 最好的
\item 通播电报
\item 信号报告
\item 电池组
\end{enumerate}
\noindent\textbf{解说:}业余无线电通信常用词语“BEST”的意思是\textbf{最好的}。\\\noindent\textbf{答案:}A

\bigskip


\noindent\textbf{试题:}业余无线电通信常用缩语“BJT”的意思是:
\begin{enumerate}[leftmargin=3em]
\item 北京时间
\item 结型场效应半导体管
\item 双基极二极管
\item 双极型半导体管
\end{enumerate}
\noindent\textbf{解说:}业余无线电通信常用缩语“BJT”的意思是\textbf{北京时间}。\\\noindent\textbf{答案:}A


\bigskip


\noindent\textbf{试题:}业余无线电通信常用缩语“BK”的意思是:
\begin{enumerate}[leftmargin=3em]
\item 插入、打断
\item 结束工作
\item 千字节(单位)
\item 请马上回答
\end{enumerate}
\noindent\textbf{解说:}业余无线电通信常用缩语“BK”的意思是\textbf{插入、打断}。\\\noindent\textbf{答案:}A

\bigskip


\noindent\textbf{试题:}“QSL卡片管理局”的业余无线电通信常用缩语是:
\begin{enumerate}[leftmargin=3em]
\item BROU
\item BOUR
\item BURO
\item BRUO
\end{enumerate}
\noindent\textbf{解说:}“QSL卡片管理局”的业余无线电通信常用缩语是\textbf{BURO}。\\\noindent\textbf{答案:}C

\bigskip


\noindent\textbf{试题:}“遇到”、“见面”的业余无线电通信常用缩语是:
\begin{enumerate}[leftmargin=3em]
\item MTRS
\item C
\item MET
\item SE
\end{enumerate}
\noindent\textbf{解说:}“遇到”、“见面”的业余无线电通信常用缩语是\textbf{C}。\\\noindent\textbf{答案:}B
%%%???莫尔斯:Yes; correct


\bigskip


\noindent\textbf{试题:}业余无线电通信常用缩语“CFM”的意思是:
\begin{enumerate}[leftmargin=3em]
\item 法拉
\item 确认
\item 调频
\item 呼叫
\end{enumerate}
\noindent\textbf{解说:}业余无线电通信常用缩语“CFM”的意思是\textbf{确认}。\\\noindent\textbf{答案:}B

\bigskip


\noindent\textbf{试题:}业余无线电通信常用词语“CHEERIO”的意思是:
\begin{enumerate}[leftmargin=3em]
\item 圣诞节
\item 英文字符
\item 再次见面
\item 再会、祝贺
\end{enumerate}
\noindent\textbf{解说:}业余无线电通信常用词语“CHEERIO”的意思是\textbf{再会、祝贺}。\\\noindent\textbf{答案:}D



\bigskip


\noindent\textbf{试题:}业余无线电通信常用缩语“CL”、“CLS”、“CLG”的意思分别是:
\begin{enumerate}[leftmargin=3em]
\item 关闭(或呼叫)、呼号、呼叫
\item 计数、云层、呼号
\item 确认、呼叫、清除
\item 呼号、清除、关闭
\end{enumerate}
\noindent\textbf{解说:}业余无线电通信常用缩语“CL”、“CLS”、“CLG”的意思分别是\textbf{关闭(或呼叫)、呼号、呼叫}。\\\noindent\textbf{答案:}A

\bigskip


\noindent\textbf{试题:}业余无线电通信常用词语“DATE”的意思是:
\begin{enumerate}[leftmargin=3em]
\item 时间
\item 日期
\item 地址
\item 频率
\end{enumerate}
\noindent\textbf{解说:}业余无线电通信常用词语“DATE”的意思是\textbf{日期}。\\\noindent\textbf{答案:}B

\bigskip


\noindent\textbf{试题:}业余无线电通信常用缩语“DR”的意思是:
\begin{enumerate}[leftmargin=3em]
\item 亲爱的
\item 从
\item 远距离
\item 二极管
\end{enumerate}
\noindent\textbf{解说:}业余无线电通信常用缩语“DR”的意思是\textbf{亲爱的}。\\\noindent\textbf{答案:}A

\bigskip


\noindent\textbf{试题:}单元(常用于天线振子)的业余无线电通信常用缩语是:
\begin{enumerate}[leftmargin=3em]
\item UNIT
\item YAGI
\item ANT
\item EL、ELE、ELS
\end{enumerate}
\noindent\textbf{解说:}单元(常用于天线振子)的业余无线电通信常用缩语是\textbf{EL、ELE、ELS}。\\\noindent\textbf{答案:}D
%%%???

\bigskip


\noindent\textbf{试题:}业余无线电CW通信常用缩语“ES”的意思是:
\begin{enumerate}[leftmargin=3em]
\item 从
\item 请等待
\item 是
\item 和
\end{enumerate}
\noindent\textbf{解说:}业余无线电CW通信常用缩语“ES”的意思是\textbf{和}。\\\noindent\textbf{答案:}D
%%%???

\bigskip


\noindent\textbf{试题:}业余无线电通信常用缩语“FB”的意思是:
\begin{enumerate}[leftmargin=3em]
\item 美好的祝愿
\item 再见
\item 很好的
\item 腐败
\end{enumerate}
\noindent\textbf{解说:}业余无线电通信常用缩语“FB”的意思是\textbf{很好的}。\\\noindent\textbf{答案:}C

\bigskip


\noindent\textbf{试题:}“频率”的业余无线电通信常用缩语是:
\begin{enumerate}[leftmargin=3em]
\item TUNE
\item FREQ
\item FIND
\item FER
\end{enumerate}
\noindent\textbf{解说:}“频率”的业余无线电通信常用缩语是\textbf{FREQ}。\\\noindent\textbf{答案:}B
%%%???

\bigskip


\noindent\textbf{试题:}业余无线电通信常用缩语“GND”的意思是:
\begin{enumerate}[leftmargin=3em]
\item 格林威治时间
\item 地线,地面
\item 高兴
\item 好运气
\end{enumerate}
\noindent\textbf{解说:}业余无线电通信常用缩语“GND”的意思是\textbf{地线,地面}”。\\\noindent\textbf{答案:}B

\bigskip


\noindent\textbf{试题:}业余无线电通信常用缩语“OM”的意思是:
\begin{enumerate}[leftmargin=3em]
\item 老人
\item 欧姆
\item 老朋友
\item 或者
\end{enumerate}
\noindent\textbf{解说:}业余无线电通信常用缩语“OM”的意思是\textbf{老朋友}。\\\noindent\textbf{答案:}C

\bigskip


\noindent\textbf{试题:}“电台设备”的业余无线电通信常用缩语是:
\begin{enumerate}[leftmargin=3em]
\item REG
\item RIG
\item SB
\item EQP
\end{enumerate}
\noindent\textbf{解说:}“电台设备”的业余无线电通信常用缩语是\textbf{RIG}。\\\noindent\textbf{答案:}B

\bigskip


\noindent\textbf{试题:}业余无线电通信常用词语“FINE”的意思是:
\begin{enumerate}[leftmargin=3em]
\item 调谐
\item 好的,精细的
\item 发现
\item 确认
\end{enumerate}
\noindent\textbf{解说:}业余无线电通信常用词语“FINE”的意思是\textbf{好的,精细的}。\\\noindent\textbf{答案:}B
\bigskip


\noindent\textbf{试题:}业余无线电通信常用缩语“FR”、“FER”的意思是:
\begin{enumerate}[leftmargin=3em]
\item 频率
\item 希望
\item 好的,精细的
\item 为了,对于
\end{enumerate}
\noindent\textbf{解说:}业余无线电通信常用缩语“FR”、“FER”的意思是\textbf{为了,对于}。\\
\noindent\textbf{答案:}D


\bigskip


\noindent\textbf{试题:}“下午好”的业余无线电通信常用缩语是:
\begin{enumerate}[leftmargin=3em]
\item GN
\item GA
\item GE
\item GM
\end{enumerate}
\noindent\textbf{解说:}“下午好”的业余无线电通信常用缩语是\textbf{GA}。\\\noindent\textbf{答案:}B

\bigskip


\noindent\textbf{试题:}“早晨好”的业余无线电通信常用缩语是:
\begin{enumerate}[leftmargin=3em]
\item GM
\item GL
\item GA
\item GB
\end{enumerate}
\noindent\textbf{解说:}“早晨好”的业余无线电通信常用缩语是\textbf{GM}。\\\noindent\textbf{答案:}A


\bigskip


\noindent\textbf{试题:}“晚上好”的业余无线电通信常用缩语是:
\begin{enumerate}[leftmargin=3em]
\item GN
\item GA
\item GE
\item GM
\end{enumerate}
\noindent\textbf{解说:}“晚上好”的业余无线电通信常用缩语是\textbf{GE}。\\\noindent\textbf{答案:}C



\bigskip


\noindent\textbf{试题:}业余无线电通信常用缩语“GN”的意思是:
\begin{enumerate}[leftmargin=3em]
\item 早晨好
\item 晚安
\item 好运气
\item 高兴
\end{enumerate}
\noindent\textbf{解说:}业余无线电通信常用缩语“GN”的意思是\textbf{晚安}。\\\noindent\textbf{答案:}B

\bigskip


\noindent\textbf{试题:}“再见”的业余无线电通信常用缩语是:
\begin{enumerate}[leftmargin=3em]
\item GE
\item GB
\item GL
\item GA
\end{enumerate}
\noindent\textbf{解说:}“再见”的业余无线电通信常用缩语是\textbf{GB}。\\\noindent\textbf{答案:}B



\bigskip


\noindent\textbf{试题:}业余无线电通信常用缩语“GL”的意思是:
\begin{enumerate}[leftmargin=3em]
\item 晚安
\item 早安
\item 好运气
\item 再见
\end{enumerate}
\noindent\textbf{解说:}业余无线电通信常用缩语“GL”的意思是\textbf{好运气}。\\\noindent\textbf{答案:}C



\bigskip


\noindent\textbf{试题:}业余无线电通信常用缩语“GLD”的意思是:
\begin{enumerate}[leftmargin=3em]
\item 高兴
\item 地线,地面
\item 再见
\item 好运气
\end{enumerate}
\noindent\textbf{解说:}业余无线电通信常用缩语“GLD”的意思是\textbf{高兴}。\\\noindent\textbf{答案:}A



\bigskip


\noindent\textbf{试题:}业余无线电通信常用缩语“GMT”的意思是:
\begin{enumerate}[leftmargin=3em]
\item 格林威治时间
\item 地线,地面
\item 好运气
\item 高兴
\end{enumerate}
\noindent\textbf{解说:}业余无线电通信常用缩语“GMT”的意思是\textbf{格林威治时间}。\\\noindent\textbf{答案:}A



\bigskip


\noindent\textbf{试题:}“抄收”的业余无线电通信常用缩语是:
\begin{enumerate}[leftmargin=3em]
\item CFM
\item CPI
\item HPI
\item HPY
\end{enumerate}
\noindent\textbf{解说:}“抄收”的业余无线电通信常用缩语是\textbf{CPI}。\\\noindent\textbf{答案:}B



\bigskip


\noindent\textbf{试题:}“希望”的业余无线电通信常用缩语是:
\begin{enumerate}[leftmargin=3em]
\item CPI
\item HPI
\item HPE
\item HPY
\end{enumerate}
\noindent\textbf{解说:}“希望”的业余无线电通信常用缩语是\textbf{HPE}。\\\noindent\textbf{答案:}C



\bigskip


\noindent\textbf{试题:}业余无线电通信常用缩语“HPY”、“HPI”的意思是:
\begin{enumerate}[leftmargin=3em]
\item 这里
\item 抄收
\item 幸福
\item 希望
\end{enumerate}
\noindent\textbf{解说:}业余无线电通信常用缩语“HPY”、“HPI”的意思是\textbf{幸福}。\\\noindent\textbf{答案:}C



\bigskip


\noindent\textbf{试题:}业余无线电通信常用缩语“HR”的意思是:
\begin{enumerate}[leftmargin=3em]
\item 这里、听到
\item 幸福
\item 号码
\item 希望
\end{enumerate}
\noindent\textbf{解说:}业余无线电通信常用缩语“HR”的意思是\textbf{这里、听到}。\\\noindent\textbf{答案:}A
%%%
% 号码为NR

\bigskip


\noindent\textbf{试题:}“怎样”、“如何”的业余无线电通信常用缩语是:
\begin{enumerate}[leftmargin=3em]
\item CW
\item HW
\item HPI
\item HR
\end{enumerate}
\noindent\textbf{解说:}“怎样”、“如何”的业余无线电通信常用缩语是\textbf{HW}。\\\noindent\textbf{答案:}B



\bigskip


\noindent\textbf{试题:}“很多”的业余无线电通信常用缩语是:
\begin{enumerate}[leftmargin=3em]
\item NAME
\item MNY、MNI
\item VY
\item ALL
\end{enumerate}
\noindent\textbf{解说:}“很多”的业余无线电通信常用缩语是\textbf{MNY、MNI}。\\\noindent\textbf{答案:}B



\bigskip


\noindent\textbf{试题:}业余无线电通信常用缩语“MTRS”的意思是:
\begin{enumerate}[leftmargin=3em]
\item 太太
\item 小姐
\item 米
\item 先生
\end{enumerate}
\noindent\textbf{解说:}业余无线电通信常用缩语“MTRS”的意思是\textbf{米}。\\\noindent\textbf{答案:}C



\bigskip


\noindent\textbf{试题:}“方式”的业余无线电通信常用英语是:
\begin{enumerate}[leftmargin=3em]
\item NAME
\item NICE
\item MTRS
\item MODE
\end{enumerate}
\noindent\textbf{解说:}“方式”的业余无线电通信常用英语是\textbf{MODE}。\\\noindent\textbf{答案:}D



\bigskip


\noindent\textbf{试题:}“名字”的业余无线电通信常用英语是:
\begin{enumerate}[leftmargin=3em]
\item NICE
\item MNI
\item NAME
\item MODE
\end{enumerate}
\noindent\textbf{解说:}“名字”的业余无线电通信常用英语是\textbf{NAME}。\\
\noindent\textbf{答案:}C



\bigskip


\noindent\textbf{试题:}业余无线电通信常用词语“NICE”的意思是:
\begin{enumerate}[leftmargin=3em]
\item 良好的
\item 名字
\item 鼠标
\item 方式
\end{enumerate}
\noindent\textbf{解说:}业余无线电通信常用词语“NICE”的意思是\textbf{良好的}。\\\noindent\textbf{答案:}A



\bigskip


\noindent\textbf{试题:}业余无线电通信常用缩语“NW”的意思是:
\begin{enumerate}[leftmargin=3em]
\item 新的
\item 怎样
\item 现在
\item 不
\end{enumerate}
\noindent\textbf{解说:}业余无线电通信常用缩语“NW”的意思是\textbf{现在}。\\\noindent\textbf{答案:}C



\bigskip


\noindent\textbf{试题:}“操作员”的业余无线电通信常用缩语是:
\begin{enumerate}[leftmargin=3em]
\item OP、OPR
\item CPI
\item OM
\item RPT
\end{enumerate}
\noindent\textbf{解说:}“操作员”的业余无线电通信常用缩语是\textbf{OP、OPR}。\\\noindent\textbf{答案:}A



\bigskip


\noindent\textbf{试题:}“邮政信箱”的业余无线电通信常用缩语是:
\begin{enumerate}[leftmargin=3em]
\item QTH
\item MAIL
\item BURO
\item P O BOX
\end{enumerate}
\noindent\textbf{解说:}“邮政信箱”的业余无线电通信常用缩语是\textbf{P O BOX}。\\\noindent\textbf{答案:}D



\bigskip


\noindent\textbf{试题:}业余无线电通信常用缩语“RMKS”的意思是:
\begin{enumerate}[leftmargin=3em]
\item 报告
\item 备注、注释
\item 业余无线电测向
\item 中继台
\end{enumerate}
\noindent\textbf{解说:}业余无线电通信常用缩语“RMKS”的意思是\textbf{备注、注释}。\\\noindent\textbf{答案:}B



\bigskip


\noindent\textbf{试题:}“收信机”的业余无线电通信常用缩语是:
\begin{enumerate}[leftmargin=3em]
\item XMTR
\item XCVR
\item RCVR,RX
\item RMKS
\end{enumerate}
\noindent\textbf{解说:}“收信机”的业余无线电通信常用缩语是\textbf{RCVR,RX}。\\\noindent\textbf{答案:}C



\bigskip


\noindent\textbf{试题:}“发信机”的业余无线电通信常用缩语是:
\begin{enumerate}[leftmargin=3em]
\item VXO
\item VXCO
\item XTL
\item TX、XMTR
\end{enumerate}
\noindent\textbf{解说:}“发信机”的业余无线电通信常用缩语是\textbf{TX、XMTR}。\\\noindent\textbf{答案:}D



\bigskip


\noindent\textbf{试题:}“收发信机”的业余无线电通信常用缩语是:
\begin{enumerate}[leftmargin=3em]
\item XTL
\item XMTR
\item XCVR
\item XVTR
\end{enumerate}
\noindent\textbf{解说:}“收发信机”的业余无线电通信常用缩语是\textbf{XCVR}。\\\noindent\textbf{答案:}C





\bigskip


\noindent\textbf{试题:}业余无线电通信常用缩语“WX”的意思是:
\begin{enumerate}[leftmargin=3em]
\item 星期
\item 天气
\item 瓦特
\item 联络、工作
\end{enumerate}
\noindent\textbf{解说:}业余无线电通信常用缩语“WX”的意思是\textbf{天气}。\\\noindent\textbf{答案:}B


\bigskip


\noindent\textbf{试题:}业余无线电通信常用缩语“73”的意思是:
\begin{enumerate}[leftmargin=3em]
\item 再见
\item 希望下次再见
\item 谢谢你
\item 向对方的致意、美好的祝愿
\end{enumerate}
\noindent\textbf{解说:}业余无线电通信常用缩语“73”的意思是\textbf{向对方的致意、美好的祝愿}。\\\noindent\textbf{答案:}D



\bigskip


\noindent\textbf{试题:}业余无线电通话常用语“Roger”的用法是:
\begin{enumerate}[leftmargin=3em]
\item 回答起始语,相当于“听到”,用于能听到对方信号、但不一定能全部抄收的情况
\item 回答起始语,表示开始发话了,任何情况都可使用
\item 惯用口头语,相当于电话的“喂”,仅提起注意,不包含任何意义
\item 回答起始语,相当于“明白”,仅在已完全抄收对方刚才发送的信息时使用
\end{enumerate}
\noindent\textbf{解说:}。业余无线电通话常用语“Roger”为\textbf{回答起始语,相当于“明白”,仅在已完全抄收对方刚才发送的信息时使用}。\\\noindent\textbf{答案:}D


\bigskip


\noindent\textbf{试题:}业余电台通信中常用到缩写“SASE”,其意义是:
\begin{enumerate}[leftmargin=3em]
\item 写好收信人地址的信封
\item 请勿通过卡片管理局交换QSL卡
\item 国际邮资券
\item 请尽快寄出QSL卡片
\end{enumerate}
\noindent\textbf{解说:}业余电台通信中常用缩写“SASE”的意义是\textbf{写好收信人地址的信封}。\\\noindent\textbf{答案:}A



\bigskip


\noindent\textbf{试题:}“报告”的业余无线电通信常用缩语是:
\begin{enumerate}[leftmargin=3em]
\item RPRT
\item RMKS
\item MSG
\item PRT
\end{enumerate}
\noindent\textbf{解说:}“报告”的业余无线电通信常用缩语是\textbf{RPRT}。\\\noindent\textbf{答案:}A





\bigskip


\noindent\textbf{试题:}业余无线电通信常用缩语“SK”(通常在CW中连在一起拍发)的意思是:
\begin{enumerate}[leftmargin=3em]
\item 下次再见
\item 谢谢
\item 开关
\item 结束通信
\end{enumerate}
\noindent\textbf{解说:}业余无线电通信常用缩语“$\overline{\rm SK}$”的意思是\textbf{结束通信}。\\\noindent\textbf{答案:}D




\bigskip


\noindent\textbf{试题:}“对不起”的业余无线电通信常用缩语是:
\begin{enumerate}[leftmargin=3em]
\item SK
\item AS
\item TNX
\item SRI,SRY
\end{enumerate}
\noindent\textbf{解说:}“对不起”的业余无线电通信常用缩语是\textbf{SRI,SRY}。\\\noindent\textbf{答案:}D



\bigskip


\noindent\textbf{试题:}“电台”的业余无线电通信常用缩语是:
\begin{enumerate}[leftmargin=3em]
\item SRY
\item QTH
\item ANT
\item STN
\end{enumerate}
\noindent\textbf{解说:}“电台”的业余无线电通信常用缩语是\textbf{STN}。\\\noindent\textbf{答案:}D



\bigskip


\noindent\textbf{试题:}业余无线电通信常用缩语“SURE”的意思是:
\begin{enumerate}[leftmargin=3em]
\item 确实
\item 电台
\item 短波收听者
\item 对不起
\end{enumerate}
\noindent\textbf{解说:}业余无线电通信常用缩语“SURE”的意思是\textbf{确实}。\\\noindent\textbf{答案:}A


\bigskip


\noindent\textbf{试题:}业余无线电常用缩语“SWL”的意思是:
\begin{enumerate}[leftmargin=3em]
\item 确认
\item 对不起
\item 短波收听者
\item 高兴
\end{enumerate}
\noindent\textbf{解说:}业余无线电常用缩语“SWL”的意思是\textbf{短波收听者}。\\\noindent\textbf{答案:}C



\bigskip


\noindent\textbf{试题:}“温度”的业余无线电通信常用缩语是:
\begin{enumerate}[leftmargin=3em]
\item TEMP
\item WX
\item TMPO
\item TUNE
\end{enumerate}
\noindent\textbf{解说:}“温度”的业余无线电通信常用缩语是\textbf{TEMP}。\\\noindent\textbf{答案:}A


\bigskip


\noindent\textbf{试题:}“谢谢”的业余无线电通信常用缩语是:
\begin{enumerate}[leftmargin=3em]
\item SRI,SRY
\item 73
\item TRY
\item TNX,TKS
\end{enumerate}
\noindent\textbf{解说:}“谢谢”的业余无线电通信常用缩语是\textbf{TNX,TKS}。\\\noindent\textbf{答案:}D



\bigskip


\noindent\textbf{试题:}业余无线电通信常用缩语“TU”的意思是:
\begin{enumerate}[leftmargin=3em]
\item 电子管
\item 天调、天线调谐器
\item 谢谢你
\item 发信机
\end{enumerate}
\noindent\textbf{解说:}业余无线电通信常用缩语“TU”的意思是\textbf{谢谢你}。\\\noindent\textbf{答案:}C
%%%



\bigskip


\noindent\textbf{试题:}“世界协调时”的业余无线电通信常用缩语是:
\begin{enumerate}[leftmargin=3em]
\item CUT
\item UTC
\item TUC
\item UCT
\end{enumerate}
\noindent\textbf{解说:}ITU《无线电规则》第1.14条:\\协调世界时(\textbf{UTC}):由ITU-R TF.460-6建议书规定的以秒(SI)为单位的时间标度。(WRC-03)。\\\noindent\textbf{答案:}B



\bigskip


\noindent\textbf{试题:}业余无线电通信常用缩语“VIA”的意思是:
\begin{enumerate}[leftmargin=3em]
\item 经、由
\item 美国之音%谁出的作死题
\item 邮寄
\item 声控
\end{enumerate}
\noindent\textbf{解说:}业余无线电通信常用缩语“VIA”的意思是\textbf{经、由}。\\\noindent\textbf{答案:}A


\bigskip


\noindent\textbf{试题:}“很”、“非常”的业余无线电通信常用缩语是:
\begin{enumerate}[leftmargin=3em]
\item MNI
\item VIA
\item VY
\item ALL
\end{enumerate}
\noindent\textbf{解说:}“很”、“非常”的业余无线电通信常用缩语是\textbf{VY}。\\\noindent\textbf{答案:}C



\bigskip


\noindent\textbf{试题:}业余无线电通信常用缩语“WK”的意思是:
\begin{enumerate}[leftmargin=3em]
\item 瓦特
\item 星期、工作
\item 插入
\item 圣诞节
\end{enumerate}
\noindent\textbf{解说:}业余无线电通信常用缩语“WK”的意思是\textbf{星期、工作}。\\\noindent\textbf{答案:}B


\bigskip


\noindent\textbf{试题:}业余无线电通信常用缩语“WKD”的意思是:
\begin{enumerate}[leftmargin=3em]
\item 圣诞节
\item 星期
\item 联络过、工作过
\item 天气
\end{enumerate}
\noindent\textbf{解说:}业余无线电通信常用缩语“WKD”的意思是\textbf{联络过、工作过}。\\\noindent\textbf{答案:}C



\bigskip


\noindent\textbf{试题:}业余无线电通信常用缩语“WTS”的意思是:
\begin{enumerate}[leftmargin=3em]
\item 瓦特
\item 工作、联络
\item 天气
\item 星期
\end{enumerate}
\noindent\textbf{解说:}业余无线电通信常用缩语“WTS”的意思是\textbf{瓦特}。\\\noindent\textbf{答案:}A


\bigskip


\noindent\textbf{试题:}业余无线电通信常用缩语“XMAS”的意思是:
\begin{enumerate}[leftmargin=3em]
\item 收发信机
\item 晶体
\item 圣诞节
\item 发信机
\end{enumerate}
\noindent\textbf{解说:}业余无线电通信常用缩语“XMAS”的意思是\textbf{圣诞节}。\\\noindent\textbf{答案:}C




\bigskip


\noindent\textbf{试题:}业余无线电通信常用缩语“XYL”的意思是:
\begin{enumerate}[leftmargin=3em]
\item 妻子、已婚女子
\item 姑娘
\item 晶体
\item 发信机
\end{enumerate}
\noindent\textbf{解说:}业余无线电通信常用缩语“XYL”的意思是\textbf{妻子、已婚女子}。\\\noindent\textbf{答案:}A



\bigskip


\noindent\textbf{试题:}业余无线电通信常用缩语“YL”的意思是:
\begin{enumerate}[leftmargin=3em]
\item 呼叫
\item 好运气
\item 小姐、女士
\item 你的
\end{enumerate}
\noindent\textbf{解说:}业余无线电通信常用缩语“YL”的意思是\textbf{小姐、女士}。\\\noindent\textbf{答案:}C




\bigskip


\noindent\textbf{试题:}业余无线电通信常用缩语“TU”的意思是:
\begin{enumerate}[leftmargin=3em]
\item 结束联络
\item 美好的祝愿
\item 再见
\item 谢谢你
\end{enumerate}
\noindent\textbf{解说:}业余无线电通信常用缩语“TU”的意思是\textbf{谢谢你}。\\\noindent\textbf{答案:}D



\bigskip


\noindent\textbf{试题:}“你的”或者“你是”的业余无线电通信常用缩语是:
\begin{enumerate}[leftmargin=3em]
\item FB
\item UR
\item US
\item TU
\end{enumerate}
\noindent\textbf{解说:}“你的”或者“你是”的业余无线电通信常用缩语是\textbf{UR}。\\即英语Your。\\\noindent\textbf{答案:}B



\bigskip


\noindent\textbf{试题:}业余无线电通信常用缩语“88”的意思是:
\begin{enumerate}[leftmargin=3em]
\item 祝对方发达、发财
\item 再见
\item 谢谢你
\item 向对方异性操作员的致意、美好的祝愿
\end{enumerate}
\noindent\textbf{解说:}业余无线电通信常用缩语“88”的意思是\textbf{向对方异性操作员的致意、美好的祝愿}。\\\noindent\textbf{答案:}D



\bigskip


\noindent\textbf{试题:}业余无线电通信中常用的天线种类的缩写DP代表:
\begin{enumerate}[leftmargin=3em]
\item 定向天线
\item 垂直天线
\item 偶极天线
\item 长线天线
\end{enumerate}
\noindent\textbf{解说:}业余无线电通信中常用的天线种类的缩写DP代表\textbf{偶极天线}。\\\noindent\textbf{答案:}C



\bigskip


\noindent\textbf{试题:}业余无线电通信中常用的天线种类的缩写LW代表:
\begin{enumerate}[leftmargin=3em]
\item 长线天线
\item 偶极天线
\item 定向天线
\item 垂直天线
\end{enumerate}
\noindent\textbf{解说:}业余无线电通信中常用的天线种类的缩写LW代表\textbf{长线天线}。\\\noindent\textbf{答案:}A



\bigskip


\noindent\textbf{试题:}业余无线电通信中常用的天线种类的缩写GP代表:
\begin{enumerate}[leftmargin=3em]
\item 垂直接地天线
\item 偶极天线
\item 对数周期天线
\item 定向天线
\end{enumerate}
\noindent\textbf{解说:}业余无线电通信中常用的天线种类的缩写GP代表\textbf{垂直接地天线}。\\\noindent\textbf{答案:}A



\bigskip


\noindent\textbf{试题:}业余无线电通信中常用的天线种类的缩写BEAM代表:
\begin{enumerate}[leftmargin=3em]
\item 定向天线
\item 专指八木天线
\item 偶极天线
\item 垂直天线
\end{enumerate}
\noindent\textbf{解说:}业余无线电通信中常用的天线种类的缩写BEAM代表\textbf{定向天线}。\\\noindent\textbf{答案:}A


\bigskip


\noindent\textbf{试题:}业余无线电通信中常用的天线种类的缩写YAGI代表:
\begin{enumerate}[leftmargin=3em]
\item 定向天线
\item 偶极天线
\item 八木天线
\item 垂直天线
\end{enumerate}
\noindent\textbf{解说:}业余无线电通信中常用的天线种类的缩写YAGI代表\textbf{八木天线}。\\\noindent\textbf{答案:}C



\bigskip


\noindent\textbf{试题:}业余无线电通信中常用的天线种类的缩写VER代表:
\begin{enumerate}[leftmargin=3em]
\item 垂直接地天线
\item 定向天线
\item 偶极天线
\item 垂直天线
\end{enumerate}
\noindent\textbf{解说:}业余无线电通信中常用的天线种类的缩写VER代表\textbf{垂直天线}。\\\noindent\textbf{答案:}D



\bigskip


\noindent\textbf{试题:}业余无线电技术常提到的天线种类的缩写LP代表:
\begin{enumerate}[leftmargin=3em]
\item 对数周期天线
\item 垂直天线
\item 垂直接地天线
\item 定向天线
\end{enumerate}
\noindent\textbf{解说:}业余无线电技术常提到的天线种类的缩写LP代表\textbf{对数周期天线}。\\\noindent\textbf{答案:}A



\bigskip


\noindent\textbf{试题:}业余无线电通信方式缩写CW的英文原词意义是:
\begin{enumerate}[leftmargin=3em]
\item 等幅电报
\item 幅度键控
\item 移频键控
\item 莫尔斯编码
\end{enumerate}
\noindent\textbf{解说:}业余无线电通信方式缩写CW的英文原词意义是\textbf{等幅电报}。\\\noindent\textbf{答案:}A



\bigskip


\noindent\textbf{试题:}为了便于计算时间,将地球划分为若干个时区,各理论时区的划分方法是:
\begin{enumerate}[leftmargin=3em]
\item 全球划分为24个时区,每个理论时区宽度为经度15度,本初子午线通过0区的中心
\item 全球划分为24个时区,每个理论时区宽度为经度15度,其边界为东西经度为15的整倍数的子午线
\item 全球划分为12个时区,每个理论时区宽度为经度30度,本初子午线通过0区的中心
\item 全球划分为12个时区,每个理论时区宽度为经度30度,其边界为东西经度为30的整倍数的子午线
\end{enumerate}
\noindent\textbf{解说:}各理论时区的划分方法是\textbf{全球划分为24个时区,每个理论时区宽度为经度15度,本初子午线通过0区的中心}。\\\noindent\textbf{答案:}A



\bigskip


\noindent\textbf{试题:}为了便于计算时间,将地球划分为若干个时区,各理论时区的命名规则是:
\begin{enumerate}[leftmargin=3em]
\item 本初子午线通过中心的为0区,向西依次为1区、2区…24区
\item 本初子午线通过中心的为0区,如向东数则依次称为东1区、东2区…东24区,如向西数则依次称为西1区、西2区…西24区
\item 本初子午线通过中心的为0区,向东依次为1区、2区…24区
\item 本初子午线通过其中心的为0区,向东依次为东1区、东2区…东12区,向西依次为西1区、西2区…西12区
\end{enumerate}
\noindent\textbf{解说:}各理论时区的命名规则是\textbf{本初子午线通过其中心的为0区,向东依次为东1区、东2区…东12区,向西依次为西1区、西2区…西12区}。\\\noindent\textbf{答案:}D




\bigskip


\noindent\textbf{试题:}为了便于计算时间,将地球划分为若干个时区,北京的情况是:
\begin{enumerate}[leftmargin=3em]
\item 北京处于东8区,地方时间比0时区的时间晚8小时
\item 北京处于西8区,地方时间比0时区的时间晚8小时
\item 北京处于西8区,地方时间比0时区的时间早8小时
\item 北京处于东8区,地方时间比0时区的时间早8小时
\end{enumerate}
\noindent\textbf{解说:}\textbf{北京处于东8区,地方时间比0时区的时间早8小时}。\\\noindent\textbf{答案:}D




\bigskip


\noindent\textbf{试题:}为了便于计算时间,将地球划分为若干个时区,其理论分区为每区宽经度15度。北京、西安和乌鲁木齐实际所属的时区应为:
\begin{enumerate}[leftmargin=3em]
\item 根据所在经度推算,北京、西安、乌鲁木齐分别处于西8区、西7区和西6区
\item 世界上实际使用法定分区,北京、西安、乌鲁木齐都属于东8区
\item 根据所在经度推算,北京、西安、乌鲁木齐分别处于东8区、东7区和东6区
\item 根据所在经度推算,北京、西安、乌鲁木齐分别处于东6区、东7区和东8区
\end{enumerate}
\noindent\textbf{解说:}\textbf{世界上实际使用法定分区,北京、西安、乌鲁木齐都属于东8区}。\\\noindent\textbf{答案:}B




\bigskip


\noindent\textbf{试题:}已知北京时间,相应的UTC时间应为:
\begin{enumerate}[leftmargin=3em]
\item 北京时间的小时数加8,如小时数大于24,则小时数减24,日期改为后一天。
\item 北京时间的小时数加8,如小时数大于24,则小时数减24,日期改为前一天。
\item 北京时间的小时数减8,如小时数小于0,则小时数加24,日期改为前一天。
\item 北京时间的小时数减8,如小时数小于0,则小时数加24,日期改为后一天。
\end{enumerate}
\noindent\textbf{解说:}已知北京时间,相应的UTC时间应为:\textbf{北京时间的小时数减8,如小时数小于0,则小时数加24,日期改为前一天。}\\\noindent\textbf{答案:}C


\bigskip


\noindent\textbf{试题:}已知UTC时间,相应的北京时间应为:
\begin{enumerate}[leftmargin=3em]
\item 北京时间的小时数减8,如小时数小于0,则小时数加24,日期改为前一天。
\item 北京时间的小时数加8,如小时数大于24,则小时数减24,日期改为前一天。
\item UTC时间的小时数加8,如小时数大于24,则小时数减24,日期改为后一天。
\item 北京时间的小时数减8,如小时数小于0,则小时数加24,日期改为后一天。
\end{enumerate}
\noindent\textbf{解说:}已知UTC时间,相应的北京时间应为:\textbf{UTC时间的小时数加8,如小时数大于24,则小时数减24,日期改为后一天。}。\\\noindent\textbf{答案:}C


\bigskip


\noindent\textbf{试题:}已知某业余电台处于西N时区(N为0-12间的整数),该台的当地时间应比北京时间:
\begin{enumerate}[leftmargin=3em]
\item 早\(8-N\)小时
\item 早\(8+N\)小时
\item 晚\(8+N\)小时
\item 晚\(8-N\)小时
\end{enumerate}
\noindent\textbf{解说:}已知某业余电台处于西N时区(N为0-12间的整数),该台的当地时间应比北京时间\textbf{晚\(8+N\)小时}。\\\noindent\textbf{答案:}C




\bigskip


\noindent\textbf{试题:}已知某业余电台处于东N时区(N为0-12间的整数),该台的当地时间应比北京时间:
\begin{enumerate}[leftmargin=3em]
\item 早\(8+N\)小时
\item 晚\(8+N\)小时
\item 晚\(8-N\)小时
\item 早\(8-N\)小时
\end{enumerate}
\noindent\textbf{解说:}已知某业余电台处于东N时区(N为0-12间的整数),该台的当地时间应比北京时间\textbf{晚\(8-N\)小时}。\\\noindent\textbf{答案:}C



\bigskip


\noindent\textbf{试题:}为划分无线电频率,国际电信联盟《无线电规则》进行了如下的区域划分:
\begin{enumerate}[leftmargin=3em]
\item 将世界划分为40个区域,中国位于第24、25区
\item 将世界划分为17个区域,中国位于第8区
\item 将世界划分为3个区域,中国位于第3区
\item 将世界划分为89个区域,中国位于第33、42、43、44、45、50区
\end{enumerate}
\noindent\textbf{解说:}ITU《无线电规则》(2020年版)第5.2款中\textbf{把世界分成三个区域,中国位于第3区}。\\\noindent\textbf{答案:}C



\bigskip


\noindent\textbf{试题:}在业余无线电通信中,经常用到把全球分为三个区域的分区办法。制定该分区的国际机构及其公布的文件分别为:
\begin{enumerate}[leftmargin=3em]
\item 国际电信联盟ITU,《无线电规则》
\item 美国《CQ》杂志,《WAZ奖状规则》
\item 美国业余无线电协会ARRL,《业余无线电手册》
\item 国际业余无线电协会IARU,《IARU新闻》
\end{enumerate}
\noindent\textbf{解说:}\textbf{国际电信联盟ITU《无线电规则》}(2020年版)第5.2款中把世界分成三个区域。\\\noindent\textbf{答案:}A




\bigskip


\noindent\textbf{试题:}ITU的区域划分有一套详细的规则,粗略地描述大体是:
\begin{enumerate}[leftmargin=3em]
\item 南北美洲为一区,欧洲、俄罗斯亚洲部分、蒙古及部分西北亚国家为二区,亚洲(除俄罗斯、蒙古和部分西北亚洲国家)和大洋洲为三区
\item 欧洲、俄罗斯亚洲部分、蒙古及部分西北亚国家为一区,亚洲(除俄罗斯、蒙古和部分西北亚洲国家)和大洋洲为二区,南北美洲为三区%,
\item 南北美洲为一区,亚洲(除俄罗斯、蒙古和部分西北亚洲国家)和大洋洲为二区,欧洲、俄罗斯亚洲部分、蒙古及部分西北亚国家为三区
\item 欧洲、俄罗斯亚洲部分、蒙古及部分西北亚国家为一区,南北美洲为二区,亚洲(除俄罗斯、蒙古和部分西北亚洲国家)和大洋洲为三区
\end{enumerate}
\noindent\textbf{解说:}粗略地描述大体是:\textbf{欧洲、俄罗斯亚洲部分、蒙古及部分西北亚国家为一区,南北美洲为二区,亚洲(除俄罗斯、蒙古和部分西北亚洲国家)和大洋洲为三区}。\\\noindent\textbf{答案:}D



\bigskip


\noindent\textbf{试题:}业余无线电通信计算成绩时,经常用到“CQ分区”。制定该分区的民间机构及其公布的文件分别为:
\begin{enumerate}[leftmargin=3em]
\item 美国《CQ》杂志,《WAZ奖状规则》
\item 国际业余无线电协会ITU,《IARU新闻》
\item 美国业余无线电协会ARRL,《业余无线电手册》
\item 英国业余无线电协会RSGB,《无线电通信》杂志
\end{enumerate}
\noindent\textbf{解说:}制定该分区的民间机构及其公布的文件分别为:\textbf{美国《CQ》杂志,《WAZ奖状规则》}。\\\noindent\textbf{答案:}A



\bigskip


\noindent\textbf{试题:}我国所属的“CQ分区”有:
\begin{enumerate}[leftmargin=3em]
\item 42、43、44、50
\item 23、24、27
\item 42、43、44
\item 23、24
\end{enumerate}
\noindent\textbf{解说:}我国所属的“CQ分区”有\textbf{23、24、27}。\\\noindent\textbf{答案:}B



\bigskip


\noindent\textbf{试题:}我国黄岩岛、东沙岛、钓鱼岛分别属于“CQ分区”的:
\begin{enumerate}[leftmargin=3em]
\item 27、27、24
\item 27、24、24
\item 24、24、25
\item 44、44、50
\end{enumerate}
\noindent\textbf{解说:}我国黄岩岛、东沙岛、钓鱼岛分别属于“CQ分区”的\textbf{27、24、24}。\\\noindent\textbf{答案:}B


\bigskip


\noindent\textbf{试题:}“ITU分区”是IARU的活动计算通信成绩的基础。我国所属的“ITU分区”有:
\begin{enumerate}[leftmargin=3em]
\item 23、24、27
\item 23、24
\item 33、42、43、44、50
\item 33、42、43、44
\end{enumerate}
\noindent\textbf{解说:}我国所属的“ITU分区”有\textbf{33、42、43、44、50}。\\\noindent\textbf{答案:}C



\bigskip


\noindent\textbf{试题:}“ITU分区”是IARU的活动计算通信成绩的基础。我国黄岩岛、东沙岛、钓鱼岛分别属于“ITU分区”的:
\begin{enumerate}[leftmargin=3em]
\item 24、24、25
\item 50、44、44
\item 44、44、44
\item 27、24、24
\end{enumerate}
\noindent\textbf{解说:}我国黄岩岛、东沙岛、钓鱼岛分别属于“ITU分区”的\textbf{50、44、44}。\\\noindent\textbf{答案:}B




\bigskip


\noindent\textbf{试题:}业余无线电通信梅登海德网格定位系统(Maidenhead Grid Square Locator)是一种:
\begin{enumerate}[leftmargin=3em]
\item 根据国际呼号系列对地球表面进行网格划分和命名,用以标示地理位置的系统
\item 根据经纬度坐标对地球表面进行网格划分和命名,用以标示地理位置的系统
\item 根据国际政治行政区划对地球表面进行网格划分和命名,用以标示地理位置的系统
\item 卫星定位系统
\end{enumerate}
\noindent\textbf{解说:}业余无线电通信梅登海德网格定位系统(Maidenhead Grid Square Locator)是一种\textbf{根据经纬度坐标对地球表面进行网格划分和命名,用以标示地理位置的系统}。\\\noindent\textbf{答案:}B



\bigskip


\noindent\textbf{试题:}业余无线电通信常用的梅登海德网格定位系统网格名称的格式为:
\begin{enumerate}[leftmargin=3em]
\item 4位数字或者6位数字
\item 2个字母和2位数字、2个字母和2位数字再加2个字母
\item 呼号前缀字母加2位数字和2个字母
\item 4个字母或者6个字母
\end{enumerate}
\noindent\textbf{解说:}业余无线电通信常用的梅登海德网格定位系统网格名称的格式为\textbf{2个字母和2位数字、2个字母和2位数字再加2个字母}。\\\noindent\textbf{答案:}B




\bigskip


\noindent\textbf{试题:}业余无线电通信常用的梅登海德网格定位系统网格名称的长度是4字符或6字符,两者定位精度不同,差别为:
\begin{enumerate}[leftmargin=3em]
\item 4字符网格根据国际呼号系列区分,6字符网格在4字符基础上加以经纬度细分
\item 4字符网格精确到国家分区,6字符网格精确到国家的城市或县乡
\item 4字符网格名称用于HF频段通信,6字符网格名称用于VHF/UHF通信
\item 两者网格大小不同,4字符网格为经度2度和纬度1度,6字符网格为经度5分和纬度2.5分
\end{enumerate}
\noindent\textbf{解说:}差别为:\textbf{两者网格大小不同,4字符网格为经度2度和纬度1度,6字符网格为经度5分和纬度2.5分}。\\\noindent\textbf{答案:}D


\bigskip

\noindent\textbf{试题:}以下呼号前缀中,所属CQ分区与埃及相同的是:
\begin{enumerate}[leftmargin=3em]
	\item \texttt{5X}
	\item \texttt{5A}
	\item \texttt{5T}
	\item \texttt{5W}
\end{enumerate}
\noindent\textbf{解说:}呼号前缀\texttt{5A}所属CQ分区与埃及相同。\cite{zitou,cqfenqu}\\
%5A(利比亚)与埃及同属CQ分区34,5W(萨摩亚)属CQ分区32,5T(毛里塔尼亚)属CQ分区35,5X(乌干达)属CQ分区37。\\
\noindent\textbf{答案:}B

\bigskip


\noindent\textbf{试题:}对于中国HAM来说,属于既稀有又困难的是:
\begin{enumerate}[leftmargin=3em]
	\item \texttt{VK0HR}
	\item \texttt{KP5A}
	\item \texttt{P5/4L4FN} % 不明,朝鲜分配给外国人的呼号?
	\item \texttt{KP1A}
\end{enumerate}
\noindent\textbf{解说:}对于中国HAM来说,属于既稀有又困难的是\texttt{\textbf{KP5A}}。\\\noindent\textbf{答案:}B
% 其余选项为:KP1A为美国呼号,VK0HR为澳大利亚呼号,P5/4L4FN为格鲁吉亚电台在朝鲜民主主义人民共和国使用的呼号
% \texttt{KP5A}为美国/德塞切奥岛?的业余无线电呼号。
%%%

\bigskip


\noindent\textbf{试题:}\texttt{3V}、\texttt{4X}、\texttt{5A}、\texttt{6Y}字头所代表的国家是:
\begin{enumerate}[leftmargin=3em]
	\item Libya、Israel、Jamaica、Guinea
	\item Jamaica、Israel、Libya、Senegal
	\item Guinea、Israel、Fiji Islands、Senegal
	\item Tunisia、Israel、Libya、Jamaica % 原文:Tunis、Israel、Libya、Jamaica
\end{enumerate}
\noindent\textbf{解说:}\texttt{3V}、\texttt{4X}、\texttt{5A}、\texttt{6Y}字头所代表的国家是\textbf{Tunisia、Israel、Libya、Jamaica}。\\\noindent\textbf{答案:}D

\bigskip


\noindent\textbf{试题:}业余无线电通信所说的“网格定位”是什么意思?
\begin{enumerate}[leftmargin=3em]
	\item 一个由一串字母和数字确定的地理位置 % 的系统
	\item 用来调谐末级功放的设备
	\item 用于无线电测向运动的设备
	\item 一个由一串字母和数字确定的方位角和仰角
\end{enumerate}
\noindent\textbf{解说:}业余无线电通信所说的“网格定位”是\textbf{一个由一串字母和数字确定地理位置}的系统。\\\noindent\textbf{答案:}A

% 第二章 无线电通信的方法 结束




















% 第三章 无线电系统理论 开始

\chapter{无线电系统理论}

\newpage


\noindent\textbf{试题:}收发信机面板上或设置菜单中的符号VOX代表什么功能?
\begin{enumerate}[leftmargin=3em]
\item 发信机声控,接入后将根据对话筒有无语音输入的判别自动控制收发转换
\item 发信自动电平控制,对射频输出电平进行检测并反馈控制,以维持其在适当限度之内
\item 发信自动音量控制,对音频输入电平进行检测并反馈控制,以维持其在适当限度之内
\item 自动天线调谐,对天线电路的电压驻波比进行检测并进行自动补偿,以维持最小驻波比
\end{enumerate}
\noindent\textbf{解说:}VOX代表\textbf{发信机声控,接入后将根据对话筒有无语音输入的判别自动控制收发转换}。\\\noindent\textbf{答案:}A



\bigskip


\noindent\textbf{试题:}收发信机中的PTT是指什么信号?
\begin{enumerate}[leftmargin=3em]
\item 按键发射,有信号(一般为对地接通)时发射机由等待转为发射
\item 发信语音压缩,对音频输入电平进行检测并反馈控制,以提升语音包络幅度较小的部分
\item 自动天线调谐,对天线电路的电压驻波比进行检测并进行自动补偿,以维持最小驻波比
\item 收信机前置放大器,在接收微弱信号时接入(此时某些技术指标可能低于额定值)
\end{enumerate}
\noindent\textbf{解说:}PTT代表\textbf{按键发射,有信号(一般为对地接通)时发射机由等待转为发射}。\\\noindent\textbf{答案:}A



\bigskip


\noindent\textbf{试题:}收发信机面板上或设置菜单中的符号SQL代表什么功能?
\begin{enumerate}[leftmargin=3em]
\item 收信机前置放大器,在接收微弱信号时接入(此时某些技术指标可能低于额定值)
\item 发信语音压缩,对音频输入电平进行检测并反馈控制,以提升语音包络幅度较小的部分
\item 自动天线调谐,对天线电路的电压驻波比进行检测并进行自动补偿,以维持最小驻波比
\item 静噪控制,检测到接收信号低于一定电平时关断音频输出
\end{enumerate}
\noindent\textbf{解说:}SQL代表\textbf{静噪控制,检测到接收信号低于一定电平时关断音频输出}。\\\noindent\textbf{答案:}D



\bigskip


\noindent\textbf{试题:}有些调频接收机的参数设置菜单有NFM和WFM两种选择。它们的含义是:
\begin{enumerate}[leftmargin=3em]
\item NFM为窄带调频方式,适用于信道带宽25\unit{\kHz}/12.5\unit{\kHz}的通信信号;WFM为宽带调频方式,适用于接收信道带宽180\unit{\kHz}左右的广播信号
\item NFM为调频通信本地方式(较低灵敏度),WFM为调频通信远程方式(最高灵敏度)
\item NFM代表数字化语音方式,WFM代表模拟语音方式
\item NFM为单频率守候方式,WFM为双频率守候方式
\end{enumerate}
\noindent\textbf{解说:NFM为窄带调频方式,适用于信道带宽25\unit{\kHz}/12.5\unit{\kHz}的通信信号;WFM为宽带调频方式,适用于接收信道带宽180\unit{\kHz}左右的广播信号}。\\\noindent\textbf{答案:}A



\bigskip


\noindent\textbf{试题:}某些对讲机具有发送DTMF码的功能。缩写DTMF指的是:
\begin{enumerate}[leftmargin=3em]
\item 数字设备识别码,即在松开PTT按键时自动发送一串代表设备代号的二进制数据
\item 自动静噪,即在接收机没有收到信号时自动关闭音频输出
\item 双音多频编码,由8个音调频率中的两个频率组合成的控制信号,代表16种状态之一,用于遥控和传输数字等简单字符
\item 亚音调静噪,即从67-250.3Hz的38个亚音调频率中选取一个作为选通信号,代表38种状态之一,接收机没有收到特定的选通信号时自动关闭音频输出
\end{enumerate}
\noindent\textbf{解说:}DTMF指\textbf{双音多频编码,由8个音调频率中的两个频率组合成的控制信号,代表16种状态之一,用于遥控和传输数字等简单字符}。\\\noindent\textbf{答案:}C



\bigskip


\noindent\textbf{试题:}某些对讲机具有发送CTCSS码的功能。缩写CTCSS指的是:
\begin{enumerate}[leftmargin=3em]
\item 自动静噪,即在接收机没有收到信号时自动关闭音频输出
\item 亚音调静噪,即从67-250.3Hz的38个亚音调频率中选取一个作为选通信号,代表38种状态之一,接收机没有收到特定的选通信号时自动关闭音频输出
\item 双音多频编码,由8个音调频率中的两个频率组合成的控制信号,代表16种状态之一,用于遥控和传输数字等简单字符
\item 数字设备识别码,即在松开PTT按键时自动发送一串代表设备代号的二进制数据
\end{enumerate}
\noindent\textbf{解说:}CTCSS指\textbf{亚音调静噪,即从67-250.3Hz的38个亚音调频率中选取一个作为选通信号,代表38种状态之一,接收机没有收到特定的选通信号时自动关闭音频输出}。\\\noindent\textbf{答案:}B



\bigskip


\noindent\textbf{试题:}关于是否可以在FM话音通信时单凭接收机听到对方语音的音量大小来准确判断对方信号的强弱,正确答案及其理由是:
\begin{enumerate}[leftmargin=3em]
\item 不能。因为信号越强,自动增益控制作用也越强,增益的急剧减小使声音反而被压低
\item 能。调频信号越强,频偏也必然越大,解调后的声音也越大
\item 不能。因为鉴频输出大小只取决于射频信号的频偏,而且正常信号的幅度会被限幅电路切齐到同样大小
\item 能。最后的信号是接收到的射频信号经过放大处理得到的,当然信号强声音越大
\end{enumerate}
\noindent\textbf{解说:}正确答案及其理由是:\textbf{不能。因为鉴频输出大小只取决于射频信号的频偏,而且正常信号的幅度会被限幅电路切齐到同样大小}。\\\noindent\textbf{答案:}C


\bigskip


\noindent\textbf{试题:}用设置在NFM方式的对讲机接收WFM信号,其效果为:
\begin{enumerate}[leftmargin=3em]
\item 可以正常听到信号,但声音比较小
\item 可以听到信号,但当调制信号幅度较大、音调较高时会发生明显非线性失真
\item 听不到信号,但接收到信号时调频噪声会变得寂静
\item 可以正常听到信号,但声音的高音频部分衰减较大,缺乏高音
\end{enumerate}
\noindent\textbf{解说:}用设置在NFM方式的对讲机接收WFM信号,其效果为\textbf{可以听到信号,但当调制信号幅度较大、音调较高时会发生明显非线性失真}。\\\noindent\textbf{答案:}B



\bigskip


\noindent\textbf{试题:}用设置在WFM方式的对讲机接收NFM信号,其效果为:
\begin{enumerate}[leftmargin=3em]
\item 可以正常听到信号,但声音的高音频部分衰减较大,缺乏高音
\item 可以听到信号,但当调制信号幅度较大、音调较高时会发生明显非线性失真
\item 可以正常听到信号,但声音比较小
\item 听不到信号,但接收到信号时调频噪声会变得寂静
\end{enumerate}
\noindent\textbf{解说:}用设置在WFM方式的对讲机接收NFM信号,其效果为\textbf{可以正常听到信号,但声音比较小}\\\noindent\textbf{答案:}C



\bigskip


\noindent\textbf{试题:}调频接收机没有接收到信号时,会输出强烈的噪声。关于这种噪声的描述是:
\begin{enumerate}[leftmargin=3em]
\item 由天线背景噪声和机内电路噪声的随机频率变化经鉴频形成,其大小与天线接收到的背景噪声幅度无关
\item 由天线接收到的背景噪声的随机幅度变化经放大形成,其大小与天线背景噪声电压的平方根成正比
\item 由天线接收到的背景噪声的随机幅度变化经放大形成,其大小与天线背景噪声电压的平方成正比
\item 由天线接收到的背景噪声的随机幅度变化经放大形成,其大小与天线背景噪声电压成正比
\end{enumerate}
\noindent\textbf{解说:}这种噪声\textbf{由天线背景噪声和机内电路噪声的随机频率变化经鉴频形成,其大小与天线接收到的背景噪声幅度无关}。\\\noindent\textbf{答案:}A


\bigskip


\noindent\textbf{试题:}如果业余中继台发射机被断断续续的干扰信号所启动,夹杂着不清楚的语音,根据覆盖区内其他业余电台的监听,确定中继台上行频率并没有电台工作。则:
\begin{enumerate}[leftmargin=3em]
\item 可能是中继台发射机发生了寄生振荡
\item 肯定是中继台接收机受到了人为恶意干扰
\item 可能是中继台接收机发生了寄生振荡
\item 可能是中继台附近的两个其他发射机的强信号在中继台上行频率造成了互调干扰
\end{enumerate}
\noindent\textbf{解说:可能是中继台附近的两个其他发射机的强信号在中继台上行频率造成了互调干扰}。\\\noindent\textbf{答案:}D



\bigskip


\noindent\textbf{试题:}业余电台在进行业余卫星通信时使用超过常规要求的发射功率,造成的结果以及对这种做法的态度是:
\begin{enumerate}[leftmargin=3em]
\item 上行功率太大造成浪费和电磁污染;不提倡
\item 上行功率超过一定值对通信效果改善不大,但并无明显坏处;无所谓
\item 上行功率越大,转发的效果越好,通信范围越大;可提倡
\item 过强的上行信号会使卫星转发器压低对其他信道的转发功率,严重影响别人通信;必须反对
\end{enumerate}
\noindent\textbf{解说:}\textbf{过强的上行信号会使卫星转发器压低对其他信道的转发功率,严重影响别人通信;必须反对}。\\\noindent\textbf{答案:}D



\bigskip


\noindent\textbf{试题:}即使在空旷平地,接收到的本地VHF/UHF信号强度也可能会随着接收位置的移动而发生变化,最主要的可能原因是:
\begin{enumerate}[leftmargin=3em]
\item 移动过程中设备与大地之间的分布电容发生微小的变动
\item 收发位置之间的空气流动造成电波折射不均匀而飘移
\item 直射和经地面反射等多条路径到达的电波相位不同,互相叠加或抵消造成衰落(多径效应)
\item 不同接收位置大地导电率有差异
\end{enumerate}
\noindent\textbf{解说:}最主要的可能原因是:\textbf{直射和经地面反射等多条路径到达的电波相位不同,互相叠加或抵消造成衰落(多径效应)}。\\\noindent\textbf{答案:}C




\bigskip


\noindent\textbf{试题:}在相距不远的两点接收同一个远方信号,信号强度发生很大差别,且差别随两点间距离的增大呈周期性变化。这是因为:
\begin{enumerate}[leftmargin=3em]
\item 多径传播,各路径到达的信号相位延迟不同而互相干涉
\item 大气扰动影响
\item 发射机到两接收点传播距离不同造成传播路径衰耗不同
\item 地磁影响影响
\end{enumerate}
\noindent\textbf{解说:}这是因为\textbf{多径传播,各路径到达的信号相位延迟不同而互相干涉}。\\\noindent\textbf{答案:}A



\bigskip


\noindent\textbf{试题:}下列哪种设备可以用来代替普通的扬声器,可在嘈杂的环境中更好地抄收语音信号?
\begin{enumerate}[leftmargin=3em]
\item 视频显示器
\item 耳机
\item 低通滤波器
\item 吊杆话筒
\end{enumerate}
\noindent\textbf{解说:耳机}可以用来代替普通的扬声器,可在嘈杂的环境中更好地抄收语音信号。\\\noindent\textbf{答案:}B



\bigskip


\noindent\textbf{试题:}收发信机中的静噪控制的目的是什么?
\begin{enumerate}[leftmargin=3em]
\item 可以进行自动增益控制
\item 使接收机的输出音量调到最大
\item 控制发射机的输出功率
\item 在没有信号的情况下,关闭音频输出,使其不会输出噪音%在没有信号的情况下,关闭音频输出,使其不会输出噪音。
\end{enumerate}
\noindent\textbf{解说:}收发信机中的静噪控制的目的是\textbf{在没有信号的情况下,关闭音频输出,使其不会输出噪音}。\\\noindent\textbf{答案:}D




\bigskip


\noindent\textbf{试题:}如果对方报告你的调频电台发射的信号听起来失真严重、可辨度差,可能的原因是:
\begin{enumerate}[leftmargin=3em]
\item 三项都可能
\item 电台所处的位置不好
\item 电台的电源电压不足
\item 电台的发射频率不准确
\end{enumerate}
\noindent\textbf{解说:}如果对方报告你的调频电台发射的信号听起来失真严重、可辨度差,可能的原因是:
\begin{enumerate}[label=, leftmargin=3em]
\item \textbf{电台的电源电压不足};
\item \textbf{电台所处的位置不好};
\item \textbf{电台的发射频率不准确}。
\end{enumerate}
\noindent\textbf{答案:}A



\bigskip


\noindent\textbf{试题:}一些VHF/UHF业余无线电调频手持对讲机或车载台的设置菜单中有“全频偏”和“半频偏”的选择,其表示的意义是:
\begin{enumerate}[leftmargin=3em]
\item 分别表示信道间隔为25\unit{\kHz}或者12.5\unit{\kHz}
\item 全频偏方式适用于收发异频的中继台通信,半频偏方式适用于同频对讲通信
\item 全频偏方式语音信号经过压缩,半频偏方式语音信号只压缩低音频分量
\item 全频偏方式下发射频率的误差比半频偏方式大一倍
\end{enumerate}
\noindent\textbf{解说:}\textbf{“全频偏”表示信道间隔为25\unit{\kHz},“半频偏”表示信道间隔为12.5\unit{\kHz}}。\\\noindent\textbf{答案:}A



\bigskip


\noindent\textbf{试题:}发现有业余电台的发射操作技巧不够规范,但还不至于造成严重的干扰和影响,正确的做法是:
\begin{enumerate}[leftmargin=3em]
\item 立即报告无线电管理机构进行干涉
\item 立即在频率上当面加以指出和纠正
\item 通过电话、邮件等方式提出善意的改进建议
\item 立即报告当地业余无线电协会,由其总部电台到频率上进行纠察
\end{enumerate}
\noindent\textbf{解说:}正确的做法是\textbf{通过电话、邮件等方式提出善意的改进建议}。\\\noindent\textbf{答案:}C


\bigskip


\noindent\textbf{试题:}接收机设置项目中缩写“NB”和“SQL”的中文简称和作用是:
\begin{enumerate}[leftmargin=3em]
\item NB为“抑噪”,切除高于平均信号的大幅度突发脉冲噪声;SQL为“静噪”,信噪比达不到一定水平时自动关闭音频输出
\item NB和SQL都是指“静噪”,切除高于平均信号的大幅度突发脉冲噪声
\item NB和SQL都是指“抑噪”,收不到有用信号时自动关断背景噪声
\item NB和SQL都是指“静噪”,收不到带有预期的特定控制信号时自动关断音频输出
\end{enumerate}
\noindent\textbf{解说:}\textbf{NB为“抑噪”,切除高于平均信号的大幅度突发脉冲噪声;SQL为“静噪”,信噪比达不到一定水平时自动关闭音频输出}。\\\noindent\textbf{答案:}A


\bigskip


\noindent\textbf{试题:}单边带发信机语音压缩的作用是:
\begin{enumerate}[leftmargin=3em]
\item 压低语音信号的低频分量,提升高频分量,以增加信号的带宽,使高音更加细腻
\item 压缩信号所占用的频谱宽度,提高无线电频谱的利用率
\item 压低较强语音信号的幅度、提升较弱信号的幅度,以改善小幅度语音在接收端的信噪比
\item 压低较弱语音信号的幅度、提升较强信号的幅度,以增加语音的动态范围和抑扬顿挫感
\end{enumerate}
\noindent\textbf{解说:}单边带发信机语音压缩的作用是\textbf{压低较强语音信号的幅度、提升较弱信号的幅度,以改善小幅度语音在接收端的信噪比}。\\\noindent\textbf{答案:}C




\bigskip


\noindent\textbf{试题:}收发信机面板上的符号ALC代表什么功能?
\begin{enumerate}[leftmargin=3em]
\item 自动天线调谐,对天线电路的电压驻波比进行检测并进行自动补偿,以维持最小驻波比
\item 发信自动电平控制,对射频输出电平进行检测并反馈控制,以维持其在适当限度之内
\item 发信自动音量控制,对音频输入电平进行检测并反馈控制,以维持其在适当限度之内
\item 自动频率控制,对发射频率的漂移进行检测并反馈控制,以维持准确的工作频率
\end{enumerate}
\noindent\textbf{解说:}ALC代表\textbf{发信自动电平控制,对射频输出电平进行检测并反馈控制,以维持其在适当限度之内}。\\\noindent\textbf{答案:}B



\bigskip


\noindent\textbf{试题:}收发信机面板上的符号AT代表什么功能?
\begin{enumerate}[leftmargin=3em]
\item 发信自动电平控制,对射频输出电平进行检测并反馈控制,以维持其在适当限度之内
\item 自动频率控制,对发射频率的漂移进行检测并反馈控制,以维持准确的工作频率
\item 发信自动音量控制,对音频输入电平进行检测并反馈控制,以维持其在适当限度之内
\item 自动天线调谐,对天线电路的电压驻波比进行检测并进行自动补偿,以维持最小驻波比
\end{enumerate}
\noindent\textbf{解说:}AT代表\textbf{自动天线调谐,对天线电路的电压驻波比进行检测并进行自动补偿,以维持最小驻波比}。\\\noindent\textbf{答案:}D




\bigskip


\noindent\textbf{试题:}收发信机面板上的符号ATT代表什么功能?
\begin{enumerate}[leftmargin=3em]
\item 发信自动电平控制,对射频输出电平进行检测并反馈控制,以维持其在适当限度之内
\item 发信自动音量控制,对音频输入电平进行检测并反馈控制,以维持其在适当限度之内
\item 收信机输入衰减器,在接收大信号时接入,使信号不致过大而使前级电路过载
\item 自动天线调谐,对天线电路的电压驻波比进行检测并进行自动补偿,以维持最小驻波比
\end{enumerate}
\noindent\textbf{解说:}ATT代表\textbf{收信机输入衰减器,在接收大信号时接入,使信号不致过大而使前级电路过载}。\\\noindent\textbf{答案:}C




\bigskip


\noindent\textbf{试题:}收发信机面板上的符号AGC代表什么功能?
\begin{enumerate}[leftmargin=3em]
\item 发信自动电平控制,对射频输出电平进行检测并反馈控制,以维持其在适当限度之内
\item 收信自动音量控制,对音频输出电平进行检测并反馈控制,以维持其在适当限度之内
\item 自动天线调谐,对天线电路的电压驻波比进行检测并进行自动补偿,以维持最小驻波比
\item 收信机自动增益控制,对中频级信号电平进行检测并反馈控制,防止电路过载
\end{enumerate}
\noindent\textbf{解说:}AGC代表\textbf{收信机自动增益控制,对中频级信号电平进行检测并反馈控制,防止电路过载}。\\\noindent\textbf{答案:}D



\bigskip


\noindent\textbf{试题:}收发信机面板上的符号PRE代表什么功能?
\begin{enumerate}[leftmargin=3em]
\item 发信语音压缩,对音频输入电平进行检测并反馈控制,以提升语音包络幅度较小的部分
\item 自动天线调谐,对天线电路的电压驻波比进行检测并进行自动补偿,以维持最小驻波比
\item 发信自动电平控制,对射频输出电平进行检测并反馈控制,以维持其在适当限度之内
\item 收信机前置放大器,在接收微弱信号时接入(此时某些技术指标可能低于额定值)
\end{enumerate}
\noindent\textbf{解说:}PRE代表\textbf{收信机前置放大器,在接收微弱信号时接入(此时某些技术指标可能低于额定值)}。\\\noindent\textbf{答案:}D


\bigskip


\noindent\textbf{试题:}收发信机面板上的符号PROC代表什么功能?
\begin{enumerate}[leftmargin=3em]
\item 自动天线调谐,对天线电路的电压驻波比进行检测并进行自动补偿,以维持最小驻波比
\item 发信自动电平控制,对射频输出电平进行检测并反馈控制,以维持其在适当限度之内
\item 发信语音压缩,对音频输入电平进行检测并反馈控制,以提升语音包络幅度较小的部分
\item 收信机前置放大器,在接收微弱信号时接入(此时某些技术指标可能低于额定值)
\end{enumerate}
\noindent\textbf{解说:}PROC代表\textbf{发信语音压缩,对音频输入电平进行检测并反馈控制,以提升语音包络幅度较小的部分}。\\\noindent\textbf{答案:}C



\bigskip


\noindent\textbf{试题:}业余收发信机面饭上RIT的中文名称和代表的意义是:
\begin{enumerate}[leftmargin=3em]
\item 清除信道频率存贮器的所有数据
\item 接收增量调谐,在接收频率的主调谐不变的基础上,对接收频率进行附加微调
\item 发射增量调谐,在发射频率的主调谐不变的基础上,对发射频率进行附加微调
\item 异频收发,接收和发射使用互相独立的频率
\end{enumerate}
\noindent\textbf{解说:}RIT的中文名称是\textbf{接收增量调谐},其代表的意义是\textbf{在接收频率的主调谐不变的基础上,对接收频率进行附加微调}。\\\noindent\textbf{答案:}B




\bigskip


\noindent\textbf{试题:}业余收发信机面饭上XIT的中文名称和代表的意义是:
\begin{enumerate}[leftmargin=3em]
\item 异频收发,接收和发射使用互相独立的频率
\item 清除信道频率存贮器的所有数据
\item 发射增量调谐,在发射频率的主调谐不变的基础上,对发射频率进行附加微调
\item 接收增量调谐,在接收频率的主调谐不变的基础上,对接收频率进行附加微调
\end{enumerate}
\noindent\textbf{解说:}XIT的中文名称是\textbf{发射增量调谐},其代表的意义是\textbf{在发射频率的主调谐不变的基础上,对发射频率进行附加微调}。\\\noindent\textbf{答案:}C




\bigskip


\noindent\textbf{试题:}收听射频/中频增益和音频增益分开控制的通信接收机时,较好的设置方法是:
\begin{enumerate}[leftmargin=3em]
\item 信号特弱时尽量把音频增益开到最大,信号特强时尽量把射频/中频增益开到最大,然后从低到高调整另一个增益以得到适当的音量
\item 任何情况下都应将音频增益放在中间位置,然后从低到高调整射频/中频增益以得到适当的音量
\item 任何情况下都应将射频/中频增益放在中间位置,然后从低到高调整音频增益以得到适当的音量
\item 信号特弱时尽量把射频/中频增益开到最大,信号特强时尽量把音频增益开到最大,然后从低到高调整另一个增益以得到适当的音量
\end{enumerate}
\noindent\textbf{解说:}较好的设置方法是\textbf{信号特弱时尽量把射频/中频增益开到最大,信号特强时尽量把音频增益开到最大,然后从低到高调整另一个增益以得到适当的音量}。\\\noindent\textbf{答案:}D



\bigskip


\noindent\textbf{试题:}如果短波业余电台附近环境有不可避免的强烈噪声源影响接收微弱信号,合理的做法是:
\begin{enumerate}[leftmargin=3em]
\item 尽量只呼叫和回答能听到的电台,必须发起CQ呼叫时应降低功率
\item 发起CQ呼叫时应增大功率,以便压倒环境噪声
\item 发起正常满功率CQ呼叫,并应尽量调小接收机射频增益
\item 发起正常满功率CQ呼叫,并设置好接收机的NB、AGC等功能
\end{enumerate}
\noindent\textbf{解说:}合理的做法是\textbf{尽量只呼叫和回答能听到的电台,必须发起CQ呼叫时应降低功率}。\\\noindent\textbf{答案:}A




\bigskip


\noindent\textbf{试题:}某业余通信接收机的中频滤波器带宽有100Hz、400Hz、2.7\unit{\kHz}和6\unit{\kHz}几挡选择。如果要为接收SSB、AM、PSK31和CW方式的信号分别从中选择合适的挡位,应该依次为:
\begin{enumerate}[leftmargin=3em]
\item 2.7\unit{\kHz}、6\unit{\kHz}、100Hz、400Hz
\item 2.7\unit{\kHz}、400Hz、6\unit{\kHz}、100Hz
\item 2.7\unit{\kHz}、100Hz、6\unit{\kHz}、400Hz
\item 6\unit{\kHz}、2.7\unit{\kHz}、400Hz、100Hz
\end{enumerate}
\noindent\textbf{解说:}应该依次为\textbf{2.7\unit{\kHz}、6\unit{\kHz}、100Hz、400Hz}。\\\noindent\textbf{答案:}A




\bigskip


\noindent\textbf{试题:}单边带发信机中发信自动电平控制ALC的主要作用是:
\begin{enumerate}[leftmargin=3em]
\item 实现天线电路阻抗的自动匹配
\item 防止过驱动带来的调制失真
\item 防止话筒过于灵敏造成的背景噪音
\item 改善发信频率的稳定度
\end{enumerate}
\noindent\textbf{解说:}ALC的主要作用是\textbf{防止过驱动带来的调制失真}。\\\noindent\textbf{答案:}B




\bigskip


\noindent\textbf{试题:}业余电台发射单边带语音信号中,语音虽然基本正常,但操作员周围噪杂的声音很响,应该:
\begin{enumerate}[leftmargin=3em]
\item 调低发射机的话筒增益
\item 调低发射机的射频输出功率
\item 重新调整发射机天线电路的匹配
\item 重新调整发信机的自动电平控制(ALC)
\end{enumerate}
\noindent\textbf{解说:}应该\textbf{调低发射机的话筒增益}。\\\noindent\textbf{答案:}A




\bigskip


\noindent\textbf{试题:}如果将发射机的话筒增益设置得过高会导致什么问题?
\begin{enumerate}[leftmargin=3em]
\item 发射机的频率会变得不稳定
\item 发射机的输出功率将会特别高
\item 发射机发射的信号可能会失真
\item 驻波比会增加
\end{enumerate}
\noindent\textbf{解说:}如果将发射机的话筒增益设置得过高会导致\textbf{发射机发射的信号可能会失真}。\\\noindent\textbf{答案:}C



\bigskip


\noindent\textbf{试题:}电台的下列哪一项控制功能可以使听起来音调偏高或偏低的SSB语音信号变得正常?
\begin{enumerate}[leftmargin=3em]
\item RIT功能
\item 带宽选择
\item 自动增益控制或限幅器
\item 哑音静噪
\end{enumerate}
\noindent\textbf{解说:}\textbf{RIT功能}可以使听起来音调偏高或偏低的SSB语音信号变得正常。\\\noindent\textbf{答案:}A




\bigskip


\noindent\textbf{试题:}下述通信不属于电信(telecommunication)范畴:
\begin{enumerate}[leftmargin=3em]
\item 光通信
\item 有线通信
\item 无线电通信
\item 邮政通信
\end{enumerate}
\noindent\textbf{解说:}\textbf{邮政通信}不属于电信范畴。\\\noindent\textbf{答案:}D



\bigskip


\noindent\textbf{试题:}下列情况会产生减幅波辐射:
\begin{enumerate}[leftmargin=3em]
\item 对讲机按键发射
\item 电路接触点打火
\item 医用高频加热器泄漏
\item 电视机本振泄漏
\end{enumerate}
\noindent\textbf{解说:}电路接触点打火\textbf{会产生减幅波辐射}。\\\noindent\textbf{答案:}B




\bigskip


\noindent\textbf{试题:}在业余无线电中,计算莫尔斯电码的WPM速度时采用的信号时值标准(以一个“点”信号的时间为比较基准)为,点信号、划信号、字符内点划信号的间隔、字符之间的间隔、单词(组)之间的间隔分别为:
\begin{enumerate}[leftmargin=3em]
\item 1、3、1、3、5
\item 1、5、2、3、3
\item 1、5、1、5、7
\item 1、3、1、3、7
\end{enumerate}
\noindent\textbf{解说:}点信号、划信号、字符内点划信号的间隔、字符之间的间隔、单词(组)之间的间隔分别为:\textbf{1、3、1、3、7}。\\\noindent\textbf{答案:}D




\bigskip


\noindent\textbf{试题:}无线电原理经常用到缩写VFO,它代表:
\begin{enumerate}[leftmargin=3em]
\item 压控振荡器
\item 可变频率振荡器
\item 可变频率石英振荡器
\item 石英晶体元件
\end{enumerate}
\noindent\textbf{解说:}VFO代表\textbf{可变频率振荡器}。\\\noindent\textbf{答案:}B



\bigskip


\noindent\textbf{试题:}无线电原理经常用到缩写XTAL,它代表:
\begin{enumerate}[leftmargin=3em]
\item 石英晶体元件
\item 晶体振荡器
\item 压控振荡器
\item 可变频率振荡器
\end{enumerate}
\noindent\textbf{解说:}XTAL代表\textbf{石英晶体元件}。\\\noindent\textbf{答案:}A



\bigskip


\noindent\textbf{试题:}业余无线电慢扫描电视传送的是:
\begin{enumerate}[leftmargin=3em]
\item 逐行扫描的活动图像
\item 交叉扫描的活动图像
\item 逐行扫描的静止图像
\item 交叉扫描的静止图像
\end{enumerate}
\noindent\textbf{解说:}业余无线电慢扫描电视传送的是\textbf{逐行扫描的静止图像}。\\\noindent\textbf{答案:}C



\bigskip


\noindent\textbf{试题:}地球同步(geosynchronous)卫星是指:
\begin{enumerate}[leftmargin=3em]
\item 运行周期等于地球自转周期的地球卫星
\item 所经过地点的地方时基本相同的卫星
\item 其轨道平面通过地球北极和南极地区的卫星
\item 瞬时轨道平面与太阳始终保持固定取向的卫星
\end{enumerate}
\noindent\textbf{解说:}地球同步卫星是指\textbf{运行周期等于地球自转周期的地球卫星}。\\\noindent\textbf{答案:}A




\bigskip


\noindent\textbf{试题:}太阳同步(轨道)(Sun-synchronousorbit)卫星是指:
\begin{enumerate}[leftmargin=3em]
\item 运行周期等于地球自转周期的地球卫星
\item 对地球保持大致相对静止的卫星
\item 瞬时轨道平面与太阳始终保持固定取向的卫星
\item 圆形及顺行轨道位于地球赤道平面上,并对地球保持相对静止的卫星
\end{enumerate}
\noindent\textbf{解说:}太阳同步(轨道)卫星是指\textbf{瞬时轨道平面与太阳始终保持固定取向的卫星}。\\\noindent\textbf{答案:}C



\bigskip


\noindent\textbf{试题:}地球(geostationary)静止卫星是指:
\begin{enumerate}[leftmargin=3em]
\item 其轨道平面通过地球北极和南极地区的卫星
\item 所有的地球同步卫星
\item 对地球保持大致相对静止的卫星
\item 所经过地点的地方时基本相同的卫星
\end{enumerate}
\noindent\textbf{解说:}地球静止卫星是指\textbf{对地球保持大致相对静止的卫星}。\\\noindent\textbf{答案:}C




\bigskip


\noindent\textbf{试题:}卫星的周期是指:
\begin{enumerate}[leftmargin=3em]
\item 卫星两次正好从地面某一点的正上方通过的间隔时间
\item 卫星随地球绕太阳一周所需的时间
\item 卫星沿轨道绕地球一周所需的时间
\item 卫星绕质心自旋一周所需的时间
\end{enumerate}
\noindent\textbf{解说:}卫星的周期是指\textbf{卫星沿轨道绕地球一周所需的时间}。\\\noindent\textbf{答案:}C



\bigskip


\noindent\textbf{试题:}业余低轨卫星的转发器覆盖范围有限。利用这类卫星进行全球性业余无线电通信的解决方法是:
\begin{enumerate}[leftmargin=3em]
\item 由地面站进行地面中继
\item 由卫星对上行数据进行存贮和转发
\item 地面业余电台换用更大功率的发射机
\item 增加地面业余电台的天线高度
\end{enumerate}
\noindent\textbf{解说:}解决方法是\textbf{由卫星对上行数据进行存贮和转发}。\\\noindent\textbf{答案:}B




\bigskip


\noindent\textbf{试题:}我国发射的第一颗业余卫星的发射年份、名称、国际OSCAR系列号和转发器模式分别为:
\begin{enumerate}[leftmargin=3em]
\item 2010年,希望一号(XW-1),HO68,V/U(J)
\item 2008年,希望一号(XW-1),无OSCAR编号,B
\item 2008年,希望一号(XW-1),HO68,V/U(J)
\item 2009年,希望一号(XW-1),HO68,V/U(J)
\end{enumerate}
\noindent\textbf{解说:}我国发射的第一颗业余卫星的发射年份、名称、国际OSCAR系列号和转发器模式分别为\textbf{2009年,希望一号(XW-1),HO68,V/U(J)}。\\\noindent\textbf{答案:}D




\bigskip


\noindent\textbf{试题:}根据数据串行通信收发两端的时钟只需要在传送一个字节的时间内保持同步还是需要在传送一整块数据的时间内保持同步,可以分为“异步”和“同步”两种方式。下列业余无线电数字通信方式中属于异步方式的例子是:
\begin{enumerate}[leftmargin=3em]
\item PACKET
\item PACTOR-II
\item QPSK31
\item RTTY
\end{enumerate}
\noindent\textbf{解说:}\textbf{RTTY}属于异步方式。\\\noindent\textbf{答案:}D



\bigskip


\noindent\textbf{试题:}根据数据串行通信收发两端的时钟只需要在传送一个字节的时间内保持同步还是需要在传送一整块数据的时间内保持同步,可以分为“异步”和“同步”两种方式。下列业余无线电数字通信方式中不属于同步方式的例子是:
\begin{enumerate}[leftmargin=3em]
\item PACKET
\item RTTY
\item BPSK31
\item AMTOR-FEC
\end{enumerate}
\noindent\textbf{解说:}\textbf{RTTY}不属于同步方式。\\\noindent\textbf{答案:}B



\bigskip


\noindent\textbf{试题:}在进行串行异步数字通信时,双方需要设置相同的波特率,数据位数,校验位数,停止位数。RTTY最常用的设置为:
\begin{enumerate}[leftmargin=3em]
\item 50(或45.45),5,N,1
\item 50(或45.45),8,2,3
\item 2295,2125,170,5
\item 31.25,7,170,0.3
\end{enumerate}
\noindent\textbf{解说:}RTTY最常用的设置为\textbf{50(或45.45),5,N,1}。\\\noindent\textbf{答案:}A



\bigskip


\noindent\textbf{试题:}PACKET是业余无线电爱好者利用X.25数据分组通信协议开发的业余无线电通信方式,用于HF频段、VHF频段和卫星通信时通常采用的信号速率分别为:
\begin{enumerate}[leftmargin=3em]
\item 300波特、1200波特、9600波特
\item 300波特、2400波特、19200波特
\item 600波特、1200波特、2400波特
\item 1200波特、9600波特、19200波特
\end{enumerate}
\noindent\textbf{解说:}用于HF频段、VHF频段和卫星通信时通常采用的信号速率分别为\textbf{300波特、1200波特、9600波特}。\\\noindent\textbf{答案:}A



\bigskip


\noindent\textbf{试题:}国际2号电报码(ITA2)的俗称、在业余无线电通信中应用场合及其与计算机常用的数据交换代码相比的主要特点是:
\begin{enumerate}[leftmargin=3em]
\item GB2312编码,用于传输汉字,由两个字节连用代表一个汉字
\item ASCII码,用于PSK31通信,每字节仅包含7位二进制数据
\item 博多码(Baudot code),用于RTTY通信,每字节仅包含5位二进制数据
\item 莫尔斯电码,用于CW通信,每个字符包含的信号数量不等(非对称码)
\end{enumerate}
\noindent\textbf{解说:}国际2号电报码(ITA2)的俗称是\textbf{博多码(Baudot code)}、在业余无线电通信中应用场合及其与计算机常用的数据交换代码相比的主要特点是\textbf{用于RTTY通信,每字节仅包含5位二进制数据}。\\\noindent\textbf{答案:}C



\bigskip


\noindent\textbf{试题:}亚音调静噪(CTCSS)是指附加在发射端信号中的一个亚音频控制音调。这个信号的频率范围大致是:
\begin{enumerate}[leftmargin=3em]
\item 220Hz - 2503Hz
\item 16Hz - 20\unit{\kHz}
\item 67Hz - 250.3Hz
\item 16\unit{\kHz} - 20\unit{\kHz}
\end{enumerate}
\noindent\textbf{解说:}这个信号的频率范围大致是\textbf{67Hz - 250.3Hz}。\\\noindent\textbf{答案:}C



\bigskip


\noindent\textbf{试题:}在超外差式收信机电路中,信号通道的有用信号频率比本振频率低(或者高)一个中频频率。但比本振频率高(或者低)一个中频频率的信号也可能窜入信号通道,称为“镜像频率干扰”或“镜频干扰”。某VHF对讲机使用说明书的技术指标部分给出了第一中频(IF)为45.05\unit{\MHz},但没有更多的资料。由此可推测当接收145.00\unit{\MHz}信号时下述频率之一的强信号可能造成镜频干扰:
\begin{enumerate}[leftmargin=3em]
\item 90.10\unit{\MHz}或.180.20\unit{\MHz}
\item 190.05\unit{\MHz}或99.95\unit{\MHz}
\item 235.10\unit{\MHz}或54.90\unit{\MHz}
\item 45.05\unit{\MHz}或90.10\unit{\MHz}
\end{enumerate}
\noindent\textbf{解说:}由此可推测当接收145.00\unit{\MHz}信号时,\textbf{235.10\unit{\MHz}或54.90\unit{\MHz}}之一的强信号可能造成镜频干扰。\\\noindent\textbf{答案:}C




\bigskip


\noindent\textbf{试题:}在超外差式收信机电路中,信号通道的有用信号频率比本振频率低(或者高)一个中频频率。但比本振频率高(或者低)一个中频频率的信号也可能窜入信号通道,称为“镜像频率干扰”或“镜频干扰”。某对讲机使用说明书的技术指标部分给出了在NFM方式时第一中频(IF)为47.25\unit{\MHz},但没有更多的资料。由此可推测当接收145.00\unit{\MHz}信号时下述频率之一的强信号可能造成镜频干扰:%:
\begin{enumerate}[leftmargin=3em]
\item 50.50\unit{\MHz}或101.00\unit{\MHz}
\item 151.50\unit{\MHz}或.202.00\unit{\MHz}
\item 192.25\unit{\MHz}或97.75\unit{\MHz}
\item 239.50\unit{\MHz}或50.50\unit{\MHz}
\end{enumerate}
\noindent\textbf{解说:}由此可推测当接收145.00\unit{\MHz}信号时,\textbf{239.50\unit{\MHz}或50.50\unit{\MHz}}之一的强信号可能造成镜频干扰。\\\noindent\textbf{答案:}D



\bigskip


\noindent\textbf{试题:}在超外差式收信机电路中,信号通道的有用信号频率比本振频率低(或者高)一个中频频率。但比本振频率高(或者低)一个中频频率的信号也可能窜入信号通道,称为“镜像频率干扰”或“镜频干扰”。某对讲机使用说明书的技术指标部分给出了接收NFM信号时第一中频(IF)为47.25\unit{\MHz},但没有更多的资料。由此可推测当接收435.00\unit{\MHz}信号时下述频率之一的强信号可能造成镜频干扰:
\begin{enumerate}[leftmargin=3em]
\item 47.25\unit{\MHz}或94.50\unit{\MHz}
\item 141.70\unit{\MHz}或.236.25\unit{\MHz}
\item 387.75\unit{\MHz}或482.25\unit{\MHz}
\item 340.50\unit{\MHz}或529.50\unit{\MHz}
\end{enumerate}
\noindent\textbf{解说:}由此可推测当接收435.00\unit{\MHz}信号时\textbf{340.50\unit{\MHz}或529.50\unit{\MHz}}之一的强信号可能造成镜频干扰。\\\noindent\textbf{答案:}D



\bigskip


\noindent\textbf{试题:}在超外差式收信机电路中,信号通道的有用信号频率比本振频率低(或者高)一个中频频率。但比本振频率高(或者低)一个中频频率的信号也可能窜入信号通道,称为“镜像频率干扰”或“镜频干扰”。某UHF对讲机的使用说明书技术指标部分给出了第一中频(IF)为58.525\unit{\MHz},但没有更多的资料。由此可推测当接收435.00\unit{\MHz}信号时下述频率之一的强信号可能造成镜频干扰:
\begin{enumerate}[leftmargin=3em]
\item 58.525\unit{\MHz}或117.05\unit{\MHz}
\item 376.475\unit{\MHz}或493.525\unit{\MHz}
\item 234.10.05\unit{\MHz}或.468.20\unit{\MHz}
\item 317.95\unit{\MHz}或552.05\unit{\MHz}
\end{enumerate}
\noindent\textbf{解说:}由此可推测当接收435.00\unit{\MHz}信号时\textbf{317.95\unit{\MHz}或552.05\unit{\MHz}}之一的强信号可能造成镜频干扰。\\\noindent\textbf{答案:}D




\bigskip


\noindent\textbf{试题:}现代超外差式业余收发信机面板上常设有选择中频滤波器带宽的控制部件。这些中频滤波器负责抑制的干扰种类为:
\begin{enumerate}[leftmargin=3em]
\item 中频频率干扰
\item 邻近频率干扰
\item 镜像频率干扰
\item 突发脉冲干扰
\end{enumerate}
\noindent\textbf{解说:}这些中频滤波器负责抑制的干扰种类为\textbf{邻近频率干扰}。\\\noindent\textbf{答案:}B




\bigskip


\noindent\textbf{试题:}超外差式业余收发信机中负责抑制镜像频率干扰的部件是:
\begin{enumerate}[leftmargin=3em]
\item 中频放大级中的限幅电路
\item 变频级之后的中频滤波器
\item 变频级之前的预选滤波器
\item 带有音调控制的音频滤波器
\end{enumerate}
\noindent\textbf{解说:}。超外差式业余收发信机中负责抑制镜像频率干扰的部件是\textbf{变频级之前的预选滤波器}。\\\noindent\textbf{答案:}C



\bigskip


\noindent\textbf{试题:}很多具有静噪功能的FM通信接收机在对方松开话筒PTT键后,会听到一声很明显的“嘶啦”或“喀拉”噪声拖尾,其原因是:
\begin{enumerate}[leftmargin=3em]
\item 此类电路根据鉴频输出中的强高音频噪声分量判断电台信号是否消失,从而关断音频输出。该项检测需占用一定时间,造成静噪的延迟,短时间漏出鉴频噪声
\item 该噪声由发射台话筒PTT接点跳动造成,发射到接收端
\item 该噪声是发射设备有目的地发射,作为结束发射的一种标志
\item 接收机自动增益电路的时间常数造成
\end{enumerate}
\noindent\textbf{解说:}。其原因是\textbf{此类电路根据鉴频输出中的强高音频噪声分量判断电台信号是否消失,从而关断音频输出。该项检测需占用一定时间,造成静噪的延迟,短时间漏出鉴频噪声}。\\\noindent\textbf{答案:}A



\bigskip


\noindent\textbf{试题:}能够确定直流电路中任何一个两端元件工作状况的基本参数包括:
\begin{enumerate}[leftmargin=3em]
\item 波长、电压、电容量
\item 频率、电荷量、电场强度
\item 驻波比、功率、阻抗
\item 通过电流、两端电压、电阻
\end{enumerate}
\noindent\textbf{解说:}能够确定直流电路中任何一个两端元件工作状况的基本参数包括\textbf{通过电流、两端电压、电阻}。\\\noindent\textbf{答案:}D



\bigskip


\noindent\textbf{试题:}物理量“电流”描述的是:
\begin{enumerate}[leftmargin=3em]
\item 电源所能供出的最大的电荷数量
\item 单位时间内流过电路的电荷数量
\item 流过电路的累计电荷数量
\item 电荷在电路导体内的运动速度
\end{enumerate}
\noindent\textbf{解说:}物理量“电流”描述的是\textbf{单位时间内流过电路的电荷数量}。\\\noindent\textbf{答案:}B



\bigskip


\noindent\textbf{试题:}物理量“电压”描述的是:
\begin{enumerate}[leftmargin=3em]
\item 电源把其它形式的能量转化为电能的能力
\item 电源加在电路两端驱动电子流动的力量大小
\item 单位时间内流过电路的电荷数量
\item 电源所能供出的最大的电荷数量
\end{enumerate}
\noindent\textbf{解说:}物理量“电压”描述的是\textbf{电源加在电路两端驱动电子流动的力量大小}。\\\noindent\textbf{答案:}B


\bigskip


\noindent\textbf{试题:}物理量“电动势”描述的是:
\begin{enumerate}[leftmargin=3em]
\item 单位时间内流过电路的电荷数量
\item 电源加在电路两端驱动电子流动的力量大小
\item 电源所能供出的最大的电荷数量
\item 电源把其它形式的能量转化为电能的能力
\end{enumerate}
\noindent\textbf{解说:}物理量“电动势”描述的是\textbf{电源把其它形式的能量转化为电能的能力}。\\\noindent\textbf{答案:}D



\bigskip


\noindent\textbf{试题:}物理量“电阻”描述的是:
\begin{enumerate}[leftmargin=3em]
\item 电路阻断电流所需要的过度时间
\item 电流克服电路阻力的能力大小
\item 电路对电流的阻碍力大小
\item 电路阻力所消耗的能量多少
\end{enumerate}
\noindent\textbf{解说:}物理量“电阻”描述的是\textbf{电路对电流的阻碍力大小}。\\\noindent\textbf{答案:}C



\bigskip


\noindent\textbf{试题:}物理量“功率”描述的是:
\begin{enumerate}[leftmargin=3em]
\item 电源总共能够做的功
\item 电源所能供出的最大的电荷数量
\item 负载总共消耗的能量
\item 单位时间里电所能够做的功
\end{enumerate}
\noindent\textbf{解说:}物理量“功率”描述的是\textbf{单位时间里电所能够做的功}。\\\noindent\textbf{答案:}D



\bigskip


\noindent\textbf{试题:}正弦交流电压或电流的峰值(peak value)是指:(”x\string^m”表示“x的m次方”)
\begin{enumerate}[leftmargin=3em]
\item 一个周期内瞬时值的平均乘以值2^(1/2)
\item 负半周最大幅度与正半周最大幅度的差值的二次方
\item 从零点算起的最大值
\item 负半周最大幅度与正半周最大幅度的平均值
\end{enumerate}
\noindent\textbf{解说:}正弦交流电压或电流的峰值是指\textbf{从零点算起的最大值}。\\\noindent\textbf{答案:}C


\bigskip


\noindent\textbf{试题:}正弦交流电压或电流的峰值峰-峰值(peak to peak)是指:
\begin{enumerate}[leftmargin=3em]
\item 负半周最大幅度与正半周最大幅度的差值的二次方
\item 从零点算起的最大值
\item 负半周最大幅度与正半周最大幅度的差值的平方根
\item 从负半周峰值到正半周峰值之间的差
\end{enumerate}
\noindent\textbf{解说:}正弦交流电压或电流的峰值峰-峰值是指\textbf{从负半周峰值到正半周峰值之间的差}。\\\noindent\textbf{答案:}D



\bigskip


\noindent\textbf{试题:}任意交流电压的有效值是指:(”x\string^m”表示“x的m次方”)
\begin{enumerate}[leftmargin=3em]
\item 最终转换成应用所需的有用能量的那部分电压值
\item 电压的峰值除以2^(1/2)
\item 电压的平均值乘以2^(1/2)
\item 在同一电阻上可以转换出与该交流电压效果相同的热量的直流电压
\end{enumerate}
\noindent\textbf{解说:}任意交流电压的有效值是指\textbf{在同一电阻上可以转换出与该交流电压效果相同的热量的直流电压}。\\\noindent\textbf{答案:}D



\bigskip


\noindent\textbf{试题:}把两个幅度相等、相位相差360°的正弦电压信号源相串联,得到的是:
\begin{enumerate}[leftmargin=3em]
\item 幅度为单个信号源的2倍、相位与原信号源相同的正弦电压
\item 幅度与单个信号源的相同、相位与原信号源相差180°的正弦电压
\item 幅度与单个信号源的相同、频率比原信号高一倍的正弦电压
\item 电压为0
\end{enumerate}
\noindent\textbf{解说:}得到的是\textbf{幅度为单个信号源的2倍、相位与原信号源相同的正弦电压}。\\\noindent\textbf{答案:}A



\bigskip


\noindent\textbf{试题:}把两个幅度相等、相位相差180°的正弦电压信号源相串联,得到的是:
\begin{enumerate}[leftmargin=3em]
\item 幅度与单个信号源的相同、频率比原信号高一倍的正弦电压
\item 幅度为单个信号源的2倍、相位与原信号源相同的正弦电压
\item 幅度与单个信号源的相同、相位与原信号源相差90°的正弦电压
\item 电压为0
\end{enumerate}
\noindent\textbf{解说:}得到的是\textbf{电压为0}。\\\noindent\textbf{答案:}D



\bigskip


\noindent\textbf{试题:}把两个幅度相等、相位相差90°的正弦电压信号源相串联,得到的是:
\begin{enumerate}[leftmargin=3em]
\item 幅度与单个信号源的相同、频率比原信号高一倍的正弦电压
\item 幅度为单个信号源的2倍、相位与原信号源相同的正弦电压
\item 幅度与单个信号源的相同、相位与原信号源相差45°的正弦电压
\item 幅度为单个信号源的1.41倍、相位与原信号源相差45°的正弦电压
\end{enumerate}
\noindent\textbf{解说:}得到的是\textbf{幅度为单个信号源的1.41倍、相位与原信号源相差45°的正弦电压}。\\\noindent\textbf{答案:}D



\bigskip


\noindent\textbf{试题:}将一个电阻为R的负载接到电压为U的电源上。负载中的电流I和负载消耗的功率P分别为:(U、I、R、P的单位分别为伏特、安培、欧姆、瓦特,”x\string^m”表示“x的m次方”)
\begin{enumerate}[leftmargin=3em]
\item \(I = U\times R\),\(P =U\string^2\times R\)
\item \(I = U\times R\),\(P =U/R\)
\item \(I = R/U\),\(P =U\times R\)
\item \(I = U/R\),\(P = U\string^2/R\)
\end{enumerate}
\noindent\textbf{解说:}负载中的电流I和负载消耗的功率P分别为:\(I = U/R\),\(P = U\string^2/R\)。\\
\noindent\textbf{答案:}D




\bigskip


\noindent\textbf{试题:}一个电阻为R的负载中流过的电流为I。其两端的电压U所消耗的功率P分别为:(U、I、R、P的单位分别为伏特、安培、欧姆、瓦特,”x\string^m”表示“x的m次方”)
\begin{enumerate}[leftmargin=3em]
\item \(U = I + R\),\(P =I ×R\)
\item \(U = I / R\),\(P = I\string^2/R\)
\item \(U = R / I\),\(P = R / I\string^2\)
\item \(U = I\times R\),\(P = I\string^2×R\)
\end{enumerate}
\noindent\textbf{解说:}其两端的电压U所消耗的功率P分别为:\(U = I\times R\),\(P = I\string^2\times R\)。\\\noindent\textbf{答案:}D


\bigskip


\noindent\textbf{试题:}一个电阻负载两端电压为U,流过的电流为I。它的电阻R和所消耗的功率P分别为:(U、I、R、P的单位分别为伏特、安培、欧姆、瓦特,”x\string^m”表示“x的m次方”)
\begin{enumerate}[leftmargin=3em]
\item \(R = I / U\),\(P = U\string^2/I\)
\item \(R = U/I\),\(P = U\times I\)
\item \(R = U\times I\),\(P = U/ I\string^2\)
\item \(R = U\string^2/I\),\(P = U\string^2\times I\)
\end{enumerate}
\noindent\textbf{解说:}它的电阻R和所消耗的功率P分别为:\(R = U/I\),\(P = U\times I\)。\\
\noindent\textbf{答案:}B




\bigskip


\noindent\textbf{试题:}一个电阻负载两端电压为U,所消耗的功率为P。流过负载的电流I和负载的电阻R分别为:(U、I、R、P的单位分别为伏特、安培、欧姆、瓦特,”x\string^m”表示“x的m次方”)
\begin{enumerate}[leftmargin=3em]
\item \(I = P\times U\),\(R = P/U\)
\item \(I = P/U\),\(R = U\string^2/P\)
\item \(I = U/P\),\(R = P/U\)
\item \(I = P/ U\string^2\),\(R = P\times U\)
\end{enumerate}
\noindent\textbf{解说:}流过负载的电流I和负载的电阻R分别为:\(I = P/U\),\(R = U\string^2/P\)。\\\noindent\textbf{答案:}B




\bigskip


\noindent\textbf{试题:}有阻值分别为R1和R2的两个负载,其中R1的电阻值是R2的\(N\)倍,把它们并联后接到电源上,则:(”x\string^m”表示“x的m次方”)
\begin{enumerate}[leftmargin=3em]
\item 流过R1的电流是R2的\(1/N\),R1消耗的功率是R2的\(1/N\)
\item 流过R1的电流与R2的相同,R1消耗的功率是R2的\(N\)倍
\item 流过R1的电流是R2的\(N\)倍,R1消耗的功率是R2的\(N\string^2\)倍
\item 流过R1的电流与R2的相同,R1消耗的功率是R2的\(1/N\string^2\)
\end{enumerate}
\noindent\textbf{解说:}流过R1的电流是R2的\(1/N\),R1消耗的功率是R2的\(1/N\)。\\
\noindent\textbf{答案:}A




\bigskip


\noindent\textbf{试题:}有阻值分别为R1和R2的两个负载,其中R1的电阻值是R2的\(N\)倍,把它们并联后接到电源上,则:(”x\string^m”表示“x的m次方”)
\begin{enumerate}[leftmargin=3em]
\item R1两端的电压是R2的\(N\)倍,R1消耗的功率是R2的\(N\string^2\)倍
\item R1两端的电压与R2的相同,R1消耗的功率是R2的\(N\string^2\)倍
\item R1两端的电压是R2的\(1/N\),R1消耗的功率是R2的\(1/N\string^2\)
\item R1两端的电压与R2的相同,R1消耗的功率是R2的\(1/N\)
\end{enumerate}
\noindent\textbf{解说:}R1两端的电压与R2的相同,R1消耗的功率是R2的\(1/N\)。\\
\noindent\textbf{答案:}D




\bigskip


\noindent\textbf{试题:}有阻值分别为R1和R2的两个负载,其中R1的电阻值是R2的\(N\)倍,把它们串联后接到电源上,则:(”x\string^m”表示“x的m次方”)
\begin{enumerate}[leftmargin=3em]
\item 流过R1的电流与R2的相同,R1消耗的功率是R2的\(N\)倍
\item 流过R1的电流是R2的\(1/N\),R1消耗的功率是R2的\(1/N\string^2\)
\item 流过R1的电流与R2的相同,R1消耗的功率是R2的\(1/N\)
\item 流过R1的电流是R2的\(N\)倍,R1消耗的功率是R2的\(N\string^2\)倍
\end{enumerate}
\noindent\textbf{解说:}流过R1的电流与R2的相同,R1消耗的功率是R2的\(N\)倍。\\
\noindent\textbf{答案:}A



\bigskip


\noindent\textbf{试题:}有阻值分别为R1和R2的两个负载,其中R1的电阻值是R2的\(N\)倍,把它们串联后接到电源上,则:(”x\string^m”表示“x的m次方”)
\begin{enumerate}[leftmargin=3em]
\item R1两端的电压是R2的\(N\)倍,R1消耗的功率是R2的\(N\)倍
\item R1两端的电压与R2的相同,R1消耗的功率是R2的\(N\string^2\)倍
\item R1两端的电压是R2的\(1/N\),R1消耗的功率是R2的\(1/N\string^2\)
\item R1两端的电压与R2的相同,R1消耗的功率是R2的\(1/N\)
\end{enumerate}
\noindent\textbf{解说:}R1两端的电压是R2的\(N\)倍,R1消耗的功率是R2的\(N\)倍。\\
\noindent\textbf{答案:}A



\bigskip


\noindent\textbf{试题:}已知A、B两个设备的工作电压相同,A的耗电功率是B的\(N\)倍。则:(”x\string^m”表示“x的m次方”)
\begin{enumerate}[leftmargin=3em]
\item A的工作电流是B的\(N\)倍
\item A的工作电流是B的\(N\string^(1/2)\)倍
\item A的工作电流是B的1/\(N\)倍
\item A的工作电流是B的\(N\string^2\)倍
\end{enumerate}
\noindent\textbf{解说:}A的工作电流是B的\(N\)倍。\\
\noindent\textbf{答案:}A



\bigskip


\noindent\textbf{试题:}已知A、B两个设备的工作电压相同,A的额定电流是B的\(N\)倍。则:(”x\string^m”表示“x的m次方”)
\begin{enumerate}[leftmargin=3em]
\item A的耗电功率是B的\(N\string^(1/2)\)倍
\item A的耗电功率是B的\(N\)倍
\item A的耗电功率是B的\(N\string^2\)倍
\item A的耗电功率是B的1/\(N\)倍
\end{enumerate}
\noindent\textbf{解说:}A的耗电功率是B的\(N\)倍。\\
\noindent\textbf{答案:}B


\bigskip


\noindent\textbf{试题:}将N个相同的电阻负载串联后接到电源上。与每个负载单独接到电源相比:(”x\string^m”表示“x的m次方”)
\begin{enumerate}[leftmargin=3em]
\item 串联后流过每个电阻的电流减少到\(1/N\),每个电阻的耗电功率减少到\(1/N\string^2\)
\item 串联后流过每个电阻的电流减少到\(1/N\),每个电阻的耗电功率减少到\(1/N\)
\item 串联后流过每个电阻的电流不变,每个电阻的耗电功率减少到\(1/N\)
\item 串联后流过每个电阻的电流增加到\(N\)倍,每个电阻的耗电功率增加到\(N\string^2\)倍
\end{enumerate}
\noindent\textbf{解说:}与每个负载单独接到电源相比:\textbf{串联后流过每个电阻的电流减少到\(1/N\),每个电阻的耗电功率减少到\(1/N\string^2\)}。\\
\noindent\textbf{答案:}A




\bigskip


\noindent\textbf{试题:}将N个相同的电阻负载串联后接到电源上。与每个负载单独接到电源相比:(”x\string^m”表示“x的m次方”)
\begin{enumerate}[leftmargin=3em]
\item 串联后每个电阻两端的电压减少到\(1/N\),每个电阻的耗电功率减少到1/ \(N\string^2\)
\item 串联后每个电阻两端的电压不变,每个电阻的耗电功率减少到\(1/N\)
\item 串联后每个电阻两端的电压增加到\(N\)倍,每个电阻的耗电功率增加到\(N\string^2\)倍
\item 串联后每个电阻两端的电压减少到\(1/N\),每个电阻的耗电功率减少到\(1/N\)
\end{enumerate}
\noindent\textbf{解说:}与每个负载单独接到电源相比:\textbf{串联后每个电阻两端的电压减少到\(1/N\),每个电阻的耗电功率减少到1/ \(N\string^2\)}。\\\noindent\textbf{答案:}A




\bigskip


\noindent\textbf{试题:}将N个相同的电阻负载并联后接到电源上。与每个负载单独接到电源相比:(”x\string^m”表示“x的m次方”)
\begin{enumerate}[leftmargin=3em]
\item 并联后流过每个电阻的电流增加到\(N\)倍,两个电阻的总耗电功率增加到\(N\string^2\)倍
\item 并联后流过每个电阻的电流减少到\(1/N\),两个电阻的总耗电功率减少到\(1/N\string^2\)
\item 并联后流过每个电阻的电流不变,所有电阻的总耗电功率增加到\(N\string^2\)倍
\item 并联后流过每个电阻的电流不变,所有电阻的总耗电功率增加到\(N\)倍
\end{enumerate}
\noindent\textbf{解说:}与每个负载单独接到电源相比:\textbf{并联后流过每个电阻的电流不变,所有电阻的总耗电功率增加到\(N\)倍}。\\\noindent\textbf{答案:}D



\bigskip


\noindent\textbf{试题:}将N个相同的电阻负载并联后接到电源上。与每个负载单独接到电源相比:(”x\string^m”表示“x的m次方”)
\begin{enumerate}[leftmargin=3em]
\item 并联后每个电阻两端的电压不变,所有电阻的总耗电功率增加到\(N\string^2\)倍
\item 并联后每个电阻两端的电压增加到\(N\)倍,两个电阻的总耗电功率增加到\(N\string^2\)倍
\item 并联后每个电阻两端的电压不变,所有电阻的总耗电功率增加到\(N\)倍
\item 并联后每个电阻两端的电压减少到\(1/N\),两个电阻的总耗电功率减少到\(1/N\string^2\)
\end{enumerate}
\noindent\textbf{解说:}与每个负载单独接到电源相比:\textbf{并联后每个电阻两端的电压不变,所有电阻的总耗电功率增加到\(N\)倍}。\\\noindent\textbf{答案:}C



\bigskip


\noindent\textbf{试题:}一个电阻负载,如果将其两端的工作电压提高百分之N,则:(”x\string^m”表示“x的m次方”)
\begin{enumerate}[leftmargin=3em]
\item 耗电量增加到原来的[百分之(100+N)]^2
\item 耗电量比原来的增加百分之N
\item 耗电量增加到原来的百分之(100+N)
\item 耗电量比原来的增加[百分之N]^2
\end{enumerate}
\noindent\textbf{解说:}\textbf{耗电量增加到原来的[百分之(100+N)]^2}。\\\noindent\textbf{答案:}A



\bigskip


\noindent\textbf{试题:}一个电阻负载,如果将其两端的工作电压降低百分之N,则:(”x\string^m”表示“x的m次方”)
\begin{enumerate}[leftmargin=3em]
\item 耗电量比原来的减少[百分之N]^2
\item 耗电量比原来的减少百分之N
\item 耗电量减少到原来的[百分之(100-N)]^2
\item 耗电量减少到原来的百分之(100-N)
\end{enumerate}
\noindent\textbf{解说:}\textbf{耗电量减少到原来的[百分之(100-N)]^2}。\\\noindent\textbf{答案:}C




\bigskip


\noindent\textbf{试题:}分别用电压为120V的蓄电池组和电压最大值为120V的交流变压器驱动同样的电阻负载,哪一个电阻每分钟发出的热量多?
\begin{enumerate}[leftmargin=3em]
\item 蓄电池驱动的电阻所发的热量是交流变压器上的电阻的1.4倍左右
\item 两个电源驱动的电阻发热相同
\item 蓄电池驱动的电阻所发的热量是交流变压器上的电阻的0.7倍左右
\item 蓄电池驱动的电阻所发的热量是交流变压器上的电阻的2倍左右
\end{enumerate}
\noindent\textbf{解说:}\textbf{蓄电池驱动的电阻所发的热量是交流变压器上的电阻的2倍左右}。\\\noindent\textbf{答案:}D



\bigskip


\noindent\textbf{试题:}分别用电压为120V的蓄电池组和电压有效值为120V的交流变压器驱动同样的电阻负载,哪一个电阻每分钟发出的热量多?
\begin{enumerate}[leftmargin=3em]
\item 蓄电池驱动的电阻所发的热量是交流变压器上的电阻的2倍左右
\item 蓄电池驱动的电阻所发的热量是交流变压器上的电阻的1.4倍左右
\item 蓄电池驱动的电阻所发的热量是交流变压器上的电阻的0.7倍左右
\item 两个电源驱动的电阻发热相同
\end{enumerate}
\noindent\textbf{解说:}\textbf{两个电源驱动的电阻发热相同}。\\\noindent\textbf{答案:}D



\bigskip


\noindent\textbf{试题:}分别用电压为120V的蓄电池组和电压有效值为120V的交流变压器串联二极管后驱动同样的电阻负载,哪一个电阻每分钟发出的热量多?(忽略二极管的电压降)
\begin{enumerate}[leftmargin=3em]
\item 蓄电池驱动的电阻所发的热量是交流变压器上的电阻的2倍左右
\item 蓄电池驱动的电阻所发的热量是交流变压器上的电阻的1.4倍左右
\item 蓄电池驱动的电阻所发的热量是交流变压器上的电阻的0.7倍左右
\item 两个电源驱动的电阻发热相同
\end{enumerate}
\noindent\textbf{解说:}\textbf{蓄电池驱动的电阻所发的热量是交流变压器上的电阻的2倍左右}。\\\noindent\textbf{答案:}A



\bigskip


\noindent\textbf{试题:}分别用电压为120V的蓄电池组和电压最大值为120V的交流变压器经过带电容滤波的全波整流电路驱动同样的电阻负载,哪一个电阻每分钟发出的热量多?(忽略整流器的电压降)
\begin{enumerate}[leftmargin=3em]
\item 蓄电池驱动的电阻所发的热量是交流变压器上的电阻的1.4倍左右
\item 蓄电池驱动的电阻所发的热量是交流变压器上的电阻的2倍左右
\item 蓄电池驱动的电阻所发的热量是交流变压器上的电阻的0.7倍左右
\item 两个电源驱动的电阻发热相同
\end{enumerate}
\noindent\textbf{解说:}\textbf{两个电源驱动的电阻发热相同}。\\\noindent\textbf{答案:}D



\bigskip


\noindent\textbf{试题:}分别用电压有效值为120V、频率为50Hz的交流电源和电压有效值为120V、频率为10\unit{\kHz}的方波电源驱动同样的电阻负载,哪一个电阻每分钟发出的热量多?
\begin{enumerate}[leftmargin=3em]
\item 10\unit{\kHz}电路电阻所发的热量是50Hz电路电阻的1/5倍左右
\item 10\unit{\kHz}电路电阻所发的热量是50Hz电路电阻的200倍左右
\item 两个电源驱动的电阻发热相同
\item 10\unit{\kHz}电路电阻所发的热量是50Hz电路电阻的5倍左右
\end{enumerate}
\noindent\textbf{解说:}\textbf{两个电源驱动的电阻发热相同}。\\\noindent\textbf{答案:}C



\bigskip


\noindent\textbf{试题:}在业余收发信机的常见元器件中,标有耐压指标的是:
\begin{enumerate}[leftmargin=3em]
\item 熔丝
\item 电容
\item 电阻
\item 电感
\end{enumerate}
\noindent\textbf{解说:}在业余收发信机的常见元器件中,标有耐压指标的是\textbf{电容}。\\\noindent\textbf{答案:}B

\bigskip


\noindent\textbf{试题:}在业余收发信机的常见元器件中,以额定耗散功率指标分类的是:
\begin{enumerate}[leftmargin=3em]
\item 电容
\item 电感
\item 电阻
\item 熔丝
\end{enumerate}
\noindent\textbf{解说:}在业余收发信机的常见元器件中,以额定耗散功率指标分类的是\textbf{电阻}。\\\noindent\textbf{答案:}C

\bigskip


\noindent\textbf{试题:}在业余收发信机的常见元器件中,标有额定电流指标的是:
\begin{enumerate}[leftmargin=3em]
\item 熔丝
\item 电阻
\item 电感
\item 电容
\end{enumerate}
\noindent\textbf{解说:}在业余收发信机的常见元器件中,标有额定电流指标的是\textbf{熔丝}。\\\noindent\textbf{答案:}A


\bigskip


\noindent\textbf{试题:}在业余收发信机电路中,经常用于隔直流或者给交流信号提供旁路的元件是:
\begin{enumerate}[leftmargin=3em]
\item 电感
\item 半导体开关器件
\item 电阻
\item 电容
\end{enumerate}
\noindent\textbf{解说:}在业余收发信机电路中,经常用于隔直流或者给交流信号提供旁路的元件是\textbf{电容}。\\\noindent\textbf{答案:}D

\bigskip


\noindent\textbf{试题:}在业余收发信机电路中,经常用谐振回路来筛选一定频率的信号。组成谐振回路的主要元器件是:
\begin{enumerate}[leftmargin=3em]
\item 半导体三极管和电阻的组合
\item 电阻和电容的组合
\item 电容和电感的组合
\item 电感和电阻的组合
\end{enumerate}
\noindent\textbf{解说:}在业余收发信机电路中,经常用谐振回路来筛选一定频率的信号。组成谐振回路的主要元器件是\textbf{电容和电感的组合}。\\\noindent\textbf{答案:}C



\bigskip


\noindent\textbf{试题:}在电容器两端施加一定幅度的正弦交流电压。流过电容器的电流幅度:
\begin{enumerate}[leftmargin=3em]
\item 与电压和电容量都成反比
\item 与电容量成正比,与电压成反比
\item 与电压和电容量都成正比
\item 与电压成正比,与电容量成反比
\end{enumerate}
\noindent\textbf{解说:}在电容器两端施加一定幅度的正弦交流电压。流过电容器的电流幅度\textbf{与电压和电容量都成正比}。\\\noindent\textbf{答案:}C


\bigskip


\noindent\textbf{试题:}在线圈两端施加一定幅度的正弦交流电压。流过线圈的电流幅度:
\begin{enumerate}[leftmargin=3em]
\item 与电感量成正比,与电压成反比
\item 与电压和电感量都成反比
\item 与电压和电感量都成正比
\item 与电压成正比,与电感量成反比
\end{enumerate}
\noindent\textbf{解说:}在线圈两端施加一定幅度的正弦交流电压。流过线圈的电流幅度\textbf{与电压成正比,与电感量成反比}。\\\noindent\textbf{答案:}D

\bigskip


\noindent\textbf{试题:}构成振荡器的必备元素是:
\begin{enumerate}[leftmargin=3em]
\item 任意放大器、LC或晶体谐振电路
\item LC或晶体谐振电路、正反馈电路
\item 放大倍数大于1的放大器、正反馈电路
\item 放大倍数大于1的放大器、负反馈电路
\end{enumerate}
\noindent\textbf{解说:}构成振荡器的必备元素是\textbf{放大倍数大于1的放大器、正反馈电路}。\\\noindent\textbf{答案:}C

\bigskip


\noindent\textbf{试题:}在直流电路中,用来阻碍电流流动的元件是?
\begin{enumerate}[leftmargin=3em]
\item 电感
\item 变压器
\item 电阻
\item 电压表
\end{enumerate}
\noindent\textbf{解说:}在直流电路中,用来阻碍电流流动的元件是\textbf{电阻}。\\\noindent\textbf{答案:}C

\bigskip


\noindent\textbf{试题:}下列哪一个元件经常用来实现音量调节的功能?
\begin{enumerate}[leftmargin=3em]
\item 变压器
\item 定值电阻
\item 功率电阻
\item 电位器
\end{enumerate}
\noindent\textbf{解说:}\textbf{电位器}经常用来实现音量调节的功能。\\\noindent\textbf{答案:}D


\bigskip


\noindent\textbf{试题:}电位器控制什么电学物理量?
\begin{enumerate}[leftmargin=3em]
\item 场强
\item 电容
\item 电阻
\item 电感
\end{enumerate}
\noindent\textbf{解说:}电位器控制\textbf{电阻}。\\\noindent\textbf{答案:}C

\bigskip


\noindent\textbf{试题:}哪一种电子元件由两个或多个使用绝缘材料分离开的片状导体组成?
\begin{enumerate}[leftmargin=3em]
\item 电容
\item 电阻
\item 电位器
\item 振荡器
\end{enumerate}
\noindent\textbf{解说:}\textbf{电容}由两个或多个使用绝缘材料分离开的片状导体组成。\\\noindent\textbf{答案:}A


\bigskip


\noindent\textbf{试题:}哪一种电子元件一般由线圈组成?
\begin{enumerate}[leftmargin=3em]
\item 二极管
\item 电容
\item 电感
\item 开关
\end{enumerate}
\noindent\textbf{解说:}\textbf{电感}一般由线圈组成。\\\noindent\textbf{答案:}C



\bigskip


\noindent\textbf{试题:}哪一种电子元件用来接通或切断电路?
\begin{enumerate}[leftmargin=3em]
\item 开关
\item 电感
\item 可变电阻
\item 齐纳二极管
\end{enumerate}
\noindent\textbf{解说:}\textbf{开关}用来接通或切断电路。\\\noindent\textbf{答案:}A


\bigskip


\noindent\textbf{试题:}一个充满电的镍镉电池的标称电压是多少?
\begin{enumerate}[leftmargin=3em]
\item 2.2伏
\item 1.5伏
\item 1.0伏
\item 1.2伏
\end{enumerate}
\noindent\textbf{解说:}一个充满电的镍镉电池的标称电压是\textbf{1.2伏}。\\\noindent\textbf{答案:}D



\bigskip


\noindent\textbf{试题:}下列哪一种元器件可以用一个较小的电流来控制较大的电流?
\begin{enumerate}[leftmargin=3em]
\item 电感
\item 电阻
\item 晶体管
\item 电容
\end{enumerate}
\noindent\textbf{解说:}\textbf{晶体管}可以用一个较小的电流来控制较大的电流。\\\noindent\textbf{答案:}C



\bigskip


\noindent\textbf{试题:}下列哪一种元器件只允许单方向的电流流动?
\begin{enumerate}[leftmargin=3em]
\item 电阻
\item 稳压元件
\item 熔断器
\item 二极管
\end{enumerate}
\noindent\textbf{解说:}\textbf{二极管}只允许单方向的电流流动。\\\noindent\textbf{答案:}D




\bigskip


\noindent\textbf{试题:}下列哪一种元器件既可以当作电子开关又可以当作放大器使用?
\begin{enumerate}[leftmargin=3em]
\item 电压表
\item 电位器
\item 单刀双掷开关
\item 晶体管
\end{enumerate}
\noindent\textbf{解说:}\textbf{晶体管}既可以当作电子开关又可以当作放大器使用。\\\noindent\textbf{答案:}D



\bigskip


\noindent\textbf{试题:}下列哪一种元器件可以放大信号?
\begin{enumerate}[leftmargin=3em]
\item 多芯电池
\item 可变电阻
\item 晶体管
\item 电解电容
\end{enumerate}
\noindent\textbf{解说:}\textbf{晶体管}可以放大信号。\\\noindent\textbf{答案:}C


\bigskip


\noindent\textbf{试题:}继电器的功能可以描述为:
\begin{enumerate}[leftmargin=3em]
\item 由电流控制的放大器
\item 一个光学传感器
\item 由电磁铁控制的开关
\item 无线电转发设备
\end{enumerate}
\noindent\textbf{解说:}继电器的功能为\textbf{由电磁铁控制的开关}。\\\noindent\textbf{答案:}C


\bigskip


\noindent\textbf{试题:}下列哪一项和电感一起使用,可以制作一个谐振电路?
\begin{enumerate}[leftmargin=3em]
\item 电位器
\item 齐纳二极管
\item 电阻
\item 电容
\end{enumerate}
\noindent\textbf{解说:}\textbf{电容}和电感一起使用,可以制作一个谐振电路。\\\noindent\textbf{答案:}D



\bigskip


\noindent\textbf{试题:}集成电路是指:
\begin{enumerate}[leftmargin=3em]
\item 将一个电路的大量元器件集合于一个单晶片上所制成的器件
\item 多极继电器
\item 把多个电阻或电容元件堆积在一起
\item 变压器
\end{enumerate}
\noindent\textbf{解说:}集成电路是\textbf{将一个电路的大量元器件集合于一个单晶片上所制成的器件}。\\\noindent\textbf{答案:}A



\bigskip


\noindent\textbf{试题:}附图中的电路元器件符号代表的是:

\begin{circuitikz}[]
	\ctikzset{grounds/scale=3, ground=european}
	\draw (0,0) node[rground]{};
\end{circuitikz}

\begin{enumerate}[leftmargin=3em]
\item 二极管
\item 接地
\item 电阻
\item 天线
\end{enumerate}%LK0497.jpg
\noindent\textbf{解说:}附图中的电路元器件符号代表的是\textbf{接地}。\\\noindent\textbf{答案:}B






\bigskip


\noindent\textbf{试题:}附图中的电路元器件符号代表的是:

\begin{circuitikz}[european]
	\draw (0,-2) -- (0,.75);
	\draw (-1,.75) -- (0,-.5) -- (1,.75);
\end{circuitikz}


\begin{enumerate}[leftmargin=3em]
\item 接地
\item 二极管
\item 天线
\item 电阻
\end{enumerate}%LK0498.jpg
\noindent\textbf{解说:}附图中的电路元器件符号代表的是\textbf{天线}。\\\noindent\textbf{答案:}C


\bigskip


\noindent\textbf{试题:}附图中的电路元器件符号代表的是:

\begin{circuitikz}[]
	\ctikzset{misc/scale=1.5}
	\draw (0,0) to [fuse]     (2,0) node [right] {};
\end{circuitikz}


\begin{enumerate}[leftmargin=3em]
\item 二极管
\item 电阻
\item 电容
\item 熔断器
\end{enumerate}%LK0499.jpg
\noindent\textbf{解说:}附图中的电路元器件符号代表的是\textbf{熔断器}。\\\noindent\textbf{答案:}D


\bigskip


\noindent\textbf{试题:}附图中的电路元器件符号代表的是:

\begin{circuitikz}[european]
	\ctikzset{capacitors/scale=1.5, ccapacitorshape=european}
	\draw (0,0) to [C](0,1.5);
\end{circuitikz}

\begin{enumerate}[leftmargin=3em]
\item 二极管
\item 熔断器
\item 电容器
\item 电阻
\end{enumerate}%LK0500.jpg
\noindent\textbf{解说:}附图中的电路元器件符号代表的是\textbf{电容器}。\\\noindent\textbf{答案:}C


\bigskip


\noindent\textbf{试题:}附图中的电路元器件符号代表的是:

\begin{circuitikz}[]
	\ctikzset{resistors/scale=1, resistor=european}
	\draw (0,0) to [R](2,0);
\end{circuitikz}

\begin{enumerate}[leftmargin=3em]
\item 电阻
\item 电容器
\item 压电晶体
\item 熔断器
\end{enumerate}%LK0501.jpg
\noindent\textbf{解说:}附图中的电路元器件符号代表的是\textbf{电阻}。\\\noindent\textbf{答案:}A


\bigskip


\noindent\textbf{试题:}附图中的电路元器件符号代表的是:

\begin{circuitikz}[]
	\ctikzset{diodes/scale=1.5, emptydiodeshape=europe}
	\draw (0,1.5) to [D-](0,0);
\end{circuitikz}

\begin{enumerate}[leftmargin=3em]
\item 线圈
\item 电容器
\item 电阻
\item 二极管
\end{enumerate}%LK0502.jpg
\noindent\textbf{解说:}附图中的电路元器件符号代表的是\textbf{二极管}。\\\noindent\textbf{答案:}D


\bigskip


\noindent\textbf{试题:}附图中的电路元器件符号代表的是:

\begin{circuitikz}[]
	\ctikzset{inductors/scale=1.5, inductor=american}
	\draw (0,0) to[L] ++(1.7,0);
\end{circuitikz}

\begin{enumerate}[leftmargin=3em]
\item 线圈
\item 电阻
\item 二极管
\item 电容器
\end{enumerate}%LK0503.jpg
\noindent\textbf{解说:}附图中的电路元器件符号代表的是\textbf{线圈}。\\\noindent\textbf{答案:}A



\bigskip


\noindent\textbf{试题:}附图中的电路元器件符号代表的是:

\begin{circuitikz}[]
	\ctikzset{batteries/scale=1.5, battery2shape=europe}
	\draw (1.5,0) to [battery2] (0,0);
\end{circuitikz}

\begin{enumerate}[leftmargin=3em]
\item 线圈
\item 二极管
\item 电池
\item 电阻
\end{enumerate}%LK0504.jpg
\noindent\textbf{解说:}附图中的电路元器件符号代表的是\textbf{电池}。\\\noindent\textbf{答案:}C


\bigskip


\noindent\textbf{试题:}附图中的电路元器件符号代表的是:

\begin{circuitikz}[]
	\ctikzset{capacitors/scale=1.5, piezoelectricshape=europe}
	\draw (0,0)  to [PZ] (0,1.5);
\end{circuitikz}


\begin{enumerate}[leftmargin=3em]
\item 电池
\item 压电晶体
\item 电阻
\item 二极管
\end{enumerate}%LK0505.jpg
\noindent\textbf{解说:}附图中的电路元器件符号代表的是\textbf{压电晶体}。\\\noindent\textbf{答案:}B


\bigskip


\noindent\textbf{试题:}附图中的电路元器件符号代表的是:

\begin{circuitikz}[]
	\ctikzset{diodes/scale=1.5, emptyzdiodeshape=europe}
	\draw (0,1.5)  to [zD-, mirror] (0,0);
\end{circuitikz}

\begin{enumerate}[leftmargin=3em]
\item 压电晶体
\item 电阻
\item 发光二极管
\item 稳压二极管
\end{enumerate}%LK0506.jpg
\noindent\textbf{解说:}附图中的电路元器件符号代表的是\textbf{稳压二极管}。\\\noindent\textbf{答案:}D



\bigskip


\noindent\textbf{试题:}附图中的电路元器件符号代表的是:

\begin{circuitikz}[]
	\ctikzset{diodes/scale=1.5, emptylediodeshape=europe}
	\draw (0,1.5)  to [leD-, mirror] (0,0);
\end{circuitikz}
%%% 光的方向不对

\begin{enumerate}[leftmargin=3em]
\item 发光二极管
\item 电阻
\item 压电晶体
\item 稳压二极管
\end{enumerate}%LK0507.jpg
\noindent\textbf{解说:}附图中的电路元器件符号代表的是\textbf{发光二极管}。\\\noindent\textbf{答案:}A
%%%???



\bigskip


\noindent\textbf{试题:}附图中的电路元器件符号代表的是:

\begin{circuitikz}[]
	\ctikzset{transistors/scale=1.5}
	\draw (0,0) node[pnp]{};
\end{circuitikz}

\begin{enumerate}[leftmargin=3em]
\item PNP双极型半导体三极管
\item NPN双极型半导体三极管
\item 绝缘栅场效应半导体三极管
\item 结型场效应半导体三极管
\end{enumerate}%LK0508.jpg
\noindent\textbf{解说:}附图中的电路元器件符号代表的是\textbf{PNP双极型半导体三极管}。\\\noindent\textbf{答案:}A


\bigskip


\noindent\textbf{试题:}附图中的电路元器件符号代表的是:

\begin{circuitikz}[]
	\ctikzset{transistors/scale=1.5}
	\draw (0,0) node[npn]{};
\end{circuitikz}

\begin{enumerate}[leftmargin=3em]
\item NPN双极型半导体三极管
\item 结型场效应半导体三极管
\item 绝缘栅场效应半导体三极管
\item PNP双极型半导体三极管
\end{enumerate}%LK0509.jpg
\noindent\textbf{解说:}附图中的电路元器件符号代表的是\textbf{NPN双极型半导体三极管}。\\\noindent\textbf{答案:}A


\bigskip


\noindent\textbf{试题:}附图中的电路元器件符号代表的是:

\begin{circuitikz}[]
	\ctikzset{transistors/scale=1.5}
	\draw (0,0) node[njfet]{};
\end{circuitikz}

\begin{enumerate}[leftmargin=3em]
\item 结型场效应半导体三极管
\item 绝缘栅场效应半导体三极管
\item NPN双极型半导体三极管
\item PNP双极型半导体三极管
\end{enumerate}  % LK0510.jpg
\noindent\textbf{解说:}附图中的电路元器件符号代表的是\textbf{结型场效应半导体三极管}。\\\noindent\textbf{答案:}A


\bigskip


\noindent\textbf{试题:}附图中的电路元器件符号代表的是:

\begin{circuitikz}[]
	\ctikzset{transistors/scale=1.5}
	\draw (0,0) node[pigfetebulk]{};
\end{circuitikz}

\begin{enumerate}[leftmargin=3em]
\item 结型场效应半导体三极管
\item 绝缘栅场效应半导体三极管
\item NPN双极型半导体三极管
\item PNP双极型半导体三极管
\end{enumerate}%LK0511.jpg
\noindent\textbf{解说:}附图中的电路元器件符号代表的是\textbf{绝缘栅场效应半导体三极管}。\\\noindent\textbf{答案:}B



\bigskip


\noindent\textbf{试题:}将电阻R和电容C串联后突然接到直流电压U上,电容C两端的电压会:
\begin{enumerate}[leftmargin=3em]
\item 从U按指数规律逐渐减小到0
\item 从0按指数规律逐渐增加到U
\item 从0突然跳到U,然后再按指数规律逐渐减小到0
\item 从U突然跳到0,然后再按指数规律逐渐增大到U
\end{enumerate}
\noindent\textbf{解说:}将电阻R和电容C串联后突然接到直流电压U上,电容C两端的电压会\textbf{从0按指数规律逐渐增加到U}。\\\noindent\textbf{答案:}B


\bigskip


\noindent\textbf{试题:}将电阻R和电容C串联后突然接到直流电压U上,电阻R两端的电压会:
\begin{enumerate}[leftmargin=3em]
\item 从U突然跳到0,然后再按直线规律逐渐减小到U
\item 从U按直线规律逐渐减小到0
\item 从0按指数规律逐渐增加到U
\item 从0突然跳到U,然后再按指数规律逐渐减小到0
\end{enumerate}
\noindent\textbf{解说:}将电阻R和电容C串联后突然接到直流电压U上,电阻R两端的电压会\textbf{从0突然跳到U,然后再按指数规律逐渐减小到0}。\\\noindent\textbf{答案:}D



\bigskip


\noindent\textbf{试题:}将电阻R和电容C串联后突然接到直流电压U上,流过电阻R的电流会:
\begin{enumerate}[leftmargin=3em]
\item 从0按指数规律逐渐增加到U/R
\item 从U/R突然跳到0并保持
\item 从0突然跳到U/R并保持
\item 从0突然跳到U/R,然后再按指数规律逐渐减小到0
\end{enumerate}
\noindent\textbf{解说:}将电阻R和电容C串联后突然接到直流电压U上,流过电阻R的电流会\textbf{从0突然跳到U/R,然后再按指数规律逐渐减小到0}。\\\noindent\textbf{答案:}D


\bigskip


\noindent\textbf{试题:}将电阻R和电容C串联后突然接到直流电压U上,流过电容C的电流会:
\begin{enumerate}[leftmargin=3em]
\item 从U/R突然跳到0并保持
\item 从0突然跳到U/R,然后再按指数规律逐渐减小到0
\item 从0突然跳到U/R并保持
\item 从0按指数规律逐渐增加到U/R
\end{enumerate}
\noindent\textbf{解说:}将电阻R和电容C串联后突然接到直流电压U上,流过电容C的电流会\textbf{从0突然跳到U/R,然后再按指数规律逐渐减小到0}。\\\noindent\textbf{答案:}B



\bigskip


\noindent\textbf{试题:}电阻R和电容C并联后接在电压为U的直流电源上。突然断开电源,电容C两端的电压会:
\begin{enumerate}[leftmargin=3em]
\item 从0按指数规律逐渐增加到U
\item 从U按指数规律逐渐减小到0
\item 从0突然跳到U并保持
\item 从U突然跳到0并保持
\end{enumerate}
\noindent\textbf{解说:}电阻R和电容C并联后接在电压为U的直流电源上。突然断开电源,电容C两端的电压会\textbf{从U按指数规律逐渐减小到0}。\\\noindent\textbf{答案:}B



\bigskip


\noindent\textbf{试题:}电阻R和电容C并联后接在电压为U的直流电源上。突然断开电源,电阻R两端的电压会:
\begin{enumerate}[leftmargin=3em]
\item 从U按指数规律逐渐减小到0
\item 从0按指数规律逐渐增加到U
\item 从U突然跳到0并保持
\item 从0突然跳到U并保持
\end{enumerate}
\noindent\textbf{解说:}电阻R和电容C并联后接在电压为U的直流电源上。突然断开电源,电阻R两端的电压会\textbf{从U按指数规律逐渐减小到0}。\\\noindent\textbf{答案:}A


\bigskip


\noindent\textbf{试题:}电阻R和电容C并联后接在电压为U的直流电源上。突然断开电源,流过电阻R的电流会:
\begin{enumerate}[leftmargin=3em]
\item 从0突然跳到U/R,然后再按指数规律逐渐减小到0
\item 从U/R按指数规律逐渐减小到0
\item 从U突然跳到0,然后再按指数规律逐渐增大到U/R
\item 从0按指数规律逐渐增加到U/R
\end{enumerate}
\noindent\textbf{解说:}电阻R和电容C并联后接在电压为U的直流电源上。突然断开电源,流过电阻R的电流会\textbf{从U/R按指数规律逐渐减小到0}。\\\noindent\textbf{答案:}B




\bigskip


\noindent\textbf{试题:}电阻R和电容C并联后接在电压为U的直流电源上。突然断开电源,流过电容C的电流会:
\begin{enumerate}[leftmargin=3em]
\item 从U突然跳到0,然后再按指数规律逐渐增大到U/R
\item 从0突然跳到U/R,然后再按指数规律逐渐减小到0
\item 从0突然跳到U/R并保持
\item 从0按指数规律逐渐增加到U/R
\end{enumerate}
\noindent\textbf{解说:}电阻R和电容C并联后接在电压为U的直流电源上。突然断开电源,流过电容C的电流会\textbf{从0突然跳到U/R,然后再按指数规律逐渐减小到0}。\\\noindent\textbf{答案:}B



\bigskip


\noindent\textbf{试题:}将电阻R和电感L串联后突然接到直流电压U上,电感L两端的电压会:
\begin{enumerate}[leftmargin=3em]
\item 从U突然跳到0并保持
\item 从0突然跳到U,然后再按指数规律逐渐减小到0
\item 从0按指数规律逐渐增加到U
\item 从U按指数规律逐渐减小到0
\end{enumerate}
\noindent\textbf{解说:}将电阻R和电感L串联后突然接到直流电压U上,电感L两端的电压会\textbf{从0突然跳到U,然后再按指数规律逐渐减小到0}。\\\noindent\textbf{答案:}B



\bigskip


\noindent\textbf{试题:}将电阻R和电感L串联后突然接到直流电压U上,电阻R两端的电压会:
\begin{enumerate}[leftmargin=3em]
\item 从U按指数规律逐渐减小到0
\item 从0按指数规律逐渐增加到U
\item 从0突然跳到U,然后再按指数规律逐渐减小到0
\item 从U突然跳到0并保持
\end{enumerate}
\noindent\textbf{解说:}将电阻R和电感L串联后突然接到直流电压U上,电阻R两端的电压会\textbf{从0按指数规律逐渐增加到U}。\\\noindent\textbf{答案:}B




\bigskip


\noindent\textbf{试题:}将电阻R和电感L串联后突然接到直流电压U上,流过电阻R的电流会:
\begin{enumerate}[leftmargin=3em]
\item 从0按指数规律逐渐增加到U/R
\item 从U/R按指数规律逐渐减小到0
\item 从0突然跳到U/R,然后再按指数规律逐渐减小到0
\item 从0突然跳到U/R并保持U/R
\end{enumerate}
\noindent\textbf{解说:}将电阻R和电感L串联后突然接到直流电压U上,流过电阻R的电流会\textbf{从0按指数规律逐渐增加到U/R}。\\\noindent\textbf{答案:}A




\bigskip


\noindent\textbf{试题:}将电阻R和电感L串联后突然接到直流电压U上,流过电感L的电流会:
\begin{enumerate}[leftmargin=3em]
\item 从0突然跳到U/R,然后再按指数规律逐渐减小到0
\item 从U/R按指数规律逐渐减小到0
\item 从0突然跳到U/R并保持
\item 从0按指数规律逐渐增加到U/R
\end{enumerate}
\noindent\textbf{解说:}将电阻R和电感L串联后突然接到直流电压U上,流过电感L的电流会\textbf{从0按指数规律逐渐增加到U/R}。\\\noindent\textbf{答案:}D




\bigskip


\noindent\textbf{试题:}电阻R和电感L并联后接在电流为I的直流电路中。突然断开电路,电感L两端的电压会:
\begin{enumerate}[leftmargin=3em]
\item 保持在0
\item 保持在I*R
\item 从I*R按指数规律逐渐减小到0
\item 从0按指数规律逐渐增加到I*R
\end{enumerate}
\noindent\textbf{解说:}电阻R和电感L并联后接在电流为I的直流电路中。突然断开电路,电感L两端的电压会\textbf{从I*R按指数规律逐渐减小到0}。\\\noindent\textbf{答案:}C




\bigskip


\noindent\textbf{试题:}用一个电压为4.2伏的低电压电池和一堆无源电子元件做电路实验,但连接电路时感觉手不慎被高电压击了一下。可能产生这个高电压的元件是:
\begin{enumerate}[leftmargin=3em]
\item 大电流高反压二极管
\item 电源变压器的绕组
\item 额定功率为50瓦的大电阻
\item 电解电容器
\end{enumerate}
\noindent\textbf{解说:}可能产生这个高电压的元件是\textbf{电源变压器的绕组}。\\\noindent\textbf{答案:}B

\bigskip


\noindent\textbf{试题:}用SSB接收机的天线引线靠近一个晶体管LC振荡器电路板,接收其信号。振荡器电路接通电源后,发现收到的信号音调会从低到高或者从高到低变化。这主要因为:
\begin{enumerate}[leftmargin=3em]
\item 随着射频能量泄露,电路的输出功率下降
\item 元器件通电发热,引起相关LC参数变化,造成谐振频率漂移
\item 接收机的声音能量反馈到电路板引起
\item 半导体晶体管处于老化过程
\end{enumerate}
\noindent\textbf{解说:}这主要因为\textbf{元器件通电发热,引起相关LC参数变化,造成谐振频率漂移}。\\\noindent\textbf{答案:}B



\bigskip


\noindent\textbf{试题:}在无线电电路中常用于产生基准频率的元件中,按频率稳定度由低到高的排列为:
\begin{enumerate}[leftmargin=3em]
\item LC回路,RC定时电路,陶瓷谐振器,石英声表面波元件,石英晶体谐振器
\item RC定时电路,LC回路,陶瓷谐振器,石英声表面波元件,石英晶体谐振器
\item RC定时电路,陶瓷谐振器,LC回路,石英晶体谐振器,石英声表面波元件
\item RC定时电路,LC回路,石英声表面波元件,石英晶体谐振器,陶瓷谐振器
\end{enumerate}
\noindent\textbf{解说:}在无线电电路中常用于产生基准频率的元件中,按频率稳定度由低到高的排列为:\textbf{RC定时电路,LC回路,陶瓷谐振器,石英声表面波元件,石英晶体谐振器}。\\\noindent\textbf{答案:}B



\bigskip


\noindent\textbf{试题:}在无线电技术中,通常把放大器分为A、B、C、D等类别,这种分类是依据:
\begin{enumerate}[leftmargin=3em]
\item 放大器件的工作点所处的范围
\item 放大器件的最高工作频率
\item 放大器件的最大输出功率
\item 放大器件的质量等级
\end{enumerate}
\noindent\textbf{解说:}在无线电技术中,通常把放大器分为A、B、C、D等类别,这种分类是依据:\textbf{放大器件的工作点所处的范围}。\\\noindent\textbf{答案:}A



\bigskip


\noindent\textbf{试题:}根据放大器的工作状态,通常把放大器分为A、B、C、D等类别。A类放大器是指:
\begin{enumerate}[leftmargin=3em]
\item 放大器件在半个信号周期内工作点处于线性区、另半个信号周期内处于截止区的放大器
\item 放大器件在整个信号周期内始终工作在线性区的放大器
\item 放大器件在半个信号周期内处于截止区,另半个周期处于饱和区的放大器
\item 放大器件在半个信号周期内处于截止区,另半个周期的部分时间候处于饱和区的放大器
\end{enumerate}
\noindent\textbf{解说:}A类放大器是指\textbf{放大器件在整个信号周期内始终工作在线性区的放大器}。\\\noindent\textbf{答案:}B


\bigskip


\noindent\textbf{试题:}根据放大器的工作状态,通常把放大器分为A、B、C、D等类别。B类放大器是指:
\begin{enumerate}[leftmargin=3em]
\item 放大器件在半个信号周期内处于截止区,另半个周期处于饱和区的放大器
\item 放大器件在整个信号周期内始终工作在线性区的放大器
\item 放大器件在半个信号周期内处于截止区,另半个周期的部分时间候处于饱和区的放大器
\item 放大器件在半个信号周期内工作点处于线性区、另半个信号周期内处于截止区的放大器
\end{enumerate}
\noindent\textbf{解说:}B类放大器是指\textbf{放大器件在半个信号周期内工作点处于线性区、另半个信号周期内处于截止区的放大器}。\\\noindent\textbf{答案:}D





\bigskip


\noindent\textbf{试题:}根据放大器的工作状态,通常把放大器分为A、B、C、D等类别。C类放大器是指:
\begin{enumerate}[leftmargin=3em]
\item 放大器件在整个信号周期内始终工作在线性区的放大器
\item 放大器件在多于半个信号周期的时间内处于截止区,另半个周期的部分时间候处于线性区的放大器
\item 放大器件在半个信号周期内处于截止区,另半个周期处于饱和区的放大器
\item 放大器件在半个信号周期内工作点处于线性区、另半个信号周期内处于截止区的放大器
\end{enumerate}
\noindent\textbf{解说:}C类放大器是指\textbf{放大器件在多于半个信号周期的时间内处于截止区,另半个周期的部分时间候处于线性区的放大器}。\\\noindent\textbf{答案:}B





\bigskip


\noindent\textbf{试题:}根据放大器的工作状态,通常把放大器分为A、B、C、D等类别。D类放大器是指:
\begin{enumerate}[leftmargin=3em]
\item 放大器件在整个信号周期内始终工作在线性区的放大器
\item 放大器件在半个信号周期内处于截止区,另半个周期处于饱和区的放大器
\item 放大器件在半个信号周期内处于截止区,另半个周期的部分时间候处于饱和区的放大器
\item 放大器件在半个信号周期内工作点处于线性区、另半个信号周期内处于截止区的放大器
\end{enumerate}
\noindent\textbf{解说:}D类放大器是指\textbf{放大器件在半个信号周期内处于截止区,另半个周期处于饱和区的放大器}。\\\noindent\textbf{答案:}B





\bigskip


\noindent\textbf{试题:}A、B、C、D四类放大器按输出波形失真由小到大的排列顺序是:
\begin{enumerate}[leftmargin=3em]
\item B、A、D、C
\item D、C、B、A
\item A、B、C、D
\item A、C、B、D
\end{enumerate}
\noindent\textbf{解说:}A、B、C、D四类放大器按输出波形失真由小到大的排列顺序是:\textbf{A、B、C、D}。\\\noindent\textbf{答案:}C





\bigskip


\noindent\textbf{试题:}A、B、C、D四类放大器用作射频功率放大时,按电源效率由高到低的排列顺序是:
\begin{enumerate}[leftmargin=3em]
\item B、A、D、C
\item D、C、B、A
\item D、C、A、B
\item A、B、C、D
\end{enumerate}
\noindent\textbf{解说:}A、B、C、D四类放大器用作射频功率放大时,按电源效率由高到低的排列顺序是:\textbf{D、C、B、A}。\\\noindent\textbf{答案:}B





\bigskip


\noindent\textbf{试题:}A、B、C、D四类放大器中,适宜于做小信号放大器的是:
\begin{enumerate}[leftmargin=3em]
\item C
\item D
\item B
\item A
\end{enumerate}
\noindent\textbf{解说:}A、B、C、D四类放大器中,适宜于做小信号放大器的是:\textbf{A}类放大器。\\\noindent\textbf{答案:}D






\bigskip


\noindent\textbf{试题:}A、B、C、D四类放大器中,属于大信号放大器的全部类别有:
\begin{enumerate}[leftmargin=3em]
\item A、B、C、D
\item B、C、D
\item A、C、D
\item C、D
\end{enumerate}
\noindent\textbf{解说:}A、B、C、D四类放大器中,属于大信号放大器的全部类别有:\textbf{B、C、D}。\\\noindent\textbf{答案:}B





\bigskip


\noindent\textbf{试题:}很多业余电台的末级和末前级射频输出放大器中采用两个并联的输出半导体功率管,这是为了:
\begin{enumerate}[leftmargin=3em]
\item 双管并联,得到双倍的器件耐压,减少损坏几率
\item 构成推挽电路,减小输出波形的失真
\item 双管并联,使每个功率管的失真互相补偿,减少失真,降低杂散发射
\item 双管并联,得到双倍的输出电流和输出功率
\end{enumerate}
\noindent\textbf{解说:}很多业余电台的末级和末前级射频输出放大器中采用两个并联的输出半导体功率管,这是为了\textbf{双管并联,得到双倍的输出电流和输出功率}。\\\noindent\textbf{答案:}D





\bigskip


\noindent\textbf{试题:}很多现代无线电设备的音频功率放大电路采用两个串联的输出半导体功率管,分别负责信号正、负半周的放大。这种电路的通用名称和作用是:
\begin{enumerate}[leftmargin=3em]
\item 双管串联电路,得到双倍的输出电流和输出功率
\item 双管串联电路,得到较高的输出阻抗以改善与负载的阻抗匹配
\item 推挽放大电路,实现极小静态工作点下的高电源效率的线性功率放大
\item 双管串联电路,得到较高的输入阻抗以改善与推动级之间的阻抗匹配
\end{enumerate}
\noindent\textbf{解说:}很多现代无线电设备的音频功率放大电路采用两个串联的输出半导体功率管,分别负责信号正、负半周的放大。这种电路的通用名称和作用是\textbf{推挽放大电路,实现极小静态工作点下的高电源效率的线性功率放大}。\\\noindent\textbf{答案:}C






\bigskip


\noindent\textbf{试题:}放大器的负反馈是指这样的电路:
\begin{enumerate}[leftmargin=3em]
\item 将放大器输入信号的一部分直通到放大器的输出端,起到加强输出信号的作用
\item 将放大器输出信号的一部分回输到放大器的输入端,起到抵消输入信号的作用
\item 将放大器输入信号的一部分直通到放大器的输出端,起到抵消输出信号的作用
\item 将放大器输出信号的一部分回输到放大器的输入端,起到加强输入信号的作用

\end{enumerate}
\noindent\textbf{解说:}放大器的负反馈是指这样的电路:\textbf{将放大器输出信号的一部分回输到放大器的输入端,起到抵消输入信号的作用}。\\\noindent\textbf{答案:}B

%%%???

\bigskip


\noindent\textbf{试题:}很多现代业余无线电收发信机的本机振荡电路采用了直接数字频率合成(DDS)方式。它的主要特点是:
\begin{enumerate}[leftmargin=3em]
\item 直接产生纯净的正弦波信号,不需要采用任何滤波器
\item 采用同样的频率源振荡器时频率稳定度优于锁相环频率合成方式
\item 电路结构简洁,无锁相捕捉范围限制,不产生相位噪声,跳换频率快
\item 与锁相环频率合成方式相比,可以使用速度较低的数字元器件
\end{enumerate}
\noindent\textbf{解说:}很多现代业余无线电收发信机的本机振荡电路采用了直接数字频率合成(DDS)方式。它的主要特点是\textbf{电路结构简洁,无锁相捕捉范围限制,不产生相位噪声,跳换频率快}。\\\noindent\textbf{答案:}C

%%%???

\bigskip


\noindent\textbf{试题:}频移电报技术(frequency-shift telegraphy)是指:电报信号控制载波频率在预定的范围之内变化的调频电报技术。下述业余通信使用的是移频电报技术:
\begin{enumerate}[leftmargin=3em]
\item CW
\item PSK31
\item SSTV
\item RTTY
\end{enumerate}
\noindent\textbf{解说:}\textbf{RTTY}使用的是移频电报技术。\\\noindent\textbf{答案:}D



\bigskip


\noindent\textbf{试题:}无线电通信选择不同调制方式的主要考虑因素是:
\begin{enumerate}[leftmargin=3em]
\item 有利于提高接收机选择性、尽量提高话筒灵敏度、防止产生谐波干扰
\item 信息在传递过程中的保真度、信号的抗干扰能力、尽量节省无线电频谱资源
\item 防止与附近发射机产生三阶互调、尽量降低本振相位噪声、采用高中频方案
\item 改善天线阻抗匹配、尽量提高发射频率稳定度、尽量减少杂散发射
\end{enumerate}
\noindent\textbf{解说:}无线电通信选择不同调制方式的主要考虑因素是\textbf{信息在传递过程中的保真度、信号的抗干扰能力、尽量节省无线电频谱资源}。\\\noindent\textbf{答案:}B



\bigskip


\noindent\textbf{试题:}接收机解调部件的作用是:
\begin{enumerate}[leftmargin=3em]
\item 从接收到的已调制射频信号中分离出原始信号
\item 对接收到的射频信号进行选频放大
\item 从接收到的已调制射频信号中提取出载频分量
\item 对接收到的射频信号进行宽带线性放大
\end{enumerate}
\noindent\textbf{解说:}接收机解调部件的作用是\textbf{从接收到的已调制射频信号中分离出原始信号}。\\\noindent\textbf{答案:}A



\bigskip


\noindent\textbf{试题:}选择解调部件的主要应考因素是:
\begin{enumerate}[leftmargin=3em]
\item 尽量提升已调制信号中的载频分量
\item 尽量忠实地还原原始信号
\item 尽量补偿接收到的射频信号的频率偏移
\item 尽量对已调制信号加以放大
\end{enumerate}
\noindent\textbf{解说:}选择解调部件的主要应考因素是\textbf{尽量忠实地还原原始信号}。\\\noindent\textbf{答案:}B




\bigskip


\noindent\textbf{试题:}在HF业余频段的数据通信段中,用收信机的SSB挡听到一个由两种音调交替组成的信号,这个信号的调制方式最可能属于下述种类:
\begin{enumerate}[leftmargin=3em]
\item SSTV或FAX
\item ASK
\item FSK
\item PSK
\end{enumerate}
\noindent\textbf{解说:}这个信号的调制方式最可能属于\textbf{FSK}。\\\noindent\textbf{答案:}C




\bigskip


\noindent\textbf{试题:}在HF业余频段的数据通信段中,用收信机的SSB挡听到一个音调不变但又似乎不断颤动的信号,这个信号的调制方式最可能属于下述种类:
\begin{enumerate}[leftmargin=3em]
\item SSTV或FAX
\item PSK
\item FSK
\item ASK
\end{enumerate}
\noindent\textbf{解说:}这个信号的调制方式最可能属于\textbf{PSK}。\\\noindent\textbf{答案:}B




\bigskip


\noindent\textbf{试题:}用收信机的SSB挡在业余频段中,听到一个音调大致以约为几分之一秒的重复周期连续变化、并夹有一种规律的“笃、笃”声的信号,这个信号的调制方式最可能属于下述种类:
\begin{enumerate}[leftmargin=3em]
\item PSK
\item SSTV或FAX
\item FSK
\item ASK
\end{enumerate}
\noindent\textbf{解说:}这个信号的调制方式最可能属于\textbf{SSTV或FAX}。\\\noindent\textbf{答案:}B



\bigskip


\noindent\textbf{试题:}对于给定的SSB发射设备,决定其输出信号实际占用带宽的因素是:
\begin{enumerate}[leftmargin=3em]
\item 所传输信号的最高频率越高,射频输出占用带宽越宽,但与其幅度和带宽无关
\item 所传输信号的带宽越宽,射频输出占用带宽越宽,但与其幅度和最高频率无关
\item 所传输信号的幅度越大,射频输出占用带宽越宽,但与其频率和带宽无关
\item 射频输出实际占用带宽为由电路决定的固定值,通信常用的是2.7\unit{\kHz}
\end{enumerate}
\noindent\textbf{解说:}对于给定的SSB发射设备,决定其输出信号实际占用带宽的因素是\textbf{所传输信号的带宽越宽,射频输出占用带宽越宽,但与其幅度和最高频率无关}。\\\noindent\textbf{答案:}B




\bigskip


\noindent\textbf{试题:}业余SSTV通信和有些模拟ATV采用调频方式而不是广播电视图像的调幅方式,主要原因是:
\begin{enumerate}[leftmargin=3em]
\item 业余电台信号较弱,调频解调可以更好地抗拒叠加在信号上的外界噪声所引起的幅度变化
\item 便于与调频通话方式所使用的设备兼容
\item 调频收信设备的灵敏度比调幅的高一个数量级
\item 调频方式占用的频带比调幅方式窄
\end{enumerate}
\noindent\textbf{解说:}业余SSTV通信和有些模拟ATV采用调频方式而不是广播电视图像的调幅方式,主要原因是\textbf{业余电台信号较弱,调频解调可以更好地抗拒叠加在信号上的外界噪声所引起的幅度变化}。\\\noindent\textbf{答案:}A




\bigskip


\noindent\textbf{试题:}如果发射机在不同工作模式时最大射频输出功率相同,无语音调制时,实际射频输出由大到小的排序为:
\begin{enumerate}[leftmargin=3em]
\item FM,AM,SSB
\item SSB,AM,FM
\item SSB,FM,AM
\item AM,SSB,FM
\end{enumerate}
\noindent\textbf{解说:}实际射频输出由大到小的排序为:\textbf{FM,AM,SSB}。\\\noindent\textbf{答案:}A




\bigskip


\noindent\textbf{试题:}下列调制得到的信号幅度恒定不变:
\begin{enumerate}[leftmargin=3em]
\item 移频键控FSK
\item 幅度调制AM
\item 幅度键控调制ASK
\item 单边带幅度调制SSB
\end{enumerate}
\noindent\textbf{解说:}\textbf{移频键控FSK}调制得到的信号幅度恒定不变。\\\noindent\textbf{答案:}A




\bigskip


\noindent\textbf{试题:}下列调制得到的信号幅度恒定不变:
\begin{enumerate}[leftmargin=3em]
\item 幅度调制AM
\item 单边带幅度调制SSB
\item 幅度键控调制ASK
\item 频率调制FM
\end{enumerate}
\noindent\textbf{解说:}\textbf{频率调制FM}调制得到的信号幅度恒定不变。\\\noindent\textbf{答案:}D




\bigskip


\noindent\textbf{试题:}下列调制得到的信号幅度恒定不变:
\begin{enumerate}[leftmargin=3em]
\item 幅度键控调制ASK
\item 相位调制PM
\item 单边带幅度调制SSB
\item 幅度调制AM
\end{enumerate}
\noindent\textbf{解说:}\textbf{相位调制PM}调制得到的信号幅度恒定不变。\\\noindent\textbf{答案:}B



\bigskip


\noindent\textbf{试题:}下列调制得到的信号幅度恒定不变:
\begin{enumerate}[leftmargin=3em]
\item 单边带幅度调制SSB
\item 幅度调制AM
\item 移相键控调制PSK
\item 幅度键控调制ASK
\end{enumerate}
\noindent\textbf{解说:}\textbf{移相键控调制PSK}调制得到的信号幅度恒定不变。\\\noindent\textbf{答案:}C




\bigskip


\noindent\textbf{试题:}下列调制得到的信号周期恒定不变:
\begin{enumerate}[leftmargin=3em]
\item 相位调制PM
\item 频率键控调制FSK
\item 频率调制FM
\item 单边带幅度调制SSB
\end{enumerate}
\noindent\textbf{解说:}\textbf{相位调制PM}调制得到的信号周期恒定不变。\\\noindent\textbf{答案:}A




\bigskip


\noindent\textbf{试题:}下列调制得到的信号中载频分量幅度恒定不变:
\begin{enumerate}[leftmargin=3em]
\item 载波抑制单边带幅度调制SSB
\item 幅度调制AM
\item 频率键控调制FSK
\item 频率调制FM
\end{enumerate}
\noindent\textbf{解说:}\textbf{幅度调制AM}调制得到的信号中载频分量幅度恒定不变。\\\noindent\textbf{答案:}B




\bigskip


\noindent\textbf{试题:}信号经过下列调制后,频带宽度可能会大于原有值:
\begin{enumerate}[leftmargin=3em]
\item 单边带幅度调制SSB
\item 幅度调制AM
\item 频率调制FM
\item 幅度键控调制ASK
\end{enumerate}
\noindent\textbf{解说:}信号经过\textbf{频率调制FM}后,频带宽度可能会大于原有值。\\\noindent\textbf{答案:}C



\bigskip


\noindent\textbf{试题:}什么是PSK31?
\begin{enumerate}[leftmargin=3em]
\item 一种高速率的数据通信模式
\item 一种压缩数字电视信号的方法
\item 一种低速率的数据通信模式
\item 一种可以减少噪音对FM信号干扰的方法
\end{enumerate}
\noindent\textbf{解说:}PSK31是\textbf{一种低速率的数据通信模式}。\\\noindent\textbf{答案:}C




\bigskip


\noindent\textbf{试题:}谐振回路的通带宽度BW是指:
\begin{enumerate}[leftmargin=3em]
\item 回路两端电压信号幅度从中心频率衰减\qty{30}{\percent}时上、下限频率的间距
\item 回路两端电压信号幅度从中心频率衰减\qty{95}{\percent}时上、下限频率的间距
\item 回路两端电压信号幅度从中心频率衰减\qty{80}{\percent}时上、下限频率的间距
\item 回路两端电压信号幅度从中心频率衰减\SI{3}{\dB}时上、下限频率的间距
\end{enumerate}
\noindent\textbf{解说:}谐振回路的通带宽度BW是指\textbf{回路两端电压信号幅度从中心频率衰减\SI{3}{\dB}时上、下限频率的间距}。\\\noindent\textbf{答案:}D



\bigskip


\noindent\textbf{试题:}滤波器的“截止频率”是指:
\begin{enumerate}[leftmargin=3em]
\item 输出频率特性曲线从通带的\SI{0}{\dB}变化到\SI{-3}{\dB}的频率
\item 低于该频率的信号将被滤波器完全切除
\item 高于该频率的信号将被滤波器完全切除
\item 高于该频率的信号将会在滤波器中发生非线性失真
\end{enumerate}
\noindent\textbf{解说:}滤波器的“截止频率”是指\textbf{输出频率特性曲线从通带的\SI{0}{\dB}变化到\SI{-3}{\dB}的频率}。\\\noindent\textbf{答案:}A



\bigskip


\noindent\textbf{试题:}滤波器的“\SI{3}{\dB}带宽”是指:
\begin{enumerate}[leftmargin=3em]
\item 滤波器维持\SI{3}{\dB}增益的频率比范围宽度
\item 输出信号相对于输入信号衰减\SI{3}{\dB}以上(含\SI{3}{\dB})的频率范围宽度
\item 输出信号相对于输入信号衰减\SI{3}{\dB}以下(含\SI{3}{\dB})的频率范围宽度
\item 输出频率特性曲线从通带的\SI{0}{\dB}变化到\SI{-3}{\dB}的频率之间的宽度
\end{enumerate}
\noindent\textbf{解说:}滤波器的“\SI{3}{\dB}带宽”是指\textbf{输出频率特性曲线从通带的\SI{0}{\dB}变化到\SI{-3}{\dB}的频率之间的宽度}。\\\noindent\textbf{答案:}D



\bigskip


\noindent\textbf{试题:}220V.AC/13.8V.DC通信开关电源的一般工作过程是:
\begin{enumerate}[leftmargin=3em]
\item 由变压器将交流输入变为低压交流,由半导体开关电路变成超音频脉冲电流,经整流滤波为低压直流
\item 将交流输入整流滤波为高压直流,由半导体开关电路变成高压脉冲电流,由变压器变成低压脉冲,整流滤波为低压直流
\item 将交流输入整流滤波为高压直流,由变压器变成低压脉冲,由半导体开关电路变成低压直流,滤波后输出
\item 由大功率半导体三极管将交流输入变为高压直流,由专用集成电路变成超音频脉冲电流,经整流滤波为低压直流
\end{enumerate}
\noindent\textbf{解说:}220V.AC/13.8V.DC通信开关电源的一般工作过程是\textbf{将交流输入整流滤波为高压直流,由半导体开关电路变成高压脉冲电流,由变压器变成低压脉冲,整流滤波为低压直流}。\\\noindent\textbf{答案:}B





\bigskip


\noindent\textbf{试题:}具有两个输入的与门(AND)是最简单的数字逻辑电路之一。如果两个输入信号的组合分别为00、01、10、11,对应的输出信号应为:
\begin{enumerate}[leftmargin=3em]
\item 1、0、1、0
\item 0、0、0、1
\item 0、1、1、1
\item 0、1、0、1
\end{enumerate}
\noindent\textbf{解说:}对应的输出信号应为\textbf{0、0、0、1}。\\\noindent\textbf{答案:}B





\bigskip


\noindent\textbf{试题:}具有两个输入的或门(OR)是最简单的数字逻辑电路之一。如果两个输入信号组合分别为00、01、10、11,对应的输出信号应为:
\begin{enumerate}[leftmargin=3em]
\item 1、0、1、0
\item 0、1、1、0
\item 0、1、0、1
\item 0、1、1、1
\end{enumerate}
\noindent\textbf{解说:}对应的输出信号应为\textbf{0、1、1、1}。\\\noindent\textbf{答案:}D



\bigskip


\noindent\textbf{试题:}具有两个输入的异或门(XOR)是最简单的数字逻辑电路之一。如果两个输入信号组合分别为00、01、10、11,对应的输出信号应为:
\begin{enumerate}[leftmargin=3em]
\item 0、1、1、0
\item 0、1、1、1
\item 0、1、0、1
\item 1、0、1、0
\end{enumerate}
\noindent\textbf{解说:}对应的输出信号应为\textbf{0、1、1、0}。\\\noindent\textbf{答案:}A




\bigskip


\noindent\textbf{试题:}具有两个输入的与非门(NAND)是最简单的数字逻辑电路之一。如果两个输入信号组合分别为00、01、10、11,对应的输出信号应为:
\begin{enumerate}[leftmargin=3em]
\item 1、0、1、0
\item 1、1、1、0
\item 0、1、0、1
\item 0、1、1、1
\end{enumerate}
\noindent\textbf{解说:}对应的输出信号应为\textbf{1、1、1、0}。\\\noindent\textbf{答案:}B






\bigskip


\noindent\textbf{试题:}具有两个输入的或非门(NOR)是最简单的数字逻辑电路之一。如果两个输入信号组合分别为00、01、10、11,对应的输出信号应为:
\begin{enumerate}[leftmargin=3em]
\item 0、1、1、1
\item 1、0、0、0
\item 1、0、1、0
\item 0、1、0、1
\end{enumerate}
\noindent\textbf{解说:}对应的输出信号应为\textbf{1、0、0、0}。\\\noindent\textbf{答案:}B





\bigskip


\noindent\textbf{试题:}具有两个输入的异或非门(NXOR)是最简单的数字逻辑电路之一。如果两个输入信号组合分别为00、01、10、11,对应的输出信号应为:
\begin{enumerate}[leftmargin=3em]
\item 0、1、1、1
\item 1、0、0、1
\item 1、1、0、0
\item 0、1、1、0
\end{enumerate}
\noindent\textbf{解说:}对应的输出信号应为\textbf{1、0、0、1}。\\\noindent\textbf{答案:}B




\bigskip


\noindent\textbf{试题:}一个重复频率为F的非正弦周期信号的频谱包含有:
\begin{enumerate}[leftmargin=3em]
\item 无穷多个连续的频率分量
\item 频率为F以外的无穷多个频率分量
\item 频率为F的整数倍的无穷多个频率分量
\item 频率为F的一个频率分量
\end{enumerate}
\noindent\textbf{解说:}一个重复频率为F的非正弦周期信号的频谱包含有\textbf{频率为F的整数倍的无穷多个频率分量}。\\\noindent\textbf{答案:}C




\bigskip


\noindent\textbf{试题:}“频率失真”是指电路的输出信号波形与输入信号相比,发生了下列变化:
\begin{enumerate}[leftmargin=3em]
\item 信号的幅度发生了改变
\item 产生了新的频率分量
\item 不同频率分量的相位延迟差发生了改变
\item 各频率分量的比例发生了改变
\end{enumerate}
\noindent\textbf{解说:}“频率失真”是指电路的输出信号波形与输入信号相比,\textbf{各频率分量的比例发生了改变}。\\\noindent\textbf{答案:}D

%%%???



\bigskip


\noindent\textbf{试题:}“非线性失真”是指电路的输出信号波形与输入信号相比,发生了下列变化:
\begin{enumerate}[leftmargin=3em]
\item 产生了新的频率分量
\item 不同频率分量的相位延迟差发生了改变
\item 各频率分量的比例发生了改变
\item 信号的幅度发生了改变
\end{enumerate}
\noindent\textbf{解说:}“非线性失真”是指电路的输出信号波形与输入信号相比,\textbf{产生了新的频率分量}。\\\noindent\textbf{答案:}A

%%%???

\bigskip


\noindent\textbf{试题:}“相位失真”是指电路的输出信号波形与输入信号相比,发生了下列变化:
\begin{enumerate}[leftmargin=3em]
\item 信号的幅度发生了改变
\item 不同频率分量的相位延迟差发生了改变
\item 各频率分量的比例发生了改变
\item 产生了新的频率分量
\end{enumerate}
\noindent\textbf{解说:}“相位失真”是指电路的输出信号波形与输入信号相比,\textbf{不同频率分量的相位延迟差发生了改变}。\\\noindent\textbf{答案:}B



\bigskip


\noindent\textbf{试题:}在射频电路分析中,能产生信号频率以外的新频率分量的元器件属于有源元器件,可能成为形成干扰的重要环节。下列元器件中属于有源元器件的有:
\begin{enumerate}[leftmargin=3em]
\item 半导体二极管
\item 电解电容器
\item 宽带变压器
\item 碱性干电池
\end{enumerate}
\noindent\textbf{解说:}下列元器件中属于有源元器件的有\textbf{半导体二极管}。\\\noindent\textbf{答案:}A



\bigskip


\noindent\textbf{试题:}在射频电路分析中,能产生信号频率以外的新频率分量的元器件属于有源元器件,可能成为形成干扰的重要环节。下列元器件中属于有源元器件的有:
\begin{enumerate}[leftmargin=3em]
\item 电阻假负载
\item 可调电感器
\item 晶体滤波器
\item 半导体三极管
\end{enumerate}
\noindent\textbf{解说:}下列元器件中属于有源元器件的有\textbf{半导体三极管}。\\\noindent\textbf{答案:}D



\bigskip


\noindent\textbf{试题:}频率为\(f1\)、\(f2\)的两个正弦交流信号流过一个非线性元件,会发生“混频”。混频产物中属于三阶互调的干扰信号的频率是:
\begin{enumerate}[leftmargin=3em]
\item \(4f1\pm f2\)、\(5f1\pm 2f2\)、\(6f1\pm 3f2\)……
\item \(f1\pm f2\)、\(2f1\pm 2f2\)、\(3f1\pm 3f2\)
\item \(2f1\pm f2\)、\(2f2\pm f1\)
\item \(2f1\)、\(3f1\)、\(2f2\)、\(3f2\)
\end{enumerate}
\noindent\textbf{解说:}混频产物中属于三阶互调的干扰信号的频率是\(2f1\pm f2\)、\(2f2\pm f1\)。\\\noindent\textbf{答案:}C


\bigskip


\noindent\textbf{试题:}“差拍”现象是指:
\begin{enumerate}[leftmargin=3em]
\item 两个不同频率信号经过非线性电路得到频率为两者之差的新频率信号
\item 凡是接收机收到的连续音频叫声都叫做“差拍”
\item 一个单频率信号经过非线性电路得到一系列谐波信号,相邻信号之间的频率差等于单频信号的频率。这样一族谐波信号的集合总称“差拍”
\item 发射机用以调制载波的连续音频信号
\end{enumerate}
\noindent\textbf{解说:}“差拍”现象是指\textbf{两个不同频率信号经过非线性电路得到频率为两者之差的新频率信号}。\\\noindent\textbf{答案:}A





\bigskip


\noindent\textbf{试题:}一个CW电报信号在频谱仪上显示为:
\begin{enumerate}[leftmargin=3em]
\item 两条水平直线
\item 一条正弦波曲线
\item 一条固定的直线
\item 一条闪动的垂直线
\end{enumerate}
\noindent\textbf{解说:}一个CW电报信号在频谱仪上显示为\textbf{一条闪动的垂直线}。\\\noindent\textbf{答案:}D




\bigskip


\noindent\textbf{试题:}一个RTTY信号在频谱仪上显示为:
\begin{enumerate}[leftmargin=3em]
\item 两条闪动的垂直线
\item 一条正弦波曲线
\item 一条垂直线
\item 一条复杂的周期性曲线
\end{enumerate}
\noindent\textbf{解说:}一个RTTY信号在频谱仪上显示为\textbf{两条闪动的垂直线}。\\\noindent\textbf{答案:}A



\bigskip


\noindent\textbf{试题:}一个SSB话音信号在频谱仪上显示为:
\begin{enumerate}[leftmargin=3em]
\item 一条复杂的周期性曲线
\item 一组随语音出现和变化的非对称垂直线
\item 一条固定的直线
\item 一条随语音闪烁的直线
\end{enumerate}
\noindent\textbf{解说:}一个SSB话音信号在频谱仪上显示为\textbf{一组随语音出现和变化的非对称垂直线}。\\\noindent\textbf{答案:}B




\bigskip


\noindent\textbf{试题:}一个AM话音信号在频谱仪上显示为:
\begin{enumerate}[leftmargin=3em]
\item 多条固定的直线
\item 一条随语音闪烁的直线
\item 一条固定的垂直线,左右伴随一组对称的随语音出现和变化的垂直线
\item 一条复杂的周期性曲线
\end{enumerate}
\noindent\textbf{解说:}一个AM话音信号在频谱仪上显示为\textbf{一条固定的垂直线,左右伴随一组对称的随语音出现和变化的垂直线}。\\\noindent\textbf{答案:}C



\bigskip


\noindent\textbf{试题:}下列几种图表中,最容易用来表达和解释LC振荡器温度漂移程度的是:
\begin{enumerate}[leftmargin=3em]
\item 相位矢量图
\item 波形图
\item 频谱瀑布图
\item 频谱图
\end{enumerate}
\noindent\textbf{解说:}最容易用来表达和解释LC振荡器温度漂移程度的是\textbf{频谱瀑布图}。\\\noindent\textbf{答案:}C



\bigskip


\noindent\textbf{试题:}正弦交流信号通过下列电路时会产生高次谐波:
\begin{enumerate}[leftmargin=3em]
\item 调谐在信号的整倍频上的谐振电路
\item 二极管整流器或三极管开关放大器
\item 复杂电容电感网络
\item 复杂电阻网络
\end{enumerate}
\noindent\textbf{解说:}正弦交流信号通过\textbf{二极管整流器或三极管开关放大器}时会产生高次谐波。\\\noindent\textbf{答案:}B



\bigskip


\noindent\textbf{试题:}下列发射模式中拥有最窄带宽的是:
\begin{enumerate}[leftmargin=3em]
\item 调频话音
\item CW
\item 单边带话音
\item 慢扫描电视
\end{enumerate}
\noindent\textbf{解说:}\textbf{CW}发射模式拥有最窄带宽。\\\noindent\textbf{答案:}B



\bigskip


\noindent\textbf{试题:}无线电发射设备参数和业余无线电原理书籍中经常出现缩写为ppm的度量单位。其中文含义和最经常的用处是:
\begin{enumerate}[leftmargin=3em]
\item “每分钟点数”,常用于描述太阳黑子的发生频度
\item “每分钟点数”,常用于描述随机干扰的出现次数
\item “百万分比”,常用于描述频率的相对稳定度
\item “每分钟平均脉冲数”,常用于描述单个脉冲的出现次数
\end{enumerate}
\noindent\textbf{解说:}\textbf{“百万分比”,常用于描述频率的相对稳定度}。\\\noindent\textbf{答案:}C

%%%???


\bigskip


\noindent\textbf{试题:}甲乙两种业余无线电台设备资料列出接收机灵敏度指标分别为0.1μV和0.15μV。正确的推论为:
\begin{enumerate}[leftmargin=3em]
\item 可以推断甲机接收微弱信号的能力比乙机的高,因为可以接收的信号更微弱
\item 凭此指标还无法比较两者接收微弱信号的能力,因没有给出测量灵敏度时的输出信号质量条件
\item 可以推断甲机接收微弱信号的能力比乙机的低,因为灵敏度数值比较小
\item 可以推断甲机承受强信号的能力比乙机的低,因为灵敏度数值比较小
\end{enumerate}
\noindent\textbf{解说:}正确的推论为:\textbf{凭此指标还无法比较两者接收微弱信号的能力,因没有给出测量灵敏度时的输出信号质量条件}。\\\noindent\textbf{答案:}B

%%%???


\bigskip


\noindent\textbf{试题:}决定接收机抗拒与工作频率相距较远的强信号干扰的主要选择性指标是:
\begin{enumerate}[leftmargin=3em]
\item 前端带宽
\item 信道带宽、信道选择性和信道滤波器特性矩形系数
\item 镜像抑制比
\item 带内波动和信道带宽
\end{enumerate}
\noindent\textbf{解说:}决定接收机抗拒与工作频率相距较远的强信号干扰的主要选择性指标是\textbf{前端带宽}。\\\noindent\textbf{答案:}A

\bigskip


\noindent\textbf{试题:}决定接收机抗拒与工作频率相距两倍于中频的频率上强信号干扰的主要选择性指标是:
\begin{enumerate}[leftmargin=3em]
\item 镜像抑制比
\item 前端带宽
\item 信道带宽、信道选择性和信道滤波器特性矩形系数
\item 带内波动和信道带宽
\end{enumerate}
\noindent\textbf{解说:}决定接收机抗拒与工作频率相距两倍于中频的频率上强信号干扰的主要选择性指标是\textbf{镜像抑制比}。\\\noindent\textbf{答案:}A



\bigskip


\noindent\textbf{试题:}制约现代无线电接收机灵敏度的主要因素是:
\begin{enumerate}[leftmargin=3em]
\item 电源噪声
\item 放大电路的稳定性
\item 放大电路的增益
\item 机内噪声
\end{enumerate}
\noindent\textbf{解说:}制约现代无线电接收机灵敏度的主要因素是\textbf{机内噪声}。\\\noindent\textbf{答案:}D


\bigskip


\noindent\textbf{试题:}业余通信接收机大多带有接收信号强度指示。VHF/UHF频段的最小刻度S1对应于输入信号功率电平-141\unit[qualifier-mode=combine]{\deci\bel\of{m}}(50Ω输入电压0.02μV)标为S1,而HF频段的S1则对应于输入信号功率电平-121\unit[qualifier-mode=combine]{\deci\bel\of{m}}(0.2μV)。这是因为:
\begin{enumerate}[leftmargin=3em]
\item V/UHF频段较寂静而HF频段外界背景噪声电平较高,前者可感知的最小信号电平比后者低约20dB
\item HF业余电台功率一般比较大,VHF/UHF电台功率比较小,因此需要不同的刻度标准
\item 由于电路技术的原因,HF频段接收机的灵敏度只能做到比VHF/UHF频段低大约20dB
\item HF业余电台主要用于DX通信,VHF/UHF电台主要做本地通信,因此需要不同的刻度标准
\end{enumerate}
\noindent\textbf{解说:}因为\textbf{V/UHF频段较寂静而HF频段外界背景噪声电平较高,前者可感知的最小信号电平比后者低约20dB}。\\\noindent\textbf{答案:}A




\bigskip


\noindent\textbf{试题:}业余调频中继台发射机只要被上行信号正常启动,就会一直继续发射载波,上行信号消失不能使其停止。可能的原因是:
\begin{enumerate}[leftmargin=3em]
\item 中继台上下行隔离不良,中继台发射的载波窜入中继台接收机造成自锁
\item 中继台发射机电源电压不稳
\item 中继台接收机电源电压不稳
\item 肯定受到人为恶意干扰
\end{enumerate}
\noindent\textbf{解说:}可能的原因是\textbf{中继台上下行隔离不良,中继台发射的载波窜入中继台接收机造成自锁}。\\\noindent\textbf{答案:}A



\bigskip


\noindent\textbf{试题:}业余调频中继台发射机被上行信号正常启动,但上行信号消失后经常会继续发射一段或长或短的时间并夹杂有一些不清楚的语音。可能的原因是:
\begin{enumerate}[leftmargin=3em]
\item 中继台接收机电源电压不稳
\item 中继台下行信号与附近的其他通信发射机形成对中继台上行频率的互调干扰
\item 肯定受到人为恶意干扰
\item 中继台发射机电源电压不稳
\end{enumerate}
\noindent\textbf{解说:}可能的原因是\textbf{中继台下行信号与附近的其他通信发射机形成对中继台上行频率的互调干扰}。\\\noindent\textbf{答案:}B




\bigskip


\noindent\textbf{试题:}业余中继台上下行共用一副天线时,需要在接收机、发信机和天线之间插入一个:
\begin{enumerate}[leftmargin=3em]
\item 收发信转换开关
\item 环形器(circulator)
\item 双工器(duplexer)
\item 功率分配器(power divider)
\end{enumerate}
\noindent\textbf{解说:}业余中继台上下行共用一副天线时,需要在接收机、发信机和天线之间插入一个\textbf{双工器(duplexer)}。\\\noindent\textbf{答案:}C



\bigskip


\noindent\textbf{试题:}业余中继台上下行共用一副天线时,需要在接收机、发信机和天线之间插入一个双工器,其基本构造和作用为:
\begin{enumerate}[leftmargin=3em]
\item 一组滤波器,阻止中继台发射信号反馈进入中继台接收机
\item 一个匹配网络,使天线、中继发射机、中继接收机三者之间都满足阻抗匹配条件
\item 一个环形器,使信号只能沿中继发射机-天线-中继接收机的方向前进
\item 一组半导体开关,在中继台发射时关断中继台接收机
\end{enumerate}
\noindent\textbf{解说:}双工器的基本构造和作用为\textbf{一组滤波器,阻止中继台发射信号反馈进入中继台接收机}。\\\noindent\textbf{答案:}A




\bigskip


\noindent\textbf{试题:}无线电接收机的灵敏度是指:
\begin{enumerate}[leftmargin=3em]
\item 接收机正常工作所需的最小输入信号强度
\item 接收机正常工作所需的最大输入信号强度
\item 接收机正常工作所需的最小电源功率
\item 接收机正常工作所需的最大电源电压
\end{enumerate}
\noindent\textbf{解说:}无线电接收机的灵敏度是指\textbf{接收机正常工作所需的最小输入信号强度}。\\\noindent\textbf{答案:}A



\bigskip


\noindent\textbf{试题:}一部业余无线电台,工作电压直流13.8伏,FM发射方式的射频输出载波功率为N瓦,电源效率约80\%。发射时的工作电流约为:
\begin{enumerate}[leftmargin=3em]
\item 13.8$\times$N(安)
\item 0.058$\times$N(安)
\item 0.091$\times$N(安)
\item 13.8 / N$\times$80\%(安)%13.8/N$\times$80\%(安)
\end{enumerate}
\noindent\textbf{解说:}发射时的工作电流约为\textbf{0.091$\times$N(安)}。\\\noindent\textbf{答案:}C




\bigskip


\noindent\textbf{试题:}一部业余无线电台,工作电压交流220伏,FM发射方式的射频输出载波功率为N瓦,电源效率约80\%。发射时的工作电流约为:
\begin{enumerate}[leftmargin=3em]
\item 0.0036$\times$N(安)
\item 200 / N$\times$80\%(安)%200/N$\times$80\%(安)
\item 220$\times$N(安)
\item 0.0057$\times$N(安)
\end{enumerate}
\noindent\textbf{解说:}发射时的工作电流约为\textbf{0.0057$\times$N(安)}。\\\noindent\textbf{答案:}D




\bigskip


\noindent\textbf{试题:}一部业余无线电台,FM发射方式的射频输出载波功率为N瓦,电源效率约80\%。通话时每发射10秒钟的电源消耗约为:
\begin{enumerate}[leftmargin=3em]
\item 0.0768 / N(千瓦小时)%0.0768 /N(千瓦小时)
\item 220 / N(千瓦小时)
\item 0.0022$\times$N(千瓦小时)
\item 0.0000035$\times$N(千瓦小时)
\end{enumerate}
\noindent\textbf{解说:}通话时每发射10秒钟的电源消耗约为\textbf{0.0000035$\times$N(千瓦小时)}。\\\noindent\textbf{答案:}D





\bigskip


\noindent\textbf{试题:}一部业余无线电台,FM发射方式的射频输出载波功率为10瓦,电源效率约80\%。连续发话10秒钟,在此期间发射到空间的平均功率:
\begin{enumerate}[leftmargin=3em]
\item 约为12.5瓦
\item 约为8瓦
\item 约为10瓦
\item 肯定高于10瓦
\end{enumerate}
\noindent\textbf{解说:}连续发话10秒钟,在此期间发射到空间的平均功率\textbf{约为10瓦}。\\\noindent\textbf{答案:}C




\bigskip


\noindent\textbf{试题:}一部业余无线电台,CW发射方式的射频输出载波功率为10瓦,电源效率约80\%。连续发报10秒钟,在此期间发射到空间的平均功率:
\begin{enumerate}[leftmargin=3em]
\item 显著低于10瓦
\item 约为12.5瓦
\item 约为8瓦
\item 约为10瓦
\end{enumerate}
\noindent\textbf{解说:}连续发报10秒钟,在此期间发射到空间的平均功率\textbf{显著低于10瓦}。\\\noindent\textbf{答案:}A




\bigskip


\noindent\textbf{试题:}从能量转换的观点,“匹配”是指:
\begin{enumerate}[leftmargin=3em]
\item 选择电路参数,使负载能够得到最高实际输出功率的状态
\item 选择电路参数,使负载能够得到最高实际输出电流的状态
\item 选择电路参数,使电源内阻能够达到最小功率损耗的状态
\item 选择电路参数,使负载能够得到最高实际输出电压的状态
\end{enumerate}
\noindent\textbf{解说:}从能量转换的观点,“匹配”是指\textbf{选择电路参数,使负载能够得到最高实际输出功率的状态}。\\\noindent\textbf{答案:}A





\bigskip


\noindent\textbf{试题:}信号源和负载达到阻抗“匹配”时,信号源内阻损耗的功率为:
\begin{enumerate}[leftmargin=3em]
\item 与负载得到的输出功率相等
\item 损耗功率为零
\item 与负载得到的输出功率相比可以忽略不计
\item 是负载得到的输出功率的一半
\end{enumerate}
\noindent\textbf{解说:}信号源和负载达到阻抗“匹配”时,信号源内阻损耗的功率\textbf{与负载得到的输出功率相等}。\\\noindent\textbf{答案:}A



\bigskip


\noindent\textbf{试题:}一个放大器具有20dB的信号增益,其意义是:
\begin{enumerate}[leftmargin=3em]
\item 放大器把相当于输入信号的100倍的能量从电源转移到了输出负载
\item 放大器把相输入信号的能量放大了100倍
\item 放大器把相输入信号的能量放大了99倍
\item 放大器产生了当于输入信号的100倍的能量,供给了输出负载
\end{enumerate}
\noindent\textbf{解说:}一个放大器具有20dB的信号增益,其意义是\textbf{放大器把相当于输入信号的100倍的能量从电源转移到了输出负载}。\\\noindent\textbf{答案:}A



\bigskip


\noindent\textbf{试题:}射频信号通过某电路时产生了20dB的损耗。这部分被损耗的能量:
\begin{enumerate}[leftmargin=3em]
\item 返回了信号源
\item 一部分在电路中消失了,一部分返回了信号源
\item 在电路中消失了
\item 在电路中被转化为热能等其他形式,或者通过电磁辐射等转移到了其他地方
\end{enumerate}
\noindent\textbf{解说:}这部分被损耗的能量\textbf{在电路中被转化为热能等其他形式,或者通过电磁辐射等转移到了其他地方}。\\\noindent\textbf{答案:}D



\bigskip


\noindent\textbf{试题:}某电路输出信号功率是输入信号功率的100倍,该电路的增益为:
\begin{enumerate}[leftmargin=3em]
\item \SI{100}{\dB}
\item \SI{1}{\dB}
\item \SI{10}{\dB}
\item \SI{20}{\dB}
\end{enumerate}
\noindent\textbf{解说:}该电路的增益为\SI{20}{\dB}。\\
套入功率增益的计算公式得:
\begin{equation*}
\begin{aligned}
\mbox{功率增益}&=10 \log \left( {\frac{P_{ \mbox{输出} }}{P_{ \mbox{输入} }}} \right) \\
& = 10 \log 100 \\
& = 10 \log 10^2 \\
& = 10 \times 2\times \log 10 \\
& = 10 \times 2\times 1 \\
& = \SI{20}{\dB}
\end{aligned}
\end{equation*}
\noindent\textbf{答案:}D



\bigskip


\noindent\textbf{试题:}某电路输出信号功率是输入信号功率的100万倍,该电路的增益为:
\begin{enumerate}[leftmargin=3em]
\item \num{100}万\si{\dB}
\item \SI{60}{\dB}
\item \SI{100}{\dB}
\item \num{99}万\si{\dB}
\end{enumerate}
\noindent\textbf{解说:}该电路的增益为\SI{60}{\dB}。\\
套入功率增益的计算公式得:
\begin{equation*}
\begin{aligned}
\mbox{功率增益}&=10 \log \left( {\frac{P_{ \mbox{输出} }}{P_{ \mbox{输入} }}} \right)
& = 10 \log 1000000 \\
& = 10 \log 10^6 \\
& = 10\times 6\times \log 10 \\
& = 10\times 6\times 1 \\
& = \SI{60}{\dB}
\end{aligned}
\end{equation*}
\noindent\textbf{答案:}B




\bigskip


\noindent\textbf{试题:}某电路输出信号功率是输入信号功率的5倍,该电路的增益约为:
\begin{enumerate}[leftmargin=3em]
\item 7dB
\item 3.5dB
\item 5dB
\item 14dB
\end{enumerate}
\noindent\textbf{解说:}该电路的增益约为\textbf{7dB}。\\
套入功率增益的计算公式得:
$$\mbox{功率增益}=10 \log \left( {\frac{P_{ \mbox{输出} }}{P_{ \mbox{输入} }}} \right)=10 \log 5 = 10 \times 0.699 = \approx 7 \ \mathrm{dB}$$
\noindent\textbf{答案:}A




\bigskip


\noindent\textbf{试题:}某电路输出信号功率是输入信号功率的2倍,该电路的增益约为:
\begin{enumerate}[leftmargin=3em]
\item 3dB
\item 0.5dB
\item 1dB
\item 2dB
\end{enumerate}
\noindent\textbf{解说:}该电路的增益约为\textbf{3dB}。\\
套入功率增益的计算公式得:
$$\mbox{功率增益}=10 \log \left( {\frac{P_{ \mbox{输出} }}{P_{ \mbox{输入} }}} \right)=10 \log 2 = 10 \times 0.301 \approx 3 \ \mathrm{dB}$$
\noindent\textbf{答案:}A




\bigskip


\noindent\textbf{试题:}某电路输出信号电压是输入信号电压的100倍,该电路的增益为:
\begin{enumerate}[leftmargin=3em]
\item 100dB
\item 20dB
\item 40dB
\item 10dB
\end{enumerate}
\noindent\textbf{解说:}该电路的增益为\textbf{40dB}。\\
套入电压增益的计算公式得:
$$\mbox{电压增益}=20 \log \left( {\frac{V_{ \mbox{输出} }}{V_{ \mbox{输入} }}} \right)=20 \log 100=20 \log 10^2= 40 \ \mathrm{dB}$$
\noindent\textbf{答案:}C




\bigskip


\noindent\textbf{试题:}某电路输出信号电压是输入信号电压的1万倍,该电路的增益为:(”x\string^m”表示“x的m次方”)
\begin{enumerate}[leftmargin=3em]
\item 80dB
\item 10,000 dB
\item 9,999 dB
\item 10^4dB
\end{enumerate}
\noindent\textbf{解说:}该电路的增益为\textbf{80dB}。\\
套入电压增益的计算公式得:
$$\mbox{电压增益}=20 \log \left( {\frac{V_{ \mbox{输出} }}{V_{ \mbox{输入} }}} \right)=20 \log 10000=20 \log 10^4= 80 \ \mathrm{dB}$$
\noindent\textbf{答案:}A



\bigskip


\noindent\textbf{试题:}某电路输出信号电压是输入信号电压的10倍,该电路的增益约为:
\begin{enumerate}[leftmargin=3em]
\item 20dB
\item 14dB
\item 15dB
\item 7dB
\end{enumerate}
\noindent\textbf{解说:}该电路的增益约为\textbf{20dB}。\\
套入电压增益的计算公式得:
$$\mbox{电压增益}=20 \log \left( {\frac{V_{ \mbox{输出} }}{V_{ \mbox{输入} }}} \right)=20 \log 10=20 \ \mathrm{dB}$$
\noindent\textbf{答案:}A



\bigskip



%%%%%%%%%%%%%%% 2021-11-27 10:21AM B类 广西南宁 超纲题 %%%%%%%%%%%%%%%
\noindent\textbf{试题:}某电路输出信号电压是输入信号电压的7倍,该电路的增益约为:
\begin{enumerate}[leftmargin=3em]
	\item \SI{7}{\dB}
	\item \SI{10}{\dB}
	\item \SI{15}{\dB}
	\item \SI{17}{\dB}  %原选项14dB?不得已改成这个以符合题目要求
\end{enumerate}
\noindent\textbf{解说:}该电路的增益约为\SI{17}{\dB}。\\
套入电压增益的计算公式得:
\begin{equation*}
\begin{aligned}
\mbox{电压增益}&=20 \log \left( {\frac{V_{ \mbox{输出} }}{V_{ \mbox{输入} }}} \right) \\
& = 20 \log 7 \\
& = 20\times 0.845 \\
& \approx \SI{17}{\dB}
\end{aligned}
\end{equation*}
\noindent\textbf{答案:}D

\bigskip
%%%%%%%%%%%%%%% 2021-11-27 10:21AM B类 广西南宁 超纲题 %%%%%%%%%%%%%%%



\noindent\textbf{试题:}某电路输出信号电压是输入信号电压的2倍,该电路的增益约为:
\begin{enumerate}[leftmargin=3em]
\item 0.5dB
\item 2dB
\item 6dB
\item 4dB
\end{enumerate}
\noindent\textbf{解说:}该电路的增益约为\textbf{6dB}。\\
套入电压增益的计算公式得:
$$\mbox{电压增益}=20 \log \left( {\frac{V_{ \mbox{输出} }}{V_{ \mbox{输入} }}} \right)=20 \log 2 = 20 \times 0.301 \approx 6 \ \mathrm{dB}$$
\noindent\textbf{答案:}C





\bigskip


\noindent\textbf{试题:}某电路输出信号功率是输入信号功率的1/100,该电路的增益为:
\begin{enumerate}[leftmargin=3em]
\item 100dB
\item -20dB
\item -100dB
\item -10dB
\end{enumerate}
\noindent\textbf{解说:}该电路的增益为\textbf{-20dB}。\\
套入功率增益的计算公式得:
$$\mbox{功率增益}=10 \log \left( {\frac{P_{ \mbox{输出} }}{P_{ \mbox{输入} }}} \right)=10 \log \frac{1}{100}=10 \log 10^{-2}=-20 \ \mathrm{dB}$$
\noindent\textbf{答案:}B




\bigskip


\noindent\textbf{试题:}某电路输出信号功率是输入信号功率的百万分之一,该电路的增益为:
\begin{enumerate}[leftmargin=3em]
\item -100dB
\item 990000dB
\item -60dB
\item -1000000dB
\end{enumerate}
\noindent\textbf{解说:}该电路的增益为\textbf{-60dB}。\\
套入功率增益的计算公式得:
$$\mbox{功率增益}=10 \log \left( {\frac{P_{ \mbox{输出} }}{P_{ \mbox{输入} }}} \right)=10 \log \frac{1}{1000000}=10 \log 10^{-6}=-60 \ \mathrm{dB}$$
\noindent\textbf{答案:}C




\bigskip


\noindent\textbf{试题:}某电路输出信号功率是输入信号功率的1/5,该电路的增益约为:
\begin{enumerate}[leftmargin=3em]
\item 3.5dB
\item -5dB
\item -14dB
\item -7dB
\end{enumerate}
\noindent\textbf{解说:}该电路的增益约为\textbf{-7dB}。\\
套入功率增益的计算公式得:
$$\mbox{功率增益}=10 \log \left( {\frac{P_{ \mbox{输出} }}{P_{ \mbox{输入} }}} \right)=10 \log \frac{1}{5}=10 \left( \log 1 - \log 5 \right) = 10 \times \left( 0 - 0.699 \right) \approx -7 \ \mathrm{dB}$$
\noindent\textbf{答案:}D





\bigskip


\noindent\textbf{试题:}某电路输出信号功率是输入信号功率的1/2,该电路的增益约为:
\begin{enumerate}[leftmargin=3em]
\item -1dB
\item -3dB
\item 0.5dB
\item -2dB
\end{enumerate}
\noindent\textbf{解说:}该电路的增益约为\textbf{-3dB}。\\
套入功率增益的计算公式得:
$$\mbox{功率增益}=10 \log \left( {\frac{P_{ \mbox{输出} }}{P_{ \mbox{输入} }}} \right)=10 \log \frac{1}{2}=10 \left( \log 1 - \log 2 \right) = 10 \times \left( 0 - 0.301 \right) \approx -3 \ \mathrm{dB}$$
\noindent\textbf{答案:}B




\bigskip


\noindent\textbf{试题:}某电路输出信号电压是输入信号电压的1/100,该电路的增益为:
\begin{enumerate}[leftmargin=3em]
\item -40dB
\item -20dB
\item -10dB
\item -100dB
\end{enumerate}
\noindent\textbf{解说:}该电路的增益为\textbf{-40dB}。\\
套入电压增益的计算公式得:
$$\mbox{电压增益}=20 \log \left( {\frac{V_{ \mbox{输出} }}{V_{ \mbox{输入} }}} \right)=20 \log \frac{1}{100}=20 \log 10^{-2}= -40 \ \mathrm{dB}$$
\noindent\textbf{答案:}A



\bigskip


\noindent\textbf{试题:}某电路输出信号电压是输入信号电压的万分之一,该电路的增益为:(”x\string^m”表示“x的m次方”)
\begin{enumerate}[leftmargin=3em]
\item -10,000 dB
\item 1/10,000 dB
\item 10^-4dB
\item -80dB
\end{enumerate}
\noindent\textbf{解说:}该电路的增益为\textbf{-80dB}。\\
套入电压增益的计算公式得:
$$\mbox{电压增益}=20 \log \left( {\frac{V_{ \mbox{输出} }}{V_{ \mbox{输入} }}} \right)=20 \log \frac{1}{10000}=20 \log 10^{-4}= -80 \ \mathrm{dB}$$
\noindent\textbf{答案:}D



\bigskip


\noindent\textbf{试题:}某电路输出信号电压是输入信号电压的1/10倍,该电路的增益约为:
\begin{enumerate}[leftmargin=3em]
\item -7dB
\item -14dB
\item 0.143dB
\item -20dB
\end{enumerate}
\noindent\textbf{解说:}该电路的增益约为\textbf{-20dB}。\\
套入电压增益的计算公式得:
$$\mbox{电压增益}=20 \log \left( {\frac{V_{ \mbox{输出} }}{V_{ \mbox{输入} }}} \right)=20 \log \frac{1}{10}=20 \log 10^{-1}= -20 \ \mathrm{dB}$$
\noindent\textbf{答案:}D



\bigskip


\noindent\textbf{试题:}某电路输出信号电压是输入信号电压的1/2,该电路的增益约为:
\begin{enumerate}[leftmargin=3em]
\item -0.5dB
\item -2dB
\item 0.5dB
\item -6dB
\end{enumerate}
\noindent\textbf{解说:}该电路的增益约为\textbf{-6dB}。\\
套入电压增益的计算公式得:
$$\mbox{电压增益}=20 \log \left( {\frac{V_{ \mbox{输出} }}{V_{ \mbox{输入} }}} \right)=20 \log \frac{1}{2}=20 \left( \log 1 - \log 2 \right) = 20 \times \left( 0 - 0.301 \right) \approx -6 \ \mathrm{dB}$$
\noindent\textbf{答案:}D



\bigskip


\noindent\textbf{试题:}信号依次通过增益分别为 x dB、y dB和 z dB的三个电路,总增益为:
\begin{enumerate}[leftmargin=3em]
\item (x × y × z)dB
\item (x + y + z)dB
\item (x × y × z)倍
\item (x + y + z)倍
\end{enumerate}
\noindent\textbf{解说:}总增益为\textbf{(x + y + z)dB}。\\\noindent\textbf{答案:}B




\bigskip


\noindent\textbf{试题:}信号依次通过增益分别为 x dB、y dB和 z dB的三个电路,总增益为:(”x\string^m”表示“x的m次方”)
\begin{enumerate}[leftmargin=3em]
\item 10^(x × y × z) 倍
\item (x + y + z)倍
\item (x × y × z)倍
\item 10^((x + y + z)/ 10) 倍%10^((x + y + z)/10) 倍
\end{enumerate}
\noindent\textbf{解说:}总增益为\textbf{10^((x + y + z)/ 10) 倍}。\\\noindent\textbf{答案:}D



\bigskip


\noindent\textbf{试题:}接收机的接收信号强度表每两档的信号强度相差6dB。接收某电台信号,发射功率为20\unit[qualifier-mode=combine]{\deci\bel\of{W}}时读数为S9。该台减小发射功率后,接收机读数变为S4。此时该台的发射功率约为(以W为单位):
\begin{enumerate}[leftmargin=3em]
\item \SI{1.73}{\W}
\item 10.24W
\item 0.098W
\item 0.156W
\end{enumerate}
\noindent\textbf{解说:}此时该台的发射功率约为\textbf{0.098W}。\\\noindent\textbf{答案:}C





\bigskip


\noindent\textbf{试题:}接收机的接收信号强度表每两档的信号强度相差\SI{6}{\dB}。接收某电台信号,发射功率为10\unit[qualifier-mode=combine]{\deci\bel\of{W}}时读数为S8。该台减小发射功率后,接收机读数变为S5。此时该台的发射功率约为(以\si{\W}为单位):
\begin{enumerate}[leftmargin=3em]
\item \SI{0.098}{\W}
\item \SI{10.24}{\W}
\item \SI{1.73}{\W}
\item \SI{0.156}{\W}
\end{enumerate}
\noindent\textbf{解说:}此时该台的发射功率约为\SI{0.156}{\W}。\\\noindent\textbf{答案:}D





\bigskip


\noindent\textbf{试题:}功率为\num{0} \unit[qualifier-mode=combine]{\deci\bel\of{m}}的射频信号通过增益为 23 dB的电路后,输出功率为:
\begin{enumerate}[leftmargin=3em]
\item \SI{0.23}{\W}
\item \SI{23}{\W}
\item \SI{0.2}{\W}
\item \SI{23}{\mW}
\end{enumerate}
\noindent\textbf{解说:}输出功率为\textbf{\SI{0.2}{\W}}。\\\noindent\textbf{答案:}C



\bigskip


\noindent\textbf{试题:}功率为\num{0}\unit[qualifier-mode=combine]{\deci\bel\of{μ}}的射频信号通过增益为\SI{36}{\dB}的电路后,输出功率为:
\begin{enumerate}[leftmargin=3em]
\item 360μW
\item \SI{4}{\mW}
\item 3.6W
\item 36mW
\end{enumerate}
\noindent\textbf{解说:}输出功率为\textbf{\SI{4}{\mW}}。\\\noindent\textbf{答案:}B


\bigskip


\noindent\textbf{试题:}功率为\num{0} \unit[qualifier-mode=combine]{\deci\bel\of{W}}的射频信号通过增益为\SI{-36}{\dB}的电路后,输出功率为:
\begin{enumerate}[leftmargin=3em]
\item \SI{25}{\mW}
\item \SI{0.25}{\mW}
\item \SI{360}{\uW}
\item \SI{3.6}{\mW}
\end{enumerate}
\noindent\textbf{解说:}输出功率为\SI{0.25}{\mW}。\\
\noindent\textbf{答案:}B



\bigskip


\noindent\textbf{试题:}功率为\num{0} \unit[qualifier-mode=combine]{\deci\bel\of{W}}的射频信号通过衰减量为\SI{40}{\dB}的衰减器后,输出功率为:
\begin{enumerate}[leftmargin=3em]
\item \SI{140}{\uW}
\item \SI{100}{\uW}
\item \SI{0.40}{\W}  % 0.40 W
\item \SI{40}{\mW}
\end{enumerate}
\noindent\textbf{解说:}输出功率为\SI{100}{\uW}。\\
\noindent\textbf{答案:}B




\bigskip


\noindent\textbf{试题:}某业余电台以\num{100}瓦功率发射时,对方接收机的信号强度指示为S8。现双方天线不变,将发射功率降到\num{25}瓦,对方接收的信号强度指示将变为:【提示:收信机信号强度指示S1至S9每档相差6dB】
\begin{enumerate}[leftmargin=3em]
\item S7
\item S6
\item S4
\item S5
\end{enumerate}
\noindent\textbf{解说:}对方接收的信号强度指示将变为\textbf{S7}。\\\noindent\textbf{答案:}A



\bigskip


\noindent\textbf{试题:}某业余电台以\num{80}瓦功率发射时,对方接收机的信号强度指示为S8。现双方天线不变,将发射功率降为\num{5}瓦QRP,对方接收的信号强度指示将变为:【提示:收信机信号强度指示S1至S9每档相差6dB】
\begin{enumerate}[leftmargin=3em]
\item S6
\item S2
\item S4
\item S7
\end{enumerate}
\noindent\textbf{解说:}对方接收的信号强度指示将变为\textbf{S6}。\\\noindent\textbf{答案:}A



\bigskip


\noindent\textbf{试题:}在特定方向上具有主辐射瓣的水平偶极天线,其振子的总长度应为:
\begin{enumerate}[leftmargin=3em]
\item \(1/2\)波长的奇数倍
\item \(1/4\)波长的奇数倍
\item \(1/2\)波长的偶数倍
\item \(1/2\)波长的任意整数倍
\end{enumerate}
\noindent\textbf{解说:}在特定方向上具有主辐射瓣的水平偶极天线,其振子的总长度应为\(1/2\)波长的奇数倍。\\\noindent\textbf{答案:}A



\bigskip


\noindent\textbf{试题:}偶极天线与工作频率发生谐振的充分和必要条件是:
\begin{enumerate}[leftmargin=3em]
\item 两臂总电气长度为工作波长的整数倍
\item 两臂总电气长度为\(1/2\)工作波长的奇数倍
\item 两臂总电气长度为\(1/2\)工作波长的整数倍
\item 两臂总电气长度为\(1/4\)工作波长的整数倍
\end{enumerate}
\noindent\textbf{解说:}偶极天线与工作频率发生谐振的充分和必要条件是两臂总电气长度为\(1/2\)工作波长的整数倍。\\\noindent\textbf{答案:}C



\bigskip


\noindent\textbf{试题:}偶极天线两臂总长度选取下列电气长度时,在垂直于天线轴线方向的增益达到峰值:
\begin{enumerate}[leftmargin=3em]
\item \(1/2\)工作波长的整数倍
\item \(1/2\)工作波长的偶数倍
\item \(1/4\)工作波长的奇数倍
\item \(1/2\)工作波长的奇数倍
\end{enumerate}
\noindent\textbf{解说:}偶极天线两臂总长度选取下列电气长度时,在垂直于天线轴线方向的增益达到峰值\(1/2\)工作波长的奇数倍。\\
\noindent\textbf{答案:}D



\bigskip


\noindent\textbf{试题:}制作工作频率为\(f\)(单位:兆赫兹)的某相控天线阵列需要长度为\(1/4\)波长的同轴电缆。其大致长度(单位:米)为:
\begin{enumerate}[leftmargin=3em]
\item \(71.3 / f\)
\item \(48.8 / f\)
\item \(149.8 / f\)
\item \(75 / f\)
\end{enumerate}
\noindent\textbf{解说:}其大致长度(单位:米)为\(48.8 / f\)。\\
\noindent\textbf{答案:}B
%%%%%%

\bigskip


\noindent\textbf{试题:}制作工作频率为\(f\)(单位:兆赫兹)的半波长偶极天线。每个振子的大致长度(单位:米)为:
\begin{enumerate}[leftmargin=3em]
\item \(71.3 / f\)
\item \(48.8 / f\)
\item \(150 / f\)
\item \(142.6 / f\)
\end{enumerate}
\noindent\textbf{解说:}每个振子的大致长度(单位:米)为\(71.3 / f\)。\\
\noindent\textbf{答案:}A
%%%%%%

\bigskip


\noindent\textbf{试题:}南北走向的水平极化偶极天线,中点馈电,通过特性阻抗为50欧的电缆连接到输入/输出阻抗为50欧的收发信机,通信对象在东西方向。选择天线长度的原则是:
\begin{enumerate}[leftmargin=3em]
\item 天线臂长为四分之一波长的奇数倍时,通信效果肯定最好
\item 天线臂长为四分之一波长的偶数倍时,通信效果肯定最好
\item 只要天线与工作频率谐振,通信效果一定好
\item 只要电压驻波比达到最低,通信效果一定好
\end{enumerate}
\noindent\textbf{解说:}选择天线长度的原则是\textbf{天线臂长为四分之一波长的奇数倍时,通信效果肯定最好}。\\\noindent\textbf{答案:}A



\bigskip


\noindent\textbf{试题:}对称半波振子每一臂的长度为波长的:
\begin{enumerate}[leftmargin=3em]
\item 4倍
\item 1/2倍
\item 1/4倍
\item 2倍
\end{enumerate}
\noindent\textbf{解说:}对称半波振子每一臂的长度为波长的\textbf{1/4倍}。\\\noindent\textbf{答案:}C

\bigskip


\noindent\textbf{试题:}对一个偶极子天线怎么做,才能让它的谐振频率升高一些?
\begin{enumerate}[leftmargin=3em]
\item 将振子截短一些
\item 在振子的两端加上电容帽
\item 将振子加长一些
\item 在振子某部位串联一个线圈
\end{enumerate}
\noindent\textbf{解说:}\textbf{将振子截短一些}能让一个偶极子天线的谐振频率升高一些。\\\noindent\textbf{答案:}A



\bigskip


\noindent\textbf{试题:}甲偶极天线的增益为6.15\unit[qualifier-mode=combine]{\deci\bel\of{i}},乙偶极天线的增益为1\unit[qualifier-mode=combine]{\deci\bel\of{d}}。当两副天线按同样条件架设、用同样功率驱动时、在它们最大发射方向的同一远方地点接收时,两天线产生的信号功率的关系为:
\begin{enumerate}[leftmargin=3em]
\item 甲天线的信号功率为乙天线的1/2
\item 甲天线的信号功率为乙天线的两倍
\item 甲天线的信号功率为乙天线的5.15倍
\item 甲天线的信号功率为乙天线的6.15倍
\end{enumerate}
\noindent\textbf{解说:}两天线产生的信号功率的关系为\textbf{甲天线的信号功率为乙天线的两倍}。\\\noindent\textbf{答案:}B


\bigskip


\noindent\textbf{试题:}甲天线的增益为0\unit[qualifier-mode=combine]{\deci\bel\of{d}},乙天线的增益为2\unit[qualifier-mode=combine]{\deci\bel\of{i}}。当两副天线按同样条件架设、用同样功率驱动时、在它们最大发射方向的同一远方地点接收并比较收到的信号功率强度,正确的说法为:
\begin{enumerate}[leftmargin=3em]
\item 甲、乙天线的效果实际相同
\item 甲天线效果为零,不能工作,乙天线效果比甲天线好2倍%。
\item 甲天线的效果与半波长偶极天线相当,乙天线发射的信号强度比甲天线好2dB%。
\item 甲天线的效果与半波长偶极天线相当,乙天线比甲天线略差%。
\end{enumerate}
\noindent\textbf{解说:}正确的说法为\textbf{甲天线的效果与半波长偶极天线相当,乙天线比甲天线略差}。\\\noindent\textbf{答案:}D



\bigskip


\noindent\textbf{试题:}业余条件测试天线增益的典型方法如图。用场强表或接收机接收设置在远处同一地点、最大辐射方向朝向自己的半波偶极天线(上)和待测天线(下)。调整送到两副天线的射频功率\(Po\)和\(P\),使接收到的场强相同。待测天线的增益\unit[qualifier-mode=combine]{\deci\bel\of{d}}值为:
\begin{enumerate}[leftmargin=3em]
\item \(10 lg(P/Po)\)
\item \(P - Po\)
\item \(10 lg(P-Po)\)
\item \(10 lg(Po/P)\)
\end{enumerate}
\noindent\textbf{解说:}待测天线的增益\unit[qualifier-mode=combine]{\deci\bel\of{d}}值为\(10 lg(Po/P)\)。\\
\noindent\textbf{答案:}D
% LK0927.jpg
%%% 插入图片

\bigskip


\noindent\textbf{试题:}业余条件测试天线增益的典型方法如图。用场强表或接收机接收设置在远处同一地点、最大辐射方向朝向自己的半波偶极天线(上)和待测天线(下)。调整送到两副天线的射频功率\(Po\)和\(P\),使接收到的场强相同。待测天线的增益\unit[qualifier-mode=combine]{\deci\bel\of{i}}值为:
\begin{enumerate}[leftmargin=3em]
\item \(20 lg(P/Po) + 2.15\)
\item \(10 lg(Po/P) + 2.15\)
\item \(10 lg(P-Po) + 2.15\)
\item \(10 lg(P/Po) + 2.15\)
\end{enumerate}
\noindent\textbf{解说:}待测天线的增益\unit[qualifier-mode=combine]{\deci\bel\of{i}}值为\(10 lg(Po/P) + 2.15\)。\\
\noindent\textbf{答案:}B
%%% 插入图片
% LK0928.jpg

\bigskip


\noindent\textbf{试题:}某业余电台使用半波长偶极天线发射时,对方接收机的信号强度指示为S4。现发射功率不变,发信端改用增益为 8.15 \unit[qualifier-mode=combine]{\deci\bel\of{i}}的八木天线(最大辐射方向不变),对方接收的信号强度指示将变为:【提示:收信机信号强度指示S1至S9每档相差6dB】
\begin{enumerate}[leftmargin=3em]
\item S7
\item S5
\item S8
\item S6
\end{enumerate}
\noindent\textbf{解说:}对方接收的信号强度指示将变为\textbf{S5}。\\\noindent\textbf{答案:}B



\bigskip


\noindent\textbf{试题:}某业余电台使用半波长偶极天线发射时,对方接收机的信号强度指示为S4。现发射功率不变,发信端改用增益为 12 \unit[qualifier-mode=combine]{\deci\bel\of{d}}的八木天线(最大辐射方向不变),对方接收的信号强度指示将变为:【提示:收信机信号强度指示S1至S9每档相差6dB】
\begin{enumerate}[leftmargin=3em]
\item S6
\item S5
\item S8
\item S7
\end{enumerate}
\noindent\textbf{解说:}对方接收的信号强度指示将变为\textbf{S6}。\\\noindent\textbf{答案:}A



\bigskip


\noindent\textbf{试题:}某业余电台使用半波长偶极天线发射时,对方亦使用半波长偶极天线接收,接收机的信号强度指示为S4。现发射功率不变,收发双方都改用增益为 8.15 \unit[qualifier-mode=combine]{\deci\bel\of{i}}的八木天线(最大辐射方向不变),对方接收的信号强度指示将变为:【提示:收信机信号强度指示S1至S9每档相差6dB】
\begin{enumerate}[leftmargin=3em]
\item S4
\item S6
\item S5
\item S7
\end{enumerate}
\noindent\textbf{解说:}对方接收的信号强度指示将变为\textbf{S6}。\\\noindent\textbf{答案:}B


\bigskip


\noindent\textbf{试题:}甲、乙业余电台相距2000公里,均使用1/2波长水平偶极天线,正在HF频段进行稳定的通信。现其中一方改用1/2波长垂直偶极天线,改变前后的通信效果的比较将是:
\begin{enumerate}[leftmargin=3em]
\item 通信效果的变化不确定,取决于当时天波反射途中极化方向的旋转情况
\item 通信效果变好
\item 通信效果没有改变
\item 通信效果变差
\end{enumerate}
\noindent\textbf{解说:}改变前后的通信效果的比较将是\textbf{通信效果的变化不确定,取决于当时天波反射途中极化方向的旋转情况}。\\\noindent\textbf{答案:}A



\bigskip


\noindent\textbf{试题:}垂直偶极天线所发射的无线电波的极化方式为:
\begin{enumerate}[leftmargin=3em]
\item 右旋圆极化波
\item 水平极化波
\item 垂直极化波
\item 左旋圆极化波
\end{enumerate}
\noindent\textbf{解说:}垂直偶极天线所发射的无线电波的极化方式为\textbf{垂直极化波}。\\\noindent\textbf{答案:}C



\bigskip


\noindent\textbf{试题:}水平偶极天线所发射的无线电波的极化方式为:
\begin{enumerate}[leftmargin=3em]
\item 左旋圆极化波
\item 右旋圆极化波
\item 垂直极化波
\item 水平极化波
\end{enumerate}
\noindent\textbf{解说:}水平偶极天线所发射的无线电波的极化方式为\textbf{水平极化波}。\\\noindent\textbf{答案:}D



\bigskip


\noindent\textbf{试题:}假设收发天线均采用半波长偶极天线。在依靠电离层反射的远距离通信中,发射天线和接收天线的最佳极化方式为:
\begin{enumerate}[leftmargin=3em]
\item 收发天线的极化方向都平行于两台之间的连线
\item 不确定,根据具体传播情况而经常变化
\item 收发天线都处于垂直于两台连线的平面内收发天线极化方向互相垂直
\item 收发天线都处于垂直于两台连线的平面内并且极化方向互相一致
\end{enumerate}
\noindent\textbf{解说:}在依靠电离层反射的远距离通信中,发射天线和接收天线的最佳极化方式\textbf{不确定,根据具体传播情况而经常变化}。\\\noindent\textbf{答案:}B




\bigskip


\noindent\textbf{试题:}短波水平偶极类天线(如偶极天线和八木天线等)的发射仰角主要由下列因素决定:
\begin{enumerate}[leftmargin=3em]
\item 由天线的辐射和大地的反射叠加造成,仰角高低与天线离地高度与波长的比值有关
\item 由八木天线主梁所指的方向决定
\item 由天线振子的长度所决定
\item 由天线振子导体所指的方向决定
\end{enumerate}
\noindent\textbf{解说:}短波水平偶极类天线(如偶极天线和八木天线等)的发射仰角主要\textbf{由天线的辐射和大地的反射叠加造成,仰角高低与天线离地高度与波长的比值有关}。\\\noindent\textbf{答案:}A



\bigskip


\noindent\textbf{试题:}架设短波天线时,天线发射仰角的大致选择原则是:
\begin{enumerate}[leftmargin=3em]
\item 近距离通信选择低发射仰角,远距离通信选择高发射仰角
\item 远距离通信选择低发射仰角,近距离通信选择高发射仰角
\item 较低频率通信选择低发射仰角,较高频率通信选择高发射仰角
\item 近处开阔时选择低发射仰角,近处有建筑物时选择高发射仰角
\end{enumerate}
\noindent\textbf{解说:}架设短波天线时,天线发射仰角的大致选择原则是\textbf{远距离通信选择低发射仰角,近距离通信选择高发射仰角}。\\\noindent\textbf{答案:}B




\bigskip


\noindent\textbf{试题:}架设短波天线时,天线高度的大致选择原则是:
\begin{enumerate}[leftmargin=3em]
\item 近处有建筑物时选择较低的高度,近处开阔时选择较高的高度
\item 较低频率通信选择较高的高度,较高频率通信选择较低的高度
\item 远距离通信选择较高的高度,近距离通信选择较低的高度
\item 远距离通信选择较低的高度,近距离通信选择较高的高度
\end{enumerate}
\noindent\textbf{解说:}架设短波天线时,天线高度的大致选择原则是\textbf{远距离通信选择较高的高度,近距离通信选择较低的高度}。\\\noindent\textbf{答案:}C



\bigskip


\noindent\textbf{试题:}在针对特定对象的DX通信中,计算天线最佳发射仰角的基本方法是:
\begin{enumerate}[leftmargin=3em]
\item 根据通信对象所在的方位、通信方向上障碍物所遮挡的仰角、本台周围的大地导电率、实际工作频率,找公式计算
\item 根据所使用电离层的大致高度、通信对象的大致距离、电波在传播途经中经电离层反射的次数,用简单几何方法计算
\item 根据通信对象所在的方位、地球半径、对方天线高度、实际工作频率、太阳平均黑子数,查表计算
\item 根据通信双方的发射功率、天线极化方向、通信方向上障碍物所遮挡的仰角、太阳10.7 cm 射电流量,找公式计算
\end{enumerate}
\noindent\textbf{解说:}在针对特定对象的DX通信中,计算天线最佳发射仰角的基本方法是\textbf{根据所使用电离层的大致高度、通信对象的大致距离、电波在传播途经中经电离层反射的次数,用简单几何方法计算}。\\\noindent\textbf{答案:}B



\bigskip


\noindent\textbf{试题:}通过目视判断全尺寸八木天线发射方向的办法是:
\begin{enumerate}[leftmargin=3em]
\item 比主振子短者为引向振子,比主振子长者为反射振子,反射振子朝向最大辐射方向
\item 比主振子短者为引向振子,比主振子长者为反射振子,引向振子朝向最大辐射方向
\item 主振子两端所指方向为最大辐射方向
\item 比主振子长者为引向振子,比主振子短者为反射振子,引向振子朝向最大辐射方向
\end{enumerate}
\noindent\textbf{解说:}通过目视判断全尺寸八木天线发射方向的办法是\textbf{比主振子短者为引向振子,比主振子长者为反射振子,引向振子朝向最大辐射方向}。\\\noindent\textbf{答案:}B

\bigskip


\noindent\textbf{试题:}北京的水平极化半波长偶极天线,通信对象为纽约的业余电台。按电波的最短传输途径考虑,天线的最佳走向应大致为:
\begin{enumerate}[leftmargin=3em]
\item 西偏南30度-东偏北30度
\item 东偏南30度-西偏北30度
\item 东-西
\item 南-北
\end{enumerate}
\noindent\textbf{解说:}按电波的最短传输途径考虑,天线的最佳走向应大致为\textbf{东-西}。\\\noindent\textbf{答案:}C



\bigskip


\noindent\textbf{试题:}在导电良好的地面上,决定短波天线辐射仰角的主要参数是:
\begin{enumerate}[leftmargin=3em]
\item 天线离海平面的绝对高度
\item 天线导线或者八木天线主梁与地面之间的夹角
\item 天线的绝对高度,与波长无关
\item 天线离地面的相对于波长的高度,即离地高度除以波长
\end{enumerate}
\noindent\textbf{解说:}在导电良好的地面上,决定短波天线辐射仰角的主要参数是\textbf{天线离地面的相对于波长的高度,即离地高度除以波长}。\\\noindent\textbf{答案:}D


\bigskip


\noindent\textbf{试题:}什么是八木天线?
\begin{enumerate}[leftmargin=3em]
\item 一种可以将接收到的信号反向的天线
\item 一种可以集中聚集某一方向信号的天线
\item 一种由日本八木秀次发明的全向天线
\item 用八根木头制作的天线
\end{enumerate}
\noindent\textbf{解说:}八木天线是\textbf{一种可以集中聚集某一方向信号的天线}。\\\noindent\textbf{答案:}B




\bigskip


\noindent\textbf{试题:}在自由空间中的半波偶极子天线,哪个方向的辐射强度最大?
\begin{enumerate}[leftmargin=3em]
\item 沿着馈线的方向
\item 垂直于导体的方向
\item 各个方向强度相同
\item 沿着导体的方向
\end{enumerate}
\noindent\textbf{解说:}在自由空间中的半波偶极子天线,\textbf{垂直于导体的方向}方向的辐射强度最大。\\\noindent\textbf{答案:}B



\bigskip


\noindent\textbf{试题:}同样材料、同样直径、同样长度的实心铜线和空心铜管,在交流电路中的发热损耗情况为:
\begin{enumerate}[leftmargin=3em]
\item 在低频率下实心铜线损耗较大,在高频率下两者损耗一样
\item 在各种频率下两者耗差都一样
\item 在低频率下实心铜线损耗较小,在高频率下两者损耗一样
\item 在不同频率两者的发热损耗大小不好说,取决于具体散热条件
\end{enumerate}
\noindent\textbf{解说:}同样材料、同样直径、同样长度的实心铜线和空心铜管,在交流电路中的发热损耗情况为\textbf{在低频率下实心铜线损耗较小,在高频率下两者损耗一样}。\\\noindent\textbf{答案:}C

\bigskip


\noindent\textbf{试题:}把实心导线接到频率为数十兆赫兹的高频率射频电路中,则会有下列现象:
\begin{enumerate}[leftmargin=3em]
\item 表层的电流沿导线方向流动,内层电流形成螺旋状涡流
\item 导线外层和内层都有电流,但两者方向相反
\item 电流集中在导线表层,导线内部没有电流
\item 导线截面各处的电流密度均匀分布
\end{enumerate}
\noindent\textbf{解说:}把实心导线接到频率为数十兆赫兹的高频率射频电路中,则会有\textbf{电流集中在导线表层,导线内部没有电流}的现象。\\\noindent\textbf{答案:}C


\bigskip


\noindent\textbf{试题:}工作在高频率下的射频部件积灰或受潮后,即使没有漏电,也可能因绝缘物体的物理变化而带来额外的:
\begin{enumerate}[leftmargin=3em]
\item 介质损耗
\item 磁滞损耗
\item 磁阻损耗
\item 涡流损耗
\end{enumerate}
\noindent\textbf{解说:}工作在高频率下的射频部件积灰或受潮后,即使没有漏电,也可能因绝缘物体的物理变化而带来额外的\textbf{介质损耗}。\\\noindent\textbf{答案:}A



\bigskip


\noindent\textbf{试题:}天线和馈线之间经常接一个俗称“巴伦(BALUN)”的部件。“巴伦”的由来是:
\begin{enumerate}[leftmargin=3em]
\item 著名天线阻抗匹配理论家的名字
\item 发明平衡不平衡转换器的人的名字
\item 宽带阻抗变压器的英文缩写
\item 平衡和不平衡两个英文字头的组合
\end{enumerate}
\noindent\textbf{解说:}“巴伦”的由来是\textbf{平衡和不平衡两个英文字头的组合}。\\\noindent\textbf{答案:}D




\bigskip


\noindent\textbf{试题:}天线和馈线之间经常接一个俗称“巴伦(BALUN)”的部件。它的主要功能是:
\begin{enumerate}[leftmargin=3em]
\item 展宽天线的工作频带
\item 降低天线的驻波比
\item 实现天线和馈线之间的自动阻抗匹配
\item 在平衡电路和不平衡电路之间传递射频能量,并阻断两者之间的任何寄生耦合
\end{enumerate}
\noindent\textbf{解说:}“巴伦”的主要功能是\textbf{在平衡电路和不平衡电路之间传递射频能量,并阻断两者之间的任何寄生耦合}。\\\noindent\textbf{答案:}D




\bigskip


\noindent\textbf{试题:}同轴电缆的绝缘介质相同时,影响特性阻抗的因素是:
\begin{enumerate}[leftmargin=3em]
\item 外导体内径越大,特性阻抗越高
\item 内导体外径越大,特性阻抗越高
\item 外导体内径和内导体外径的比越大,特性阻抗越高
\item 电缆的长度越长,特性阻抗越高
\end{enumerate}
\noindent\textbf{解说:}同轴电缆的绝缘介质相同时,影响特性阻抗的因素是:\textbf{外导体内径和内导体外径的比越大,特性阻抗越高}。\\\noindent\textbf{答案:}C



\bigskip


\noindent\textbf{试题:}天线调谐器(俗称“天调”)的作用是:
\begin{enumerate}[leftmargin=3em]
\item 对天馈系统进行优化调谐,使整个系统的辐射功率获得一个附加的增益
\item 对不匹配的天馈系统进行补偿,使整个天馈系统的传输和辐射效率达到匹配天线的水平
\item 对不匹配的天馈系统进行补偿,虽然不能改善馈线的损耗,但能使天线本身的辐射效率达到匹配天线的水平
\item 补偿不匹配系统,向收发信机提供谐振的、阻抗匹配的负载,但不能改善天线本身的辐射效率
\end{enumerate}
\noindent\textbf{解说:}天线调谐器(俗称“天调”)的作用是\textbf{补偿不匹配系统,向收发信机提供谐振的、阻抗匹配的负载,但不能改善天线本身的辐射效率}。\\\noindent\textbf{答案:}D



\bigskip


\noindent\textbf{试题:}天线通过50欧同轴馈线与输出阻抗为50欧的收发信机相连接,并打算在天线电路中串入天线调谐器和通过式驻波功率计来监测和补偿天线的失配。理论上最理想的连接顺序为:
\begin{enumerate}[leftmargin=3em]
\item 天线-驻波功率计-天线调谐器-馈线-收发信机
\item 天线-天线调谐器-驻波功率计-馈线-收发信机
\item 天线-馈线-天线调谐器-驻波功率计-收发信机
\item 天线-天线调谐器-馈线-驻波功率计-收发信机
\end{enumerate}
\noindent\textbf{解说:}理论上最理想的连接顺序为\textbf{天线-天线调谐器-驻波功率计-馈线-收发信机}。\\\noindent\textbf{答案:}B




\bigskip


\noindent\textbf{试题:}塔上的天线通过50欧同轴馈线与输出阻抗为50欧的收发信机相连接,在天线电路中串入天线调谐器ATU和通过式驻波功率计M来监测和补偿天线的失配。有四种方案:1、ATU和M均在塔顶,2、ATU和M均在塔底,3、ATU在塔底、M在机房,4、ATU和M均在机房。当ATU调到最佳状态时,各方案按天线系统发射效率由高到低的排序为:
\begin{enumerate}[leftmargin=3em]
\item 方案4最好,方案2、3其次,方案1最差
\item 方案1最好,方案2、3其次,方案4最差
\item 方案3最好,方案4其次,方案2再其次,方案1最差
\item 方案2最好,方案1其次,方案4再其次,方案3最差
\end{enumerate}
\noindent\textbf{解说:}当ATU调到最佳状态时,各方案按天线系统发射效率由高到低的排序为\textbf{方案1最好,方案2、3其次,方案4最差}。\\\noindent\textbf{答案:}B

%%%???
%LK0938.jpg

\bigskip


\noindent\textbf{试题:}下列哪一种导体最适合射频接地使用?
\begin{enumerate}[leftmargin=3em]
\item 双绞线
\item 圆形多股线
\item 镀银软铜丝编织扁带
\item 圆形铜包钢单股线
\end{enumerate}
\noindent\textbf{解说:}\textbf{镀银软铜丝编织扁带}最适合射频接地使用。\\\noindent\textbf{答案:}C



\bigskip


\noindent\textbf{试题:}如果在驻波表上读到了4:1,这意味着?
\begin{enumerate}[leftmargin=3em]
\item 良好的阻抗匹配
\item 天线有4dB的损失
\item 阻抗匹配得不好
\item 天线的增益是4dB
\end{enumerate}
\noindent\textbf{解说:}如果在驻波表上读到了4:1,这意味着\textbf{阻抗匹配得不好}。\\\noindent\textbf{答案:}C



\bigskip


\noindent\textbf{试题:}收发信机天线调谐器(天调)的作用是什么?
\begin{enumerate}[leftmargin=3em]
\item 它可以使一部天线既能作为发射天线,又能作为接收天线
\item 它能够依照发射机当前工作的波段自动连接合适的天线
\item 它将发射机的输出阻抗和天线的输入阻抗进行良好的匹配
\item 它帮助接收机在遇到弱信号时能自动调准频率
\end{enumerate}
\noindent\textbf{解说:}收发信机天线调谐器(天调)的作用是\textbf{它将发射机的输出阻抗和天线的输入阻抗进行良好的匹配}。\\\noindent\textbf{答案:}C



\bigskip


\noindent\textbf{试题:}影响短波电离层传播的主要因素有:
\begin{enumerate}[leftmargin=3em]
\item 当地天气
\item 太阳黑子活动、太阳耀斑活动和地磁活动
\item 发射机功率
\item 地表温度
\end{enumerate}
\noindent\textbf{解说:}影响短波电离层传播的主要因素有\textbf{太阳黑子活动、太阳耀斑活动和地磁活动}。\\\noindent\textbf{答案:}B



\bigskip


\noindent\textbf{试题:}影响短波电离层传播的主要因素有:
\begin{enumerate}[leftmargin=3em]
\item 当地地面气压
\item 接收天线增益
\item 季节和昼夜
\item 发射机功率
\end{enumerate}
\noindent\textbf{解说:}影响短波电离层传播的主要因素有\textbf{季节和昼夜}。\\\noindent\textbf{答案:}C



\bigskip


\noindent\textbf{试题:}影响短波电离层传播的主要因素有:
\begin{enumerate}[leftmargin=3em]
\item 对流层气压
\item 发射机功率
\item 工作频率和通信距离
\item 高空云量
\end{enumerate}
\noindent\textbf{解说:}影响短波电离层传播的主要因素有\textbf{工作频率和通信距离}。\\\noindent\textbf{答案:}C



\bigskip


\noindent\textbf{试题:}对短波电离层传播发生主要影响的各电离层按高度自低到高分别称为:
\begin{enumerate}[leftmargin=3em]
\item D、E、F1、F2
\item F2、F1、E2、E1
\item A、B、C、D
\item F、E2、E1、D
\end{enumerate}
\noindent\textbf{解说:}对短波电离层传播发生主要影响的各电离层按高度自低到高分别称为\textbf{D、E、F1、F2}。\\\noindent\textbf{答案:}A



\bigskip


\noindent\textbf{试题:}各电离层对短波电离层传播所起的主要影响为:
\begin{enumerate}[leftmargin=3em]
\item D、E层可反射电波,F1、F2层不能反射但衰减电波
\item F1、E层可反射电波,D、F2层不能反射但衰减电波
\item F2、D层可反射电波,E、F1层不能反射但衰减电波
\item F2、F1、E层可反射电波,D层不能反射但衰减电波
\end{enumerate}
\noindent\textbf{解说:}\textbf{F2、F1、E层可反射电波,D层不能反射但衰减电波}。\\\noindent\textbf{答案:}D



\bigskip


\noindent\textbf{试题:}太阳黑子活动的平均周期约为:
\begin{enumerate}[leftmargin=3em]
\item 38年
\item 6.7年
\item 11.2年
\item 5.7年
\end{enumerate}
\noindent\textbf{解说:}太阳黑子活动的平均周期约为\textbf{11.2年}。\\
\noindent\textbf{答案:}C
%%% 关于周期:
%%% 日文维基百科:9.5年から12年,英文维基百科:约11年


\bigskip


\noindent\textbf{试题:}太阳黑子活动的强弱用“太阳黑子平均数(SSN)”来描述。一般说来:
\begin{enumerate}[leftmargin=3em]
\item 只有在发生太阳耀斑的情况下SSN 才会影响短波远程通信效果
\item SSN与短波远程通信效果无直接关系
\item SSN小,有利于短波远程通信
\item SSN大,有利于短波远程通信
\end{enumerate}
\noindent\textbf{解说:}一般说来\textbf{SSN大,有利于短波远程通信}。\\\noindent\textbf{答案:}D


\bigskip


\noindent\textbf{试题:}当最高可用频率(MUF)为\SI{20}{\MHz}时,具有较大DX通联机会的业余频段是:
\begin{enumerate}[leftmargin=3em]
\item \SI{18}{\MHz}
\item \SI{21}{\MHz}
\item \SI{24}{\MHz}
\item \SI{14}{\MHz}
\end{enumerate}
\noindent\textbf{解说:}当最高可用频率(MUF)为\SI{20}{\MHz}时,具有较大DX通联机会的业余频段是\SI{18}{\MHz}。\\
\noindent\textbf{答案:}A



\bigskip


\noindent\textbf{试题:}太阳耀斑引起的电离层扰动(SID)对短波通信的影响是:
\begin{enumerate}[leftmargin=3em]
\item 地球黑夜一面受到的影响会超过白昼面
\item 卫星通信受到的影响会超过地面直射波通信
\item 高纬度的传播路径影响会超过低纬度
\item 低频率受到的影响超过高频率
\end{enumerate}
\noindent\textbf{解说:}太阳耀斑引起的电离层扰动(SID)对短波通信的影响是\textbf{低频率受到的影响超过高频率}。\\\noindent\textbf{答案:}D



\bigskip


\noindent\textbf{试题:}HF频段远距离通信主要依靠下列传播方式:
\begin{enumerate}[leftmargin=3em]
\item 电离层反射
\item 中继台网转发
\item 对流层散射传播
\item 业余卫星转发
\end{enumerate}
\noindent\textbf{解说:}HF频段远距离通信主要依靠\textbf{电离层反射}传播。\\\noindent\textbf{答案:}A


\bigskip


\noindent\textbf{试题:}“静寂区”或者“越距”是指:
\begin{enumerate}[leftmargin=3em]
\item HF频段天波和地波都传播不到的中间区域
\item VHF/UHF频段超过视距电波传播不到的区域
\item VHF/UHF频段视距范围内但受障碍物阻挡电波传播不到的区域
\item 卫星通信中覆盖区以外电波传播不到的区域
\end{enumerate}
\noindent\textbf{解说:}“静寂区”或者“越距”是指\textbf{HF频段天波和地波都传播不到的中间区域}。\\\noindent\textbf{答案:}A



\bigskip


\noindent\textbf{试题:}大气层中的哪一部分使得无线电信号可以在全世界范围内传播?
\begin{enumerate}[leftmargin=3em]
\item 对流层
\item 电离层
\item 磁层
\item 平流层
\end{enumerate}
\noindent\textbf{解说:}大气层中的\textbf{电离层}使得无线电信号可以在全世界范围内传播。\\\noindent\textbf{答案:}B





% 第三章 无线电系统理论 结束

% 第四章 无线电的安全性 开始
\chapter{无线电的安全性}

\newpage

\bigskip



\noindent\textbf{试题:}用甲乙两块电压表检查一节新干电池两端电压,均测得1.5伏。检查一节旧干电池,读数分别为1.2伏和1.3伏。正确的结论是:
\begin{enumerate}[leftmargin=3em]
	\item 甲电压表的内阻小于乙电压表
	\item 甲电压表的内阻大于乙电压表
	\item 甲电压表的精度低于乙电压表
	\item 甲电压表的精度高于乙电压表
\end{enumerate}
\noindent\textbf{解说:}正确的结论是\textbf{甲电压表的内阻小于乙电压表}。\\\noindent\textbf{答案:}A


\bigskip



\noindent\textbf{试题:}用万用电表的交流电压档测量简单正弦交流电压,得到的读数是该电压的:
\begin{enumerate}[leftmargin=3em]
	\item 峰-峰值
	\item 有效值
	\item 平均值
	\item 最大值
\end{enumerate}
\noindent\textbf{解说:}用万用电表的交流电压档测量简单正弦交流电压,得到的读数是该电压的\textbf{有效值}。\\\noindent\textbf{答案:}B


\bigskip



\noindent\textbf{试题:}万用电表设在电阻挡,将表笔分别接到尚未连接BALUN和电缆等其它部件的每臂长度均为四分之一波长的偶极天线的中心馈电点两端,读数应为:
\begin{enumerate}[leftmargin=3em]
	\item 读数受振子周围物体的影响
	\item 50欧姆
	\item 读数受表笔连线造成的驻波比的影响
	\item 无穷大
\end{enumerate}
\noindent\textbf{解说:}读数应为\textbf{无穷大}。\\\noindent\textbf{答案:}D


\bigskip



\noindent\textbf{试题:}万用电表设在电阻挡,将表笔分别接到一条终端短路的任意长度的理想50欧同轴电缆的中心导体和屏蔽层,读数应为:
\begin{enumerate}[leftmargin=3em]
	\item 50欧姆
	\item 0欧姆
	\item 读数与电缆长度有关
	\item 无穷大
\end{enumerate}
\noindent\textbf{解说:}读数应为\textbf{0欧姆}。\\\noindent\textbf{答案:}B


\bigskip



\noindent\textbf{试题:}无线电通信和测试设备、电视设备和音频设备常用的传输接口标准阻抗分别为:
\begin{enumerate}[leftmargin=3em]
	\item 50欧、50欧和75欧
	\item 50欧、75欧和600欧
	\item 75欧、50欧和16欧
	\item 50欧、600欧和75欧
\end{enumerate}
\noindent\textbf{解说:}无线电通信和测试设备、电视设备和音频设备常用的传输接口标准阻抗分别为\textbf{50欧、75欧和600欧}。\\\noindent\textbf{答案:}B


\bigskip



\noindent\textbf{试题:}在使用欧姆表测量一个电路的某两个端点间电阻值之前,要先注意什么?
\begin{enumerate}[leftmargin=3em]
	\item 确保电路已接通了正常工作时需要的电源
	\item 确保电路没有连接至电源
	\item 确保电路已经正常接地
	\item 确保电路在正常的频率下工作
\end{enumerate}
\noindent\textbf{解说:}在使用欧姆表测量一个电路的某两个端点间电阻值之前,要先\textbf{确保电路没有连接至电源}。\\\noindent\textbf{答案:}B


\bigskip



\noindent\textbf{试题:}防雷装置的作用是防止雷电危害。传统防雷装置的主要组成部分是:
\begin{enumerate}[leftmargin=3em]
	\item 接闪器(避雷针)、引下线、接地体
	\item 天线、限流器、地线
	\item 避雷针、高压指示灯、过流保护器
	\item 避雷针、过压保护器、熔丝
\end{enumerate}
\noindent\textbf{解说:}传统防雷装置的主要组成部分是\textbf{接闪器(避雷针)、引下线、接地体}。\\\noindent\textbf{答案:}A


\bigskip



\noindent\textbf{试题:}防雷接地的作用是:
\begin{enumerate}[leftmargin=3em]
	\item 用接闪器感应到的雷电高压启动过压保护电路
	\item 有效地阻断接闪器引入的雷击电流使其不致流入大地
	\item 把接闪器引入的雷击电流有效地泄入大地
	\item 当接闪器引入雷击电流时迅速烧断熔丝,阻断其流动
\end{enumerate}
\noindent\textbf{解说:}防雷接地的作用是\textbf{把接闪器引入的雷击电流有效地泄入大地}。\\\noindent\textbf{答案:}C


\bigskip



\noindent\textbf{试题:}关于对防雷接地基本要求的正确说法:
\begin{enumerate}[leftmargin=3em]
	\item 要有单独的接地体,接地电阻要小,接闪器到接地体之间的引下线应尽量短而粗
	\item 接闪器到接地体之间的引下线平时没有电流流过,采用直径0.5毫米的导线为好
	\item 交流电网的“零线”在配电系统中已经接地,因此可代替防雷接地体及其引下线
	\item 可以利用与埋地金属管线相连的自来水管作为接地体
\end{enumerate}
\noindent\textbf{解说:}关于对防雷接地基本要求的正确说法是\textbf{要有单独的接地体,接地电阻要小,接闪器到接地体之间的引下线应尽量短而粗}。\\\noindent\textbf{答案:}A


\bigskip



\noindent\textbf{试题:}单支避雷针的保护范围大致有多大:
\begin{enumerate}[leftmargin=3em]
	\item 避雷针周围水平方圆30米内的任何物体
	\item 避雷针周围所有比避雷针低的空间
	\item 以避雷针为顶点的45度圆锥体以内空间
	\item 以避雷针为顶点、避雷针高度为半径的半球体以内空间
\end{enumerate}
\noindent\textbf{解说:}单支避雷针的保护范围大致为\textbf{以避雷针为顶点的45度圆锥体以内空间}。\\\noindent\textbf{答案:}C


\bigskip



\noindent\textbf{试题:}安全电压是指不致使人直接致死或致残的电压。标准国家标准GB3805-83《安全电压》,一般环境条件下允许持续接触的“安全特低电压”为:
\begin{enumerate}[leftmargin=3em]
	\item 48V
	\item 24V
	\item 72V
	\item 6V
\end{enumerate}
\noindent\textbf{解说:}国家标准GB3805-83《安全电压》规定:一般环境条件下允许持续接触的“安全特低电压”上限为\textbf{24V}。\\\noindent\textbf{答案:}B
%%%


\bigskip



\noindent\textbf{试题:}触及裸露的射频导线时,与触及相同电压的直流或50Hz交流导线相比,对人身安全危险的大致差别为:
\begin{enumerate}[leftmargin=3em]
	\item 致死危险性下降,但皮肤容易灼伤
	\item 射频电流对人体没有危险
	\item 更容易出现呼吸或心脏麻痹、更易致死
	\item 触及不同频率、相同电压的导线时,对人体的危害没有差别
\end{enumerate}
\noindent\textbf{解说:}触及裸露的射频导线时,与触及相同电压的直流或50Hz交流导线相比,对人身安全危险的大致差别是\textbf{致死危险性下降,但皮肤容易灼伤}。\\\noindent\textbf{答案:}A


\bigskip



\noindent\textbf{试题:}如遇设备、电线或者电源引起失火,正确的处置为:
\begin{enumerate}[leftmargin=3em]
	\item 立即切断电源,使用二氧化碳灭火器灭火
	\item 立即切断电源,用水灭火
	\item 立即切断电源,使用化学泡沫灭火器灭火
	\item 迅速起身逃离火场
\end{enumerate}
\noindent\textbf{解说:}如遇设备、电线或者电源引起失火,正确的处置方法是\textbf{立即切断电源,使用二氧化碳灭火器灭火}。\\\noindent\textbf{答案:}A


\bigskip



\noindent\textbf{试题:}必须带电检修由市电供电的无线电设备时,应做到:
\begin{enumerate}[leftmargin=3em]
	\item 只要设备外壳良好接地,双脚是否与地绝缘、双手是否同时操作都没有关系
	\item 双脚与地绝缘,单手操作,另一只手不触摸机壳等任何与电路设备有关的金属物品
	\item 只要确保设备外壳与地绝缘,双脚是否与地绝缘、双手是否同时操作都没有关系
	\item 双脚与地绝缘,单手操作,另一只手通过触摸机壳或水管等途径良好接地
\end{enumerate}
\noindent\textbf{解说:}必须带电检修由市电供电的无线电设备时,应做到\textbf{双脚与地绝缘,单手操作,另一只手不触摸机壳等任何与电路设备有关的金属物品}。\\\noindent\textbf{答案:}B


\bigskip



\noindent\textbf{试题:}两手分别接触电压有效值相同但频率不同的电路两端时,对人体生命安全威胁由大到小的排序为:
\begin{enumerate}[leftmargin=3em]
	\item 工频交流电、HF射频交流电、UHF射频交流电
	\item HF射频交流电、工频交流电、UHF射频交流电
	\item UHF射频交流电、HF射频交流电、工频交流电
	\item HF射频交流电、UHF射频交流电、工频交流电
\end{enumerate}
\noindent\textbf{解说:}当两手分别接触电压有效值相同但频率不同的电路两端时,对人体生命安全威胁由大到小的排序为:\textbf{工频交流电、HF射频交流电、UHF射频交流电}。\\\noindent\textbf{答案:}A


\bigskip



\noindent\textbf{试题:}电路中的保险丝起到什么作用?
\begin{enumerate}[leftmargin=3em]
	\item 过载时切断电路
	\item 以上三项全部正确
	\item 防止电源纹波损害电路
	\item 限制电流,防止触电
\end{enumerate}
\noindent\textbf{解说:}电路中的保险丝起到\textbf{过载时切断电路}的作用。\\\noindent\textbf{答案:}A


\bigskip



\noindent\textbf{试题:}为什么在需要安装5安培保险丝的地方安装一个20安培的保险丝是不可取的?
\begin{enumerate}[leftmargin=3em]
	\item 过大的电流可能导致火灾
	\item 其他三项全部正确
	\item 因为20安培的保险丝电流更大,所以它更容易熔断
	\item 电源的纹波会显著增大
\end{enumerate}
\noindent\textbf{解说:}电路中的保险丝起到过载时切断电路的作用。把较小熔断电流的保险丝擅自换成大熔断电流的保险丝是错误的,这样做有可能造成\textbf{在过大的电流时不能切断电路,甚至可能导致火灾}。\\\noindent\textbf{答案:}A


\bigskip



\noindent\textbf{试题:}防止设备外壳带电危险的措施包括:
\begin{enumerate}[leftmargin=3em]
	\item 安装漏电保护断路器
	\item 将所有的交流供电设备全部连接至一个安全地线
	\item 其他三项全部正确
	\item 所有使用交流供电的设备的电源线都使用带有单独保护地线端的三线插头
\end{enumerate}
\noindent\textbf{解说:}防止设备外壳带电危险的措施包括:将所有的交流供电设备全部连接至一个安全地线、安装漏电保护断路器、所有使用交流供电的设备的电源线都使用带有单独保护地线端的三线插头等措施。因此\textbf{其他三项全部正确}。\\\noindent\textbf{答案:}C


\bigskip



\noindent\textbf{试题:}在为同轴电缆馈线安装避雷器时应当注意什么?
\begin{enumerate}[leftmargin=3em]
	\item 将所有避雷器的地线接到同一个金属板上,然后将这个金属板接到室外的接地极
	\item 将每一个避雷器单独引出接地线,并且将它们都连接至电台的地线
	\item 要在每一个避雷器处两端并联一个开关,以便在使用大功率输出的时候可以将避雷器旁路掉
	\item 在每一个避雷器的接地线上安装开关,以防止射频过载损伤避雷器
\end{enumerate}
\noindent\textbf{解说:}在为同轴电缆馈线安装避雷器时应当注意要\textbf{将所有避雷器的地线接到同一个金属板上,然后将这个金属板接到室外的接地极}。\\\noindent\textbf{答案:}A


\bigskip



\noindent\textbf{试题:}常规的12伏酸铅蓄电池通常有什么潜在的危险?
\begin{enumerate}[leftmargin=3em]
	\item 如果通风不良,有爆炸风险的气体会聚集
	\item 有高电压,存在触电的风险
	\item 长时间不使用可能会引起自燃
	\item 它会释放臭氧,进而污染大气层
\end{enumerate}
\noindent\textbf{解说:}\textbf{如果通风不良,有爆炸风险的气体会聚集}是常规的12伏酸铅蓄电池潜在的危险。\\\noindent\textbf{答案:}A


\bigskip



\noindent\textbf{试题:}以下哪一项是天线铁塔安装防雷接地的良好措施?
\begin{enumerate}[leftmargin=3em]
	\item 为接地系统做好防水处理
	\item 接地连接要尽可能地短而直接
	\item 打入地面的接地极要尽量远离天线塔
	\item 确保接地线的所有需要弯折的地方都是干净的直角弯
\end{enumerate}
\noindent\textbf{解说:}\textbf{接地连接要尽可能地短而直接}是天线铁塔安装防雷接地的良好措施。\\\noindent\textbf{答案:}B


\bigskip



\noindent\textbf{试题:}设备电源拔掉电源线以后,检修时还有什么安全风险?
\begin{enumerate}[leftmargin=3em]
	\item 充满高电压的电容器可能造成电击
	\item 地磁场可能在变压器中激起感应电流导致电源损坏
	\item 静电可能损坏接地系统
	\item 打开电源外壳可能引起保险丝烧断
\end{enumerate}
\noindent\textbf{解说:}检修时应注意,设备电源拔掉电源线以后,\textbf{充满高电压的电容器可能造成电击}。\\\noindent\textbf{答案:}A


\bigskip



\noindent\textbf{试题:}自制一台由220伏交流供电的设备,推荐采用的安全措施是:
\begin{enumerate}[leftmargin=3em]
	\item 在交流供电入口处并联安装一个电容
	\item 在交流供电入口处并联安装一个交流电压表
	\item 在交流供电入口处串联安装一个电感
	\item 交流电源入口火线端串联安装保险丝
\end{enumerate}
\noindent\textbf{解说:}在自制由220伏交流供电的设备时,推荐采用的安全措施是在\textbf{交流电源入口火线端串联安装保险丝}。\\\noindent\textbf{答案:}D



% 第四章 无线电的安全性 结束

% 第五章 避免和解决射频干扰 开始

\chapter{避免和解决射频干扰}

\newpage

\noindent\textbf{试题:}我国业余电台应该遵守的关于电磁辐射污染的具体管理规定文件为:
\begin{enumerate}[leftmargin=3em]
\item 美国FCC《射频电磁场人员暴露准则的测评方法规》
\item 国家标准GB/T 51391《通信工程建设环境保护技术标准》
\item 《业余电台管理办法》
\item 国际非电离辐射防护委员会《限制时变电场和磁场暴露的导则》
\end{enumerate}
\noindent\textbf{解说:}我国业余电台应该遵守的关于电磁辐射污染的具体管理规定文件为\textbf{国家标准GB/T 51391《通信工程建设环境保护技术标准》}。\\\textbf{答案:}B



\bigskip


\noindent\textbf{试题:}按照我国国家标准GB/T 51391《通信工程建设环境保护技术标准》,向没有屏蔽空间发射电磁场可以免于管理的有:
\begin{enumerate}[leftmargin=3em]
\item 发射频率30\unit{\MHz}以下的所有业余电台
\item 发射频率30\unit{\MHz}以上的所有业余电台
\item 等效辐射功率在0.1-3\unit{\MHz}不大于300W、在3\unit{\MHz}-300\unit{\GHz}不大于100瓦的无线通信设施(设备)
\item 所有业余电台
\end{enumerate}
\noindent\textbf{解说:}按照我国国家标准GB/T 51391《通信工程建设环境保护技术标准》,向没有屏蔽空间发射电磁场可以免于管理的有\textbf{等效辐射功率在0.1-3\unit{\MHz}不大于300W、在3\unit{\MHz}-300\unit{\GHz}不大于100瓦的无线通信设施(设备)}。\\\textbf{答案:}C



\bigskip


%\noindent\textbf{试题:}按照我国国家标准《电磁辐射防护规定》,凡其功率大于豁免水平(3\unit{\MHz}以上等效辐射功率100瓦)的一切电磁波辐射体的所有者,必须:
%\begin{enumerate}[leftmargin=3em]
%\item 业余电台可不受环境保护部门的监督管理
%\item 向所在地区的无线电管理机构交纳环境电磁辐射监测报告
%\item 向所在地区的环境保护部门缴纳环境保护费
%\item 向所在地区的环境保护部门申报、登记,并接受监督
%\end{enumerate}
%\noindent\textbf{解说:}按照我国国家标准《电磁辐射防护规定》,凡其功率大于豁免水平(3\unit{\MHz}以上等效辐射功率100瓦)的一切电磁波辐射体的所有者,必须\textbf{向所在地区的环境保护部门申报、登记,并接受监督}。\\\textbf{答案:}D
%
%
%\bigskip


%\noindent\textbf{试题:}按照我国国家标准《电磁辐射防护规定》,负责对超过豁免水平(3\unit{\MHz}以上等效辐射功率100瓦)电磁波辐射体所在的场所以及周围环境的电磁辐射水平进行监测的是:
%\begin{enumerate}[leftmargin=3em]
%\item 所在地区无线电管理机构
%\item 所在地区业余无线电协会
%\item 所在地区环境保护部门
%\item 其拥有者
%\end{enumerate}
%\noindent\textbf{解说:}按照我国国家标准《电磁辐射防护规定》,\textbf{其拥有者}负责对超过豁免水平(3\unit{\MHz}以上等效辐射功率100瓦)电磁波辐射体所在的场所以及周围环境的电磁辐射水平进行监测。\\\textbf{答案:}D
%
%
%
%\bigskip


%\noindent\textbf{试题:}按照我国国家标准《电磁辐射防护规定》,当监测到超过豁免水平的电磁辐射体使环境电磁辐射水平超过规定的限值时,必须:
%\begin{enumerate}[leftmargin=3em]
%\item 立即向无线电管理机构报告
%\item 立即向当地城管机构报告
%\item 尽快采取措施降低辐射水平,同时向环境保护部门报告
%\item 立即到环境保护部门缴纳电磁污染治理费
%\end{enumerate}
%\noindent\textbf{解说:}按照我国国家标准《电磁辐射防护规定》,当监测到超过豁免水平的电磁辐射体使环境电磁辐射水平超过规定的限值时,必须\textbf{尽快采取措施降低辐射水平,同时向环境保护部门报告}。\\\textbf{答案:}C
%
%
%\bigskip


%\noindent\textbf{试题:}按照我国国家标准《电磁辐射防护规定》,对超过豁免水平的电磁辐射体的环境电磁辐射水平监测应在下列地点进行:
%\begin{enumerate}[leftmargin=3em]
%\item 在发射机射频输出端口进行测量
%\item 在距辐射体天线2000米以内选择任意位置测量
%\item 在距辐射体天线100米以内最小辐射方向上选点测量
%\item 在距辐射体天线2000米以内最大辐射方向上选点测量
%\end{enumerate}
%\noindent\textbf{解说:}按照我国国家标准《电磁辐射防护规定》,对超过豁免水平的电磁辐射体的环境电磁辐射水平监测应\textbf{在距辐射体天线2000米以内最大辐射方向上选点测量}。\\\textbf{答案:}D
%
%
%
%\bigskip


%\noindent\textbf{试题:}我国国家标准《电磁辐射防护规定》所规定的电磁辐射防护限值的公众照射基本限值,其基本计量方法规是:
%\begin{enumerate}[leftmargin=3em]
%\item 一天24小时内任意6分钟内全身平均的比吸收率(SAR)应小于每公斤体重限值
%\item 任何时刻的全身平均的比吸收率(SAR)都应小于每公斤体重限值
%\item 任何时刻的瞬时辐射场强都应小于与频率无关的固定限值
%\item 任何时刻的瞬时辐射场强都应小于与频率有关的限值
%\end{enumerate}
%
%\bigskip


\noindent\textbf{试题:}我国国家标准《电磁环境控制限值》规定公众暴露控制限值中,对环境电磁辐射场强任意连续6分钟内的方均根值要求最严格的频率范围为:
\begin{enumerate}[leftmargin=3em]
\item 300\unit{\kHz}-30\unit{\MHz}
\item 3\unit{\MHz}- 30\unit{\MHz}
\item 30\unit{\MHz}- 3\unit{\GHz}
\item 1200\unit{\MHz}-30\unit{\GHz}
\end{enumerate}
\noindent\textbf{解说:}我国国家标准《电磁环境控制限值》规定公众暴露控制限值中,对环境电磁辐射场强任意连续6分钟内的方均根值要求最严格的频率范围为\textbf{30\unit{\MHz}- 3\unit{\GHz}}。\\\textbf{答案:}C


\bigskip


%%%???电磁辐射防护规定 未修改
\noindent\textbf{试题:}为什么电磁辐射防护规定国家标准中的照射限值随着频率的变化而不同?
\begin{enumerate}[leftmargin=3em]
\item 较低频率的无线电波相对高频率的无线电波拥有更高的能量
\item 人体会对某些特定频率的电磁波吸收量更大
\item 在自然中高频电磁波不常见
\item 较低频率的无线电波不会穿透人体
\end{enumerate}
\noindent\textbf{解说:}因为\textbf{人体会对某些特定频率的电磁波吸收量更大}。\\\textbf{答案:}B

\bigskip


\noindent\textbf{试题:}要防止HF发射机的杂散发射干扰天线附近的VHF电视机,应该发射机和天线之间串联:
\begin{enumerate}[leftmargin=3em]
\item 截止频率为300\unit{\MHz}左右的低通滤波器
\item 截止频率为30\unit{\MHz}左右的高通滤波器
\item 中心频率为30\unit{\MHz}左右的带通滤波器
\item 截止频率为30\unit{\MHz}左右的低通滤波器
\end{enumerate}
\noindent\textbf{解说:}要防止HF发射机的杂散发射干扰天线附近的VHF电视机,应该发射机和天线之间串联\textbf{截止频率为30\unit{\MHz}左右的低通滤波器}。\\\textbf{答案:}D


%%%???

\bigskip


\noindent\textbf{试题:}假设中继台的收、发信机共用天线,上下行频率分别为F1和F2。要防止中继台发射机对接收机产生干扰,应该对中继台设备采取下列措施:
\begin{enumerate}[leftmargin=3em]
\item 在发信机与天线间串联中心频率为F1的带阻滤波器,在收信机与天线间串接中心频率为F1的带阻滤波器
\item 在发信机与天线间串联中心频率为F2的带阻滤波器,在收信机与天线间串接中心频率为F2的带阻滤波器
\item 在发信机与天线间串联中心频率为F1的带阻滤波器,在收信机与天线间串接中心频率为F2的带阻滤波器
\item 在发信机与天线间串联中心频率为F1的带通滤波器,在收信机与天线间串接中心频率为F2的带通滤波器
\end{enumerate}
\noindent\textbf{解说:}应该\textbf{在中继台设备的发信机与天线间串联中心频率为F1的带阻滤波器,在收信机与天线间串接中心频率为F2的带阻滤波器}。\\\noindent\textbf{答案:}C


\bigskip


\noindent\textbf{试题:}要防止业余HF发射机的杂散发射干扰天线附近的电话机,应该在电话机和电话线之间之间串联:
\begin{enumerate}[leftmargin=3em]
\item 截止频率约为3\unit{\kHz}的高通滤波器
\item 截止频率约为3\unit{\kHz}的带阻滤波器
\item 截止频率不高于1\unit{\MHz}的低通滤波器
\item 中心频率约为3\unit{\kHz}的带通滤波器
\end{enumerate}
\noindent\textbf{解说:}要防止业余HF发射机的杂散发射干扰天线附近的电话机,应该在电话机和电话线之间之间串联\textbf{截止频率不高于1\unit{\MHz}的低通滤波器}。\\\noindent\textbf{答案:}C



\bigskip


\noindent\textbf{试题:}A、B两部HF业余电台相距很近,分别工作在A、B两个频段。为减少B电台受到来自A电台的干扰,可以在B电台与天线之间串联:
\begin{enumerate}[leftmargin=3em]
\item 中心频率为A的带阻滤波器
\item 截止频率为B的高通滤波器
\item 截止频率为A的高通滤波器
\item 中心频率为A的带通滤波器
\end{enumerate}
\noindent\textbf{解说:}可以在B电台与天线之间串联\textbf{中心频率为A的带阻滤波器}。\\\noindent\textbf{答案:}A


\bigskip


\noindent\textbf{试题:}为了减少发射设备的谐波干扰近在咫尺的接收机,可以在发射设备和天线之间串联一个LC低通滤波器。正确的说法是:
\begin{enumerate}[leftmargin=3em]
\item 滤波器的阶数越低,抑制倍频干扰的效果越好
\item 滤波器的阶数越高,抑制倍频干扰的效果越好
\item 滤波器的阶数越高,损耗的功率越小
\item 滤波器的阶数越高,耐受的功率越大
\end{enumerate}
\noindent\textbf{解说:}正确的说法是:\textbf{滤波器的阶数越高,抑制倍频干扰的效果越好}。\\\noindent\textbf{答案:}B



\bigskip


\noindent\textbf{试题:}架设业余中继台前应确定台址附近没有能与中继台下行频率形成三阶互调的发射台。如果中继台的上、下行频率分别为\(fR\)和\(fT\),可能造成这种三阶互调的干扰频率\(fX\)是:
\begin{enumerate}[leftmargin=3em]
\item \(fT - fR\) 或 \(fT + fR\)
\item \(2fT - fR\) 或 \((fT + fR ) / 2\)
\item \(2fT\) 或 \(2fR\)
\item \(2(fT - fR)\) 或 \(2(fT + fR )\) % 原文:2(fT - fR) 或 2(fT + fR )
\end{enumerate}
\noindent\textbf{解说:}可能造成这种三阶互调的干扰频率\(fX\)是\(2fT - fR\) 或 \((fT + fR ) / 2\)。\\
\noindent\textbf{答案:}B



\bigskip


\noindent\textbf{试题:}如果别人报告说你的发射干扰了相邻频率的通信,此时你应当做的是:
\begin{enumerate}[leftmargin=3em]
\item 加大发射功率
\item 换用另一种调制模式工作
\item 将这种情况通知你的设备制造商
\item 检查发射机的频率指示是否准确、发射机的杂散发射指标是否合格
\end{enumerate}
\noindent\textbf{解说:}如果别人报告说你的发射干扰了相邻频率的通信,此时你应当做的是\textbf{检查发射机的频率指示是否准确、发射机的杂散发射指标是否合格}。\\\noindent\textbf{答案:}D



\bigskip


\noindent\textbf{试题:}滤去杂散发射的滤波器应该安装在什么地方?
\begin{enumerate}[leftmargin=3em]
\item 发信机和天线之间
\item 话筒上
\item 收信机和发信机之间
\item 电台的电源处
\end{enumerate}
\noindent\textbf{解说:}滤去杂散发射的滤波器应该安装在\textbf{发信机和天线之间}。\\\noindent\textbf{答案:}A


\bigskip


\noindent\textbf{试题:}在解决电视接收机被附近的144\unit{\MHz}业余电台的过载干扰问题的时候,应当先尝试什么措施?
\begin{enumerate}[leftmargin=3em]
\item 在电视接收机的天线端子前安装144\unit{\MHz}带阻滤波器
\item 在业余电台的射频输出端安装144\unit{\MHz}带阻滤波器
\item 在业余电台的射频输出端安装144\unit{\MHz}带通滤波器
\item 在电视接收机的天线端子前安装144\unit{\MHz}带通波器
\end{enumerate}
\noindent\textbf{解说:}应当先尝试\textbf{在电视接收机的天线端子前安装144\unit{\MHz}带阻滤波器}。\\\noindent\textbf{答案:}A



\bigskip


\noindent\textbf{试题:}在汽车上安装的移动电台中能听到的随着引擎转速变化的高频啸叫声的来源是?
\begin{enumerate}[leftmargin=3em]
\item 火花塞系统
\item 防抱死刹车装置的控制器
\item 发电机
\item 电动油泵
\end{enumerate}
\noindent\textbf{解说:}在汽车上安装的移动电台中能听到的随着引擎转速变化的高频啸叫声的来源是\textbf{发电机}。\\\noindent\textbf{答案:}C



\bigskip


\noindent\textbf{试题:}移动车载台的直流电源负极应当接在哪里?
\begin{enumerate}[leftmargin=3em]
\item 连接在固定住电台的挂置架上
\item 连接在天线座上
\item 连接在电池的负极或发动机的接地带
\item 可以连接在汽车的任意的金属部分
\end{enumerate}
\noindent\textbf{解说:}移动车载台的直流电源负极应当接在连接在\textbf{电池的负极或发动机的接地带}。\\\noindent\textbf{答案:}C



\bigskip


\noindent\textbf{试题:}下列哪一项可以有效减小火花塞干扰?
\begin{enumerate}[leftmargin=3em]
\item 将频率稍稍偏离一些
\item 打开电台的抑噪(NB)功能
\item 降低静噪(SQL)阀值
\item 调节电台的RIT旋钮
\end{enumerate}
\noindent\textbf{解说:}\textbf{打开电台的抑噪(NB)功能}可以有效减小火花塞干扰。\\\noindent\textbf{答案:}B



\bigskip


\noindent\textbf{试题:}一般来说,如果要解决发射机对附近有线电话的干扰,最先做的应当是:
\begin{enumerate}[leftmargin=3em]
\item 在有线电话进线处安装射频滤波器
\item 在发射机射频输出端安装高通滤波器
\item 改善发射机的接地情况
\item 在发射机射频输出端安装低通滤波器
\end{enumerate}
\noindent\textbf{解说:}一般来说,如果要解决发射机对附近有线电话的干扰,最先做的应当是\textbf{在有线电话进线处安装射频滤波器}。\\\noindent\textbf{答案:}A



\bigskip


\noindent\textbf{试题:}下列哪一种方法可以用来定位无线电噪音源或者恶意干扰源?
\begin{enumerate}[leftmargin=3em]
\item 回波定位
\item 相位锁定
\item 无线电测向
\item 多普勒雷达
\end{enumerate}
\noindent\textbf{解说:}\textbf{无线电测向}可以用来定位无线电噪音源或者恶意干扰源。\\\noindent\textbf{答案:}C


% 第五章 避免和解决射频干扰 结束

%所有附录
\chapter{附录}

\section{附录1 字母解释法}

\begin{tabular}{cc}%
\begin{tabular}[t]{|c|l|}
	\hline
	\textbf{字母} & \textbf{单词} \\
	\hline
	A & Alfa \\
	\hline
	B & Bravo \\
	\hline
	C & Charlie \\
	\hline
	D & Delta \\
	\hline
	E & Echo \\
	\hline
	F & Foxtrot \\
	\hline
	G & Golf \\
	\hline
	H & Hotel \\
	\hline
	I & India \\
	\hline
	J & Juliett \\
	\hline
	K & Kilo \\
	\hline
	L & Lima \\
	\hline
	M & Mike \\
	\hline
	N & November \\
	\hline
	O & Oscar \\
	\hline
	P & Papa \\
	\hline
	Q & Quebec \\
	\hline
	R & Romeo \\
	\hline
	S & Sierra \\
	\hline
	T & Tango \\
	\hline
	U & Uniform \\
	\hline
	V & Victor \\
	\hline
	W & Whiskey \\
	\hline
	X & X-ray \\
	\hline
	Y & Yankee \\
	\hline
	Z & Zulu \\
	\hline
\end{tabular} &
\begin{tabular}[t]{|c|l|}
	\hline
	\textbf{字母} & \textbf{单词} \\
	\hline
	0 & Zero \\
	\hline
	1 & One \\
	\hline
	2 & Two \\
	\hline
	3 & Three \\
	\hline
	4 & Four \\
	\hline
	5 & Five \\
	\hline
	6 & Six \\
	\hline
	7 & Seven \\
	\hline
	8 & Eight \\
	\hline
	9 & Nine \\
	\hline
\end{tabular} \tabularnewline
\end{tabular}

\newpage

\section{附录2 业余通信中常用的Q简语}

\begin{tabular}{|l|l|l|}
	\hline
	    & \textbf{提问} & \textbf{回答或建议} \\
	\hline
	QRL & 你正忙着吗 & 我正忙着 \\
	\hline
	QRM & 你遇到他台干扰吗 & 我遇到他台干扰 \\
	\hline
	QRN & 你遇到天电干扰吗 & 我遇到天电干扰 \\
	\hline
	QRO & 要我增加功率吗 & \\
	\hline
	QRP & 要我减小功率吗 & \\
	\hline
	QRQ & 要我加快发送速度吗 & 请加快发送速度 \\
	\hline
	QRS & 要我减慢发送速度吗 & 请减慢发送速度 \\
	\hline
	QRT & 要我停止发送吗 & 请停止发送 \\
	\hline
	QRU & 你和我还有事吗 & 我和你无事了 \\
	\hline
	QRV & 你是否已准备好 & 我已准备好 \\
	\hline
	QRZ & 谁在呼叫我 & \\
	\hline
	QSA & 我的信号强度如何 & 你的信号强度为 × 级(1-5 级) \\
	\hline
	QSB & 我的信号有衰落吗 & 你的信号有衰落 \\
	\hline
	QSD & 我发报的手法有毛病吗 & 你发报的手法有毛病 \\
	\hline
	QSK & 能在你的信号间隙中接收吗 & 我在发射的信号间隙中接收 \\
	& (即 QSK 插入方式)    & (即 QSK 插入方式) \\
	\hline
	QSL & 你能给我收据(或 QSL 卡片)吗 & 我给你收据(QSL 卡片)、我已收妥 \\
	\hline
	QSO & 你能直接和 ××× 电台通信吗 & 我能直接和 ××× 电台通信 \\
	\hline
	QSP & 你能传信到 ××× 电台吗 & 我能传信到 ××× 电台 \\
	\hline
	QSX & 你将在 nnnn KHz(或 MHz)频率 & 我将在 nnnn KHz(或 MHz)频率 \\
	    & 守听 ××× 电台吗 & 守听 ××× 电台 \\
	\hline
	QSY & 要我将频率改到 nnnn 频率吗 & 请将频率改到 nnnn 频率 \\
	\hline
	QTH & 你的电台位置在哪里 & 我的电台位置是 ×××× \\
	\hline
\end{tabular}

\newpage



\section{附录3 业余通信常用缩语表}

\newpage

\section{附录4 发射类别的表示方法}

\begin{tabular}{|c|c|}
	\hline
	\multicolumn{2}{|c|}{\textbf{主载波类型(第一个符号)}} \\
	\hline
	\textbf{符号} & \textbf{描述} \\
	\hline
	A & 双边带 \\
	\hline
	F & 调频 \\
	\hline
	G & 调相 \\
	\hline
	J & 单边带、抑制载波 \\
	\hline
\end{tabular}

\bigskip

\begin{tabular}{|c|c|}
	\hline
	\multicolumn{2}{|c|}{\textbf{调制主载波的信号的性质(第二个符号)}} \\
	\hline
	\textbf{符号} & \textbf{描述} \\
	\hline
	1 & 不用调制副载波但包含量化或数字信息的单个通路 \\
	\hline
	2 & 利用调制副载波且包含量化或数字信息的单个通路 \\
	\hline
	3 & 包含模拟信息的单个通路 \\
	\hline
\end{tabular}

\bigskip

\begin{tabular}{|c|c|}
	\hline
	\multicolumn{2}{|c|}{\textbf{被发送信息类型(第三个符号)}} \\
	\hline
	\textbf{符号} & \textbf{描述} \\
	\hline
	A & 电报──用于人工收听 \\
	\hline
	B & 电报──用于自动接收 \\
	\hline
	E & 电话(包括声音广播) \\
	\hline
	F & 电视(视频) \\
	\hline
\end{tabular}

\newpage

\section{附录5 计算公式}

\subsection{真值表}

\begin{tabular}{cc}%
\begin{tabular}{|c|c|c|}
	\hline
	\multicolumn{3}{|c|}{\textbf{与门(AND)}} \\
	\hline
	\multicolumn{2}{|c|}{\textbf{输入}} & \textbf{输出} \\
	\hline
	A & B & A AND B \\
	\hline
	0 & 0 & 0 \\
	\hline
	0 & 1 & 0 \\
	\hline
	1 & 0 & 0 \\
	\hline
	1 & 1 & 1 \\
	\hline
\end{tabular} &
\begin{tabular}{|c|c|c|}
	\hline
	\multicolumn{3}{|c|}{\textbf{与非门(NAND)}} \\
	\hline
	\multicolumn{2}{|c|}{\textbf{输入}} & \textbf{输出} \\
	\hline
	A & B & A NAND B \\
	\hline
	0 & 0 & 1 \\
	\hline
	0 & 1 & 1 \\
	\hline
	1 & 0 & 1 \\
	\hline
	1 & 1 & 0 \\
	\hline
\end{tabular} \tabularnewline
\end{tabular}

\bigskip

\begin{tabular}{cc}%
\begin{tabular}{|c|c|c|}
	\hline
	\multicolumn{3}{|c|}{\textbf{或门(OR)}} \\
	\hline
	\multicolumn{2}{|c|}{\textbf{输入}} & \textbf{输出} \\
	\hline
	A & B & A OR B \\
	\hline
	0 & 0 & 0 \\
	\hline
	0 & 1 & 1 \\
	\hline
	1 & 0 & 1 \\
	\hline
	1 & 1 & 1 \\
	\hline
\end{tabular} &
\begin{tabular}{|c|c|c|}
	\hline
	\multicolumn{3}{|c|}{\textbf{或非门(NOR)}} \\
	\hline
	\multicolumn{2}{|c|}{\textbf{输入}} & \textbf{输出} \\
	\hline
	A & B & A NOR B \\
	\hline
	0 & 0 & 1 \\
	\hline
	0 & 1 & 0 \\
	\hline
	1 & 0 & 0 \\
	\hline
	1 & 1 & 0 \\
	\hline
\end{tabular}  \tabularnewline
\end{tabular}

\bigskip

\begin{tabular}{cc}%
\begin{tabular}{|c|c|c|}
	\hline
	\multicolumn{3}{|c|}{\textbf{异或门(XOR)}} \\
	\hline
	\multicolumn{2}{|c|}{\textbf{输入}} & \textbf{输出} \\
	\hline
	A & B & A XOR B \\
	\hline
	0 & 0 & 0 \\
	\hline
	0 & 1 & 1 \\
	\hline
	1 & 0 & 1 \\
	\hline
	1 & 1 & 0 \\
	\hline
\end{tabular} &
\begin{tabular}{|c|c|c|}
	\hline
	\multicolumn{3}{|c|}{\textbf{异或非门(NXOR)}} \\
	\hline
	\multicolumn{2}{|c|}{\textbf{输入}} & \textbf{输出} \\
	\hline
	A & B & A XNOR B \\
	\hline
	0 & 0 & 1 \\
	\hline
	0 & 1 & 0 \\
	\hline
	1 & 0 & 0 \\
	\hline
	1 & 1 & 1 \\
	\hline
\end{tabular} \tabularnewline
\end{tabular}

\newpage

% 后记
\chapter*{后记}

本书在撰写过程中得到了家人的大力支持,在此感谢家人的付出,如果没有他们无微不至的照顾,是不会有这本书的。

作为非无线电相关人员出身,笔者能力有限,此书仍有许多不足,望读者指正。

本书的最新版PDF文件可以在\url{https://github.com/mike2718/ham}免费下载。

\section*{许可}

%\textregistered\textcopyright
%\sffamily\textregistered\textcopyright

\sffamily{\textcopyright 2021 BG7XTQ}

% 参考文献
% \chapter*{参考文献}
\bibliography{cankaowenxian}

% 许可
\chapter*{许可}

\noindent \textcopyright 2021 BG7XTQ

\section*{字体许可}

\noindent PDF文件中使用了以下字体。
非常感谢各位字体开发者们。

\begin{verbatim}
Noto Sans, Noto Serif, Noto Sans SC, Noto Serif SC, Noto Sans Mono CJK SC, Noto Sans Mono

Copyright 2022 The Noto Project Authors (https://github.com/notofonts/)

This Font Software is licensed under the SIL Open Font License,
Version 1.1.

This license is copied below, and is also available with a FAQ at:
http://scripts.sil.org/OFL

\end{verbatim}


\end{document}
